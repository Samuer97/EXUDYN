% definition of structures


%++++++++++++++++++++++++++++++++++++++
\mysubsection{Structures for structural elements}
This section includes data structures for structural elements, surch as beams (and plates in future). These classes are used as interface between Python libraries for structural elements and Exudyn internal classes.

%+++++++++++++++++++++++++++++++++++

\mysubsubsubsection{PyBeamSection}
\label{sec:PyBeamSection}
Data structure for definition of 2D and 3D beam (cross) section mechanical properties. The beam has local coordinates, in which $X$ represents the beam centerline (beam axis) coordinate, being the neutral fiber w.r.t.\ bending; $Y$ and $Z$ are the local cross section coordinates. Note that most elements do not accept all parameters, which results in an error if those parameters (e.g., stiffness parameters) are non-zero.

\noindent PyBeamSection has the following items:
%reference manual TABLE
\begin{center}
  \footnotesize
  \begin{longtable}{| p{4.2cm} | p{2.5cm} | p{0.3cm} | p{3.0cm} | p{6cm} |}
    \hline
    \bf Name & \bf type / function return type & \bf size & \bf default value / function args & \bf description \\ \hline
    dampingMatrix &     Matrix6D &      &     np.zeros((6,6)) &     $\LU{c}{\Dm} \in \Rcal^{6 \times 6}\,$ [SI:Nsm$^2$, Nsm and Ns (mixed)] sectional linear damping matrix related to $\vp{\LU{c}{\nv}}{\LU{c}{\mv}} = \LU{c}{\Dm} \vp{\LU{c}{\tepsDot}}{\LU{c}{\tkappaDot}}$; note that this damping models is highly simplified and usually, it cannot be derived from material parameters; however, it can be used to adjust model damping to observed damping behavior.\\ \hline
    inertia &     Matrix3D &      &     [[0,0,0], [0,0,0], [0,0,0]] &     \tabnewline $\LU{c}{\Jm} \in \Rcal^{3 \times 3}\,$ [SI:kg$\,$m$^2$] sectional inertia for shear-deformable beams.\\ \hline
    massPerLength &     UReal &      &     0. &     $\rho A\,$ [SI:kg/m] mass per unit length of the beam\\ \hline
    stiffnessMatrix &     Matrix6D &      &     np.zeros((6,6)) &     $\LU{c}{\Cm} \in \Rcal^{6 \times 6}\,$ [SI:Nm$^2$, Nm and N (mixed)] sectional stiffness matrix related to $\vp{\LU{c}{\nv}}{\LU{c}{\mv}} = \LU{c}{\Cm} \vp{\LU{c}{\teps}}{\LU{c}{\tkappa}}$ with sectional normal force $\LU{c}{\nv}$, torque $\LU{c}{\mv}$, strain $\LU{c}{\teps}$ and curvature $\LU{c}{\tkappa}$, all quantities expressed in the cross section frame $c$.\\ \hline
	  \end{longtable}
	\end{center}


%+++++++++++++++++++++++++++++++++++

\mysubsubsubsection{BeamSectionGeometry}
\label{sec:BeamSectionGeometry}
Data structure for definition of 2D and 3D beam (cross) section geometrical properties. Used for visualization and contact.

\noindent BeamSectionGeometry has the following items:
%reference manual TABLE
\begin{center}
  \footnotesize
  \begin{longtable}{| p{4.2cm} | p{2.5cm} | p{0.3cm} | p{3.0cm} | p{6cm} |}
    \hline
    \bf Name & \bf type / function return type & \bf size & \bf default value / function args & \bf description \\ \hline
    crossSectionRadiusY &     UReal &      &     0. &     $c_Y\,$ [SI:m] $Y$ radius for circular cross section\\ \hline
    crossSectionRadiusZ &     UReal &      &     0. &     $c_Z\,$ [SI:m] $Z$ radius for circular cross section\\ \hline
    crossSectionType &     CrossSectionType &      &     CrossSectionType::Polygon &     \tabnewline Type of cross section: Polygon, Circular, etc.\\ \hline
    polygonalPoints &     Vector2DList &      &      &     $\pv_{pg}\,$ [SI: (m,m) ] list of polygonal ($Y,Z$) points in local beam cross section coordinates, defined in positive rotation direction\\ \hline
	  \end{longtable}
	\end{center}


%++++++++++++++++++++++++++++++++++++++
\mysubsection{Simulation settings}
This section includes hierarchical structures for simulation settings, e.g., time integration, static solver, Newton iteration and solution file export.

%+++++++++++++++++++++++++++++++++++

\mysubsubsubsection{SolutionSettings}
\label{sec:SolutionSettings}
General settings for exporting the solution (results) of a simulation.

\noindent SolutionSettings has the following items:
%reference manual TABLE
\begin{center}
  \footnotesize
  \begin{longtable}{| p{4.2cm} | p{2.5cm} | p{0.3cm} | p{3.0cm} | p{6cm} |}
    \hline
    \bf Name & \bf type / function return type & \bf size & \bf default value / function args & \bf description \\ \hline
    appendToFile &     bool &      &     False &     flag (true/false); if true, solution and solverInformation is appended to existing file (otherwise created); in BINARY mode, files are always replaced and this parameter is ineffective!\\ \hline
    binarySolutionFile &     bool &      &     False &     if true, the solution file is written in binary format for improved speed and smaller file sizes; setting outputPrecision >= 8 uses double (8 bytes), otherwise float (4 bytes) is used; note that appendToFile is ineffective and files are always replaced without asking! If not provided, file ending will read .sol in case of binary files and .txt in case of text files\\ \hline
    coordinatesSolutionFileName &     FileName &      &     'coordinatesSolution' &     \tabnewline filename and (relative) path of solution file (coordinatesSolutionFile) containing all multibody system coordinates versus time; directory will be created if it does not exist; character encoding of string is up to your filesystem, but for compatibility, it is recommended to use letters, numbers and '\_' only; filename ending will be added automatically if not provided: .txt in case of text mode and .sol in case of binary solution files (binarySolutionFile=True)\\ \hline
    exportAccelerations &     bool &      &     True &     add \hac{ODE2} accelerations to solution file (coordinatesSolutionFile)\\ \hline
    exportAlgebraicCoordinates &     bool &      &     True &     add algebraicCoordinates (=Lagrange multipliers) to solution file (coordinatesSolutionFile)\\ \hline
    exportDataCoordinates &     bool &      &     True &     add DataCoordinates to solution file (coordinatesSolutionFile)\\ \hline
    exportODE1Velocities &     bool &      &     True &     add coordinatesODE1\_t to solution file (coordinatesSolutionFile)\\ \hline
    exportVelocities &     bool &      &     True &     add \hac{ODE2} velocities to solution file (coordinatesSolutionFile)\\ \hline
    flushFilesDOF &     PInt &      &     10000 &     number of DOF, above which solution file (coordinatesSolutionFile) buffers are always flushed, irrespectively of whether flushFilesImmediately is set True or False (see also flushFilesImmediately); for larger files, writing takes so much time that flushing does not add considerable time\\ \hline
    flushFilesImmediately &     bool &      &     False &     flush file buffers after every solution period written (coordinatesSolutionFile and sensor files); if set False, the output is written through a buffer, which is highly efficient, but during simulation, files may be always in an incomplete state; if set True, this may add a large amount of CPU time as the process waits until files are really written to hard disc (especially for simulation of small scale systems, writing 10.000s of time steps; at least 5us per step/file, depending on hardware)\\ \hline
    outputPrecision &     UInt &      &     10 &     precision for floating point numbers written to solution and sensor files\\ \hline
    recordImagesInterval &     Real &      &     -1. &     record frames (images) during solving: amount of time to wait until next image (frame) is recorded; set recordImages = -1. if no images shall be recorded; set, e.g., recordImages = 0.01 to record an image every 10 milliseconds (requires that the time steps / load steps are sufficiently small!); for file names, etc., see VisualizationSettings.exportImages\\ \hline
    restartFileName &     FileName &      &     'restartFile.txt' &     filename and (relative) path of text file for storing solution after every restartWritePeriod if writeRestartFile=True; backup file is created with ending .bck, which should be used if restart file is crashed; use Python utility function InitializeFromRestartFile(...) to consistently restart\\ \hline
    restartWritePeriod &     UReal &      &     0.01 &     time span (period), determines how often the restart file is updated; this should be often enough to enable restart without too much loss of data; too low values may influence performance\\ \hline
    sensorsAppendToFile &     bool &      &     False &     flag (true/false); if true, sensor output is appended to existing file (otherwise created) or in case of internal storage, it is appended to existing currently stored data; this allows storing sensor values over different simulations\\ \hline
    sensorsWriteFileFooter &     bool &      &     False &     flag (true/false); if true, file footer is written for sensor output (turn off, e.g. for multiple runs of time integration)\\ \hline
    sensorsWriteFileHeader &     bool &      &     True &     flag (true/false); if true, file header is written for sensor output (turn off, e.g. for multiple runs of time integration)\\ \hline
    sensorsWritePeriod &     UReal &      &     0.01 &     time span (period), determines how often the sensor output is written to file or internal storage during a simulation\\ \hline
    solutionInformation &     String &      &     '' &     special information added to header of solution file (e.g. parameters and settings, modes, ...); character encoding my be UTF-8, restricted to characters in \refSection{sec:utf8}, but for compatibility, it is recommended to use ASCII characters only (95 characters, see wiki)\\ \hline
    solutionWritePeriod &     UReal &      &     0.01 &     time span (period), determines how often the solution file (coordinatesSolutionFile) is written during a simulation\\ \hline
    solverInformationFileName &     FileName &      &     'solverInformation.txt' &     \tabnewline filename and (relative) path of text file showing detailed information during solving; detail level according to yourSolver.verboseModeFile; if solutionSettings.appendToFile is true, the information is appended in every solution step; directory will be created if it does not exist; character encoding of string is up to your filesystem, but for compatibility, it is recommended to use letters, numbers and '\_' only\\ \hline
    writeFileFooter &     bool &      &     True &     flag (true/false); if true, information at end of simulation is written: convergence, total solution time, statistics\\ \hline
    writeFileHeader &     bool &      &     True &     flag (true/false); if true, file header is written (turn off, e.g. for multiple runs of time integration)\\ \hline
    writeInitialValues &     bool &      &     True &     flag (true/false); if true, initial values are exported for the start time; applies to coordinatesSolution and sensor files; this may not be wanted in the append file mode if the initial values are identical to the final values of a previous computation\\ \hline
    writeRestartFile &     bool &      &     False &     flag (true/false), which determines if restart file is written regularly, see restartFileName for details\\ \hline
    writeSolutionToFile &     bool &      &     True &     flag (true/false), which determines if (global) solution vector is written to the solution file (coordinatesSolutionFile); standard quantities that are written are: solution is written as displacements and coordinatesODE1; for additional coordinates in the solution file, see the options below\\ \hline
	  \end{longtable}
	\end{center}


%+++++++++++++++++++++++++++++++++++

\mysubsubsubsection{NumericalDifferentiationSettings}
\label{sec:NumericalDifferentiationSettings}
Settings for numerical differentiation of a function (needed for computation of numerical jacobian e.g. in implizit integration).

\noindent NumericalDifferentiationSettings has the following items:
%reference manual TABLE
\begin{center}
  \footnotesize
  \begin{longtable}{| p{4.2cm} | p{2.5cm} | p{0.3cm} | p{3.0cm} | p{6cm} |}
    \hline
    \bf Name & \bf type / function return type & \bf size & \bf default value / function args & \bf description \\ \hline
    addReferenceCoordinatesToEpsilon &     \tabnewline bool &      &     False &     True: for the size estimation of the differentiation parameter, the reference coordinate $q^{Ref}_i$ is added to \hac{ODE2} coordinates --> see; False: only the current coordinate is used for size estimation of the differentiation parameter\\ \hline
    doSystemWideDifferentiation &     bool &      &     False &     True: system wide differentiation (e.g. all \hac{ODE2} equations w.r.t. all \hac{ODE2} coordinates); False: only local (object) differentiation\\ \hline
    forAE &     bool &      &     False &     flag (true/false); false = perform direct computation of jacobian for algebraic equations (AE), true = use numerical differentiation; as there must always exist an analytical implemented jacobian for AE, 'true' should only be used for verification\\ \hline
    forODE2 &     bool &      &     False &     flag (true/false); false = perform direct computation (e.g., using autodiff) of jacobian for ODE2 equations, true = use numerical differentiation; numerical differentiation is less efficient and may lead to numerical problems, but may smoothen problems of analytical derivatives; sometimes the analytical derivative may neglect terms\\ \hline
    forODE2connectors &     bool &      &     False &     flag (true/false); false: if also forODE2==false, perform direct computation of jacobian for ODE2 terms for connectors; else: use numerical differentiation; NOTE: THIS FLAG IS FOR DEVELOPMENT AND WILL BE ERASED IN FUTURE\\ \hline
    jacobianConnectorDerivative &     bool &      &     True &     True: for analytic Jacobians of connectors, the Jacobian derivative is computed, causing additional CPU costs and not beeing available for all connectors or markers (thus switching to numerical differentiation); False: Jacobian derivative is neglected in analytic Jacobians (but included in numerical Jacobians), which often has only minor influence on convergence\\ \hline
    minimumCoordinateSize &     UReal &      &     1e-2 &     minimum size of coordinates in relative differentiation parameter\\ \hline
    relativeEpsilon &     UReal &      &     1e-7 &     relative differentiation parameter epsilon; the numerical differentiation parameter $\varepsilon$ follows from the formula ($\varepsilon = \varepsilon_\mathrm{relative}*max(q_{min}, |q_i + [q^{Ref}_i]|)$, with $\varepsilon_\mathrm{relative}$=relativeEpsilon, $q_{min} = $minimumCoordinateSize, $q_i$ is the current coordinate which is differentiated, and $qRef_i$ is the reference coordinate of the current coordinate\\ \hline
	  \end{longtable}
	\end{center}


%+++++++++++++++++++++++++++++++++++

\mysubsubsubsection{DiscontinuousSettings}
\label{sec:DiscontinuousSettings}
Settings for discontinuous iterations, as in contact, friction, plasticity and general switching phenomena.

\noindent DiscontinuousSettings has the following items:
%reference manual TABLE
\begin{center}
  \footnotesize
  \begin{longtable}{| p{4.2cm} | p{2.5cm} | p{0.3cm} | p{3.0cm} | p{6cm} |}
    \hline
    \bf Name & \bf type / function return type & \bf size & \bf default value / function args & \bf description \\ \hline
    ignoreMaxIterations &     bool &      &     True &     continue solver if maximum number of discontinuous (post Newton) iterations is reached (ignore tolerance)\\ \hline
    iterationTolerance &     UReal &      &     1 &     absolute tolerance for discontinuous (post Newton) iterations; the errors represent absolute residuals and can be quite high\\ \hline
    maxIterations &     UInt &      &     5 &     maximum number of discontinuous (post Newton) iterations\\ \hline
	  \end{longtable}
	\end{center}


%+++++++++++++++++++++++++++++++++++

\mysubsubsubsection{NewtonSettings}
\label{sec:NewtonSettings}
Settings for Newton method used in static or dynamic simulation.

\noindent NewtonSettings has the following items:
%reference manual TABLE
\begin{center}
  \footnotesize
  \begin{longtable}{| p{4.2cm} | p{2.5cm} | p{0.3cm} | p{3.0cm} | p{6cm} |}
    \hline
    \bf Name & \bf type / function return type & \bf size & \bf default value / function args & \bf description \\ \hline
    numericalDifferentiation &     NumericalDifferentiationSettings &      &      &     numerical differentiation parameters for numerical jacobian (e.g. Newton in static solver or implicit time integration)\\ \hline
    absoluteTolerance &     UReal &      &     1e-10 &     absolute tolerance of residual for Newton (needed e.g. if residual is fulfilled right at beginning); condition: sqrt(q*q)/numberOfCoordinates <= absoluteTolerance\\ \hline
    adaptInitialResidual &     bool &      &     True &     flag (true/false); false = standard; True: if initialResidual is very small (or zero), it may increase significantely in the first Newton iteration; to achieve relativeTolerance, the initialResidual will by updated by a higher residual within the first Newton iteration\\ \hline
    maximumSolutionNorm &     UReal &      &     1e38 &     this is the maximum allowed value for solutionU.L2NormSquared() which is the square of the square norm (i.e., value=$u_1^2$+$u_2^2$+...), and solutionV/A...; if the norm of solution vectors is larger, Newton method is stopped; the default value is chosen such that it would still work for single precision numbers (float)\\ \hline
    maxIterations &     UInt &      &     25 &     maximum number of iterations (including modified + restart Newton iterations); after that total number of iterations, the static/dynamic solver refines the step size or stops with an error\\ \hline
    maxModifiedNewtonIterations &     UInt &      &     8 &     maximum number of iterations for modified Newton (without Jacobian update); after that number of iterations, the modified Newton method gets a jacobian update and is further iterated\\ \hline
    maxModifiedNewtonRestartIterations &     \tabnewline UInt &      &     7 &     maximum number of iterations for modified Newton after a Jacobian update; after that number of iterations, the full Newton method is started for this step\\ \hline
    modifiedNewtonContractivity &     PReal &      &     0.5 &     maximum contractivity (=reduction of error in every Newton iteration) accepted by modified Newton; if contractivity is greater, a Jacobian update is computed\\ \hline
    modifiedNewtonJacUpdatePerStep &     \tabnewline bool &      &     False &     True: compute Jacobian at every time step (or static step), but not in every Newton iteration (except for bad convergence ==> switch to full Newton)\\ \hline
    newtonResidualMode &     UInt &      &     0 &     0 ... use residual for computation of error (standard); 1 ... use \hac{ODE2} and \hac{ODE1} newton increment for error (set relTol and absTol to same values!) ==> may be advantageous if residual is zero, e.g., in kinematic analysis; TAKE CARE with this flag\\ \hline
    relativeTolerance &     UReal &      &     1e-8 &     relative tolerance of residual for Newton (general goal of Newton is to decrease the residual by this factor)\\ \hline
    useModifiedNewton &     bool &      &     False &     True: compute Jacobian only at first call to solver; the Jacobian (and its factorizations) is not computed in each Newton iteration, even not in every (time integration) step; False: Jacobian (and factorization) is computed in every Newton iteration (default, but may be costly)\\ \hline
    useNewtonSolver &     bool &      &     True &     flag (true/false); false = linear computation, true = use Newton solver for nonlinear solution\\ \hline
    weightTolerancePerCoordinate &     bool &      &     False &     flag (true/false); false = compute error as L2-Norm of residual; true = compute error as (L2-Norm of residual) / (sqrt(number of coordinates)), which can help to use common tolerance independent of system size\\ \hline
	  \end{longtable}
	\end{center}


%+++++++++++++++++++++++++++++++++++

\mysubsubsubsection{GeneralizedAlphaSettings}
\label{sec:GeneralizedAlphaSettings}
Settings for generalized-alpha, implicit trapezoidal or Newmark time integration methods.

\noindent GeneralizedAlphaSettings has the following items:
%reference manual TABLE
\begin{center}
  \footnotesize
  \begin{longtable}{| p{4.2cm} | p{2.5cm} | p{0.3cm} | p{3.0cm} | p{6cm} |}
    \hline
    \bf Name & \bf type / function return type & \bf size & \bf default value / function args & \bf description \\ \hline
    computeInitialAccelerations &     bool &      &     True &     True: compute initial accelerations from system EOM in acceleration form; NOTE that initial accelerations that are following from user functions in constraints are not considered for now! False: use zero accelerations\\ \hline
    lieGroupAddTangentOperator &     bool &      &     True &     True: for Lie group nodes, the integrator adds the tangent operator for stiffness and constraint matrices, for improved Newton convergence; not available for sparse matrix mode (EigenSparse)\\ \hline
    newmarkBeta &     UReal &      &     0.25 &     value beta for Newmark method; default value beta = $\frac 1 4$ corresponds to (undamped) trapezoidal rule\\ \hline
    newmarkGamma &     UReal &      &     0.5 &     value gamma for Newmark method; default value gamma = $\frac 1 2$ corresponds to (undamped) trapezoidal rule\\ \hline
    resetAccelerations &     bool &      &     False &     this flag only affects if computeInitialAccelerations=False: if resetAccelerations=True, accelerations are set zero in the solver function InitializeSolverInitialConditions; this may be unwanted in case of repeatedly called SolveSteps() and in cases where solutions shall be prolonged from previous computations\\ \hline
    spectralRadius &     UReal &      &     0.9 &     spectral radius for Generalized-alpha solver; set this value to 1 for no damping or to 0 < spectralRadius < 1 for damping of high-frequency dynamics; for position-level constraints (index 3), spectralRadius must be < 1\\ \hline
    useIndex2Constraints &     bool &      &     False &     set useIndex2Constraints = true in order to use index2 (velocity level constraints) formulation\\ \hline
    useNewmark &     bool &      &     False &     if true, use Newmark method with beta and gamma instead of generalized-Alpha\\ \hline
	  \end{longtable}
	\end{center}


%+++++++++++++++++++++++++++++++++++

\mysubsubsubsection{ExplicitIntegrationSettings}
\label{sec:ExplicitIntegrationSettings}
Settings for explicit solvers, like Explicit Euler, RK44, ODE23, DOPRI5 and others. The settings may significantely influence performance.

\noindent ExplicitIntegrationSettings has the following items:
%reference manual TABLE
\begin{center}
  \footnotesize
  \begin{longtable}{| p{4.2cm} | p{2.5cm} | p{0.3cm} | p{3.0cm} | p{6cm} |}
    \hline
    \bf Name & \bf type / function return type & \bf size & \bf default value / function args & \bf description \\ \hline
    computeEndOfStepAccelerations &     \tabnewline bool &      &     True &     accelerations are computed at stages of the explicit integration scheme; if the user needs accelerations at the end of a step, this flag needs to be activated; if True, this causes a second call to the RHS of the equations, which may DOUBLE COMPUTATIONAL COSTS for one-step-methods; if False, the accelerations are re-used from the last stage, being slightly different\\ \hline
    computeMassMatrixInversePerBody &     \tabnewline bool &      &     False &     If true, the solver assumes the bodies to be independent and computes the inverse of the mass matrix for all bodies independently; this may lead to WRONG RESULTS, if bodies share nodes, e.g., two MassPoint objects put on the same node or a beam with a mass point attached at a shared node; however, it may speed up explicit time integration for large systems significantly (multi-threaded)\\ \hline
    dynamicSolverType &     DynamicSolverType &      &     DynamicSolverType::DOPRI5 &     \tabnewline selection of explicit solver type (DOPRI5, ExplicitEuler, ExplicitMidpoint, RK44, RK67, ...), for detailed description see DynamicSolverType, \refSection{sec:DynamicSolverType}, but only referring to explicit solvers.\\ \hline
    eliminateConstraints &     bool &      &     True &     True: make explicit solver work for simple CoordinateConstraints, which are eliminated for ground constraints (e.g. fixed nodes in finite element models). False: incompatible constraints are ignored (BE CAREFUL)!\\ \hline
    useLieGroupIntegration &     bool &      &     True &     True: use Lie group integration for rigid body nodes; must be turned on for Lie group nodes (without data coordinates) to work properly; does not work for nodes with data coordinates!\\ \hline
	  \end{longtable}
	\end{center}


%+++++++++++++++++++++++++++++++++++

\mysubsubsubsection{TimeIntegrationSettings}
\label{sec:TimeIntegrationSettings}
General parameters used in time integration; specific parameters are provided in the according solver settings, e.g. for generalizedAlpha.

\noindent TimeIntegrationSettings has the following items:
%reference manual TABLE
\begin{center}
  \footnotesize
  \begin{longtable}{| p{4.2cm} | p{2.5cm} | p{0.3cm} | p{3.0cm} | p{6cm} |}
    \hline
    \bf Name & \bf type / function return type & \bf size & \bf default value / function args & \bf description \\ \hline
    discontinuous &     DiscontinuousSettings &      &      &     parameters for treatment of discontinuities\\ \hline
    explicitIntegration &     ExplicitIntegrationSettings &      &      &     special parameters for explicit time integration\\ \hline
    generalizedAlpha &     GeneralizedAlphaSettings &      &      &     parameters for generalized-alpha, implicit trapezoidal rule or Newmark (options only apply for these methods)\\ \hline
    newton &     NewtonSettings &      &      &     parameters for Newton method; used for implicit time integration methods only\\ \hline
    absoluteTolerance &     UReal &      &     1e-8 &     $a_{tol}$: if automaticStepSize=True, absolute tolerance for the error control; must fulfill $a_{tol} > 0$; see \refSection{sec:ExplicitSolver}\\ \hline
    adaptiveStep &     bool &      &     True &     True: the step size may be reduced if step fails; no automatic stepsize control\\ \hline
    adaptiveStepDecrease &     UReal &      &     0.5 &     Multiplicative factor (MUST BE: 0 < factor < 1) for step size to decrese due to discontinuousIteration or Newton errors\\ \hline
    adaptiveStepIncrease &     UReal &      &     2 &     Multiplicative factor (MUST BE > 1) for step size to increase after previous step reduction due to discontinuousIteration or Newton errors\\ \hline
    adaptiveStepRecoveryIterations &     \tabnewline UInt &      &     7 &     Number of max. (Newton iterations + discontinuous iterations) at which a step increase is considered; in order to immediately increase steps after reduction, chose a high value\\ \hline
    adaptiveStepRecoverySteps &     UInt &      &     10 &     Number of steps needed after which steps will be increased after previous step reduction due to discontinuousIteration or Newton errors\\ \hline
    automaticStepSize &     bool &      &     True &     True: for specific integrators with error control (e.g., DOPRI5), compute automatic step size based on error estimation; False: constant step size (step may be reduced if adaptiveStep=True); the maximum stepSize reads $h = h_{max} = \frac{t_{end} - t_{start}}{n_{steps}}$\\ \hline
    endTime &     UReal &      &     1 &     $t_{end}$: end time of time integration\\ \hline
    initialStepSize &     UReal &      &     0 &     $h_{init}$: if automaticStepSize=True, initial step size; if initialStepSize==0, max. stepSize, which is (endTime-startTime)/numberOfSteps, is used as initial guess; a good choice of initialStepSize may help the solver to start up faster.\\ \hline
    minimumStepSize &     PReal &      &     1e-8 &     $h_{min}$: if automaticStepSize=True or adaptiveStep=True: lower limit of time step size, before integrator stops with adaptiveStep; lower limit of automaticStepSize control (continues but raises warning)\\ \hline
    numberOfSteps &     PInt &      &     100 &     $n_{steps}$: number of steps in time integration; (maximum) stepSize $h$ is computed from $h = \frac{t_{end} - t_{start}}{n_{steps}}$; for automatic stepsize control, this stepSize is the maximum steps size, $h_{max} = h$\\ \hline
    realtimeFactor &     PReal &      &     1 &     if simulateInRealtime=True, this factor is used to make the simulation slower than realtime (factor < 1) or faster than realtime (factor > 1)\\ \hline
    realtimeWaitMicroseconds &     PInt &      &     1000 &     if simulateInRealtime=True, a loop runs which waits realtimeWaitMicroseconds until checking again if the realtime is reached; using larger values leads to less CPU usage but less accurate realtime accuracy; smaller values (< 1000) increase CPU usage but improve realtime accuracy\\ \hline
    relativeTolerance &     UReal &      &     1e-8 &     $r_{tol}$: if automaticStepSize=True, relative tolerance for the error control; must fulfill $r_{tol} \ge 0$; see \refSection{sec:ExplicitSolver}\\ \hline
    reuseConstantMassMatrix &     bool &      &     True &     True: does not recompute constant mass matrices (e.g. of some finite elements, mass points, etc.); if False, it always recomputes the mass matrix (e.g. needed, if user changes mass parameters via Python)\\ \hline
    simulateInRealtime &     bool &      &     False &     True: simulate in realtime; the solver waits for computation of the next step until the CPU time reached the simulation time; if the simulation is slower than realtime, it simply continues\\ \hline
    startTime &     UReal &      &     0 &     $t_{start}$: start time of time integration (usually set to zero)\\ \hline
    stepInformation &     UInt &      &     67 &     add up the following binary flags: 0 ... show only step time, 1 ... show time to go, 2 ... show newton iterations (Nit) per step or period, 4 ... show Newton jacobians (jac) per step or period, 8 ... show discontinuous iterations (Dit) per step or period, 16 ... show step size (dt), 32 ... show CPU time spent; 64 ... show adaptive step reduction warnings; 128 ... show step increase information; 1024 ... show every time step; time is usually shown in fractions of seconds (s), hours (h), or days\\ \hline
    stepSizeMaxIncrease &     UReal &      &     2 &     $f_{maxInc}$: if automaticStepSize=True, maximum increase of step size per step, see \refSection{sec:ExplicitSolver}; make this factor smaller (but $> 1$) if too many rejected steps\\ \hline
    stepSizeSafety &     UReal &      &     0.90 &     $r_{sfty}$: if automaticStepSize=True, a safety factor added to estimated optimal step size, in order to prevent from many rejected steps, see \refSection{sec:ExplicitSolver}. Make this factor smaller if many steps are rejected.\\ \hline
    verboseMode &     UInt &      &     0 &     0 ... no output, 1 ... show short step information every 2 seconds (every 30 seconds after 1 hour CPU time), 2 ... show every step information, 3 ... show also solution vector, 4 ... show also mass matrix and jacobian (implicit methods), 5 ... show also Jacobian inverse (implicit methods)\\ \hline
    verboseModeFile &     UInt &      &     0 &     same behaviour as verboseMode, but outputs all solver information to file\\ \hline
	  \end{longtable}
	\end{center}


%+++++++++++++++++++++++++++++++++++

\mysubsubsubsection{StaticSolverSettings}
\label{sec:StaticSolverSettings}
Settings for static solver linear or nonlinear (Newton).

\noindent StaticSolverSettings has the following items:
%reference manual TABLE
\begin{center}
  \footnotesize
  \begin{longtable}{| p{4.2cm} | p{2.5cm} | p{0.3cm} | p{3.0cm} | p{6cm} |}
    \hline
    \bf Name & \bf type / function return type & \bf size & \bf default value / function args & \bf description \\ \hline
    discontinuous &     DiscontinuousSettings &      &      &     parameters for treatment of discontinuities\\ \hline
    newton &     NewtonSettings &      &      &     parameters for Newton method (e.g. in static solver or time integration)\\ \hline
    adaptiveStep &     bool &      &     True &     True: use step reduction if step fails; False: fixed step size\\ \hline
    adaptiveStepDecrease &     UReal &      &     0.25 &     Multiplicative factor (MUST BE: 0 < factor < 1) for step size to decrese due to discontinuousIteration or Newton errors\\ \hline
    adaptiveStepIncrease &     UReal &      &     2 &     Multiplicative factor (MUST BE > 1) for step size to increase after previous step reduction due to discontinuousIteration or Newton errors\\ \hline
    adaptiveStepRecoveryIterations &     \tabnewline UInt &      &     7 &     Number of max. (Newton iterations + discontinuous iterations) at which a step increase is considered; in order to immediately increase steps after reduction, chose a high value\\ \hline
    adaptiveStepRecoverySteps &     UInt &      &     4 &     Number of steps needed after which steps will be increased after previous step reduction due to discontinuousIteration or Newton errors\\ \hline
    loadStepDuration &     PReal &      &     1 &     quasi-time for all load steps (added to current time in load steps)\\ \hline
    loadStepGeometric &     bool &      &     False &     if loadStepGeometric=false, the load steps are incremental (arithmetic series, e.g. 0.1,0.2,0.3,...); if true, the load steps are increased in a geometric series, e.g. for $n=8$ numberOfLoadSteps and $d = 1000$ loadStepGeometricRange, it follows: $1000^{1/8}/1000=0.00237$, $1000^{2/8}/1000=0.00562$, $1000^{3/8}/1000=0.0133$, ..., $1000^{7/8}/1000=0.422$, $1000^{8/8}/1000=1$\\ \hline
    loadStepGeometricRange &     PReal &      &     1000 &     if loadStepGeometric=true, the load steps are increased in a geometric series, see loadStepGeometric\\ \hline
    loadStepStart &     UReal &      &     0 &     a quasi time, which can be used for the output (first column) as well as for time-dependent forces; quasi-time is increased in every step i by loadStepDuration/numberOfLoadSteps; loadStepTime = loadStepStart + i*loadStepDuration/numberOfLoadSteps, but loadStepStart untouched ==> increment by user\\ \hline
    minimumStepSize &     PReal &      &     1e-8 &     lower limit of step size, before nonlinear solver stops\\ \hline
    numberOfLoadSteps &     PInt &      &     1 &     number of load steps; if numberOfLoadSteps=1, no load steps are used and full forces are applied at once\\ \hline
    stabilizerODE2term &     UReal &      &     0 &     add mass-proportional stabilizer term in \hac{ODE2} part of jacobian for stabilization (scaled ), e.g. of badly conditioned problems; the diagnoal terms are scaled with $stabilizer = (1-loadStepFactor^2)$, and go to zero at the end of all load steps: $loadStepFactor=1$ -> $stabilizer = 0$\\ \hline
    stepInformation &     UInt &      &     67 &     add up the following binary flags: 0 ... show only step time, 1 ... show time to go, 2 ... show newton iterations (Nit) per step or period, 4 ... show Newton jacobians (jac) per step or period, 8 ... show discontinuous iterations (Dit) per step or period, 16 ... show step size (dt), 32 ... show CPU time spent; 64 ... show adaptive step reduction warnings; 128 ... show step increase information; 1024 ... show every time step; time is usually shown in fractions of seconds (s), hours (h), or days\\ \hline
    useLoadFactor &     bool &      &     True &     True: compute a load factor $\in [0,1]$ from static step time; all loads are scaled by the load factor; False: loads are always scaled with 1 -- use this option if time dependent loads use a userFunction\\ \hline
    verboseMode &     UInt &      &     1 &     0 ... no output, 1 ... show errors and load steps, 2 ... show short Newton step information (error), 3 ... show also solution vector, 4 ... show also jacobian, 5 ... show also Jacobian inverse\\ \hline
    verboseModeFile &     UInt &      &     0 &     same behaviour as verboseMode, but outputs all solver information to file\\ \hline
	  \end{longtable}
	\end{center}


%+++++++++++++++++++++++++++++++++++

\mysubsubsubsection{LinearSolverSettings}
\label{sec:LinearSolverSettings}
Settings for linear solver, both dense and sparse (Eigen).

\noindent LinearSolverSettings has the following items:
%reference manual TABLE
\begin{center}
  \footnotesize
  \begin{longtable}{| p{4.2cm} | p{2.5cm} | p{0.3cm} | p{3.0cm} | p{6cm} |}
    \hline
    \bf Name & \bf type / function return type & \bf size & \bf default value / function args & \bf description \\ \hline
    ignoreRedundantConstraints &     bool &      &     False &     [ONLY implemented for dense matrices] False: standard way, fails if redundant equations or singular matrices occur; True: if redundant constraints appear, the solver tries to resolve them by setting according Lagrange multipliers to zero; in case of redundant constraints, this may help, but it may lead to erroneous behaviour\\ \hline
    ignoreSingularJacobian &     bool &      &     False &     [ONLY implemented for dense matrices] False: standard way, fails if jacobian is singular; True: if singularities appear in jacobian (e.g. no equation attributed to a node, redundant equations, zero mass matrix, zero eigenvalue for static problem, etc.), the jacobian inverse is resolved such that according solution variables are set to zero; this may help, but it MAY LEAD TO ERRONEOUS BEHAVIOUR; for static problems, this may suppress static motion or resolve problems in case of instabilities, but should in general be considered with care!\\ \hline
    pivotTreshold &     PReal &      &     0 &     treshold for dense linear solver, can be used to detect close to singular solutions, setting this to, e.g., 1e-12; solver then reports on equations that are causing close to singularity\\ \hline
    reuseAnalyzedPattern &     bool &      &     False &     [ONLY available for sparse matrices] True: the Eigen SparseLU solver offers the possibility to reuse an analyzed pattern of a previous factorization; this may reduce total factorization time by a factor of 2 or 3, depending on the matrix type; however, if the matrix patterns heavily change between computations, this may even slow down performance; this flag is set for SparseMatrices in InitializeSolverData(...) and should be handled with care!\\ \hline
    showCausingItems &     bool &      &     True &     False: no output, if solver fails; True: if redundant equations appear, they are resolved such that according solution variables are set to zero; in case of redundant constraints, this may help, but it may lead to erroneous behaviour; for static problems, this may suppress static motion or resolve problems in case of instabilities, but should in general be considered with care!\\ \hline
	  \end{longtable}
	\end{center}


%+++++++++++++++++++++++++++++++++++

\mysubsubsubsection{Parallel}
\label{sec:Parallel}
Settings for linear solver, both dense and sparse (Eigen).

\noindent Parallel has the following items:
%reference manual TABLE
\begin{center}
  \footnotesize
  \begin{longtable}{| p{4.2cm} | p{2.5cm} | p{0.3cm} | p{3.0cm} | p{6cm} |}
    \hline
    \bf Name & \bf type / function return type & \bf size & \bf default value / function args & \bf description \\ \hline
    multithreadedLLimitJacobians &     PInt &      &     20 &     compute jacobians (ODE2, AE, ...) multi-threaded; this is the limit number of according objects from which on parallelization is used; flag is copied into MainSystem internal flag at InitializeSolverData(...)\\ \hline
    multithreadedLLimitLoads &     PInt &      &     20 &     compute loads multi-threaded; this is the limit number of loads from which on parallelization is used; flag is copied into MainSystem internal flag at InitializeSolverData(...)\\ \hline
    multithreadedLLimitMassMatrices &     \tabnewline PInt &      &     20 &     compute bodies mass matrices multi-threaded; this is the limit number of bodies from which on parallelization is used; flag is copied into MainSystem internal flag at InitializeSolverData(...)\\ \hline
    multithreadedLLimitResiduals &     PInt &      &     20 &     compute RHS vectors, AE, and reaction forces multi-threaded; this is the limit number of objects from which on parallelization is used; flag is copied into MainSystem internal flag at InitializeSolverData(...)\\ \hline
    numberOfThreads &     PInt &      &     1 &     number of threads used for parallel computation (1 == scalar processing); do not use more threads than available threads (in most cases it is good to restrict to the number of cores); currently, only one solver can be started with multithreading; if you use several mbs in parallel (co-simulation), you should use serial computing\\ \hline
    taskSplitMinItems &     PInt &      &     50 &     number of items from which on the tasks are split into subtasks (which slightly increases threading performance; this may be critical for smaller number of objects, should be roughly between 50 and 5000; flag is copied into MainSystem internal flag at InitializeSolverData(...)\\ \hline
    taskSplitTasksPerThread &     PInt &      &     16 &     this is the number of subtasks that every thread receives; minimum is 1, the maximum should not be larger than 100; this factor is 1 as long as the taskSplitMinItems is not reached; flag is copied into MainSystem internal flag at InitializeSolverData(...)\\ \hline
	  \end{longtable}
	\end{center}


%+++++++++++++++++++++++++++++++++++

\mysubsubsubsection{SimulationSettings}
\label{sec:SimulationSettings}
General Settings for simulation; according settings for solution and solvers are given in subitems of this structure. 

\noindent SimulationSettings has the following items:
%reference manual TABLE
\begin{center}
  \footnotesize
  \begin{longtable}{| p{4.2cm} | p{2.5cm} | p{0.3cm} | p{3.0cm} | p{6cm} |}
    \hline
    \bf Name & \bf type / function return type & \bf size & \bf default value / function args & \bf description \\ \hline
    linearSolverSettings &     LinearSolverSettings &      &      &     linear solver parameters (used for dense and sparse solvers)\\ \hline
    parallel &     Parallel &      &      &     parameters for vectorized and parallelized (multi-threaded) computations\\ \hline
    solutionSettings &     SolutionSettings &      &      &     settings for solution files\\ \hline
    staticSolver &     StaticSolverSettings &      &      &     static solver parameters\\ \hline
    timeIntegration &     TimeIntegrationSettings &      &      &     time integration parameters\\ \hline
    cleanUpMemory &     bool &      &     False &     True: solvers will free memory at exit (recommended for large systems); False: keep allocated memory for repeated computations to increase performance\\ \hline
    displayComputationTime &     bool &      &     False &     display computation time statistics at end of solving\\ \hline
    displayGlobalTimers &     bool &      &     True &     display global timer statistics at end of solving (e.g., for contact, but also for internal timings during development)\\ \hline
    displayStatistics &     bool &      &     False &     display general computation information at end of time step (steps, iterations, function calls, step rejections, ...\\ \hline
    linearSolverType &     LinearSolverType &      &     LinearSolverType::EXUdense &     \tabnewline selection of numerical linear solver: exu.LinearSolverType.EXUdense (dense matrix inverse), exu.LinearSolverType.EigenSparse (sparse matrix LU-factorization), ... (enumeration type)\\ \hline
    outputPrecision &     UInt &      &     6 &     precision for floating point numbers written to console; e.g. values written by solver\\ \hline
    pauseAfterEachStep &     bool &      &     False &     pause after every time step or static load step(user press SPACE)\\ \hline
	  \end{longtable}
	\end{center}


%++++++++++++++++++++++++++++++++++++++
\mysubsection{Visualization settings}
This section includes hierarchical structures for visualization settings, e.g., drawing of nodes, bodies, connectors, loads and markers and furthermore openGL, window and save image options. For further information, see \refSection{sec:overview:basics:visualizationsettings}.

%+++++++++++++++++++++++++++++++++++

\mysubsubsubsection{VSettingsGeneral}
\label{sec:VSettingsGeneral}
General settings for visualization.

\noindent VSettingsGeneral has the following items:
%reference manual TABLE
\begin{center}
  \footnotesize
  \begin{longtable}{| p{4.2cm} | p{2.5cm} | p{0.3cm} | p{3.0cm} | p{6cm} |}
    \hline
    \bf Name & \bf type / function return type & \bf size & \bf default value / function args & \bf description \\ \hline
    autoFitScene &     bool &      &     True &     automatically fit scene within startup after StartRenderer()\\ \hline
    axesTiling &     PInt &      &     12 &     global number of segments for drawing axes cylinders and cones (reduce this number, e.g. to 4, if many axes are drawn)\\ \hline
    backgroundColor &     Float4 &     4 &     [1.0,1.0,1.0,1.0] &     \tabnewline red, green, blue and alpha values for background color of render window (white=[1,1,1,1]; black = [0,0,0,1])\\ \hline
    backgroundColorBottom &     Float4 &     4 &     [0.8,0.8,1.0,1.0] &     \tabnewline red, green, blue and alpha values for bottom background color in case that useGradientBackground = True\\ \hline
    circleTiling &     PInt &      &     16 &     global number of segments for circles; if smaller than 2, 2 segments are used (flat)\\ \hline
    coordinateSystemSize &     float &      &     5. &     size of coordinate system relative to font size\\ \hline
    cylinderTiling &     PInt &      &     16 &     global number of segments for cylinders; if smaller than 2, 2 segments are used (flat)\\ \hline
    drawCoordinateSystem &     bool &      &     True &     false = no coordinate system shown\\ \hline
    drawWorldBasis &     bool &      &     False &     true = draw world basis coordinate system at (0,0,0)\\ \hline
    graphicsUpdateInterval &     float &      &     0.1 &     interval of graphics update during simulation in seconds; 0.1 = 10 frames per second; low numbers might slow down computation speed\\ \hline
    minSceneSize &     float &      &     0.1 &     minimum scene size for initial scene size and for autoFitScene, to avoid division by zero; SET GREATER THAN ZERO\\ \hline
    pointSize &     float &      &     0.01 &     global point size (absolute)\\ \hline
    rendererPrecision &     PInt &      &     4 &     precision of general floating point numbers shown in render window: total number of digits used  (max. 16)\\ \hline
    renderWindowString &     String &      &     '' &     string shown in render window (use this, e.g., for debugging, etc.; written below EXUDYN, similar to solutionInformation in SimulationSettings.solutionSettings)\\ \hline
    showComputationInfo &     bool &      &     True &     true = show (hide) all computation information including Exudyn and version\\ \hline
    showHelpOnStartup &     PInt &      &     5 &     seconds to show help message on startup (0=deactivate)\\ \hline
    showSolutionInformation &     bool &      &     True &     true = show solution information (from simulationSettings.solution)\\ \hline
    showSolverInformation &     bool &      &     True &     true = solver name and further information shown in render window\\ \hline
    showSolverTime &     bool &      &     True &     true = solver current time shown in render window\\ \hline
    sphereTiling &     PInt &      &     6 &     global number of segments for spheres; if smaller than 2, 2 segments are used (flat)\\ \hline
    textAlwaysInFront &     bool &      &     True &     if true, text for item numbers and other item-related text is drawn in front; this may be unwanted in case that you only with to see numbers of objects in front; currently does not work with perspective\\ \hline
    textColor &     Float4 &     4 &     [0.,0.,0.,1.0] &     \tabnewline general text color (default); used for system texts in render window\\ \hline
    textHasBackground &     bool &      &     False &     if true, text for item numbers and other item-related text have a background (depending on text color), allowing for better visibility if many numbers are shown; the text itself is black; therefore, dark background colors are ignored and shown as white\\ \hline
    textOffsetFactor &     UFloat &      &     0.005 &     This is an additional out of plane offset for item texts (node number, etc.); the factor is relative to the maximum scene size and is only used, if textAlwaysInFront=False; this factor allows to draw text, e.g., in front of nodes\\ \hline
    textSize &     float &      &     12. &     general text size (font size) in pixels if not overwritten; if useWindowsDisplayScaleFactor=True, the the textSize is multplied with the windows display scaling (monitor scaling; content scaling) factor for larger texts on on high resolution displays; for bitmap fonts, the maximum size of any font (standard/large/huge) is limited to 256 (which is not recommended, especially if you do not have a powerful graphics card)\\ \hline
    threadSafeGraphicsUpdate &     bool &      &     True &     true = updating of visualization is threadsafe, but slower for complicated models; deactivate this to speed up computation, but activate for generation of animations; may be improved in future by adding a safe visualizationUpdate state\\ \hline
    useBitmapText &     bool &      &     True &     if true, texts are displayed using pre-defined bitmaps for the text; may increase the complexity of your scene, e.g., if many (>10000) node numbers shown\\ \hline
    useGradientBackground &     bool &      &     False &     true = use vertical gradient for background; \\ \hline
    useMultiThreadedRendering &     bool &      &     True &     true = rendering is done in separate thread; false = no separate thread, which may be more stable but has lagging interaction for large models (do not interact with models during simulation); set this parameter before call to exudyn.StartRenderer(); MAC OS: uses always false, because MAC OS does not support multi threaded GLFW\\ \hline
    useWindowsDisplayScaleFactor &     bool &      &     True &     the Windows display scaling (monitor scaling; content scaling) factor is used for increased visibility of texts on high resolution displays; based on GLFW glfwGetWindowContentScale; deactivated on linux compilation as it leads to crashes (adjust textSize manually!)\\ \hline
    worldBasisSize &     float &      &     1.0 &     size of world basis coordinate system\\ \hline
	  \end{longtable}
	\end{center}


%+++++++++++++++++++++++++++++++++++

\mysubsubsubsection{VSettingsContour}
\label{sec:VSettingsContour}
Settings for contour plots; use these options to visualize field data, such as displacements, stresses, strains, etc. for bodies, nodes and finite elements.

\noindent VSettingsContour has the following items:
%reference manual TABLE
\begin{center}
  \footnotesize
  \begin{longtable}{| p{4.2cm} | p{2.5cm} | p{0.3cm} | p{3.0cm} | p{6cm} |}
    \hline
    \bf Name & \bf type / function return type & \bf size & \bf default value / function args & \bf description \\ \hline
    automaticRange &     bool &      &     True &     if true, the contour plot value range is chosen automatically to the maximum range\\ \hline
    colorBarPrecision &     PInt &      &     4 &     precision of floating point values shown in color bar; total number of digits used (max. 16)\\ \hline
    colorBarTiling &     PInt &     1 &     12 &     number of tiles (segements) shown in the colorbar for the contour plot\\ \hline
    maxValue &     float &     1 &     1 &     maximum value for contour plot; set manually, if automaticRange == False\\ \hline
    minValue &     float &     1 &     0 &     minimum value for contour plot; set manually, if automaticRange == False\\ \hline
    nodesColored &     bool &      &     True &     if true, the contour color is also applied to nodes (except mesh nodes), otherwise node drawing is not influenced by contour settings\\ \hline
    outputVariable &     OutputVariableType &      &     OutputVariableType::\_None &     \tabnewline selected contour plot output variable type; select OutputVariableType.\_None to deactivate contour plotting.\\ \hline
    outputVariableComponent &     Int &     1 &     0 &     select the component of the chosen output variable; e.g., for displacements, 3 components are available: 0 == x, 1 == y, 2 == z component; for stresses, 6 components are available, see OutputVariableType description; to draw the norm of a outputVariable, set component to -1; if a certain component is not available by certain objects or nodes, no value is drawn (using default color)\\ \hline
    reduceRange &     bool &      &     True &     if true, the contour plot value range is also reduced; better for static computation; in dynamic computation set this option to false, it can reduce visualization artifacts; you should also set minVal to max(float) and maxVal to min(float)\\ \hline
    rigidBodiesColored &     bool &      &     True &     if true, the contour color is also applied to triangular faces of rigid bodies and mass points, otherwise the rigid body drawing are not influenced by contour settings; for general rigid bodies (except for ObjectGround), Position, Displacement, DisplacementLocal(=0), Velocity, VelocityLocal, AngularVelocity, and AngularVelocityLocal are available; may slow down visualization!\\ \hline
    showColorBar &     bool &      &     True &     show the colour bar with minimum and maximum values for the contour plot\\ \hline
	  \end{longtable}
	\end{center}


%+++++++++++++++++++++++++++++++++++

\mysubsubsubsection{VSettingsNodes}
\label{sec:VSettingsNodes}
Visualization settings for nodes.

\noindent VSettingsNodes has the following items:
%reference manual TABLE
\begin{center}
  \footnotesize
  \begin{longtable}{| p{4.2cm} | p{2.5cm} | p{0.3cm} | p{3.0cm} | p{6cm} |}
    \hline
    \bf Name & \bf type / function return type & \bf size & \bf default value / function args & \bf description \\ \hline
    basisSize &     float &      &     0.2 &     size of basis for nodes\\ \hline
    defaultColor &     Float4 &     4 &     [0.2,0.2,1.,1.] &     \tabnewline default cRGB color for nodes; 4th value is alpha-transparency\\ \hline
    defaultSize &     float &      &     -1. &     global node size; if -1.f, node size is relative to openGL.initialMaxSceneSize\\ \hline
    drawNodesAsPoint &     bool &      &     True &     simplified/faster drawing of nodes; uses general->pointSize as drawing size; if drawNodesAsPoint==True, the basis of the node will be drawn with lines\\ \hline
    show &     bool &      &     True &     flag to decide, whether the nodes are shown\\ \hline
    showBasis &     bool &      &     False &     show basis (three axes) of coordinate system in 3D nodes\\ \hline
    showNodalSlopes &     UInt &      &     False &     draw nodal slope vectors, e.g. in ANCF beam finite elements\\ \hline
    showNumbers &     bool &      &     False &     flag to decide, whether the node number is shown\\ \hline
    tiling &     PInt &      &     4 &     tiling for node if drawn as sphere; used to lower the amount of triangles to draw each node; if drawn as circle, this value is multiplied with 4\\ \hline
	  \end{longtable}
	\end{center}


%+++++++++++++++++++++++++++++++++++

\mysubsubsubsection{VSettingsBeams}
\label{sec:VSettingsBeams}
Visualization settings for beam finite elements.

\noindent VSettingsBeams has the following items:
%reference manual TABLE
\begin{center}
  \footnotesize
  \begin{longtable}{| p{4.2cm} | p{2.5cm} | p{0.3cm} | p{3.0cm} | p{6cm} |}
    \hline
    \bf Name & \bf type / function return type & \bf size & \bf default value / function args & \bf description \\ \hline
    axialTiling &     PInt &      &     8 &     number of segments to discretise the beams axis\\ \hline
    crossSectionFilled &     bool &      &     True &     if implemented for element, cross section is drawn as solid (filled) instead of wire-frame; NOTE: some quantities may not be interpolated correctly over cross section in visualization\\ \hline
    crossSectionTiling &     PInt &      &     4 &     number of quads drawn over height of beam, if drawn as flat objects; leads to higher accuracy of components drawn over beam height or with, but also to larger CPU costs for drawing\\ \hline
    drawVertical &     bool &      &     False &     draw contour plot outputVariables 'vertical' along beam height; contour.outputVariable must be set accordingly\\ \hline
    drawVerticalColor &     Float4 &     4 &     [0.2,0.2,0.2,1.] &     \tabnewline color for outputVariable to be drawn along cross section (vertically)\\ \hline
    drawVerticalFactor &     float &      &     1. &     factor for outputVariable to be drawn along cross section (vertically)\\ \hline
    drawVerticalLines &     bool &      &     True &     draw additional vertical lines for better visibility\\ \hline
    drawVerticalOffset &     float &      &     0. &     offset for vertical drawn lines; offset is added before multiplication with drawVerticalFactor\\ \hline
    drawVerticalValues &     bool &      &     False &     show values at vertical lines; note that these numbers are interpolated values and may be different from values evaluated directly at this point!\\ \hline
    reducedAxialInterploation &     bool &      &     True &     if True, the interpolation along the beam axis may be lower than the beam element order; this may be, however, show more consistent values than a full interpolation, e.g. for strains or forces\\ \hline
	  \end{longtable}
	\end{center}


%+++++++++++++++++++++++++++++++++++

\mysubsubsubsection{VSettingsKinematicTree}
\label{sec:VSettingsKinematicTree}
Visualization settings for kinematic trees.

\noindent VSettingsKinematicTree has the following items:
%reference manual TABLE
\begin{center}
  \footnotesize
  \begin{longtable}{| p{4.2cm} | p{2.5cm} | p{0.3cm} | p{3.0cm} | p{6cm} |}
    \hline
    \bf Name & \bf type / function return type & \bf size & \bf default value / function args & \bf description \\ \hline
    frameSize &     float &      &     0.2 &     size of COM and joint frames\\ \hline
    showCOMframes &     bool &      &     False &     if True, a frame is attached to every center of mass\\ \hline
    showFramesNumbers &     bool &      &     True &     if True, numbers are drawn for joint frames (O[i]J[j]) and COM frames (O[i]COM[j]) for object [i] and local joint [j]\\ \hline
    showJointFrames &     bool &      &     True &     if True, a frame is attached to the origin of every joint frame\\ \hline
	  \end{longtable}
	\end{center}


%+++++++++++++++++++++++++++++++++++

\mysubsubsubsection{VSettingsBodies}
\label{sec:VSettingsBodies}
Visualization settings for bodies.

\noindent VSettingsBodies has the following items:
%reference manual TABLE
\begin{center}
  \footnotesize
  \begin{longtable}{| p{4.2cm} | p{2.5cm} | p{0.3cm} | p{3.0cm} | p{6cm} |}
    \hline
    \bf Name & \bf type / function return type & \bf size & \bf default value / function args & \bf description \\ \hline
    beams &     VSettingsBeams &      &      &     visualization settings for beams (e.g. ANCFCable or other beam elements)\\ \hline
    kinematicTree &     VSettingsKinematicTree &      &      &     visualization settings for kinematic tree\\ \hline
    defaultColor &     Float4 &     4 &     [0.3,0.3,1.,1.] &     \tabnewline default cRGB color for bodies; 4th value is \\ \hline
    defaultSize &     Float3 &     3 &     [1.,1.,1.] &     \tabnewline global body size of xyz-cube\\ \hline
    deformationScaleFactor &     float &      &     1 &     global deformation scale factor; also applies to nodes, if drawn; used for scaled drawing of (linear) finite elements, beams, etc.\\ \hline
    show &     bool &      &     True &     flag to decide, whether the bodies are shown\\ \hline
    showNumbers &     bool &      &     False &     flag to decide, whether the body(=object) number is shown\\ \hline
	  \end{longtable}
	\end{center}


%+++++++++++++++++++++++++++++++++++

\mysubsubsubsection{VSettingsConnectors}
\label{sec:VSettingsConnectors}
Visualization settings for connectors.

\noindent VSettingsConnectors has the following items:
%reference manual TABLE
\begin{center}
  \footnotesize
  \begin{longtable}{| p{4.2cm} | p{2.5cm} | p{0.3cm} | p{3.0cm} | p{6cm} |}
    \hline
    \bf Name & \bf type / function return type & \bf size & \bf default value / function args & \bf description \\ \hline
    contactPointsDefaultSize &     float &      &     0.02 &     DEPRECATED: do not use! global contact points size; if -1.f, connector size is relative to maxSceneSize\\ \hline
    defaultColor &     Float4 &     4 &     [0.2,0.2,1.,1.] &     \tabnewline default cRGB color for connectors; 4th value is alpha-transparency\\ \hline
    defaultSize &     float &      &     0.1 &     global connector size; if -1.f, connector size is relative to maxSceneSize\\ \hline
    jointAxesLength &     float &      &     0.2 &     global joint axes length\\ \hline
    jointAxesRadius &     float &      &     0.02 &     global joint axes radius\\ \hline
    show &     bool &      &     True &     flag to decide, whether the connectors are shown\\ \hline
    showContact &     bool &      &     False &     flag to decide, whether contact points, lines, etc. are shown\\ \hline
    showJointAxes &     bool &      &     False &     flag to decide, whether contact joint axes of 3D joints are shown\\ \hline
    showNumbers &     bool &      &     False &     flag to decide, whether the connector(=object) number is shown\\ \hline
    springNumberOfWindings &     PInt &      &     8 &     number of windings for springs drawn as helical spring\\ \hline
	  \end{longtable}
	\end{center}


%+++++++++++++++++++++++++++++++++++

\mysubsubsubsection{VSettingsMarkers}
\label{sec:VSettingsMarkers}
Visualization settings for markers.

\noindent VSettingsMarkers has the following items:
%reference manual TABLE
\begin{center}
  \footnotesize
  \begin{longtable}{| p{4.2cm} | p{2.5cm} | p{0.3cm} | p{3.0cm} | p{6cm} |}
    \hline
    \bf Name & \bf type / function return type & \bf size & \bf default value / function args & \bf description \\ \hline
    defaultColor &     Float4 &     4 &     [0.1,0.5,0.1,1.] &     \tabnewline default cRGB color for markers; 4th value is alpha-transparency\\ \hline
    defaultSize &     float &      &     -1. &     global marker size; if -1.f, marker size is relative to maxSceneSize\\ \hline
    drawSimplified &     bool &      &     True &     draw markers with simplified symbols\\ \hline
    show &     bool &      &     True &     flag to decide, whether the markers are shown\\ \hline
    showNumbers &     bool &      &     False &     flag to decide, whether the marker numbers are shown\\ \hline
	  \end{longtable}
	\end{center}


%+++++++++++++++++++++++++++++++++++

\mysubsubsubsection{VSettingsLoads}
\label{sec:VSettingsLoads}
Visualization settings for loads.

\noindent VSettingsLoads has the following items:
%reference manual TABLE
\begin{center}
  \footnotesize
  \begin{longtable}{| p{4.2cm} | p{2.5cm} | p{0.3cm} | p{3.0cm} | p{6cm} |}
    \hline
    \bf Name & \bf type / function return type & \bf size & \bf default value / function args & \bf description \\ \hline
    defaultColor &     Float4 &     4 &     [0.7,0.1,0.1,1.] &     \tabnewline default cRGB color for loads; 4th value is alpha-transparency\\ \hline
    defaultRadius &     float &      &     0.005 &     global radius of load axis if drawn in 3D\\ \hline
    defaultSize &     float &      &     0.2 &     global load size; if -1.f, load size is relative to maxSceneSize\\ \hline
    drawSimplified &     bool &      &     True &     draw markers with simplified symbols\\ \hline
    fixedLoadSize &     bool &      &     True &     if true, the load is drawn with a fixed vector length in direction of the load vector, independently of the load size\\ \hline
    loadSizeFactor &     float &      &     0.1 &     if fixedLoadSize=false, then this scaling factor is used to draw the load vector\\ \hline
    show &     bool &      &     True &     flag to decide, whether the loads are shown\\ \hline
    showNumbers &     bool &      &     False &     flag to decide, whether the load numbers are shown\\ \hline
	  \end{longtable}
	\end{center}


%+++++++++++++++++++++++++++++++++++

\mysubsubsubsection{VSettingsSensors}
\label{sec:VSettingsSensors}
Visualization settings for sensors.

\noindent VSettingsSensors has the following items:
%reference manual TABLE
\begin{center}
  \footnotesize
  \begin{longtable}{| p{4.2cm} | p{2.5cm} | p{0.3cm} | p{3.0cm} | p{6cm} |}
    \hline
    \bf Name & \bf type / function return type & \bf size & \bf default value / function args & \bf description \\ \hline
    defaultColor &     Float4 &     4 &     [0.6,0.6,0.1,1.] &     \tabnewline default cRGB color for sensors; 4th value is alpha-transparency\\ \hline
    defaultSize &     float &      &     -1. &     global sensor size; if -1.f, sensor size is relative to maxSceneSize\\ \hline
    drawSimplified &     bool &      &     True &     draw sensors with simplified symbols\\ \hline
    show &     bool &      &     True &     flag to decide, whether the sensors are shown\\ \hline
    showNumbers &     bool &      &     False &     flag to decide, whether the sensor numbers are shown\\ \hline
	  \end{longtable}
	\end{center}


%+++++++++++++++++++++++++++++++++++

\mysubsubsubsection{VSettingsContact}
\label{sec:VSettingsContact}
Global visualization settings for GeneralContact. This allows to easily switch on/off during visualization. 

\noindent VSettingsContact has the following items:
%reference manual TABLE
\begin{center}
  \footnotesize
  \begin{longtable}{| p{4.2cm} | p{2.5cm} | p{0.3cm} | p{3.0cm} | p{6cm} |}
    \hline
    \bf Name & \bf type / function return type & \bf size & \bf default value / function args & \bf description \\ \hline
    colorBoundingBoxes &     Float4 &     4 &     [0.9,0.1,0.1,1.] &     \tabnewline cRGB color\\ \hline
    colorSearchTree &     Float4 &     4 &     [0.1,0.1,0.9,1.] &     \tabnewline cRGB color\\ \hline
    contactForcesFactor &     float &      &     0.001 &     factor used for scaling of contact forces is showContactForces=True\\ \hline
    contactPointsDefaultSize &     float &      &     0.001 &     global contact points size; if -1.f, connector size is relative to maxSceneSize; used for some contacts, e.g., in ContactFrictionCircle\\ \hline
    showBoundingBoxes &     bool &      &     False &     show bounding boxes of all GeneralContacts\\ \hline
    showContactForces &     bool &      &     False &     if True, contact forces are drawn for certain contact models\\ \hline
    showContactForcesValues &     bool &      &     False &     if True and showContactForces=True, numerical values for  contact forces are shown at certain points\\ \hline
    showSearchTree &     bool &      &     False &     show search tree of all GeneralContacts\\ \hline
    showSearchTreeCells &     bool &      &     False &     show cells inside search tree\\ \hline
	  \end{longtable}
	\end{center}


%+++++++++++++++++++++++++++++++++++

\mysubsubsubsection{VSettingsWindow}
\label{sec:VSettingsWindow}
OpenGL Window and interaction settings for visualization; handle changes with care, as they might lead to unexpected results or crashes.

\noindent VSettingsWindow has the following items:
%reference manual TABLE
\begin{center}
  \footnotesize
  \begin{longtable}{| p{4.2cm} | p{2.5cm} | p{0.3cm} | p{3.0cm} | p{6cm} |}
    \hline
    \bf Name & \bf type / function return type & \bf size & \bf default value / function args & \bf description \\ \hline
    alwaysOnTop &     bool &      &     False &     True: OpenGL render window will be always on top of all other windows\\ \hline
    ignoreKeys &     bool &      &     False &     True: ignore keyboard input except escape and 'F2' keys; used for interactive mode, e.g., to perform kinematic analysis; This flag can be switched with key 'F2'\\ \hline
    keyPressUserFunction &     KeyPressUserFunction &      &     0 &     add a Python function f(key, action, mods) here, which is called every time a key is pressed; function shall return true, if key has been processed; Example: \tabnewline def f(key, action, mods):\tabnewline \phantom{XXX} print('key=',key);\tabnewline use chr(key) to convert key codes [32 ...96] to ascii; special key codes (>256) are provided in the exudyn.KeyCode enumeration type; key action needs to be checked (0=released, 1=pressed, 2=repeated); mods provide information (binary) for SHIFT (1), CTRL (2), ALT (4), Super keys (8), CAPSLOCK (16)\\ \hline
    limitWindowToScreenSize &     bool &      &     True &     True: render window size is limited to screen size; False: larger window sizes (e.g. for rendering) allowed according to renderWindowSize\\ \hline
    maximize &     bool &      &     False &     True: OpenGL render window will be maximized at startup\\ \hline
    renderWindowSize &     Index2 &     2 &     [1024,768] &     initial size of OpenGL render window in pixel\\ \hline
    ResetKeyPressUserFunction() &     void &      &      &     set keyPressUserFunction to zero (no function); because this cannot be assign to the variable itself\\ \hline
    showMouseCoordinates &     bool &      &     False &     True: show OpenGL coordinates and distance to last left mouse button pressed position; switched on/off with key 'F3'\\ \hline
    showWindow &     bool &      &     True &     True: OpenGL render window is shown on startup; False: window will be iconified at startup (e.g. if you are starting multiple computations automatically)\\ \hline
    startupTimeout &     PInt &      &     2500 &     OpenGL render window startup timeout in ms (change might be necessary if CPU is very slow)\\ \hline
	  \end{longtable}
	\end{center}


%+++++++++++++++++++++++++++++++++++

\mysubsubsubsection{VSettingsDialogs}
\label{sec:VSettingsDialogs}
Settings related to dialogs (e.g., visualization settings dialog).

\noindent VSettingsDialogs has the following items:
%reference manual TABLE
\begin{center}
  \footnotesize
  \begin{longtable}{| p{4.2cm} | p{2.5cm} | p{0.3cm} | p{3.0cm} | p{6cm} |}
    \hline
    \bf Name & \bf type / function return type & \bf size & \bf default value / function args & \bf description \\ \hline
    alphaTransparency &     UFloat &      &     0.94 &     alpha-transparency of dialogs; recommended range 0.7 (very transparent) - 1 (not transparent at all)\\ \hline
    alwaysTopmost &     bool &      &     True &     True: dialogs are always topmost (otherwise, they are sometimes hidden)\\ \hline
    fontScalingMacOS &     UFloat &      &     1.35 &     font scaling value for MacOS systems (on Windows, system display scaling is used)\\ \hline
    multiThreadedDialogs &     bool &      &     True &     True: During dialogs, the OpenGL render window will still get updates of changes in dialogs, etc., which may cause problems on some platforms or for some (complicated) models; False: changes of dialogs will take effect when dialogs are closed\\ \hline
    openTreeView &     bool &      &     False &     True: all sub-trees of the visusalization dialog are opened when opening the dialog; False: only some sub-trees are opened\\ \hline
	  \end{longtable}
	\end{center}


%+++++++++++++++++++++++++++++++++++

\mysubsubsubsection{VSettingsOpenGL}
\label{sec:VSettingsOpenGL}
OpenGL settings for 2D and 2D rendering. For further details, see the OpenGL functionality. 

\noindent VSettingsOpenGL has the following items:
%reference manual TABLE
\begin{center}
  \footnotesize
  \begin{longtable}{| p{4.2cm} | p{2.5cm} | p{0.3cm} | p{3.0cm} | p{6cm} |}
    \hline
    \bf Name & \bf type / function return type & \bf size & \bf default value / function args & \bf description \\ \hline
    drawFaceNormals &     bool &     1 &     False &     draws triangle normals, e.g. at center of triangles; used for debugging of faces\\ \hline
    drawNormalsLength &     PFloat &     1 &     0.1 &     length of normals; used for debugging\\ \hline
    drawVertexNormals &     bool &     1 &     False &     draws vertex normals; used for debugging\\ \hline
    enableLight0 &     bool &     1 &     True &     turn on/off light0\\ \hline
    enableLight1 &     bool &     1 &     True &     turn on/off light1\\ \hline
    enableLighting &     bool &     1 &     True &     generally enable lighting (otherwise, colors of objects are used); OpenGL: glEnable(GL\_LIGHTING)\\ \hline
    faceEdgesColor &     Float4 &     4 &     [0.2,0.2,0.2,1.] &     \tabnewline global RGBA color for face edges\\ \hline
    facesTransparent &     bool &     1 &     False &     True: show faces transparent independent of transparency (A)-value in color of objects; allow to show otherwise hidden node/marker/object numbers\\ \hline
    initialCenterPoint &     Float3 &     3 &     [0.,0.,0.] &     \tabnewline centerpoint of scene (3D) at renderer startup; overwritten if autoFitScene = True\\ \hline
    initialMaxSceneSize &     PFloat &      &     1. &     initial maximum scene size (auto: diagonal of cube with maximum scene coordinates); used for 'zoom all' functionality and for visibility of objects; overwritten if autoFitScene = True\\ \hline
    initialModelRotation &     StdArray33F &     3x3 &     [Matrix3DF[3,3,1.,0.,0., 0.,1.,0., 0.,0.,1.]] &     \tabnewline initial model rotation matrix for OpenGl; in python use e.g.: initialModelRotation=[[1,0,0],[0,1,0],[0,0,1]]\\ \hline
    initialZoom &     UFloat &      &     1. &     initial zoom of scene; overwritten/ignored if autoFitScene = True\\ \hline
    light0ambient &     float &     1 &     0.3 &     ambient value of GL\_LIGHT0\\ \hline
    light0constantAttenuation &     float &     1 &     1.0 &     constant attenuation coefficient of GL\_LIGHT0, this is a constant factor that attenuates the light source; attenuation factor = 1/(kx +kl*d + kq*d*d); (kc,kl,kq)=(1,0,0) means no attenuation; only used for lights, where last component of light position is 1\\ \hline
    light0diffuse &     float &     1 &     0.6 &     diffuse value of GL\_LIGHT0\\ \hline
    light0linearAttenuation &     float &     1 &     0.0 &     linear attenuation coefficient of GL\_LIGHT0, this is a linear factor for attenuation of the light source with distance\\ \hline
    light0position &     Float4 &     4 &     [0.2,0.2,10.,0.] &     \tabnewline 4f position vector of GL\_LIGHT0; 4th value should be 0 for lights like sun, but 1 for directional lights (and for attenuation factor being calculated); light0 is also used for shadows, so you need to adjust this position; see opengl manuals\\ \hline
    light0quadraticAttenuation &     float &     1 &     0.0 &     quadratic attenuation coefficient of GL\_LIGHT0, this is a quadratic factor for attenuation of the light source with distance\\ \hline
    light0specular &     float &     1 &     0.5 &     specular value of GL\_LIGHT0\\ \hline
    light1ambient &     float &     1 &     0.0  &     ambient value of GL\_LIGHT1\\ \hline
    light1constantAttenuation &     float &     1 &     1.0 &     constant attenuation coefficient of GL\_LIGHT1, this is a constant factor that attenuates the light source; attenuation factor = 1/(kx +kl*d + kq*d*d); only used for lights, where last component of light position is 1\\ \hline
    light1diffuse &     float &     1 &     0.5 &     diffuse value of GL\_LIGHT1\\ \hline
    light1linearAttenuation &     float &     1 &     0.0 &     linear attenuation coefficient of GL\_LIGHT1, this is a linear factor for attenuation of the light source with distance\\ \hline
    light1position &     Float4 &     4 &     [1.,1.,-10.,0.] &     \tabnewline 4f position vector of GL\_LIGHT0; 4th value should be 0 for lights like sun, but 1 for directional lights (and for attenuation factor being calculated); see opengl manuals\\ \hline
    light1quadraticAttenuation &     float &     1 &     0.0 &     quadratic attenuation coefficient of GL\_LIGHT1, this is a quadratic factor for attenuation of the light source with distance\\ \hline
    light1specular &     float &     1 &     0.6 &     specular value of GL\_LIGHT1\\ \hline
    lightModelAmbient &     Float4 &     4 &     [0.,0.,0.,1.] &     \tabnewline global ambient light; maps to OpenGL glLightModeli(GL\_LIGHT\_MODEL\_AMBIENT,[r,g,b,a])\\ \hline
    lightModelLocalViewer &     bool &     1 &     False &     select local viewer for light; maps to OpenGL glLightModeli(GL\_LIGHT\_MODEL\_LOCAL\_VIEWER,...)\\ \hline
    lightModelTwoSide &     bool &     1 &     False &     enlighten also backside of object; may cause problems on some graphics cards and lead to slower performance; maps to OpenGL glLightModeli(GL\_LIGHT\_MODEL\_TWO\_SIDE,...)\\ \hline
    lineSmooth &     bool &     1 &     True &     draw lines smooth\\ \hline
    lineWidth &     UFloat &     1 &     1. &     width of lines used for representation of lines, circles, points, etc.\\ \hline
    materialAmbientAndDiffuse &     Float4 &     4 &     [0.6,0.6,0.6,1.] &     \tabnewline 4f ambient color of material\\ \hline
    materialShininess &     float &     1 &     32. &     shininess of material\\ \hline
    materialSpecular &     Float4 &     4 &     [0.6,0.6,0.6,1.] &     \tabnewline 4f specular color of material\\ \hline
    multiSampling &     PInt &     1 &     1 &     NOTE: this parameter must be set before starting renderer; later changes are not affecting visualization; multi sampling turned off (<=1) or turned on to given values (2, 4, 8 or 16); increases the graphics buffers and might crash due to graphics card memory limitations; only works if supported by hardware; if it does not work, try to change 3D graphics hardware settings!\\ \hline
    perspective &     UFloat &      &     0. &     parameter prescribes amount of perspective (0=no perspective=orthographic projection; positive values increase perspective; feasible values are 0.001 (little perspective) ... 0.5 (large amount of perspective); mouse coordinates will not work with perspective\\ \hline
    polygonOffset &     float &      &     0.01 &     general polygon offset for polygons, except for shadows; use this parameter to draw polygons behind lines to reduce artifacts for very large or small models\\ \hline
    shadeModelSmooth &     bool &     1 &     True &     True: turn on smoothing for shaders, which uses vertex normals to smooth surfaces\\ \hline
    shadow &     UFloat &      &     0. &     parameter $\in [0 ... 1]$ prescribes amount of shadow for light0 (using light0position, etc.); if this parameter is different from 1, rendering of triangles becomes approx.\ 5 times more expensive, so take care in case of complex scenes; for complex object, such as spheres with fine resolution or for particle systems, the present approach has limitations and leads to artifacts and unrealistic shadows\\ \hline
    shadowPolygonOffset &     PFloat &      &     0.1 &     some special drawing parameter for shadows which should be handled with care; defines some offset needed by openGL to avoid aritfacts for shadows and depends on maxSceneSize; this value may need to be reduced for larger models in order to achieve more accurate shadows, it may be needed to be increased for thin bodies\\ \hline
    showFaceEdges &     bool &     1 &     False &     show edges of faces; using the options showFaces=false and showFaceEdges=true gives are wire frame representation\\ \hline
    showFaces &     bool &     1 &     True &     show faces of triangles, etc.; using the options showFaces=false and showFaceEdges=true gives are wire frame representation\\ \hline
    showLines &     bool &     1 &     True &     show lines (different from edges of faces)\\ \hline
    showMeshEdges &     bool &     1 &     True &     show edges of finite elements; independent of showFaceEdges\\ \hline
    showMeshFaces &     bool &     1 &     True &     show faces of finite elements; independent of showFaces\\ \hline
    textLineSmooth &     bool &     1 &     False &     draw lines for representation of text smooth\\ \hline
    textLineWidth &     UFloat &     1 &     1. &     width of lines used for representation of text\\ \hline
	  \end{longtable}
	\end{center}


%+++++++++++++++++++++++++++++++++++

\mysubsubsubsection{VSettingsExportImages}
\label{sec:VSettingsExportImages}
Functionality to export images to files (PNG or TGA format) which can be used to create animations; to activate image recording during the solution process, set SolutionSettings.recordImagesInterval accordingly.

\noindent VSettingsExportImages has the following items:
%reference manual TABLE
\begin{center}
  \footnotesize
  \begin{longtable}{| p{4.2cm} | p{2.5cm} | p{0.3cm} | p{3.0cm} | p{6cm} |}
    \hline
    \bf Name & \bf type / function return type & \bf size & \bf default value / function args & \bf description \\ \hline
    heightAlignment &     PInt &      &     2 &     alignment of exported image height; using a value of 2 helps to reduce problems with video conversion (additional horizontal lines are lost)\\ \hline
    saveImageAsTextCircles &     bool &      &     True &     export circles in save image (only in TXT format)\\ \hline
    saveImageAsTextLines &     bool &      &     True &     export lines in save image (only in TXT format)\\ \hline
    saveImageAsTextTexts &     bool &      &     False &     export text in save image (only in TXT format)\\ \hline
    saveImageAsTextTriangles &     bool &      &     False &     export triangles in save image (only in TXT format)\\ \hline
    saveImageFileCounter &     UInt &      &     0 &     current value of the counter which is used to consecutively save frames (images) with consecutive numbers\\ \hline
    saveImageFileName &     FileName &      &     'images/frame' &     filename (without extension!) and (relative) path for image file(s) with consecutive numbering (e.g., frame0000.png, frame0001.png,...); ; directory will be created if it does not exist\\ \hline
    saveImageFormat &     String &      &     'PNG' &     format for exporting figures: currently only PNG, TGA and TXT available; while PNG and TGA represent the according image file formats, the TXT format results in a text file containing the 3D graphics data information as lists of lines, triangles, etc; PNG is not available for Ubuntu18.04 (check  use TGA has highest compatibility with all platforms\\ \hline
    saveImageSingleFile &     bool &      &     False &     True: only save single files with given filename, not adding numbering; False: add numbering to files, see saveImageFileName\\ \hline
    saveImageTimeOut &     PInt &      &     5000 &     timeout in milliseconds for saving a frame as image to disk; this is the amount of time waited for redrawing; increase for very complex scenes\\ \hline
    widthAlignment &     PInt &      &     4 &     alignment of exported image width; using a value of 4 helps to reduce problems with video conversion (additional vertical lines are lost)\\ \hline
	  \end{longtable}
	\end{center}


%+++++++++++++++++++++++++++++++++++

\mysubsubsubsection{VSettingsOpenVR}
\label{sec:VSettingsOpenVR}
Functionality to interact openVR; requires special hardware or software emulator, see steam / openVR descriptions. 

\noindent VSettingsOpenVR has the following items:
%reference manual TABLE
\begin{center}
  \footnotesize
  \begin{longtable}{| p{4.2cm} | p{2.5cm} | p{0.3cm} | p{3.0cm} | p{6cm} |}
    \hline
    \bf Name & \bf type / function return type & \bf size & \bf default value / function args & \bf description \\ \hline
    actionManifestFileName &     FileName &      &     'C:/openVRactionsManifest.json' &     \tabnewline This string must contain a string representing a valid absolute path to a vr\_actions.json manifest, which describes all HMD, tracker, etc. devices as given by openVR\\ \hline
    enable &     bool &      &     False &     True: openVR enabled (if compiled with according flag and installed openVR)\\ \hline
    logLevel &     Int &      &     1 &     integer value setting log level of openVR: -1 (no output), 0 (error), 1 (warning), 2 (info), 3 (debug); increase log level to get more output\\ \hline
    showCompanionWindow &     bool &      &     True &     True: openVR will show companion window containing left and right eye view\\ \hline
	  \end{longtable}
	\end{center}


%+++++++++++++++++++++++++++++++++++

\mysubsubsubsection{VSettingsInteractive}
\label{sec:VSettingsInteractive}
Functionality to interact with render window; will include left and right mouse press actions and others in future.

\noindent VSettingsInteractive has the following items:
%reference manual TABLE
\begin{center}
  \footnotesize
  \begin{longtable}{| p{4.2cm} | p{2.5cm} | p{0.3cm} | p{3.0cm} | p{6cm} |}
    \hline
    \bf Name & \bf type / function return type & \bf size & \bf default value / function args & \bf description \\ \hline
    openVR &     VSettingsOpenVR &      &      &     openVR visualization settings\\ \hline
    highlightColor &     Float4 &     4 &     [0.8,0.05,0.05,0.75] &     \tabnewline cRGB color for highlighted item; 4th value is alpha-transparency\\ \hline
    highlightItemIndex &     Int &      &     -1 &     index of item that shall be highlighted (e.g., need to find item due to errors); if set -1, no item is highlighted\\ \hline
    highlightItemType &     ItemType &      &     ItemType::\_None &     item type (Node, Object, ...) that shall be highlighted (e.g., need to find item due to errors)\\ \hline
    highlightMbsNumber &     UInt &      &     0 &     index of main system (mbs) for which the item shall be highlighted; number is related to the ID in SystemContainer (first mbs = 0, second = 1, ...)\\ \hline
    highlightOtherColor &     Float4 &     4 &     [0.5,0.5,0.5,0.4] &     \tabnewline cRGB color for other items (which are not highlighted); 4th value is alpha-transparency\\ \hline
    joystickScaleRotation &     float &      &     200. &     rotation scaling factor for joystick input\\ \hline
    joystickScaleTranslation &     float &      &     6. &     translation scaling factor for joystick input\\ \hline
    keypressRotationStep &     float &      &     5. &     rotation increment per keypress in degree (full rotation = 360 degree)\\ \hline
    keypressTranslationStep &     float &      &     0.1 &     translation increment per keypress relative to window size\\ \hline
    lockModelView &     bool &      &     False &     True: all movements (with mouse/keys), rotations, zoom are disabled; initial values are considered ==> initial zoom, rotation and center point need to be adjusted, approx. 0.4*maxSceneSize is a good value\\ \hline
    mouseMoveRotationFactor &     float &      &     1. &     rotation increment per 1 pixel mouse movement in degree\\ \hline
    pauseWithSpacebar &     bool &      &     True &     True: during simulation, space bar can be pressed to pause simulation\\ \hline
    selectionHighlights &     bool &      &     True &     True: mouse click highlights item (default: red)\\ \hline
    selectionLeftMouse &     bool &      &     True &     True: left mouse click on items and show basic information\\ \hline
    selectionRightMouse &     bool &      &     True &     True: right mouse click on items and show dictionary (read only!)\\ \hline
    selectionRightMouseGraphicsData &     \tabnewline bool &      &     False &     True: right mouse click on items also shows GraphicsData information for inspectation (may sometimes be very large and may not fit into dialog for large graphics objects!)\\ \hline
    trackMarker &     Int &      &     -1 &     if valid marker index is provided and marker provides position (and orientation), the centerpoint of the scene follows the marker (and orientation); depends on trackMarkerPosition and trackMarkerOrientation; by default, only position is tracked\\ \hline
    trackMarkerMbsNumber &     Index &      &     0 &     number of main system which is used to track marker; if only 1 mbs is in the SystemContainer, use 0; if there are several mbs, it needs to specify the number\\ \hline
    trackMarkerOrientation &     Float3 &     3 &     [0.,0.,0.] &     \tabnewline choose which orientation axes (x,y,z) are tracked; currently can only be all zero or all one\\ \hline
    trackMarkerPosition &     Float3 &     3 &     [1.,1.,1.] &     \tabnewline choose which coordinates or marker are tracked (x,y,z)\\ \hline
    useJoystickInput &     bool &      &     True &     True: read joystick input (use 6-axis joystick with lowest ID found when starting renderer window) and interpret as (x,y,z) position and (rotx, roty, rotz) rotation: as available from 3Dconnexion space mouse and maybe others as well; set to False, if external joystick makes problems ...\\ \hline
    zoomStepFactor &     float &      &     1.15 &     change of zoom per keypress (keypad +/-) or mouse wheel increment\\ \hline
	  \end{longtable}
	\end{center}


%+++++++++++++++++++++++++++++++++++

\mysubsubsubsection{VisualizationSettings}
\label{sec:VisualizationSettings}
Settings for visualization. 

\noindent VisualizationSettings has the following items:
%reference manual TABLE
\begin{center}
  \footnotesize
  \begin{longtable}{| p{4.2cm} | p{2.5cm} | p{0.3cm} | p{3.0cm} | p{6cm} |}
    \hline
    \bf Name & \bf type / function return type & \bf size & \bf default value / function args & \bf description \\ \hline
    bodies &     VSettingsBodies &      &      &     body visualization settings\\ \hline
    connectors &     VSettingsConnectors &      &      &     connector visualization settings\\ \hline
    contact &     VSettingsContact &      &      &     contact visualization settings\\ \hline
    contour &     VSettingsContour &      &      &     contour plot visualization settings\\ \hline
    dialogs &     VSettingsDialogs &      &      &     dialogs settings\\ \hline
    exportImages &     VSettingsExportImages &      &      &     settings for exporting (saving) images to files in order to create animations\\ \hline
    general &     VSettingsGeneral &      &      &     general visualization settings\\ \hline
    interactive &     VSettingsInteractive &      &      &     Settings for interaction with renderer\\ \hline
    loads &     VSettingsLoads &      &      &     load visualization settings\\ \hline
    markers &     VSettingsMarkers &      &      &     marker visualization settings\\ \hline
    nodes &     VSettingsNodes &      &      &     node visualization settings\\ \hline
    openGL &     VSettingsOpenGL &      &      &     OpenGL rendering settings\\ \hline
    sensors &     VSettingsSensors &      &      &     sensor visualization settings\\ \hline
    window &     VSettingsWindow &      &      &     visualization window and interaction settings\\ \hline
	  \end{longtable}
	\end{center}


%++++++++++++++++++++++++++++++++++++++
\mysubsectionlabel{Solver substructures}{sec:solverSubstructures}
This section includes structures contained in the solver, which can be accessed via the Python interface during solution or for building a customized solver in Python.
There is plenty of possibilities to interact with the solvers, being it the extraction of data at the end (such as .it or .conv), computation of mass matrix or system matrices, and finally the modification of solver structures (which may have effect or not). In any case, there is no full description for all these methods and the user must always consider the according C++ function to verify the desired behavior.

%+++++++++++++++++++++++++++++++++++

\mysubsubsubsection{CSolverTimer}
\label{sec:CSolverTimer}
Structure for timing in solver. Each Real variable is used to measure the CPU time which certain parts of the solver need. This structure is only active if the code is not compiled with the \_\_FAST\_EXUDYN\_LINALG option and if displayComputationTime is set True. Timings will only be filled, if useTimer is True.

\noindent CSolverTimer has the following items:
%reference manual TABLE
\begin{center}
  \footnotesize
  \begin{longtable}{| p{4.2cm} | p{2.5cm} | p{0.3cm} | p{3.0cm} | p{6cm} |}
    \hline
    \bf Name & \bf type / function return type & \bf size & \bf default value / function args & \bf description \\ \hline
    AERHS &     Real &      &     0. &     time for residual evaluation of algebraic equations right-hand-side\\ \hline
    errorEstimator &     Real &      &     0. &     for explicit solvers, additional evaluation\\ \hline
    factorization &     Real &      &     0. &     solve or inverse\\ \hline
    integrationFormula &     Real &      &     0. &     time spent for evaluation of integration formulas\\ \hline
    jacobianAE &     Real &      &     0. &     jacobian of algebraic equations (not counted in sum)\\ \hline
    jacobianODE1 &     Real &      &     0. &     jacobian w.r.t. coordinates of \hac{ODE1} equations (not counted in sum)\\ \hline
    jacobianODE2 &     Real &      &     0. &     jacobian w.r.t. coordinates of \hac{ODE2} equations (not counted in sum)\\ \hline
    jacobianODE2\_t &     Real &      &     0. &     jacobian w.r.t. coordinates\_t of \hac{ODE2} equations (not counted in sum)\\ \hline
    massMatrix &     Real &      &     0. &     mass matrix computation\\ \hline
    newtonIncrement &     Real &      &     0. &     Jac$^{-1}$ * RHS; backsubstitution\\ \hline
    ODE1RHS &     Real &      &     0. &     time for residual evaluation of \hac{ODE1} right-hand-side\\ \hline
    ODE2RHS &     Real &      &     0. &     time for residual evaluation of \hac{ODE2} right-hand-side\\ \hline
    overhead &     Real &      &     0. &     overhead, such as initialization, copying and some matrix-vector multiplication\\ \hline
    postNewton &     Real &      &     0. &     discontinuous iteration / PostNewtonStep\\ \hline
    python &     Real &      &     0. &     time spent for Python functions\\ \hline
    reactionForces &     Real &      &     0. &     CqT * lambda\\ \hline
    Reset(...) &     void &      &     useSolverTimer &     reset solver timings to initial state by assigning default values; useSolverTimer sets the useTimer flag\\ \hline
    StartTimer(...) &     void &      &     value &     start timer function for a given variable; subtracts current CPU time from value\\ \hline
    StopTimer(...) &     void &      &     value &     stop timer function for a given variable; adds current CPU time to value\\ \hline
    Sum() &     Real &      &      &     compute sum of all timers (except for those counted multiple, e.g., jacobians\\ \hline
    ToString() &     String &      &      &     converts the current timings to a string\\ \hline
    total &     Real &      &     0. &     total time measured between start and end of computation (static/dynamics)\\ \hline
    totalJacobian &     Real &      &     0. &     time for all jacobian computations\\ \hline
    useTimer &     bool &      &     True &     flag to decide, whether the timer is used (true) or not\\ \hline
    visualization &     Real &      &     0. &     time spent for visualization in computation thread\\ \hline
    writeSolution &     Real &      &     0. &     time for writing solution\\ \hline
	  \end{longtable}
	\end{center}


%+++++++++++++++++++++++++++++++++++

\mysubsubsubsection{SolverLocalData}
\label{sec:SolverLocalData}
Solver local data structure for solution vectors, system matrices and temporary vectors and data structures.

\noindent SolverLocalData has the following items:
%reference manual TABLE
\begin{center}
  \footnotesize
  \begin{longtable}{| p{4.2cm} | p{2.5cm} | p{0.3cm} | p{3.0cm} | p{6cm} |}
    \hline
    \bf Name & \bf type / function return type & \bf size & \bf default value / function args & \bf description \\ \hline
    aAlgorithmic &     ResizableVectorParallel &      &      &     additional term needed for generalized alpha (current state)\\ \hline
    CleanUpMemory() &     void &      &      &     if desired, temporary data is cleaned up to safe memory\\ \hline
    GetLinearSolverType() &     LinearSolverType &      &      &     return current linear solver type (dense/sparse)\\ \hline
    nAE &     Index &      &     0 &     number of algebraic coordinates\\ \hline
    nData &     Index &      &     0 &     number of data coordinates\\ \hline
    newtonSolution &     ResizableVectorParallel &      &      &     Newton decrement (computed from residual and jacobian)\\ \hline
    nODE1 &     Index &      &     0 &     number of first order ordinary diff. eq. coordinates\\ \hline
    nODE2 &     Index &      &     0 &     number of second order ordinary diff. eq. coordinates\\ \hline
    nSys &     Index &      &     0 &     number of system (unknown) coordinates = nODE2+nODE1+nAE\\ \hline
    SetLinearSolverType(...) &     void &      &     linearSolverType, reuseAnalyzedPattern &     set linear solver type and matrix version: links system matrices to according dense/sparse versions\\ \hline
    startAE &     Index &      &     0 &     start of algebraic coordinates, but set to zero if nAE==0\\ \hline
    startOfStepStateAAlgorithmic &     ResizableVectorParallel &      &      &     additional term needed for generalized alpha (startOfStep state)\\ \hline
    systemResidual &     ResizableVectorParallel &      &      &     system residual vector (vectors will be linked to this vector!)\\ \hline
    temp2ODE2 &     ResizableVectorParallel &      &      &     second temporary vector for \hac{ODE2} quantities; use in static computation\\ \hline
    tempODE1F0 &     ResizableVectorParallel &      &      &     temporary vector for \hac{ODE1} Jacobian\\ \hline
    tempODE1F1 &     ResizableVectorParallel &      &      &     temporary vector for \hac{ODE1} Jacobian\\ \hline
    tempODE2 &     ResizableVectorParallel &      &      &     temporary vector for \hac{ODE2} quantities; use in initial accelerations and during Newton\\ \hline
    tempODE2F0 &     ResizableVectorParallel &      &      &     temporary vector for \hac{ODE2} Jacobian\\ \hline
    tempODE2F1 &     ResizableVectorParallel &      &      &     temporary vector for \hac{ODE2} Jacobian\\ \hline
	  \end{longtable}
	\end{center}


%+++++++++++++++++++++++++++++++++++

\mysubsubsubsection{SolverIterationData}
\label{sec:SolverIterationData}
Solver internal structure for counters, steps, step size, time, etc.; solution vectors, residuals, etc. are SolverLocalData. The given default values are overwritten by the simulationSettings when initializing the solver.

\noindent SolverIterationData has the following items:
%reference manual TABLE
\begin{center}
  \footnotesize
  \begin{longtable}{| p{4.2cm} | p{2.5cm} | p{0.3cm} | p{3.0cm} | p{6cm} |}
    \hline
    \bf Name & \bf type / function return type & \bf size & \bf default value / function args & \bf description \\ \hline
    adaptiveStep &     bool &      &     True &     True: the step size may be reduced if step fails; no automatic stepsize control\\ \hline
    automaticStepSize &     bool &      &     True &     True: if timeIntegration.automaticStepSize == True AND chosen integrators supports automatic step size control (e.g., DOPRI5); False: constant step size used (step may be reduced if adaptiveStep=True)\\ \hline
    automaticStepSizeError &     Real &      &     0 &     estimated error (relative to atol + rtol*solution) of last step; must be $\le 1$  for a step to be accepted\\ \hline
    currentStepIndex &     Index &      &     0 &     current step index; $i$\\ \hline
    currentStepSize &     Real &      &     0. &     stepSize of current step\\ \hline
    currentTime &     Real &      &     0. &     holds the current simulation time, copy of state.current.time; interval is [startTime,tEnd]; in static solver, duration is loadStepDuration\\ \hline
    discontinuousIteration &     Index &      &     0 &     number of current discontinuous iteration\\ \hline
    discontinuousIterationsCount &     Index &      &     0 &     count total number of discontinuous iterations (min. 1 per step)\\ \hline
    endTime &     Real &      &     0. &     end time of static/dynamic solver\\ \hline
    initialStepSize &     Real &      &     1e-6 &     initial stepSize for dynamic solver; only used, if automaticStepSize is activated\\ \hline
    lastStepSize &     Real &      &     0. &     stepSize suggested from last step or by initial step size; only used, if automaticStepSize is activated\\ \hline
    maxStepSize &     Real &      &     0. &     constant or maximum stepSize\\ \hline
    minStepSize &     Real &      &     0. &     minimum stepSize for static/dynamic solver; only used, if automaticStepSize is activated\\ \hline
    newtonJacobiCount &     Index &      &     0 &     count total Newton jacobian computations\\ \hline
    newtonSteps &     Index &      &     0 &     number of current newton steps\\ \hline
    newtonStepsCount &     Index &      &     0 &     count total Newton steps\\ \hline
    numberOfSteps &     Index &      &     0 &     number of time steps (if fixed size); $n$\\ \hline
    recommendedStepSize &     Real &      &     -1. &     recommended step size $h_{recom}$ after PostNewton(...): $h_{recom} < 0$: no recommendation, $h_{recom}==0$: use minimum step size, $h_{recom}>0$: use specific step size, if no smaller size requested by other reason\\ \hline
    rejectedAutomaticStepSizeSteps &     \tabnewline Index &      &     0 &     count the number of rejected steps in case of automatic step size control (rejected steps are repeated with smaller step size)\\ \hline
    rejectedModifiedNewtonSteps &     Index &      &     0 &     count the number of rejected modified Newton steps (switch to full Newton)\\ \hline
    startTime &     Real &      &     0. &     time at beginning of time integration\\ \hline
    ToString() &     String &      &      &     convert iteration statistics to string; used for displayStatistics option\\ \hline
	  \end{longtable}
	\end{center}


%+++++++++++++++++++++++++++++++++++

\mysubsubsubsection{SolverConvergenceData}
\label{sec:SolverConvergenceData}
Solver internal structure for convergence information: residua, iteration loop errors and error flags. For detailed behavior of these flags, visit the source code!. 

\noindent SolverConvergenceData has the following items:
%reference manual TABLE
\begin{center}
  \footnotesize
  \begin{longtable}{| p{4.2cm} | p{2.5cm} | p{0.3cm} | p{3.0cm} | p{6cm} |}
    \hline
    \bf Name & \bf type / function return type & \bf size & \bf default value / function args & \bf description \\ \hline
    contractivity &     Real &      &     0. &     Newton contractivity = geometric decay of error in every step\\ \hline
    discontinuousIterationError &     Real &      &     0. &     error of discontinuous iterations (contact, friction, ...) outside of Newton iteration\\ \hline
    discontinuousIterationSuccessful &     \tabnewline bool &      &     True &     true, if last discontinuous iteration had success (failure may be recovered by adaptive step)\\ \hline
    errorCoordinateFactor &     Real &      &     1. &     factor may include the number of system coordinates to reduce the residual\\ \hline
    InitializeData() &     void &      &      &     initialize SolverConvergenceData by assigning default values\\ \hline
    jacobianUpdateRequested &     bool &      &     True &     true, if a jacobian update is requested in modified Newton (determined in previous step)\\ \hline
    lastResidual &     Real &      &     0. &     last Newton residual to determine contractivity\\ \hline
    linearSolverCausingRow &     Index &      &     -1 &     -1 if successful, 0 ... n-1, the system equation (=coordinate) index which may have caused the problem, at which the linear solver failed\\ \hline
    linearSolverFailed &     bool &      &     False &     true, if linear solver failed to factorize\\ \hline
    massMatrixNotInvertible &     bool &      &     False &     true, if mass matrix is not invertable during initialization or solution (explicit solver)\\ \hline
    newtonConverged &     bool &      &     False &     true, if Newton has (finally) converged\\ \hline
    newtonSolutionDiverged &     bool &      &     False &     true, if Newton diverged (may be recovered)\\ \hline
    residual &     Real &      &     0. &     current Newton residual\\ \hline
    stepReductionFailed &     bool &      &     False &     true, if iterations over time/static steps failed (finally, cannot be recovered)\\ \hline
    stopNewton &     bool &      &     False &     set true by Newton, if Newton was stopped, e.g., because of exceeding iterations or linear solver failed\\ \hline
	  \end{longtable}
	\end{center}


%+++++++++++++++++++++++++++++++++++

\mysubsubsubsection{SolverOutputData}
\label{sec:SolverOutputData}
Solver internal structure for output modes, output timers and counters.

\noindent SolverOutputData has the following items:
%reference manual TABLE
\begin{center}
  \footnotesize
  \begin{longtable}{| p{4.2cm} | p{2.5cm} | p{0.3cm} | p{3.0cm} | p{6cm} |}
    \hline
    \bf Name & \bf type / function return type & \bf size & \bf default value / function args & \bf description \\ \hline
    cpuLastTimePrinted &     Real &      &     0. &     CPU time when output has been printed last time\\ \hline
    cpuStartTime &     Real &      &     0. &     CPU start time of computation (starts counting at computation of initial conditions)\\ \hline
    finishedSuccessfully &     bool &      &     False &     flag is false until solver finshed successfully (can be used as external trigger)\\ \hline
    InitializeData() &     void &      &      &     initialize SolverOutputData by assigning default values\\ \hline
    lastDiscontinuousIterationsCount &     \tabnewline Index &      &     0 &     discontinuous iterations count when written to console (or file) last time\\ \hline
    lastImageRecorded &     Real &      &     0. &     simulation time when last image has been recorded\\ \hline
    lastNewtonJacobiCount &     Index &      &     0 &     jacobian update count when written to console (or file) last time\\ \hline
    lastNewtonStepsCount &     Index &      &     0 &     newton steps count when written to console (or file) last time\\ \hline
    lastSensorsWritten &     Real &      &     0. &     simulation time when last sensors have been written\\ \hline
    lastSolutionWritten &     Real &      &     0. &     simulation time when last solution has been written\\ \hline
    lastVerboseStepIndex &     Index &      &     0 &     step index when last time written to console (or file)\\ \hline
    multiThreadingMode &     Index &      &     0 &     multithreading mode that has been used: 0=None (serial), 1=NGsolve taskmanager, 2=MicroThreading (Exudyn)\\ \hline
    numberOfThreadsUsed &     Index &      &     0 &     number of threads that have been used in simulation\\ \hline
    sensorValuesTemp &     ResizableVector &      &      &     temporary vector for per sensor values (overwritten for every sensor; usually contains last sensor values)\\ \hline
    sensorValuesTemp2 &     ResizableVector &      &      &     additional temporary vector for per sensor values (overwritten for every sensor; usually contains time+last sensor values)\\ \hline
    stepInformation &     Index &      &     0 &     this is a copy of the solvers stepInformation used for console output\\ \hline
    verboseMode &     Index &      &     0 &     this is a copy of the solvers verboseMode used for console output\\ \hline
    verboseModeFile &     Index &      &     0 &     this is a copy of the solvers verboseModeFile used for file\\ \hline
    writeToSolutionFile &     bool &      &     False &     if false, no solution file is generated and no file is written\\ \hline
    writeToSolverFile &     bool &      &     False &     if false, no solver output file is generated and no file is written\\ \hline
	  \end{longtable}
	\end{center}



%+++++++++++++++++++++++++++++++++++

\mysubsubsubsection{MainSolverStatic}
\label{sec:MainSolverStatic}
PyBind interface (trampoline) class for static solver. With this interface, the static solver and its substructures can be accessed via Python. NOTE that except from SolveSystem(...), these functions are only intended for experienced users and they need to be handled with care, as unexpected crashes may happen if used inappropriate. Furthermore, the functions have a lot of overhead (performance much lower than internal solver) due to Python interfaces, and should thus be used for small systems. To access the solver in Python, write: \\ 
 \texttt{solver = MainSolverStatic()} \\ 
and hereafter you can access all data and functions via 'solver'.

\noindent MainSolverStatic has the following items:
%reference manual TABLE
\begin{center}
  \footnotesize
  \begin{longtable}{| p{4.2cm} | p{2.5cm} | p{0.3cm} | p{3.0cm} | p{6cm} |}
    \hline
    \bf Name & \bf type / function return type & \bf size & \bf default value / function args & \bf description \\ \hline
    conv &     SolverConvergenceData &      &      &     all information about tolerances, errors and residua\\ \hline
    it &     SolverIterationData &      &      &     all information about iterations (steps, discontinuous iteration, newton,...)\\ \hline
    newton &     NewtonSettings &      &      &     copy of newton settings from timeint or staticSolver\\ \hline
    output &     SolverOutputData &      &      &     output modes and timers for exporting solver information and solution\\ \hline
    timer &     CSolverTimer &      &      &     timer which measures the CPU time of solver sub functions\\ \hline
    CheckInitialized(...) &     bool &      &     mainSystem &     check if MainSolver and MainSystem are correctly initialized ==> otherwise raise SysError\\ \hline
    ComputeAlgebraicEquations(...) &     \tabnewline \tabnewline void &      &     mainSystem, velocityLevel=false &     compute the algebraic equations in systemResidual in range(nODE2+nODE1, nODE2+nODE1+nAE)\\ \hline
    ComputeJacobianAE(...) &     void &      &     mainSystem, scalarFactor\_ODE2=1., scalarFactor\_ODE2\_t=0., scalarFactor\_ODE1=1., velocityLevel=false &     add jacobian of algebraic equations (multiplied with factor) to systemJacobian in cSolver; the scalarFactors are scaling the derivatives w.r.t. \hac{ODE2} coordinates, ODE2\_t (velocity) coordinates and ODE1 coordinates; if velocityLevel == true, the constraints are evaluated at velocity level; the scalar factors scalarFactor\_ODE2=0 and scalarFactor\_ODE2 are used for the same ODE2 block in the jacobian\\ \hline
    ComputeJacobianODE1RHS(...) &     void &      &     mainSystem, scalarFactor\_ODE2=1., scalarFactor\_ODE2\_t=0., scalarFactor\_ODE1=1. &     ADD jacobian of ODE1RHS (multiplied with factors for ODE2 and ODE1 coordinates) to the according rows (nODE2:nODE2+nODE1) of the exising systemJacobian in cSolver; it requires a prior call to ComputeJacobianODE2RHS(...); the scalar factors scalarFactor\_ODE2=0 and scalarFactor\_ODE2 are used for the same ODE2 block in the jacobian\\ \hline
    ComputeJacobianODE2RHS(...) &     void &      &     mainSystem, scalarFactor\_ODE2=1., scalarFactor\_ODE2\_t=0., scalarFactor\_ODE1=1. &     set systemJacobian to zero, size = (nODE2+nODE1+nAE) x (nODE2+nODE1+nAE), and add jacobian (multiplied with factors for ODE2 and ODE1 coordinates) of ODE2RHS to systemJacobian in cSolver; using (scalarFactor\_ODE2=-1,scalarFactor\_ODE2=0) gives the stiffness matrix (=derivatives of ODE2 coords) in the nODE2 x nODE2 part, while using (scalarFactor\_ODE2=0,scalarFactor\_ODE2=-1) gives the damping matrix (= derivatives of ODE2 velocity coordinates) in the same part; a superposition of these two parts makes sense for implicit solvers\\ \hline
    ComputeLoadFactor(...) &     Real &      &     simulationSettings &     for static solver, this is a factor in interval [0,1]; MUST be overwritten\\ \hline
    ComputeMassMatrix(...) &     void &      &     mainSystem, scalarFactor=1. &     compute systemMassMatrix (multiplied with factor) in cSolver and return mass nODE2 x nODE2 matrix\\ \hline
    ComputeNewtonJacobian(...) &     void &      &     mainSystem, simulationSettings &     compute jacobian for newton method of given solver method; store result in systemJacobian\\ \hline
    ComputeNewtonResidual(...) &     Real &      &     mainSystem, simulationSettings &     compute residual for Newton method (e.g. static or time step); store residual vector in systemResidual and return scalar residual (specific computation may depend on solver types)\\ \hline
    ComputeNewtonUpdate(...) &     void &      &     mainSystem, simulationSettings, initial=true &     compute update for currentState from newtonSolution (decrement from residual and jacobian); if initial, this is for the initial update with newtonSolution=0\\ \hline
    ComputeODE2RHS(...) &     void &      &     mainSystem &     compute the RHS of \hac{ODE2} equations in systemResidual in range(0,nODE2)\\ \hline
    DiscontinuousIteration(...) &     bool &      &     mainSystem, simulationSettings &     perform discontinuousIteration for static step / time step; CALLS ComputeNewtonResidual\\ \hline
    FinalizeSolver(...) &     void &      &     mainSystem, simulationSettings &     write concluding information (timer statistics, messages) and close files\\ \hline
    FinishStep(...) &     void &      &     mainSystem, simulationSettings &     finish static step / time step; write output of results to file\\ \hline
    GetAEsize() &     Index &      &      &     number of algebraic equations in solver\\ \hline
    GetDataSize() &     Index &      &      &     number of data (history) variables in solver\\ \hline
    GetNewtonSolution() &     NumpyVector &      &      &     get locally stored / last computed solution (=increment) of Newton\\ \hline
    GetODE1size() &     Index &      &      &     number of \hac{ODE1} equations in solver (not yet implemented)\\ \hline
    GetODE2size() &     Index &      &      &     number of \hac{ODE2} equations in solver\\ \hline
    GetSimulationEndTime(...) &     Real &      &     simulationSettings &     compute simulation end time (depends on static or time integration solver)\\ \hline
    GetSolverName() &     std::string &      &      &     get solver name - needed for output file header and visualization window\\ \hline
    GetSystemJacobian() &     NumpyMatrix &      &      &     get locally stored / last computed system jacobian of solver\\ \hline
    GetSystemMassMatrix() &     NumpyMatrix &      &      &     get locally stored / last computed mass matrix of solver\\ \hline
    GetSystemResidual() &     NumpyVector &      &      &     get locally stored / last computed system residual\\ \hline
    HasAutomaticStepSizeControl(...) &     \tabnewline \tabnewline bool &      &     mainSystem, simulationSettings &     return true, if solver supports automatic stepsize control, otherwise false\\ \hline
    IncreaseStepSize(...) &     void &      &     mainSystem, simulationSettings &     increase step size if convergence is good\\ \hline
    InitializeSolver(...) &     bool &      &     mainSystem, simulationSettings &     initialize solverSpecific,data,it,conv; set/compute initial conditions (solver-specific!); initialize output files\\ \hline
    InitializeSolverData(...) &     void &      &     mainSystem, simulationSettings &     initialize all data,it,conv; called from InitializeSolver()\\ \hline
    InitializeSolverInitialConditions(...) &     \tabnewline \tabnewline void &      &     mainSystem, simulationSettings &     set/compute initial conditions (solver-specific!); called from InitializeSolver()\\ \hline
    InitializeSolverOutput(...) &     void &      &     mainSystem, simulationSettings &     initialize output files; called from InitializeSolver()\\ \hline
    InitializeSolverPreChecks(...) &     \tabnewline \tabnewline bool &      &     mainSystem, simulationSettings &     check if system is solvable; initialize dense/sparse computation modes\\ \hline
    InitializeStep(...) &     void &      &     mainSystem, simulationSettings &     initialize static step / time step; Python-functions; do some outputs, checks, etc.\\ \hline
    IsStaticSolver() &     bool &      &      &     return true, if static solver; needs to be overwritten in derived class\\ \hline
    IsVerboseCheck(...) &     bool &      &     level &     return true, if file or console output is at or above the given level\\ \hline
    loadStepGeometricFactor &     Real &      &      &     multiplicative load step factor; this factor is computed from loadStepGeometric parameters in SolveSystem(...)\\ \hline
    Newton(...) &     bool &      &     mainSystem, simulationSettings &     perform Newton method for given solver method\\ \hline
    PostInitializeSolverSpecific(...) &     \tabnewline \tabnewline void &      &     mainSystem, simulationSettings &     post-initialize for solver specific tasks; called at the end of InitializeSolver\\ \hline
    PreInitializeSolverSpecific(...) &     \tabnewline \tabnewline void &      &     mainSystem, simulationSettings &     pre-initialize for solver specific tasks; called at beginning of InitializeSolver, right after Solver data reset\\ \hline
    ReduceStepSize(...) &     bool &      &     mainSystem, simulationSettings, severity &     reduce step size (1..normal, 2..severe problems); return true, if reduction was successful\\ \hline
    SetSystemJacobian(...) &     void &      &     systemJacobian &     set locally stored system jacobian of solver; must have size nODE2+nODE1+nAE\\ \hline
    SetSystemMassMatrix(...) &     void &      &     systemMassMatrix &     set locally stored mass matrix of solver; must have size nODE2+nODE1+nAE\\ \hline
    SetSystemResidual(...) &     void &      &     systemResidual &     set locally stored system residual; must have size nODE2+nODE1+nAE\\ \hline
    SolveSteps(...) &     bool &      &     mainSystem, simulationSettings &     main solver part: calls multiple InitializeStep(...)/ DiscontinuousIteration(...)/ FinishStep(...); do step reduction if necessary; return true if success, false else\\ \hline
    SolveSystem(...) &     bool &      &     mainSystem, simulationSettings &     solve System: InitializeSolver, SolveSteps, FinalizeSolver\\ \hline
    UpdateCurrentTime(...) &     void &      &     mainSystem, simulationSettings &     update currentTime (and load factor); MUST be overwritten in special solver class\\ \hline
    VerboseWrite(...) &     void &      &     level, str &     write to console and/or file in case of level\\ \hline
    WriteCoordinatesToFile(...) &     void &      &     mainSystem, simulationSettings &     write unique coordinates solution file\\ \hline
    WriteSolutionFileHeader(...) &     void &      &     mainSystem, simulationSettings &     write unique file header, depending on static/ dynamic simulation\\ \hline
	  \end{longtable}
	\end{center}


%+++++++++++++++++++++++++++++++++++

\mysubsubsubsection{MainSolverImplicitSecondOrder}
\label{sec:MainSolverImplicitSecondOrder}
PyBind interface (trampoline) class for dynamic implicit solver. Note that this solver includes the classical Newmark method (set useNewmark True; with option of index 2 reduction) as well as the generalized-alpha method. With the interface, the dynamic implicit solver and its substructures can be accessed via Python. NOTE that except from SolveSystem(...), these functions are only intended for experienced users and they need to be handled with care, as unexpected crashes may happen if used inappropriate. Furthermore, the functions have a lot of overhead (still fast, but performance much lower than internal solver) due to Python interfaces, and should thus be used for small systems. To access the solver in Python, write: \\ 
 \texttt{solver = MainSolverImplicitSecondOrder()} \\ 
and hereafter you can access all data and functions via 'solver'.
 In this solver, user functions are possible to extend the solver at certain parts, while keeping the overal C++ performance. User functions, which are added with SetUserFunction...(...), have the arguments (MainSolver, MainSystem, simulationSettings), except for ComputeNewtonUpdate which adds the initial flag as an additional argument and ComputeNewtonResidual, which returns the scalar residual.

\noindent MainSolverImplicitSecondOrder has the following items:
%reference manual TABLE
\begin{center}
  \footnotesize
  \begin{longtable}{| p{4.2cm} | p{2.5cm} | p{0.3cm} | p{3.0cm} | p{6cm} |}
    \hline
    \bf Name & \bf type / function return type & \bf size & \bf default value / function args & \bf description \\ \hline
    conv &     SolverConvergenceData &      &      &     all information about tolerances, errors and residua\\ \hline
    it &     SolverIterationData &      &      &     all information about iterations (steps, discontinuous iteration, newton,...)\\ \hline
    newton &     NewtonSettings &      &      &     copy of newton settings from timeint or staticSolver\\ \hline
    output &     SolverOutputData &      &      &     output modes and timers for exporting solver information and solution\\ \hline
    timer &     CSolverTimer &      &      &     timer which measures the CPU time of solver sub functions; note that solver structures can only be written indirectly, e.g.,  timer=dynamicSolver.timer; timer.useTimer = False; dynamicSolver.timer=timer; however, dynamicSolver.timer.useTimer cannot be written.\\ \hline
    alphaF &     Real &      &      &     copy of parameter in timeIntegration.generalizedAlpha\\ \hline
    alphaM &     Real &      &      &     copy of parameter in timeIntegration.generalizedAlpha\\ \hline
    CheckInitialized(...) &     bool &      &     mainSystem &     check if MainSolver and MainSystem are correctly initialized ==> otherwise raise SysError\\ \hline
    ComputeAlgebraicEquations(...) &     \tabnewline \tabnewline void &      &     mainSystem, velocityLevel=false &     compute the algebraic equations in systemResidual in range(nODE2+nODE1, nODE2+nODE1+nAE)\\ \hline
    ComputeJacobianAE(...) &     void &      &     mainSystem, scalarFactor\_ODE2=1., scalarFactor\_ODE2\_t=0., scalarFactor\_ODE1=1., velocityLevel=false &     add jacobian of algebraic equations (multiplied with factor) to systemJacobian in cSolver; the scalarFactors are scaling the derivatives w.r.t. \hac{ODE2} coordinates, ODE2\_t (velocity) coordinates and ODE1 coordinates; if velocityLevel == true, the constraints are evaluated at velocity level; the scalar factors scalarFactor\_ODE2=0 and scalarFactor\_ODE2 are used for the same ODE2 block in the jacobian\\ \hline
    ComputeJacobianODE1RHS(...) &     void &      &     mainSystem, scalarFactor\_ODE2=1., scalarFactor\_ODE2\_t=0., scalarFactor\_ODE1=1. &     ADD jacobian of ODE1RHS (multiplied with factors for ODE2 and ODE1 coordinates) to the according rows (nODE2:nODE2+nODE1) of the exising systemJacobian in cSolver; it requires a prior call to ComputeJacobianODE2RHS(...); the scalar factors scalarFactor\_ODE2=0 and scalarFactor\_ODE2 are used for the same ODE2 block in the jacobian\\ \hline
    ComputeJacobianODE2RHS(...) &     void &      &     mainSystem, scalarFactor\_ODE2=1., scalarFactor\_ODE2\_t=0., scalarFactor\_ODE1=1. &     set systemJacobian to zero, size = (nODE2+nODE1+nAE) x (nODE2+nODE1+nAE), and add jacobian (multiplied with factors for ODE2 and ODE1 coordinates) of ODE2RHS to systemJacobian in cSolver; using (scalarFactor\_ODE2=-1,scalarFactor\_ODE2=0) gives the stiffness matrix (=derivatives of ODE2 coords) in the nODE2 x nODE2 part, while using (scalarFactor\_ODE2=0,scalarFactor\_ODE2=-1) gives the damping matrix (= derivatives of ODE2 velocity coordinates) in the same part; a superposition of these two parts makes sense for implicit solvers\\ \hline
    ComputeLoadFactor(...) &     Real &      &     simulationSettings &     for static solver, this is a factor in interval [0,1]; MUST be overwritten\\ \hline
    ComputeMassMatrix(...) &     void &      &     mainSystem, scalarFactor=1. &     compute systemMassMatrix (multiplied with factor) in cSolver and return mass nODE2 x nODE2 matrix\\ \hline
    ComputeNewtonJacobian(...) &     void &      &     mainSystem, simulationSettings &     compute jacobian for newton method of given solver method; store result in systemJacobian\\ \hline
    ComputeNewtonResidual(...) &     Real &      &     mainSystem, simulationSettings &     compute residual for Newton method (e.g. static or time step); store residual vector in systemResidual and return scalar residual (specific computation may depend on solver types)\\ \hline
    ComputeNewtonUpdate(...) &     void &      &     mainSystem, simulationSettings, initial=true &     compute update for currentState from newtonSolution (decrement from residual and jacobian); if initial, this is for the initial update with newtonSolution=0\\ \hline
    ComputeODE1RHS(...) &     void &      &     mainSystem &     compute the RHS of \hac{ODE1} equations in systemResidual in range(0,nODE1)\\ \hline
    ComputeODE2RHS(...) &     void &      &     mainSystem &     compute the RHS of \hac{ODE2} equations in systemResidual in range(0,nODE2)\\ \hline
    DiscontinuousIteration(...) &     bool &      &     mainSystem, simulationSettings &     perform discontinuousIteration for static step / time step; CALLS ComputeNewtonResidual\\ \hline
    factJacAlgorithmic &     Real &      &      &     locally computed parameter from generalizedAlpha parameters\\ \hline
    FinalizeSolver(...) &     void &      &     mainSystem, simulationSettings &     write concluding information (timer statistics, messages) and close files\\ \hline
    FinishStep(...) &     void &      &     mainSystem, simulationSettings &     finish static step / time step; write output of results to file\\ \hline
    GetAAlgorithmic() &     NumpyVector &      &      &     get locally stored / last computed algorithmic accelerations\\ \hline
    GetAEsize() &     Index &      &      &     number of algebraic equations in solver\\ \hline
    GetDataSize() &     Index &      &      &     number of data (history) variables in solver\\ \hline
    GetNewtonSolution() &     NumpyVector &      &      &     get locally stored / last computed solution (=increment) of Newton\\ \hline
    GetODE1size() &     Index &      &      &     number of \hac{ODE1} equations in solver (not yet implemented)\\ \hline
    GetODE2size() &     Index &      &      &     number of \hac{ODE2} equations in solver\\ \hline
    GetSimulationEndTime(...) &     Real &      &     simulationSettings &     compute simulation end time (depends on static or time integration solver)\\ \hline
    GetSolverName() &     std::string &      &      &     get solver name - needed for output file header and visualization window\\ \hline
    GetStartOfStepStateAAlgorithmic() &     \tabnewline \tabnewline NumpyVector &      &      &     get locally stored / last computed algorithmic accelerations at start of step\\ \hline
    GetSystemJacobian() &     NumpyMatrix &      &      &     get locally stored / last computed system jacobian of solver\\ \hline
    GetSystemMassMatrix() &     NumpyMatrix &      &      &     get locally stored / last computed mass matrix of solver\\ \hline
    GetSystemResidual() &     NumpyVector &      &      &     get locally stored / last computed system residual\\ \hline
    HasAutomaticStepSizeControl(...) &     \tabnewline \tabnewline bool &      &     mainSystem, simulationSettings &     return true, if solver supports automatic stepsize control, otherwise false\\ \hline
    IncreaseStepSize(...) &     void &      &     mainSystem, simulationSettings &     increase step size if convergence is good\\ \hline
    InitializeSolver(...) &     bool &      &     mainSystem, simulationSettings &     initialize solverSpecific,data,it,conv; set/compute initial conditions (solver-specific!); initialize output files\\ \hline
    InitializeSolverData(...) &     void &      &     mainSystem, simulationSettings &     initialize all data,it,conv; called from InitializeSolver()\\ \hline
    InitializeSolverInitialConditions(...) &     \tabnewline \tabnewline void &      &     mainSystem, simulationSettings &     set/compute initial conditions (solver-specific!); called from InitializeSolver()\\ \hline
    InitializeSolverOutput(...) &     void &      &     mainSystem, simulationSettings &     initialize output files; called from InitializeSolver()\\ \hline
    InitializeSolverPreChecks(...) &     \tabnewline \tabnewline bool &      &     mainSystem, simulationSettings &     check if system is solvable; initialize dense/sparse computation modes\\ \hline
    InitializeStep(...) &     void &      &     mainSystem, simulationSettings &     initialize static step / time step; Python-functions; do some outputs, checks, etc.\\ \hline
    IsStaticSolver() &     bool &      &      &     return true, if static solver; needs to be overwritten in derived class\\ \hline
    IsVerboseCheck(...) &     bool &      &     level &     return true, if file or console output is at or above the given level\\ \hline
    newmarkBeta &     Real &      &      &     copy of parameter in timeIntegration.generalizedAlpha\\ \hline
    newmarkGamma &     Real &      &      &     copy of parameter in timeIntegration.generalizedAlpha\\ \hline
    Newton(...) &     bool &      &     mainSystem, simulationSettings &     perform Newton method for given solver method\\ \hline
    PostInitializeSolverSpecific(...) &     \tabnewline \tabnewline void &      &     mainSystem, simulationSettings &     post-initialize for solver specific tasks; called at the end of InitializeSolver\\ \hline
    PostNewton(...) &     Real &      &     mainSystem, simulationSettings &     call PostNewton for all relevant objects (contact, friction, ... iterations); returns error for discontinuous iteration\\ \hline
    PreInitializeSolverSpecific(...) &     \tabnewline \tabnewline void &      &     mainSystem, simulationSettings &     pre-initialize for solver specific tasks; called at beginning of InitializeSolver, right after Solver data reset\\ \hline
    ReduceStepSize(...) &     bool &      &     mainSystem, simulationSettings, severity &     reduce step size (1..normal, 2..severe problems); return true, if reduction was successful\\ \hline
    SetSystemJacobian(...) &     void &      &     systemJacobian &     set locally stored system jacobian of solver; must have size nODE2+nODE1+nAE\\ \hline
    SetSystemMassMatrix(...) &     void &      &     systemMassMatrix &     set locally stored mass matrix of solver; must have size nODE2+nODE1+nAE\\ \hline
    SetSystemResidual(...) &     void &      &     systemResidual &     set locally stored system residual; must have size nODE2+nODE1+nAE\\ \hline
    SetUserFunctionComputeNewtonJacobian(...) &     \tabnewline \tabnewline void &      &     mainSystem, userFunction &     set user function\\ \hline
    SetUserFunctionComputeNewtonResidual(...) &     \tabnewline \tabnewline void &      &     mainSystem, userFunction &     set user function\\ \hline
    SetUserFunctionComputeNewtonUpdate(...) &     \tabnewline \tabnewline void &      &     mainSystem, userFunction &     set user function\\ \hline
    SetUserFunctionDiscontinuousIteration(...) &     \tabnewline \tabnewline void &      &     mainSystem, userFunction &     set user function\\ \hline
    SetUserFunctionFinishStep(...) &     \tabnewline \tabnewline void &      &     mainSystem, userFunction &     set user function\\ \hline
    SetUserFunctionInitializeStep(...) &     \tabnewline \tabnewline void &      &     mainSystem, userFunction &     set user function\\ \hline
    SetUserFunctionNewton(...) &     void &      &     mainSystem, userFunction &     set user function\\ \hline
    SetUserFunctionPostNewton(...) &     \tabnewline \tabnewline void &      &     mainSystem, userFunction &     set user function\\ \hline
    SetUserFunctionUpdateCurrentTime(...) &     \tabnewline \tabnewline void &      &     mainSystem, userFunction &     set user function\\ \hline
    SolveSteps(...) &     bool &      &     mainSystem, simulationSettings &     main solver part: calls multiple InitializeStep(...)/ DiscontinuousIteration(...)/ FinishStep(...); do step reduction if necessary; return true if success, false else\\ \hline
    SolveSystem(...) &     bool &      &     mainSystem, simulationSettings &     solve System: InitializeSolver, SolveSteps, FinalizeSolver\\ \hline
    spectralRadius &     Real &      &      &     copy of parameter in timeIntegration.generalizedAlpha\\ \hline
    UpdateCurrentTime(...) &     void &      &     mainSystem, simulationSettings &     update currentTime (and load factor); MUST be overwritten in special solver class\\ \hline
    VerboseWrite(...) &     void &      &     level, str &     write to console and/or file in case of level\\ \hline
    WriteCoordinatesToFile(...) &     void &      &     mainSystem, simulationSettings &     write unique coordinates solution file\\ \hline
    WriteSolutionFileHeader(...) &     void &      &     mainSystem, simulationSettings &     write unique file header, depending on static/ dynamic simulation\\ \hline
	  \end{longtable}
	\end{center}


%+++++++++++++++++++++++++++++++++++

\mysubsubsubsection{MainSolverExplicit}
\label{sec:MainSolverExplicit}
PyBind interface (trampoline) class for dynamic explicit solver. Note that this solver includes the 1st order explicit Euler scheme and the 4th order Runge-Kutta scheme with 5th order error estimation (DOPRI5). With the interface, the solver and its substructures can be accessed via Python. NOTE that except from SolveSystem(...), these functions are only intended for experienced users and they need to be handled with care, as unexpected crashes may happen if used inappropriate. Furthermore, the functions have a lot of overhead (still fast, but performance much lower than internal solver) due to Python interfaces, and should thus be used for small systems. To access the solver in Python, write \\ 
 \texttt{solver = MainSolverExplicit()} \\ 
and hereafter you can access all data and functions via 'solver'.
 In this solver, no user functions are possible, but you can use SolverImplicitSecondOrder instead (turning off Newton gives explicit scheme ...).

\noindent MainSolverExplicit has the following items:
%reference manual TABLE
\begin{center}
  \footnotesize
  \begin{longtable}{| p{4.2cm} | p{2.5cm} | p{0.3cm} | p{3.0cm} | p{6cm} |}
    \hline
    \bf Name & \bf type / function return type & \bf size & \bf default value / function args & \bf description \\ \hline
    conv &     SolverConvergenceData &      &      &     all information about tolerances, errors and residua\\ \hline
    it &     SolverIterationData &      &      &     all information about iterations (steps, discontinuous iteration, newton,...)\\ \hline
    output &     SolverOutputData &      &      &     output modes and timers for exporting solver information and solution\\ \hline
    timer &     CSolverTimer &      &      &     timer which measures the CPU time of solver sub functions\\ \hline
    ComputeLoadFactor(...) &     Real &      &     simulationSettings &     for static solver, this is a factor in interval [0,1]; MUST be overwritten\\ \hline
    ComputeMassMatrix(...) &     void &      &     mainSystem, scalarFactor=1. &     compute systemMassMatrix (multiplied with factor) in cSolver and return mass matrix\\ \hline
    ComputeNewtonJacobian(...) &     void &      &     mainSystem, simulationSettings &     compute jacobian for newton method of given solver method; store result in systemJacobian\\ \hline
    ComputeNewtonResidual(...) &     Real &      &     mainSystem, simulationSettings &     compute residual for Newton method (e.g. static or time step); store residual vector in systemResidual and return scalar residual (specific computation may depend on solver types)\\ \hline
    ComputeNewtonUpdate(...) &     void &      &     mainSystem, simulationSettings, initial=true &     compute update for currentState from newtonSolution (decrement from residual and jacobian); if initial, this is for the initial update with newtonSolution=0\\ \hline
    ComputeODE1RHS(...) &     void &      &     mainSystem &     compute the RHS of \hac{ODE1} equations in systemResidual in range(0,nODE1)\\ \hline
    ComputeODE2RHS(...) &     void &      &     mainSystem &     compute the RHS of \hac{ODE2} equations in systemResidual in range(0,nODE2)\\ \hline
    DiscontinuousIteration(...) &     bool &      &     mainSystem, simulationSettings &     perform discontinuousIteration for static step / time step; CALLS ComputeNewtonResidual\\ \hline
    FinalizeSolver(...) &     void &      &     mainSystem, simulationSettings &     write concluding information (timer statistics, messages) and close files\\ \hline
    FinishStep(...) &     void &      &     mainSystem, simulationSettings &     finish static step / time step; write output of results to file\\ \hline
    GetAEsize() &     Index &      &      &     number of algebraic equations in solver\\ \hline
    GetDataSize() &     Index &      &      &     number of data (history) variables in solver\\ \hline
    GetMethodOrder() &     Index &      &      &     return order of method (higher value in methods with automatic step size, e.g., DOPRI5=5)\\ \hline
    GetNumberOfStages() &     Index &      &      &     return number of stages in current method\\ \hline
    GetODE1size() &     Index &      &      &     number of \hac{ODE1} equations in solver (not yet implemented)\\ \hline
    GetODE2size() &     Index &      &      &     number of \hac{ODE2} equations in solver\\ \hline
    GetSimulationEndTime(...) &     Real &      &     simulationSettings &     compute simulation end time (depends on static or time integration solver)\\ \hline
    GetSolverName() &     std::string &      &      &     get solver name - needed for output file header and visualization window\\ \hline
    GetSystemMassMatrix() &     NumpyMatrix &      &      &     get locally stored / last computed mass matrix of solver\\ \hline
    GetSystemResidual() &     NumpyVector &      &      &     get locally stored / last computed system residual\\ \hline
    HasAutomaticStepSizeControl(...) &     \tabnewline \tabnewline bool &      &     mainSystem, simulationSettings &     return true, if solver supports automatic stepsize control, otherwise false\\ \hline
    IncreaseStepSize(...) &     void &      &     mainSystem, simulationSettings &     increase step size if convergence is good\\ \hline
    InitializeSolver(...) &     bool &      &     mainSystem, simulationSettings &     initialize solverSpecific,data,it,conv; set/compute initial conditions (solver-specific!); initialize output files\\ \hline
    InitializeSolverData(...) &     void &      &     mainSystem, simulationSettings &     initialize all data,it,conv; called from InitializeSolver()\\ \hline
    InitializeSolverInitialConditions(...) &     \tabnewline \tabnewline void &      &     mainSystem, simulationSettings &     set/compute initial conditions (solver-specific!); called from InitializeSolver()\\ \hline
    InitializeSolverOutput(...) &     void &      &     mainSystem, simulationSettings &     initialize output files; called from InitializeSolver()\\ \hline
    InitializeSolverPreChecks(...) &     \tabnewline \tabnewline bool &      &     mainSystem, simulationSettings &     check if system is solvable; initialize dense/sparse computation modes\\ \hline
    InitializeStep(...) &     void &      &     mainSystem, simulationSettings &     initialize static step / time step; Python-functions; do some outputs, checks, etc.\\ \hline
    IsStaticSolver() &     bool &      &      &     return true, if static solver; needs to be overwritten in derived class\\ \hline
    IsVerboseCheck(...) &     bool &      &     level &     return true, if file or console output is at or above the given level\\ \hline
    Newton(...) &     bool &      &     mainSystem, simulationSettings &     perform Newton method for given solver method\\ \hline
    PostInitializeSolverSpecific(...) &     \tabnewline \tabnewline void &      &     mainSystem, simulationSettings &     post-initialize for solver specific tasks; called at the end of InitializeSolver\\ \hline
    PreInitializeSolverSpecific(...) &     \tabnewline \tabnewline void &      &     mainSystem, simulationSettings &     pre-initialize for solver specific tasks; called at beginning of InitializeSolver, right after Solver data reset\\ \hline
    ReduceStepSize(...) &     bool &      &     mainSystem, simulationSettings, severity &     reduce step size (1..normal, 2..severe problems); return true, if reduction was successful\\ \hline
    SetSystemMassMatrix(...) &     void &      &     systemMassMatrix &     set locally stored mass matrix of solver; must have size nODE2+nODE1+nAE\\ \hline
    SetSystemResidual(...) &     void &      &     systemResidual &     set locally stored system residual; must have size nODE2+nODE1+nAE\\ \hline
    SolveSteps(...) &     bool &      &     mainSystem, simulationSettings &     main solver part: calls multiple InitializeStep(...)/ DiscontinuousIteration(...)/ FinishStep(...); do step reduction if necessary; return true if success, false else\\ \hline
    SolveSystem(...) &     bool &      &     mainSystem, simulationSettings &     solve System: InitializeSolver, SolveSteps, FinalizeSolver\\ \hline
    UpdateCurrentTime(...) &     void &      &     mainSystem, simulationSettings &     update currentTime (and load factor); MUST be overwritten in special solver class\\ \hline
    VerboseWrite(...) &     void &      &     level, str &     write to console and/or file in case of level\\ \hline
    WriteCoordinatesToFile(...) &     void &      &     mainSystem, simulationSettings &     write unique coordinates solution file\\ \hline
    WriteSolutionFileHeader(...) &     void &      &     mainSystem, simulationSettings &     write unique file header, depending on static/ dynamic simulation\\ \hline
	  \end{longtable}
	\end{center}

This section includes the reference manual for structures (such as for solvers, helper structures, etc.) 
and settings which are available in the python interface, e.g., simulation settings, visualization settings. 
The data is auto-generated from the according interfaces in order to keep fully up-to-date with changes.
