% ++++++++++++++++++++++
% description of manual pybind interfaces; generated by Johannes Gerstmayr
% ++++++++++++++++++++++

\mysection{Python-C++ command interface}
\label{sec:PCpp:command:interface}

This chapter lists the basic interface functions which can be used to set up a \codeName\ model in Python.

\mysubsection{General information on Python-C++ interface}
\label{sec:generalPythonInterface}
This chapter lists the basic interface functions which can be used to set up 
a \codeName\ model in Python. Note that some functions or classes will be used in examples, which are explained in detail later on.
In the following, some basic steps and concepts for usage are shown, references to all functions are placed hereafter:

To import the module, just include the \codeName\ module in Python:
\bi
  \item[] \texttt{import exudyn as exu}
\ei
For compatibility with examples and other users, we recommend to use the \texttt{exu} abbreviation throughout. In addition, you may work with a convenient interface for your items, therefore also always include:
\bi
  \item[] \texttt{from exudyn.itemInterface import *}
\ei
Note that including \texttt{exudyn.utilities} will cover \texttt{itemInterface}. Also note that \texttt{from ... import *} is not recommended in general and it will not work in certain cases, e.g., if you like to compute on a cluster. However, it greatly simplifies life for smaller models and you may replace imports in your files afterwards by removing the star import.

The general hub to multibody dynamics models is provided by the classes \texttt{SystemContainer} and \texttt{MainSystem}, except for some very basic system functionality (which is inside the \codeName\ module). 

You can create a new \texttt{SystemContainer}, which is a class that is initialized by assigning a system container to a variable, usually denoted as \texttt{SC}:
\bi
  \item[] \texttt{SC = exu.SystemContainer()}
\ei
Note that creating a second \texttt{exu.SystemContainer()} will be independent of \texttt{SC} and therefore usually makes no sense.

To add a MainSystem to system container SC and store as variable mbs, write:
\bi
  \item[] \texttt{mbs = SC.AddSystem()}
\ei
Furthermore, there are a couple of commands available directly in the \texttt{exudyn} module, given in the following subsections.Regarding the \mybold{(basic) module access}, functions are related to the \texttt{exudyn = exu} module, see these examples:
\pythonstyle
\begin{lstlisting}[language=Python, firstnumber=1]

#  import exudyn module:
import exudyn as exu
#  print detailed exudyn version, Python version (at which it is compiled):
exu.GetVersionString(addDetails = True)
#  set precision of C++ output to console
exu.SetOutputPrecision(numberOfDigits)
#  turn on/off output to console
exu.SetWriteToConsole(False)
#  invalid index, may depend on compilation settings:
nInvalid = exu.InvalidIndex() #the invalid index, depends on architecture and version
\end{lstlisting}


Understanding the usage of functions for python object \texttt{SystemContainer} of the module \texttt{exudyn}, the following examples might help:
\pythonstyle
\begin{lstlisting}[language=Python, firstnumber=1]

#import exudyn module:
import exudyn as exu
#  import utilities (includes itemInterface, basicUtilities, 
#                  advancedUtilities, rigidBodyUtilities, graphicsDataUtilities):
from exudyn.utilities import *
#  create system container and store in SC:
SC = exu.SystemContainer()
#  add a MainSystem (multibody system) to system container SC and store as mbs:
mbs = SC.AddSystem()
#  add a second MainSystem to system container SC and store as mbs2:
mbs2 = SC.AddSystem()
#  print number of systems available:
nSys = SC.NumberOfSystems()
exu.Print(nSys) #or just print(nSys)
#  reset system container (mbs becomes invalid):
SC.Reset()
#  delete mbs (usually not necessary):
del mbs
\end{lstlisting}


If you run a parameter variation (check \texttt{Examples/parameterVariationExample.py}), you may reset or delete the created \texttt{MainSystem} \texttt{mbs} and the \texttt{SystemContainer} \texttt{SC} before creating new instances in order to avoid memory growth.

\mysubsubsection{Item index}
\label{sec:itemIndex}
Many functions will work with node numbers (\texttt{NodeIndex}), object numbers (\texttt{ObjectIndex}),marker numbers (\texttt{MarkerIndex}) and others. These numbers are special objects, which have been introduced in order to avoid mixing up, e.g., node and object numbers. 

For example, the command \texttt{mbs.AddNode(...)} returns a \texttt{NodeIndex}. For these indices, the following rules apply:
\bi
  \item[] \texttt{mbs.Add[Node|Object|...](...)} returns a specific \texttt{NodeIndex}, \texttt{ObjectIndex}, ...
  \item[] You can create any item index, e.g., using \texttt{ni = NodeIndex(42)} or \texttt{oi = ObjectIndex(42)}
  \item[] The benefit of these indices comes as they may not be mixed up, e.g., using an object index instead of a node index.
  \item[] You can convert any item index, e.g., NodeIndex \texttt{ni} into an integer number using \texttt{int(ni)} of \texttt{ni.GetIndex()}
  \item[] Still, you can use integers as initialization for item numbers, e.g.:\\\texttt{mbs.AddObject(MassPoint(nodeNumber=13, ...))}\\However, it must be a pure integer type.
  \item[] You can make integer calculations with such indices, e.g., \texttt{oi = 2*ObjectIndex(42)+1} restricing to addition, subtraction and multiplication. Currently, the result of such calculations is a \texttt{int} type andoperating on mixed indices is not checked (but may raise exceptions in future).
  \item[] You can also print item indices, e.g., \texttt{print(ni)} as it converts to string by default.
  \item[] If you are unsure about the type of an index, use \texttt{ni.GetTypeString()} to show the index type.
\ei
\mysubsubsection{Copying and referencing C++ objects}
\label{sec:generalPythonInterface:copyref}
As a key concept to working with \codeName\ , most data which is retrieved by C++ interface functions is copied.
Experienced Python users may know that it is a key concept to Python to often use references instead of copying, which is
sometimes error-prone but offers a computationally efficient behavior.
There are only a few very important cases where data is references in \codeName\ , the main ones are 
\texttt{SystemContainer}, 
\texttt{MainSystem}, 
\texttt{VisualizationSettings}, and
\texttt{SimulationSettings} which are always references to internal C++ classes.
The following code snippets and comments should explain this behavior:
\pythonstyle
\begin{lstlisting}[language=Python, firstnumber=1]

#create system container, referenced from SC:
SC = exu.SystemContainer()
SC2 = SC                           #this will only put a reference to SC
                                   #SC2 and SC represent the SAME C++ object
#add a MainSystem (multibody system):
mbs = SC.AddSystem()               #get reference mbs to C++ system
mbs2=mbs                           #again, mbs2 and mbs refer to the same C++ object
og = mbs.AddObject(ObjectGround()) #copy data of ObjectGround() into C++
o0 = mbs.GetObject(0)              #get copy of internal data as dictionary
del o0                             #delete the local dictionary; C++ data not affected
\end{lstlisting}


\mysubsubsection{Exceptions and Error Messages}
\label{sec:cinterface:exceptions}
There are several levels of type and argument checks, leading to different types of errors and exceptions. The according error messages are non-unique, because they may be raised in Python modules or in C++, and they may be raised on different levels of the code. Error messages depend on Python version and on your iPython console. Very often the exception may be called \texttt{ValueError}, but it mustnot mean that it is a wrong error, but it could also be, e.g., a wrong order of function calls.

As an example, a type conversion error is raised when providing wrong argument types, e.g., try \texttt{exu.GetVersionString('abs')}:
\pythonstyle
\begin{lstlisting}[language=Python, firstnumber=1]

Traceback (most recent call last):

File "C:\Users\username\AppData\Local\Temp\ipykernel_24988\2212168679.py", line 1, in <module>
    exu.GetVersionString('abs')

TypeError: GetVersionString(): incompatible function arguments. The following argument types are supported:
    1. (addDetails: bool = False) -> str

Invoked with: 'abs'
\end{lstlisting}


Note that your particular error message may be different.

Another error results from internal type and range checking, saying User ERROR, as it is due to a wrong input of the user. For this, we try
\pythonstyle
\begin{lstlisting}[language=Python, firstnumber=1]
mbs.AddObject('abc')
\end{lstlisting}


Which results in a error message similar to:
\pythonstyle
\begin{lstlisting}[language=Python, firstnumber=1]

=========================================
User ERROR [file 'C:\Users\username\AppData\Local\Temp\ipykernel_24988\2838049308.py', line 1]: 
Error in AddObject(...):
Check your python code (negative indices, invalid or undefined parameters, ...)

=========================================

Traceback (most recent call last):

  File "C:\Users\username\AppData\Local\Temp\ipykernel_24988\2838049308.py", line 1, in <module>
    mbs.AddObject('abc')

RuntimeError: Exudyn: parsing of Python file terminated due to Python (user) error
\end{lstlisting}


Finally, there may be system errors. They may be caused due to previous wrong input, but if there is no reason seen, it may be appropriate to report this error on \exuUrl{https://github.com/jgerstmayr/EXUDYN}{github.com/jgerstmayr/EXUDYN/} .

Be careful in reading and interpreting such error messages. You should \mybold{read them from top to bottom}, as the cause may be in the beginning. Often files and line numbers of errors are provided (e.g., if you have a longer script). In the ultimate case, try to comment parts of your code or deactivate items to see where the error comes from. See also section on Trouble shooting and FAQ.

%++++++++++++++++++++
\mysubsection{\codeName}



These are the access functions to the \codeName\ module. General usage is explained in \refSection{sec:generalPythonInterface} and examples are provided there. The C++ module \texttt{exudyn} is the root level object linked between Python and C++.In the installed site-packages, the according file is usually denoted as \texttt{exudynCPP.pyd} for the regular module, \texttt{exudynCPPfast.pyd} for the module without range checks and \texttt{exudynCPPnoAVX.pyd} for the module compiled without AVX vector extensions (may depend on your installation).
\pythonstyle
\begin{lstlisting}[language=Python, firstnumber=1]

#import exudyn module:
import exudyn as exu
#create systemcontainer and mbs:
SC = exu.SystemContainer()
mbs = SC.AddSystem()
\end{lstlisting}

\begin{center}
\footnotesize
\begin{longtable}{| p{8cm} | p{8cm} |} 
\hline
{\bf function/structure name} & {\bf description}\\ \hline
  GetVersionString(addDetails = False) & Get Exudyn built version as string (if addDetails=True, adds more information on compilation Python version, platform, etc.; the Python micro version may differ from that you are working with; AVX2 shows that you are running a AVX2 compiled version)\\ \hline 
  Help() & Show basic help information\\ \hline 
  RequireVersion(requiredVersionString) & Checks if the installed version is according to the required version. Major, micro and minor version must agree the required level. This function is defined in the \texttt{\_\_init\_\_.py} file\tabnewline 
    \textcolor{steelblue}{{\bf EXAMPLE}: \tabnewline 
    \texttt{exu.RequireVersion("1.0.31")}}\\ \hline 
  StartRenderer(verbose = 0) & Start OpenGL rendering engine (in separate thread) for visualization of rigid or flexible multibody system; use verbose=1 to output information during OpenGL window creation; verbose=2 produces more output and verbose=3 gives a debug level; some of the information will only be seen in windows command (powershell) windows or linux shell, but not inside iPython of e.g. Spyder\\ \hline 
  StopRenderer() & Stop OpenGL rendering engine\\ \hline 
  IsRendererActive() & returns True if GLFW renderer is available and running; otherwise False\\ \hline 
  DoRendererIdleTasks(waitSeconds = 0) & Call this function in order to interact with Renderer window; use waitSeconds in order to run this idle tasks while animating a model (e.g. waitSeconds=0.04), use waitSeconds=0 without waiting, or use waitSeconds=-1 to wait until window is closed\\ \hline 
  SolveStatic(mbs, simulationSettings = exudyn.SimulationSettings(), updateInitialValues = False, storeSolver = True) & Static solver function, mapped from module \texttt{solver}, to solve static equations (without inertia terms) of constrained rigid or flexible multibody system; for details on the Python interface see \refSection{sec:solver:SolveStatic}; for background on solvers, see \refSection{sec:solvers}\\ \hline 
  SolveDynamic(mbs, simulationSettings = exudyn.SimulationSettings(), solverType = exudyn.DynamicSolverType.GeneralizedAlpha, updateInitialValues = False, storeSolver = True) & Dynamic solver function, mapped from module \texttt{solver}, to solve equations of motion of constrained rigid or flexible multibody system; for details on the Python interface see \refSection{sec:solver:SolveDynamic}; for background on solvers, see \refSection{sec:solvers}\\ \hline 
  ComputeODE2Eigenvalues(mbs, simulationSettings = exudyn.SimulationSettings(), useSparseSolver = False, numberOfEigenvalues = -1, setInitialValues = True, convert2Frequencies = False) & Simple interface to scipy eigenvalue solver for eigenvalue analysis of the second order differential equations part in mbs, mapped from module \texttt{solver}; for details on the Python interface see \refSection{sec:solver:ComputeODE2Eigenvalues}\\ \hline 
  SetOutputPrecision(numberOfDigits) & Set the precision (integer) for floating point numbers written to console (reset when simulation is started!); NOTE: this affects only floats converted to strings inside C++ exudyn; if you print a float from Python, it is usually printed with 16 digits; if printing numpy arrays, 8 digits are used as standard, to be changed with numpy.set\_printoptions(precision=16); alternatively convert into a list\\ \hline 
  SetLinalgOutputFormatPython(flagPythonFormat) & True: use Python format for output of vectors and matrices; False: use matlab format\\ \hline 
  SetWriteToConsole(flag) & set flag to write (True) or not write to console; default = True\\ \hline 
  SetWriteToFile(filename, flagWriteToFile = True, flagAppend = False) & set flag to write (True) or not write to console; default value of flagWriteToFile = False; flagAppend appends output to file, if set True; in order to finalize the file, write \texttt{exu.SetWriteToFile('', False)} to close the output file\tabnewline 
    \textcolor{steelblue}{{\bf EXAMPLE}: \tabnewline 
    \texttt{exu.SetWriteToConsole(False) \#no output to console\tabnewline
    exu.SetWriteToFile(filename={\textquotesingle}testOutput.log{\textquotesingle}, flagWriteToFile=True, flagAppend=False)\tabnewline
    exu.Print({\textquotesingle}print this to file{\textquotesingle})\tabnewline
    exu.SetWriteToFile({\textquotesingle}{\textquotesingle}, False) \#terminate writing to file which closes the file}}\\ \hline 
  SetPrintDelayMilliSeconds(delayMilliSeconds) & add some delay (in milliSeconds) to printing to console, in order to let Spyder process the output; default = 0\\ \hline 
  Print() & this allows printing via exudyn with similar syntax as in Python print(args) except for keyword arguments: print('test=',42); allows to redirect all output to file given by SetWriteToFile(...); does not output in case that SetWriteToConsole is set to False\\ \hline 
  SuppressWarnings(flag) & set flag to suppress (=True) or enable (=False) warnings\\ \hline 
  InfoStat(writeOutput = True) & Retrieve list of global information on memory allocation and other counts as list:[array\_new\_counts, array\_delete\_counts, vector\_new\_counts, vector\_delete\_counts, matrix\_new\_counts, matrix\_delete\_counts, linkedDataVectorCast\_counts]; May be extended in future; if writeOutput==True, it additionally prints the statistics; counts for new vectors and matrices should not depend on numberOfSteps, except for some objects such as ObjectGenericODE2 and for (sensor) output to files; Not available if code is compiled with \_\_FAST\_EXUDYN\_LINALG flag\\ \hline 
  Go() & Creates a SystemContainer SC and a main multibody system mbs\\ \hline 
  InvalidIndex() & This function provides the invalid index, which may depend on the kind of 32-bit, 64-bit signed or unsigned integer; e.g. node index or item index in list; currently, the InvalidIndex() gives -1, but it may be changed in future versions, therefore you should use this function\\ \hline 
  variables & this dictionary may be used by the user to store exudyn-wide data in order to avoid global Python variables; usage: exu.variables["myvar"] = 42 \\ \hline  
  sys & this dictionary is used and reserved by the system, e.g. for testsuite, graphics or system function to store module-wide data in order to avoid global Python variables; the variable exu.sys['renderState'] contains the last render state after exu.StopRenderer() and can be used for subsequent simulations \\ \hline  
\end{longtable}
\end{center}

%++++++++++++++++++++
\mysubsection{SystemContainer}



The SystemContainer is the top level of structures in \codeName. The container holds all (multibody) systems, solvers and all other data structures for computation. Currently, only one container shall be used. In future, multiple containers might be usable at the same time.Regarding the \mybold{(basic) module access}, functions are related to the \texttt{exudyn = exu} module, see also the introduction of this chapter and this example:
\pythonstyle
\begin{lstlisting}[language=Python, firstnumber=1]

import exudyn as exu
#create system container and store by reference in SC:
SC = exu.SystemContainer() 
#add MainSystem to SC:
mbs = SC.AddSystem()
\end{lstlisting}

\begin{center}
\footnotesize
\begin{longtable}{| p{8cm} | p{8cm} |} 
\hline
{\bf function/structure name} & {\bf description}\\ \hline
  Reset() & delete all multibody systems and reset SystemContainer (including graphics); this also releases SystemContainer from the renderer, which requires SC.AttachToRenderEngine() to be called in order to reconnect to rendering; a safer way is to delete the current SystemContainer and create a new one (SC=SystemContainer() )\\ \hline 
  AddSystem() & add a new computational system\\ \hline 
  NumberOfSystems() & obtain number of multibody systems available in system container\\ \hline 
  GetSystem(systemNumber) & obtain multibody systems with index from system container\\ \hline 
  visualizationSettings & this structure is read/writeable and contains visualization settings, which are immediately applied to the rendering window. \tabnewline
    EXAMPLE:\tabnewline
    SC = exu.SystemContainer()\tabnewline
    SC.visualizationSettings.autoFitScene=False  \\ \hline  
  GetRenderState() & Get dictionary with current render state (openGL zoom, modelview, etc.); will have no effect if GLFW\_GRAPHICS is deactivated\tabnewline 
    \textcolor{steelblue}{{\bf EXAMPLE}: \tabnewline 
    \texttt{SC = exu.SystemContainer()\tabnewline
    renderState = SC.GetRenderState() \tabnewline
    print(renderState[{\textquotesingle}zoom{\textquotesingle}])}}\\ \hline 
  SetRenderState(renderState) & Set current render state (openGL zoom, modelview, etc.) with given dictionary; usually, this dictionary has been obtained with GetRenderState; will have no effect if GLFW\_GRAPHICS is deactivated\tabnewline 
    \textcolor{steelblue}{{\bf EXAMPLE}: \tabnewline 
    \texttt{SC = exu.SystemContainer()\tabnewline
    SC.SetRenderState(renderState)}}\\ \hline 
  RedrawAndSaveImage() & Redraw openGL scene and save image (command waits until process is finished)\\ \hline 
  WaitForRenderEngineStopFlag() & Wait for user to stop render engine (Press 'Q' or Escape-key); this command is used to have active response of the render window, e.g., to open the visualization dialog or use the right-mouse-button; behaves similar as mbs.WaitForUserToContinue()\\ \hline 
  RenderEngineZoomAll() & Send zoom all signal, which will perform zoom all at next redraw request\\ \hline 
  RedrawAndSaveImage() & Redraw openGL scene and save image (command waits until process is finished)\\ \hline 
  AttachToRenderEngine() & Links the SystemContainer to the render engine, such that the changes in the graphics structure drawn upon updates, etc.; done automatically on creation of SystemContainer; return False, if no renderer exists (e.g., compiled without GLFW) or cannot be linked (if other SystemContainer already linked)\\ \hline 
  DetachFromRenderEngine() & Releases the SystemContainer from the render engine; return True if successfully released, False if no GLFW available or detaching failed\\ \hline 
  SendRedrawSignal() & This function is used to send a signal to the renderer that all MainSystems (mbs) shall be redrawn\\ \hline 
  GetCurrentMouseCoordinates(useOpenGLcoordinates = False) & Get current mouse coordinates as list [x, y]; x and y being floats, as returned by GLFW, measured from top left corner of window; use GetCurrentMouseCoordinates(useOpenGLcoordinates=True) to obtain OpenGLcoordinates of projected plane\\ \hline 
\end{longtable}
\end{center}

%++++++++++++++++++++
\mysubsection{MainSystem}



This is the structure which defines a (multibody) system. In C++, there is a MainSystem (links to Python) and a System (computational part). For that reason, the name is MainSystem on the Python side, but it is often just called 'system'. For compatibility, it is recommended to denote the variable holding this system as mbs, the multibody dynamics system. It can be created, visualized and computed. Use the following functions for system manipulation.
\pythonstyle
\begin{lstlisting}[language=Python, firstnumber=1]

import exudyn as exu
SC = exu.SystemContainer()
mbs = SC.AddSystem()
\end{lstlisting}

\begin{center}
\footnotesize
\begin{longtable}{| p{8cm} | p{8cm} |} 
\hline
{\bf function/structure name} & {\bf description}\\ \hline
  Assemble() & assemble items (nodes, bodies, markers, loads, ...) of multibody system; Calls CheckSystemIntegrity(...), AssembleCoordinates(), AssembleLTGLists(), AssembleInitializeSystemCoordinates(), and AssembleSystemInitialize()\\ \hline 
  AssembleCoordinates() & assemble coordinates: assign computational coordinates to nodes and constraints (algebraic variables)\\ \hline 
  AssembleLTGLists() & build \ac{LTG} coordinate lists for objects (used to build global ODE2RHS, MassMatrix, etc. vectors and matrices) and store special object lists (body, connector, constraint, ...)\\ \hline 
  AssembleInitializeSystemCoordinates() & initialize all system-wide coordinates based on initial values given in nodes\\ \hline 
  AssembleSystemInitialize() & initialize some system data, e.g., generalContact objects (searchTree, etc.)\\ \hline 
  Reset() & reset all lists of items (nodes, bodies, markers, loads, ...) and temporary vectors; deallocate memory\\ \hline 
  GetSystemContainer() & return the systemContainer where the mainSystem (mbs) was created\\ \hline 
  WaitForUserToContinue(printMessage = True) & interrupt further computation until user input --> 'pause' function; this command runs a loop in the background to have active response of the render window, e.g., to open the visualization dialog or use the right-mouse-button; behaves similar as SC.WaitForRenderEngineStopFlagthis()\\ \hline 
  SendRedrawSignal() & this function is used to send a signal to the renderer that the scene shall be redrawn because the visualization state has been updated\\ \hline 
  GetRenderEngineStopFlag() & get the current stop simulation flag; True=user wants to stop simulation\\ \hline 
  SetRenderEngineStopFlag() & set the current stop simulation flag; set to False, in order to continue a previously user-interrupted simulation\\ \hline 
  ActivateRendering(flag = True) & activate (flag=True) or deactivate (flag=False) rendering for this system\\ \hline 
  SetPreStepUserFunction() & Sets a user function PreStepUserFunction(mbs, t) executed at beginning of every computation step; in normal case return True; return False to stop simulation after current step\tabnewline 
    \textcolor{steelblue}{{\bf EXAMPLE}: \tabnewline 
    \texttt{def PreStepUserFunction(mbs, t):\tabnewline
     \phantom{XXXX} print(mbs.systemData.NumberOfNodes())\tabnewline
     \phantom{XXXX} if(t>1): \tabnewline
     \phantom{XXXX}  \phantom{XXXX} return False \tabnewline
     \phantom{XXXX} return True \tabnewline
     mbs.SetPreStepUserFunction(PreStepUserFunction)}}\\ \hline 
  SetPostNewtonUserFunction() & Sets a user function PostNewtonUserFunction(mbs, t) executed after successful Newton iteration in implicit or static solvers and after step update of explicit solvers, but BEFORE PostNewton functions are called by the solver; function returns list [discontinuousError, recommendedStepSize], containing a error of the PostNewtonStep, which is compared to [solver].discontinuous.iterationTolerance. The recommendedStepSize shall be negative, if no recommendation is given, 0 in order to enforce minimum step size or a specific value to which the current step size will be reduced and the step will be repeated; use this function, e.g., to reduce step size after impact or change of data variables\tabnewline 
    \textcolor{steelblue}{{\bf EXAMPLE}: \tabnewline 
    \texttt{def PostNewtonUserFunction(mbs, t):\tabnewline
     \phantom{XXXX} if(t>1): \tabnewline
     \phantom{XXXX}  \phantom{XXXX} return [0, 1e-6] \tabnewline
     \phantom{XXXX} return [0,0] \tabnewline
     mbs.SetPostNewtonUserFunction(PostNewtonUserFunction)}}\\ \hline 
  AddGeneralContact() & add a new general contact, used to enable efficient contact computation between objects (nodes or markers)\\ \hline 
  GetGeneralContact(generalContactNumber) & get read/write access to GeneralContact with index generalContactNumber stored in mbs; Examples shows how to access the GeneralContact object added with last AddGeneralContact() command:\tabnewline 
    \textcolor{steelblue}{{\bf EXAMPLE}: \tabnewline 
    \texttt{gc=mbs.GetGeneralContact(mbs.NumberOfGeneralContacts()-1)}}\\ \hline 
  DeleteGeneralContact(generalContactNumber) & delete GeneralContact with index generalContactNumber in mbs; other general contacts are resorted (index changes!)\\ \hline 
  NumberOfGeneralContacts() & Return number of GeneralContact objects in mbs\\ \hline 
  \_\_repr\_\_() & return the representation of the system, which can be, e.g., printed\tabnewline 
    \textcolor{steelblue}{{\bf EXAMPLE}: \tabnewline 
    \texttt{print(mbs)}}\\ \hline 
  systemIsConsistent & this flag is used by solvers to decide, whether the system is in a solvable state; this flag is set to False as long as Assemble() has not been called; any modification to the system, such as Add...(), Modify...(), etc. will set the flag to False again; this flag can be modified (set to True), if a change of e.g.~an object (change of stiffness) or load (change of force) keeps the system consistent, but would normally lead to systemIsConsistent=False\\ \hline  
  interactiveMode & set this flag to True in order to invoke a Assemble() command in every system modification, e.g. AddNode, AddObject, ModifyNode, ...; this helps that the system can be visualized in interactive mode.\\ \hline  
  variables & this dictionary may be used by the user to store model-specific data, in order to avoid global Python variables in complex models; mbs.variables["myvar"] = 42 \\ \hline  
  sys & this dictionary is used by exudyn Python libraries, e.g., solvers, to avoid global Python variables \\ \hline  
  solverSignalJacobianUpdate & this flag is used by solvers to decide, whether the jacobian should be updated; at beginning of simulation and after jacobian computation, this flag is set automatically to False; use this flag to indicate system changes, e.g. during time integration  \\ \hline  
  systemData & Access to SystemData structure; enables access to number of nodes, objects, ... and to (current, initial, reference, ...) state variables (ODE2, AE, Data,...)\\ \hline  
\end{longtable}
\end{center}

%++++++++++++++++++++
\mysubsubsection{MainSystem: Node}
\label{sec:mainsystem:node}



This section provides functions for adding, reading and modifying nodes. Nodes are used to define coordinates (unknowns to the static system and degrees of freedom if constraints are not present). Nodes can provide various types of coordinates for second/first order differential equations (ODE2/ODE1), algebraic equations (AE) and for data (history) variables -- which are not providing unknowns in the nonlinear solver but will be solved in an additional nonlinear iteration for e.g. contact, friction or plasticity.
\pythonstyle
\begin{lstlisting}[language=Python, firstnumber=1]

import exudyn as exu               #EXUDYN package including C++ core part
from exudyn.itemInterface import * #conversion of data to exudyn dictionaries
SC = exu.SystemContainer()         #container of systems
mbs = SC.AddSystem()               #add a new system to work with
nMP = mbs.AddNode(NodePoint2D(referenceCoordinates=[0,0]))
\end{lstlisting}

\begin{center}
\footnotesize
\begin{longtable}{| p{8cm} | p{8cm} |} 
\hline
{\bf function/structure name} & {\bf description}\\ \hline
  AddNode(pyObject) & add a node with nodeDefinition from Python node class; returns (global) node index (type NodeIndex) of newly added node; use int(nodeIndex) to convert to int, if needed (but not recommended in order not to mix up index types of nodes, objects, markers, ...)\tabnewline 
    \textcolor{steelblue}{{\bf EXAMPLE}: \tabnewline 
    \texttt{item = Rigid2D( referenceCoordinates= [1,0.5,0], initialVelocities= [10,0,0]) \tabnewline
    mbs.AddNode(item) \tabnewline
    nodeDict = \{{\textquotesingle}nodeType{\textquotesingle}: {\textquotesingle}Point{\textquotesingle}, \tabnewline
    {\textquotesingle}referenceCoordinates{\textquotesingle}: [1.0, 0.0, 0.0], \tabnewline
    {\textquotesingle}initialCoordinates{\textquotesingle}: [0.0, 2.0, 0.0], \tabnewline
    {\textquotesingle}name{\textquotesingle}: {\textquotesingle}example node{\textquotesingle}\} \tabnewline
    mbs.AddNode(nodeDict)}}\\ \hline 
  GetNodeNumber(nodeName) & get node's number by name (string)\tabnewline 
    \textcolor{steelblue}{{\bf EXAMPLE}: \tabnewline 
    \texttt{n = mbs.GetNodeNumber({\textquotesingle}example node{\textquotesingle})}}\\ \hline 
  GetNode(nodeNumber) & get node's dictionary by node number (type NodeIndex)\tabnewline 
    \textcolor{steelblue}{{\bf EXAMPLE}: \tabnewline 
    \texttt{nodeDict = mbs.GetNode(0)}}\\ \hline 
  ModifyNode(nodeNumber, nodeDict) & modify node's dictionary by node number (type NodeIndex)\tabnewline 
    \textcolor{steelblue}{{\bf EXAMPLE}: \tabnewline 
    \texttt{mbs.ModifyNode(nodeNumber, nodeDict)}}\\ \hline 
  GetNodeDefaults(typeName) & get node's default values for a certain nodeType as (dictionary)\tabnewline 
    \textcolor{steelblue}{{\bf EXAMPLE}: \tabnewline 
    \texttt{nodeType = {\textquotesingle}Point{\textquotesingle}\tabnewline
    nodeDict = mbs.GetNodeDefaults(nodeType)}}\\ \hline 
  GetNodeOutput(nodeNumber, variableType, configuration = ConfigurationType.Current) & get the ouput of the node specified with the OutputVariableType; output may be scalar or array (e.g. displacement vector)\tabnewline 
    \textcolor{steelblue}{{\bf EXAMPLE}: \tabnewline 
    \texttt{mbs.GetNodeOutput(nodeNumber=0, variableType=exu.OutputVariableType.Displacement)}}\\ \hline 
  GetNodeODE2Index(nodeNumber) & get index in the global ODE2 coordinate vector for the first node coordinate of the specified node\tabnewline 
    \textcolor{steelblue}{{\bf EXAMPLE}: \tabnewline 
    \texttt{mbs.GetNodeODE2Index(nodeNumber=0)}}\\ \hline 
  GetNodeODE1Index(nodeNumber) & get index in the global ODE1 coordinate vector for the first node coordinate of the specified node\tabnewline 
    \textcolor{steelblue}{{\bf EXAMPLE}: \tabnewline 
    \texttt{mbs.GetNodeODE1Index(nodeNumber=0)}}\\ \hline 
  GetNodeAEIndex(nodeNumber) & get index in the global AE coordinate vector for the first node coordinate of the specified node\tabnewline 
    \textcolor{steelblue}{{\bf EXAMPLE}: \tabnewline 
    \texttt{mbs.GetNodeAEIndex(nodeNumber=0)}}\\ \hline 
  GetNodeParameter(nodeNumber, parameterName) & get nodes's parameter from node number (type NodeIndex) and parameterName; parameter names can be found for the specific items in the reference manual; for visualization parameters, use a 'V' as a prefix\tabnewline 
    \textcolor{steelblue}{{\bf EXAMPLE}: \tabnewline 
    \texttt{mbs.GetNodeParameter(0, {\textquotesingle}referenceCoordinates{\textquotesingle})}}\\ \hline 
  SetNodeParameter(nodeNumber, parameterName, value) & set parameter 'parameterName' of node with node number (type NodeIndex) to value; parameter names can be found for the specific items in the reference manual; for visualization parameters, use a 'V' as a prefix\tabnewline 
    \textcolor{steelblue}{{\bf EXAMPLE}: \tabnewline 
    \texttt{mbs.SetNodeParameter(0, {\textquotesingle}Vshow{\textquotesingle}, True)}}\\ \hline 
\end{longtable}
\end{center}

%++++++++++++++++++++
\mysubsubsection{MainSystem: Object}
\label{sec:mainsystem:object}



This section provides functions for adding, reading and modifying objects, which can be bodies (mass point, rigid body, finite element, ...), connectors (spring-damper or joint) or general objects. Objects provided terms to the residual of equations resulting from every coordinate given by the nodes. Single-noded objects (e.g.~mass point) provides exactly residual terms for its nodal coordinates. Connectors constrain or penalize two markers, which can be, e.g., position, rigid or coordinate markers. Thus, the dependence of objects is either on the coordinates of the marker-objects/nodes or on nodes which the objects possess themselves.
\pythonstyle
\begin{lstlisting}[language=Python, firstnumber=1]

import exudyn as exu               #EXUDYN package including C++ core part
from exudyn.itemInterface import * #conversion of data to exudyn dictionaries
SC = exu.SystemContainer()         #container of systems
mbs = SC.AddSystem()               #add a new system to work with
nMP = mbs.AddNode(NodePoint2D(referenceCoordinates=[0,0]))
mbs.AddObject(ObjectMassPoint2D(physicsMass=10, nodeNumber=nMP ))
\end{lstlisting}

\begin{center}
\footnotesize
\begin{longtable}{| p{8cm} | p{8cm} |} 
\hline
{\bf function/structure name} & {\bf description}\\ \hline
  AddObject(pyObject) & add an object with objectDefinition from Python object class; returns (global) object number (type ObjectIndex) of newly added object\tabnewline 
    \textcolor{steelblue}{{\bf EXAMPLE}: \tabnewline 
    \texttt{item = MassPoint(name={\textquotesingle}heavy object{\textquotesingle}, nodeNumber=0, physicsMass=100) \tabnewline
    mbs.AddObject(item) \tabnewline
    objectDict = \{{\textquotesingle}objectType{\textquotesingle}: {\textquotesingle}MassPoint{\textquotesingle}, \tabnewline
    {\textquotesingle}physicsMass{\textquotesingle}: 10, \tabnewline
    {\textquotesingle}nodeNumber{\textquotesingle}: 0, \tabnewline
    {\textquotesingle}name{\textquotesingle}: {\textquotesingle}example object{\textquotesingle}\} \tabnewline
    mbs.AddObject(objectDict)}}\\ \hline 
  GetObjectNumber(objectName) & get object's number by name (string)\tabnewline 
    \textcolor{steelblue}{{\bf EXAMPLE}: \tabnewline 
    \texttt{n = mbs.GetObjectNumber({\textquotesingle}heavy object{\textquotesingle})}}\\ \hline 
  GetObject(objectNumber, addGraphicsData = False) & get object's dictionary by object number (type ObjectIndex); NOTE: visualization parameters have a prefix 'V'; in order to also get graphicsData written, use addGraphicsData=True (which is by default False, as it would spoil the information)\tabnewline 
    \textcolor{steelblue}{{\bf EXAMPLE}: \tabnewline 
    \texttt{objectDict = mbs.GetObject(0)}}\\ \hline 
  ModifyObject(objectNumber, objectDict) & modify object's dictionary by object number (type ObjectIndex); NOTE: visualization parameters have a prefix 'V'\tabnewline 
    \textcolor{steelblue}{{\bf EXAMPLE}: \tabnewline 
    \texttt{mbs.ModifyObject(objectNumber, objectDict)}}\\ \hline 
  GetObjectDefaults(typeName) & get object's default values for a certain objectType as (dictionary)\tabnewline 
    \textcolor{steelblue}{{\bf EXAMPLE}: \tabnewline 
    \texttt{objectType = {\textquotesingle}MassPoint{\textquotesingle}\tabnewline
    objectDict = mbs.GetObjectDefaults(objectType)}}\\ \hline 
  GetObjectOutput(objectNumber, variableType, configuration = ConfigurationType.Current) & get object's current output variable from object number (type ObjectIndex) and OutputVariableType; for connectors, it can only be computed for exu.ConfigurationType.Current configuration!\\ \hline 
  GetObjectOutputBody(objectNumber, variableType, localPosition = [0,0,0], configuration = ConfigurationType.Current) & get body's output variable from object number (type ObjectIndex) and OutputVariableType, using the localPosition as defined in the body, and as used in MarkerBody and SensorBody\tabnewline 
    \textcolor{steelblue}{{\bf EXAMPLE}: \tabnewline 
    \texttt{u = mbs.GetObjectOutputBody(objectNumber = 1, variableType = exu.OutputVariableType.Position, localPosition=[1,0,0], configuration = exu.ConfigurationType.Initial)}}\\ \hline 
  GetObjectOutputSuperElement(objectNumber, variableType, meshNodeNumber, configuration = ConfigurationType.Current) & get output variable from mesh node number of object with type SuperElement (GenericODE2, FFRF, FFRFreduced - CMS) with specific OutputVariableType; the meshNodeNumber is the object's local node number, not the global node number!\tabnewline 
    \textcolor{steelblue}{{\bf EXAMPLE}: \tabnewline 
    \texttt{u = mbs.GetObjectOutputSuperElement(objectNumber = 1, variableType = exu.OutputVariableType.Position, meshNodeNumber = 12, configuration = exu.ConfigurationType.Initial)}}\\ \hline 
  GetObjectParameter(objectNumber, parameterName) & get objects's parameter from object number (type ObjectIndex) and parameterName; parameter names can be found for the specific items in the reference manual; for visualization parameters, use a 'V' as a prefix; NOTE that BodyGraphicsData cannot be get or set, use dictionary access instead\tabnewline 
    \textcolor{steelblue}{{\bf EXAMPLE}: \tabnewline 
    \texttt{mbs.GetObjectParameter(objectNumber = 0, parameterName = {\textquotesingle}nodeNumber{\textquotesingle})}}\\ \hline 
  SetObjectParameter(objectNumber, parameterName, value) & set parameter 'parameterName' of object with object number (type ObjectIndex) to value;; parameter names can be found for the specific items in the reference manual; for visualization parameters, use a 'V' as a prefix; NOTE that BodyGraphicsData cannot be get or set, use dictionary access instead\tabnewline 
    \textcolor{steelblue}{{\bf EXAMPLE}: \tabnewline 
    \texttt{mbs.SetObjectParameter(objectNumber = 0, parameterName = {\textquotesingle}Vshow{\textquotesingle}, value=True)}}\\ \hline 
\end{longtable}
\end{center}

%++++++++++++++++++++
\mysubsubsection{MainSystem: Marker}
\label{sec:mainsystem:marker}



This section provides functions for adding, reading and modifying markers. Markers define how to measure primal kinematical quantities on objects or nodes (e.g., position, orientation or coordinates themselves), and how to act on the quantities which are dual to the kinematical quantities (e.g., force, torque and generalized forces). Markers provide unique interfaces for loads, sensors and constraints in order to address these quantities independently of the structure of the object or node (e.g., rigid or flexible body).
\pythonstyle
\begin{lstlisting}[language=Python, firstnumber=1]

import exudyn as exu               #EXUDYN package including C++ core part
from exudyn.itemInterface import * #conversion of data to exudyn dictionaries
SC = exu.SystemContainer()         #container of systems
mbs = SC.AddSystem()               #add a new system to work with
nMP = mbs.AddNode(NodePoint2D(referenceCoordinates=[0,0]))
mbs.AddObject(ObjectMassPoint2D(physicsMass=10, nodeNumber=nMP ))
mMP = mbs.AddMarker(MarkerNodePosition(nodeNumber = nMP))
\end{lstlisting}

\begin{center}
\footnotesize
\begin{longtable}{| p{8cm} | p{8cm} |} 
\hline
{\bf function/structure name} & {\bf description}\\ \hline
  AddMarker(pyObject) & add a marker with markerDefinition from Python marker class; returns (global) marker number (type MarkerIndex) of newly added marker\tabnewline 
    \textcolor{steelblue}{{\bf EXAMPLE}: \tabnewline 
    \texttt{item = MarkerNodePosition(name={\textquotesingle}my marker{\textquotesingle},nodeNumber=1) \tabnewline
    mbs.AddMarker(item)\tabnewline
    markerDict = \{{\textquotesingle}markerType{\textquotesingle}: {\textquotesingle}NodePosition{\textquotesingle}, \tabnewline
      {\textquotesingle}nodeNumber{\textquotesingle}: 0, \tabnewline
      {\textquotesingle}name{\textquotesingle}: {\textquotesingle}position0{\textquotesingle}\}\tabnewline
    mbs.AddMarker(markerDict)}}\\ \hline 
  GetMarkerNumber(markerName) & get marker's number by name (string)\tabnewline 
    \textcolor{steelblue}{{\bf EXAMPLE}: \tabnewline 
    \texttt{n = mbs.GetMarkerNumber({\textquotesingle}my marker{\textquotesingle})}}\\ \hline 
  GetMarker(markerNumber) & get marker's dictionary by index\tabnewline 
    \textcolor{steelblue}{{\bf EXAMPLE}: \tabnewline 
    \texttt{markerDict = mbs.GetMarker(0)}}\\ \hline 
  ModifyMarker(markerNumber, markerDict) & modify marker's dictionary by index\tabnewline 
    \textcolor{steelblue}{{\bf EXAMPLE}: \tabnewline 
    \texttt{mbs.ModifyMarker(markerNumber, markerDict)}}\\ \hline 
  GetMarkerDefaults(typeName) & get marker's default values for a certain markerType as (dictionary)\tabnewline 
    \textcolor{steelblue}{{\bf EXAMPLE}: \tabnewline 
    \texttt{markerType = {\textquotesingle}NodePosition{\textquotesingle}\tabnewline
    markerDict = mbs.GetMarkerDefaults(markerType)}}\\ \hline 
  GetMarkerParameter(markerNumber, parameterName) & get markers's parameter from markerNumber and parameterName; parameter names can be found for the specific items in the reference manual\\ \hline 
  SetMarkerParameter(markerNumber, parameterName, value) & set parameter 'parameterName' of marker with markerNumber to value; parameter names can be found for the specific items in the reference manual\\ \hline 
  GetMarkerOutput(markerNumber, variableType, configuration = ConfigurationType.Current) & get the ouput of the marker specified with the OutputVariableType; currently only provides Displacement, Position and Velocity for position based markers, and RotationMatrix, Rotation and AngularVelocity(Local) for markers providing orientation; Coordinates and Coordinates\\_t available for coordinate markers\tabnewline 
    \textcolor{steelblue}{{\bf EXAMPLE}: \tabnewline 
    \texttt{mbs.GetMarkerOutput(markerNumber=0, variableType=exu.OutputVariableType.Position)}}\\ \hline 
\end{longtable}
\end{center}

%++++++++++++++++++++
\mysubsubsection{MainSystem: Load}
\label{sec:mainsystem:load}



This section provides functions for adding, reading and modifying operating loads. Loads are used to act on the quantities which are dual to the primal kinematic quantities, such as displacement and rotation. Loads represent, e.g., forces, torques or generalized forces.
\pythonstyle
\begin{lstlisting}[language=Python, firstnumber=1]

import exudyn as exu               #EXUDYN package including C++ core part
from exudyn.itemInterface import * #conversion of data to exudyn dictionaries
SC = exu.SystemContainer()         #container of systems
mbs = SC.AddSystem()               #add a new system to work with
nMP = mbs.AddNode(NodePoint2D(referenceCoordinates=[0,0]))
mbs.AddObject(ObjectMassPoint2D(physicsMass=10, nodeNumber=nMP ))
mMP = mbs.AddMarker(MarkerNodePosition(nodeNumber = nMP))
mbs.AddLoad(Force(markerNumber = mMP, loadVector=[0.001,0,0]))
\end{lstlisting}

\begin{center}
\footnotesize
\begin{longtable}{| p{8cm} | p{8cm} |} 
\hline
{\bf function/structure name} & {\bf description}\\ \hline
  AddLoad(pyObject) & add a load with loadDefinition from Python load class; returns (global) load number (type LoadIndex) of newly added load\tabnewline 
    \textcolor{steelblue}{{\bf EXAMPLE}: \tabnewline 
    \texttt{item = mbs.AddLoad(LoadForceVector(loadVector=[1,0,0], markerNumber=0, name={\textquotesingle}heavy load{\textquotesingle})) \tabnewline
    mbs.AddLoad(item)\tabnewline
    loadDict = \{{\textquotesingle}loadType{\textquotesingle}: {\textquotesingle}ForceVector{\textquotesingle},\tabnewline
      {\textquotesingle}markerNumber{\textquotesingle}: 0,\tabnewline
      {\textquotesingle}loadVector{\textquotesingle}: [1.0, 0.0, 0.0],\tabnewline
      {\textquotesingle}name{\textquotesingle}: {\textquotesingle}heavy load{\textquotesingle}\} \tabnewline
    mbs.AddLoad(loadDict)}}\\ \hline 
  GetLoadNumber(loadName) & get load's number by name (string)\tabnewline 
    \textcolor{steelblue}{{\bf EXAMPLE}: \tabnewline 
    \texttt{n = mbs.GetLoadNumber({\textquotesingle}heavy load{\textquotesingle})}}\\ \hline 
  GetLoad(loadNumber) & get load's dictionary by index\tabnewline 
    \textcolor{steelblue}{{\bf EXAMPLE}: \tabnewline 
    \texttt{loadDict = mbs.GetLoad(0)}}\\ \hline 
  ModifyLoad(loadNumber, loadDict) & modify load's dictionary by index\tabnewline 
    \textcolor{steelblue}{{\bf EXAMPLE}: \tabnewline 
    \texttt{mbs.ModifyLoad(loadNumber, loadDict)}}\\ \hline 
  GetLoadDefaults(typeName) & get load's default values for a certain loadType as (dictionary)\tabnewline 
    \textcolor{steelblue}{{\bf EXAMPLE}: \tabnewline 
    \texttt{loadType = {\textquotesingle}ForceVector{\textquotesingle}\tabnewline
    loadDict = mbs.GetLoadDefaults(loadType)}}\\ \hline 
  GetLoadValues(loadNumber) & Get current load values, specifically if user-defined loads are used; can be scalar or vector-valued return value\\ \hline 
  GetLoadParameter(loadNumber, parameterName) & get loads's parameter from loadNumber and parameterName; parameter names can be found for the specific items in the reference manual\\ \hline 
  SetLoadParameter(loadNumber, parameterName, value) & set parameter 'parameterName' of load with loadNumber to value; parameter names can be found for the specific items in the reference manual\\ \hline 
\end{longtable}
\end{center}

%++++++++++++++++++++
\mysubsubsection{MainSystem: Sensor}
\label{sec:mainsystem:sensor}



This section provides functions for adding, reading and modifying operating sensors. Sensors are used to measure information in nodes, objects, markers, and loads for output in a file.
\pythonstyle
\begin{lstlisting}[language=Python, firstnumber=1]

import exudyn as exu               #EXUDYN package including C++ core part
from exudyn.itemInterface import * #conversion of data to exudyn dictionaries
SC = exu.SystemContainer()         #container of systems
mbs = SC.AddSystem()               #add a new system to work with
nMP = mbs.AddNode(NodePoint(referenceCoordinates=[0,0,0]))
mbs.AddObject(ObjectMassPoint(physicsMass=10, nodeNumber=nMP ))
mMP = mbs.AddMarker(MarkerNodePosition(nodeNumber = nMP))
mbs.AddLoad(Force(markerNumber = mMP, loadVector=[2,0,5]))
sMP = mbs.AddSensor(SensorNode(nodeNumber=nMP, storeInternal=True,
                               outputVariableType=exu.OutputVariableType.Position))
mbs.Assemble()
exu.SolveDynamic(mbs, exu.SimulationSettings())
from exudyn.plot import PlotSensor
PlotSensor(mbs, sMP, components=[0,1,2])
\end{lstlisting}

\begin{center}
\footnotesize
\begin{longtable}{| p{8cm} | p{8cm} |} 
\hline
{\bf function/structure name} & {\bf description}\\ \hline
  AddSensor(pyObject) & add a sensor with sensor definition from Python sensor class; returns (global) sensor number (type SensorIndex) of newly added sensor\tabnewline 
    \textcolor{steelblue}{{\bf EXAMPLE}: \tabnewline 
    \texttt{item = mbs.AddSensor(SensorNode(sensorType= exu.SensorType.Node, nodeNumber=0, name={\textquotesingle}test sensor{\textquotesingle})) \tabnewline
    mbs.AddSensor(item)\tabnewline
    sensorDict = \{{\textquotesingle}sensorType{\textquotesingle}: {\textquotesingle}Node{\textquotesingle},\tabnewline
      {\textquotesingle}nodeNumber{\textquotesingle}: 0,\tabnewline
      {\textquotesingle}fileName{\textquotesingle}: {\textquotesingle}sensor.txt{\textquotesingle},\tabnewline
      {\textquotesingle}name{\textquotesingle}: {\textquotesingle}test sensor{\textquotesingle}\} \tabnewline
    mbs.AddSensor(sensorDict)}}\\ \hline 
  GetSensorNumber(sensorName) & get sensor's number by name (string)\tabnewline 
    \textcolor{steelblue}{{\bf EXAMPLE}: \tabnewline 
    \texttt{n = mbs.GetSensorNumber({\textquotesingle}test sensor{\textquotesingle})}}\\ \hline 
  GetSensor(sensorNumber) & get sensor's dictionary by index\tabnewline 
    \textcolor{steelblue}{{\bf EXAMPLE}: \tabnewline 
    \texttt{sensorDict = mbs.GetSensor(0)}}\\ \hline 
  ModifySensor(sensorNumber, sensorDict) & modify sensor's dictionary by index\tabnewline 
    \textcolor{steelblue}{{\bf EXAMPLE}: \tabnewline 
    \texttt{mbs.ModifySensor(sensorNumber, sensorDict)}}\\ \hline 
  GetSensorDefaults(typeName) & get sensor's default values for a certain sensorType as (dictionary)\tabnewline 
    \textcolor{steelblue}{{\bf EXAMPLE}: \tabnewline 
    \texttt{sensorType = {\textquotesingle}Node{\textquotesingle}\tabnewline
    sensorDict = mbs.GetSensorDefaults(sensorType)}}\\ \hline 
  GetSensorValues(sensorNumber, configuration = ConfigurationType.Current) & get sensors's values for configuration; can be a scalar or vector-valued return value!\\ \hline 
  GetSensorStoredData(sensorNumber) & get sensors's internally stored data as matrix (all time points stored); rows are containing time and sensor values as obtained by sensor (e.g., time, and x, y, and z value of position)\\ \hline 
  GetSensorParameter(sensorNumber, parameterName) & get sensors's parameter from sensorNumber and parameterName; parameter names can be found for the specific items in the reference manual\\ \hline 
  SetSensorParameter(sensorNumber, parameterName, value) & set parameter 'parameterName' of sensor with sensorNumber to value; parameter names can be found for the specific items in the reference manual\\ \hline 
\end{longtable}
\end{center}

%++++++++++++++++++++
\mysubsection{SystemData}
\label{sec:mbs:systemData}



This is the data structure of a system which contains Objects (bodies/constraints/...), Nodes, Markers and Loads. The SystemData structure allows advanced access to this data, which HAS TO BE USED WITH CARE, as unexpected results and system crash might happen.
\pythonstyle
\begin{lstlisting}[language=Python, firstnumber=1]

import exudyn as exu               #EXUDYN package including C++ core part
from exudyn.itemInterface import * #conversion of data to exudyn dictionaries
SC = exu.SystemContainer()         #container of systems
mbs = SC.AddSystem()               #add a new system to work with
nMP = mbs.AddNode(NodePoint(referenceCoordinates=[0,0,0]))
mbs.AddObject(ObjectMassPoint(physicsMass=10, nodeNumber=nMP ))
mMP = mbs.AddMarker(MarkerNodePosition(nodeNumber = nMP))
mbs.AddLoad(Force(markerNumber = mMP, loadVector=[2,0,5]))
mbs.Assemble()
exu.SolveDynamic(mbs, exu.SimulationSettings())

#obtain current ODE2 system vector (e.g. after static simulation finished):
u = mbs.systemData.GetODE2Coordinates()
#set initial ODE2 vector for next simulation:
mbs.systemData.SetODE2Coordinates(coordinates=u,
               configuration=exu.ConfigurationType.Initial)
#get detailed information as dictionary:
mbs.systemData.Info()
\end{lstlisting}

\begin{center}
\footnotesize
\begin{longtable}{| p{8cm} | p{8cm} |} 
\hline
{\bf function/structure name} & {\bf description}\\ \hline
  NumberOfLoads() & return number of loads in system\tabnewline 
    \textcolor{steelblue}{{\bf EXAMPLE}: \tabnewline 
    \texttt{print(mbs.systemData.NumberOfLoads())}}\\ \hline 
  NumberOfMarkers() & return number of markers in system\tabnewline 
    \textcolor{steelblue}{{\bf EXAMPLE}: \tabnewline 
    \texttt{print(mbs.systemData.NumberOfMarkers())}}\\ \hline 
  NumberOfNodes() & return number of nodes in system\tabnewline 
    \textcolor{steelblue}{{\bf EXAMPLE}: \tabnewline 
    \texttt{print(mbs.systemData.NumberOfNodes())}}\\ \hline 
  NumberOfObjects() & return number of objects in system\tabnewline 
    \textcolor{steelblue}{{\bf EXAMPLE}: \tabnewline 
    \texttt{print(mbs.systemData.NumberOfObjects())}}\\ \hline 
  NumberOfSensors() & return number of sensors in system\tabnewline 
    \textcolor{steelblue}{{\bf EXAMPLE}: \tabnewline 
    \texttt{print(mbs.systemData.NumberOfSensors())}}\\ \hline 
  ODE2Size(configurationType = exu.ConfigurationType.Current) & get size of ODE2 coordinate vector for given configuration (only works correctly after mbs.Assemble() )\tabnewline 
    \textcolor{steelblue}{{\bf EXAMPLE}: \tabnewline 
    \texttt{print({\textquotesingle}ODE2 size={\textquotesingle},mbs.systemData.ODE2Size())}}\\ \hline 
  ODE1Size(configurationType = exu.ConfigurationType.Current) & get size of ODE1 coordinate vector for given configuration (only works correctly after mbs.Assemble() )\tabnewline 
    \textcolor{steelblue}{{\bf EXAMPLE}: \tabnewline 
    \texttt{print({\textquotesingle}ODE1 size={\textquotesingle},mbs.systemData.ODE1Size())}}\\ \hline 
  AEsize(configurationType = exu.ConfigurationType.Current) & get size of AE coordinate vector for given configuration (only works correctly after mbs.Assemble() )\tabnewline 
    \textcolor{steelblue}{{\bf EXAMPLE}: \tabnewline 
    \texttt{print({\textquotesingle}AE size={\textquotesingle},mbs.systemData.AEsize())}}\\ \hline 
  DataSize(configurationType = exu.ConfigurationType.Current) & get size of Data coordinate vector for given configuration (only works correctly after mbs.Assemble() )\tabnewline 
    \textcolor{steelblue}{{\bf EXAMPLE}: \tabnewline 
    \texttt{print({\textquotesingle}Data size={\textquotesingle},mbs.systemData.DataSize())}}\\ \hline 
  SystemSize(configurationType = exu.ConfigurationType.Current) & get size of System coordinate vector for given configuration (only works correctly after mbs.Assemble() )\tabnewline 
    \textcolor{steelblue}{{\bf EXAMPLE}: \tabnewline 
    \texttt{print({\textquotesingle}System size={\textquotesingle},mbs.systemData.SystemSize())}}\\ \hline 
  GetTime(configurationType = exu.ConfigurationType.Current) & get configuration dependent time.\tabnewline 
    \textcolor{steelblue}{{\bf EXAMPLE}: \tabnewline 
    \texttt{mbs.systemData.GetTime(exu.ConfigurationType.Initial)}}\\ \hline 
  SetTime(newTime, configurationType = exu.ConfigurationType.Current) & set configuration dependent time; use this access with care, e.g. in user-defined solvers.\tabnewline 
    \textcolor{steelblue}{{\bf EXAMPLE}: \tabnewline 
    \texttt{mbs.systemData.SetTime(10., exu.ConfigurationType.Initial)}}\\ \hline 
  Info() & print detailed system information for every item; for short information use print(mbs)\tabnewline 
    \textcolor{steelblue}{{\bf EXAMPLE}: \tabnewline 
    \texttt{mbs.systemData.Info()}}\\ \hline 
\end{longtable}
\end{center}

%++++++++++++++++++++
\mysubsubsection{SystemData: Access coordinates}
\label{sec:mbs:systemData:coordinates}



This section provides access functions to global coordinate vectors. Assigning invalid values or using wrong vector size might lead to system crash and unexpected results.
\begin{center}
\footnotesize
\begin{longtable}{| p{8cm} | p{8cm} |} 
\hline
{\bf function/structure name} & {\bf description}\\ \hline
  GetODE2Coordinates(configuration = exu.ConfigurationType.Current) & get ODE2 system coordinates (displacements) for given configuration (default: exu.Configuration.Current)\tabnewline 
    \textcolor{steelblue}{{\bf EXAMPLE}: \tabnewline 
    \texttt{uCurrent = mbs.systemData.GetODE2Coordinates()}}\\ \hline 
  SetODE2Coordinates(coordinates, configuration = exu.ConfigurationType.Current) & set ODE2 system coordinates (displacements) for given configuration (default: exu.Configuration.Current); invalid vector size may lead to system crash!\tabnewline 
    \textcolor{steelblue}{{\bf EXAMPLE}: \tabnewline 
    \texttt{mbs.systemData.SetODE2Coordinates(uCurrent)}}\\ \hline 
  GetODE2Coordinates\_t(configuration = exu.ConfigurationType.Current) & get ODE2 system coordinates (velocities) for given configuration (default: exu.Configuration.Current)\tabnewline 
    \textcolor{steelblue}{{\bf EXAMPLE}: \tabnewline 
    \texttt{vCurrent = mbs.systemData.GetODE2Coordinates\_t()}}\\ \hline 
  SetODE2Coordinates\_t(coordinates, configuration = exu.ConfigurationType.Current) & set ODE2 system coordinates (velocities) for given configuration (default: exu.Configuration.Current); invalid vector size may lead to system crash!\tabnewline 
    \textcolor{steelblue}{{\bf EXAMPLE}: \tabnewline 
    \texttt{mbs.systemData.SetODE2Coordinates\_t(vCurrent)}}\\ \hline 
  GetODE2Coordinates\_tt(configuration = exu.ConfigurationType.Current) & get ODE2 system coordinates (accelerations) for given configuration (default: exu.Configuration.Current)\tabnewline 
    \textcolor{steelblue}{{\bf EXAMPLE}: \tabnewline 
    \texttt{vCurrent = mbs.systemData.GetODE2Coordinates\_tt()}}\\ \hline 
  SetODE2Coordinates\_tt(coordinates, configuration = exu.ConfigurationType.Current) & set ODE2 system coordinates (accelerations) for given configuration (default: exu.Configuration.Current); invalid vector size may lead to system crash!\tabnewline 
    \textcolor{steelblue}{{\bf EXAMPLE}: \tabnewline 
    \texttt{mbs.systemData.SetODE2Coordinates\_tt(aCurrent)}}\\ \hline 
  GetODE1Coordinates(configuration = exu.ConfigurationType.Current) & get ODE1 system coordinates (displacements) for given configuration (default: exu.Configuration.Current)\tabnewline 
    \textcolor{steelblue}{{\bf EXAMPLE}: \tabnewline 
    \texttt{qCurrent = mbs.systemData.GetODE1Coordinates()}}\\ \hline 
  SetODE1Coordinates(coordinates, configuration = exu.ConfigurationType.Current) & set ODE1 system coordinates (velocities) for given configuration (default: exu.Configuration.Current); invalid vector size may lead to system crash!\tabnewline 
    \textcolor{steelblue}{{\bf EXAMPLE}: \tabnewline 
    \texttt{mbs.systemData.SetODE1Coordinates\_t(qCurrent)}}\\ \hline 
  GetODE1Coordinates\_t(configuration = exu.ConfigurationType.Current) & get ODE1 system coordinates (velocities) for given configuration (default: exu.Configuration.Current)\tabnewline 
    \textcolor{steelblue}{{\bf EXAMPLE}: \tabnewline 
    \texttt{qCurrent = mbs.systemData.GetODE1Coordinates\_t()}}\\ \hline 
  SetODE1Coordinates\_t(coordinates, configuration = exu.ConfigurationType.Current) & set ODE1 system coordinates (displacements) for given configuration (default: exu.Configuration.Current); invalid vector size may lead to system crash!\tabnewline 
    \textcolor{steelblue}{{\bf EXAMPLE}: \tabnewline 
    \texttt{mbs.systemData.SetODE1Coordinates(qCurrent)}}\\ \hline 
  GetAECoordinates(configuration = exu.ConfigurationType.Current) & get algebraic equations (AE) system coordinates for given configuration (default: exu.Configuration.Current)\tabnewline 
    \textcolor{steelblue}{{\bf EXAMPLE}: \tabnewline 
    \texttt{lambdaCurrent = mbs.systemData.GetAECoordinates()}}\\ \hline 
  SetAECoordinates(coordinates, configuration = exu.ConfigurationType.Current) & set algebraic equations (AE) system coordinates for given configuration (default: exu.Configuration.Current); invalid vector size may lead to system crash!\tabnewline 
    \textcolor{steelblue}{{\bf EXAMPLE}: \tabnewline 
    \texttt{mbs.systemData.SetAECoordinates(lambdaCurrent)}}\\ \hline 
  GetDataCoordinates(configuration = exu.ConfigurationType.Current) & get system data coordinates for given configuration (default: exu.Configuration.Current)\tabnewline 
    \textcolor{steelblue}{{\bf EXAMPLE}: \tabnewline 
    \texttt{dataCurrent = mbs.systemData.GetDataCoordinates()}}\\ \hline 
  SetDataCoordinates(coordinates, configuration = exu.ConfigurationType.Current) & set system data coordinates for given configuration (default: exu.Configuration.Current); invalid vector size may lead to system crash!\tabnewline 
    \textcolor{steelblue}{{\bf EXAMPLE}: \tabnewline 
    \texttt{mbs.systemData.SetDataCoordinates(dataCurrent)}}\\ \hline 
  GetSystemState(configuration = exu.ConfigurationType.Current) & get system state for given configuration (default: exu.Configuration.Current); state vectors do not include the non-state derivatives ODE1\_t and ODE2\_tt and the time; function is copying data - not highly efficient; format of pyList: [ODE2Coords, ODE2Coords\_t, ODE1Coords, AEcoords, dataCoords]\tabnewline 
    \textcolor{steelblue}{{\bf EXAMPLE}: \tabnewline 
    \texttt{sysStateList = mbs.systemData.GetSystemState()}}\\ \hline 
  SetSystemState(systemStateList, configuration = exu.ConfigurationType.Current) & set system data coordinates for given configuration (default: exu.Configuration.Current); invalid list of vectors / vector size may lead to system crash; write access to state vectors (but not the non-state derivatives ODE1\_t and ODE2\_tt and the time); function is copying data - not highly efficient; format of pyList: [ODE2Coords, ODE2Coords\_t, ODE1Coords, AEcoords, dataCoords]\tabnewline 
    \textcolor{steelblue}{{\bf EXAMPLE}: \tabnewline 
    \texttt{mbs.systemData.SetSystemState(sysStateList, configuration = exu.ConfigurationType.Initial)}}\\ \hline 
\end{longtable}
\end{center}

%++++++++++++++++++++
\mysubsubsection{SystemData: Get object LTG coordinate mappings}
\label{sec:systemData:ObjectLTG}



This section provides access functions the \ac{LTG}-lists for every object (body, constraint, ...) in the system. For details on the \ac{LTG} mapping, see \refSection{sec:overview:ltgmapping}
\begin{center}
\footnotesize
\begin{longtable}{| p{8cm} | p{8cm} |} 
\hline
{\bf function/structure name} & {\bf description}\\ \hline
  GetObjectLTGODE2(objectNumber) & get local-to-global coordinate mapping (list of global coordinate indices) for ODE2 coordinates; only available after Assemble()\tabnewline 
    \textcolor{steelblue}{{\bf EXAMPLE}: \tabnewline 
    \texttt{ltgObject4 = mbs.systemData.GetObjectLTGODE2(4)}}\\ \hline 
  GetObjectLTGODE1(objectNumber) & get local-to-global coordinate mapping (list of global coordinate indices) for ODE1 coordinates; only available after Assemble()\tabnewline 
    \textcolor{steelblue}{{\bf EXAMPLE}: \tabnewline 
    \texttt{ltgObject4 = mbs.systemData.GetObjectLTGODE1(4)}}\\ \hline 
  GetObjectLTGAE(objectNumber) & get local-to-global coordinate mapping (list of global coordinate indices) for algebraic equations (AE) coordinates; only available after Assemble()\tabnewline 
    \textcolor{steelblue}{{\bf EXAMPLE}: \tabnewline 
    \texttt{ltgObject4 = mbs.systemData.GetObjectLTGODE2(4)}}\\ \hline 
  GetObjectLTGData(objectNumber) & get local-to-global coordinate mapping (list of global coordinate indices) for data coordinates; only available after Assemble()\tabnewline 
    \textcolor{steelblue}{{\bf EXAMPLE}: \tabnewline 
    \texttt{ltgObject4 = mbs.systemData.GetObjectLTGData(4)}}\\ \hline 
\end{longtable}
\end{center}

%++++++++++++++++++++
\mysubsection{GeneralContact}
\label{sec:GeneralContact}



Structure to define general and highly efficient contact functionality in multibody systems\footnote{Note that GeneralContact is still developed, use with care.}. For further explanations and theoretical backgrounds, see \refSection{secContactTheory}.
\pythonstyle
\begin{lstlisting}[language=Python, firstnumber=1]

#...
#code snippet, must be placed anywhere before mbs.Assemble()
#Add GeneralContact to mbs:
gContact = mbs.AddGeneralContact()
#Add contact elements, e.g.:
gContact.AddSphereWithMarker(...) #use appropriate arguments
gContact.SetFrictionPairings(...) #set friction pairings and adjust searchTree if needed.
\end{lstlisting}

\begin{center}
\footnotesize
\begin{longtable}{| p{8cm} | p{8cm} |} 
\hline
{\bf function/structure name} & {\bf description}\\ \hline
  GetPythonObject() & convert member variables of GeneralContact into dictionary; use this for debug only!\\ \hline 
  Reset(freeMemory = True) & remove all contact objects and reset contact parameters\\ \hline 
  isActive & default = True (compute contact); if isActive=False, no contact computation is performed for this contact set \\ \hline  
  verboseMode & default = 0; verboseMode = 1 or higher outputs useful information on the contact creation and computation \\ \hline  
  visualization & access visualization data structure \\ \hline  
  resetSearchTreeInterval & (default=10000) number of search tree updates (contact computation steps) after which the search tree cells are re-created; this costs some time, will free memory in cells that are not needed any more \\ \hline  
  sphereSphereContact & activate/deactivate contact between spheres \\ \hline  
  sphereSphereFrictionRecycle & False: compute static friction force based on tangential velocity; True: recycle friction from previous PostNewton step, which greatly improves convergence, but may lead to unphysical artifacts; will be solved in future by step reduction \\ \hline  
  minRelDistanceSpheresTriangles & (default=1e-10) tolerance (relative to sphere radiues) below which the contact between triangles and spheres is ignored; used for spheres directly attached to triangles \\ \hline  
  frictionProportionalZone & (default=0.001) velocity $v\_\{\mu,reg\}$ upon which the dry friction coefficient is interpolated linearly (regularized friction model); must be greater 0; very small values cause oscillations in friction force \\ \hline  
  frictionVelocityPenalty & (default=1e3) regularization factor for friction [N/(m$^2 \cdot$m/s) ];$k\_\{\mu,reg\}$, multiplied with tangential velocity to compute friciton force as long as it is smaller than $\mu$ times contact force; large values cause oscillations in friction force \\ \hline  
  excludeOverlappingTrigSphereContacts & (default=True) for consistent, closed meshes, we can exclude overlapping contact triangles (which would cause holes if mesh is overlapping and not consistent!!!) \\ \hline  
  excludeDuplicatedTrigSphereContactPoints & (default=False) run additional checks for double contacts at edges or vertices, being more accurate but can cause additional costs if many contacts \\ \hline  
  ancfCableUseExactMethod & (default=True) if True, uses exact computation of intersection of 3rd order polynomials and contacting circles \\ \hline  
  ancfCableNumberOfContactSegments & (default=1) number of segments to be used in case that ancfCableUseExactMethod=False; maximum number of segments=3 \\ \hline  
  ancfCableMeasuringSegments & (default=20) number of segments used to approximate geometry for ANCFCable2D elements for measuring with ShortestDistanceAlongLine; with 20 segments the relative error due to approximation as compared to 10 segments usually stays below 1e-8 \\ \hline  
  SetFrictionPairings(frictionPairings) & set Coulomb friction coefficients for pairings of materials (e.g., use material 0,1, then the entries (0,1) and (1,0) define the friction coefficients for this pairing); matrix should be symmetric!\tabnewline 
    \textcolor{steelblue}{{\bf EXAMPLE}: \tabnewline 
    \texttt{\#set 3 surface friction types, all being 0.1:\tabnewline
    gContact.SetFrictionPairings(0.1*np.ones((3,3)));}}\\ \hline 
  SetFrictionProportionalZone(frictionProportionalZone) & regularization for friction (m/s); used for all contacts\\ \hline 
  SetSearchTreeCellSize(numberOfCells) & set number of cells of search tree (boxed search) in x, y and z direction\tabnewline 
    \textcolor{steelblue}{{\bf EXAMPLE}: \tabnewline 
    \texttt{gContact.SetSearchTreeInitSize([10,10,10])}}\\ \hline 
  SetSearchTreeBox(pMin, pMax) & set geometric dimensions of searchTreeBox (point with minimum coordinates and point with maximum coordinates); if this box becomes smaller than the effective contact objects, contact computations may slow down significantly\tabnewline 
    \textcolor{steelblue}{{\bf EXAMPLE}: \tabnewline 
    \texttt{gContact.SetSearchTreeBox(pMin=[-1,-1,-1],\tabnewline
     \phantom{XXXX} pMax=[1,1,1])}}\\ \hline 
  AddSphereWithMarker(markerIndex, radius, contactStiffness, contactDamping, frictionMaterialIndex) & add contact object using a marker (Position or Rigid), radius and contact/friction parameters and return localIndex of the contact item in GeneralContact; frictionMaterialIndex refers to frictionPairings in GeneralContact; contact is possible between spheres (circles in 2D) (if intraSphereContact = True), spheres and triangles and between sphere (=circle) and ANCFCable2D; contactStiffness is computed as serial spring between contacting objects, while damping is computed as a parallel damper\\ \hline 
  AddANCFCable(objectIndex, halfHeight, contactStiffness, contactDamping, frictionMaterialIndex) & add contact object for an ANCF cable element, using the objectIndex of the cable element and the cable's half height as an additional distance to contacting objects (currently not causing additional torque in case of friction), and return localIndex of the contact item in GeneralContact; currently only contact with spheres (circles in 2D) possible; contact computed using exact geometry of elements, finding max 3 intersecting contact regions\\ \hline 
  AddTrianglesRigidBodyBased(rigidBodyMarkerIndex, contactStiffness, contactDamping, frictionMaterialIndex, pointList, triangleList) & add contact object using a rigidBodyMarker (of a body), contact/friction parameters, a list of points (as 3D numpy arrays or lists; coordinates relative to rigidBodyMarker) and a list of triangles (3 indices as numpy array or list) according to a mesh attached to the rigidBodyMarker; returns starting local index of trigsRigidBodyBased at which the triangles are stored; mesh can be produced with GraphicsData2TrigsAndPoints(...); contact is possible between sphere (circle) and Triangle but yet not between triangle and triangle; frictionMaterialIndex refers to frictionPairings in GeneralContact; contactStiffness is computed as serial spring between contacting objects, while damping is computed as a parallel damper (otherwise the smaller damper would always dominate); the triangle normal must point outwards, with the normal of a triangle given with local points (p0,p1,p2) defined as n=(p1-p0) x (p2-p0), see function ComputeTriangleNormal(...)\\ \hline 
  GetItemsInBox(pMin, pMax) & Get all items in box defined by minimum coordinates given in pMin and maximum coordinates given by pMax, accepting 3D lists or numpy arrays; in case that no objects are found, False is returned; otherwise, a dictionary is returned, containing numpy arrays with indices of obtained MarkerBasedSpheres, TrigsRigidBodyBased, ANCFCable2D, ...; the indices refer to the local index in GeneralContact which can be evaluated e.g. by GetMarkerBasedSphere(localIndex)\tabnewline 
    \textcolor{steelblue}{{\bf EXAMPLE}: \tabnewline 
    \texttt{gContact.GetItemsInBox(pMin=[0,1,1],\tabnewline
     \phantom{XXXX} pMax=[2,3,2])}}\\ \hline 
  GetMarkerBasedSphere(localIndex) & Get dictionary with position, radius and markerIndex for markerBasedSphere index, as returned e.g. from GetItemsInBox\\ \hline 
  ShortestDistanceAlongLine(pStart = [0,0,0], direction = [1,0,0], minDistance = -1e-7, maxDistance = 1e7, asDictionary = False, cylinderRadius = 0, typeIndex = Contact.IndexEndOfEnumList) & Find shortest distance to contact objects in GeneralContact along line with pStart (given as 3D list or numpy array) and direction (as 3D list or numpy array with no need to be normalized); the function returns the distance which is >= minDistance and < maxDistance; in case of beam elements, it measures the distance to the beam centerline; the distance is measured from pStart along given direction and can also be negative; if no item is found along line, the maxDistance is returned; if asDictionary=False, the result is a float, while otherwise details are returned as dictionary (including distance, velocityAlongLine (which is the object velocity in given direction and may be different from the time derivative of the distance; works similar to a laser Doppler vibrometer - LDV), itemIndex and itemType in GeneralContact); the cylinderRadius, if not equal to 0, will be used for spheres to find closest sphere along cylinder with given point and direction; the typeIndex can be set to a specific contact type, e.g., which are searched for (otherwise all objects are considered)\\ \hline 
  \_\_repr\_\_() & return the string representation of the GeneralContact, containing basic information and statistics\\ \hline 
\end{longtable}
\end{center}

%++++++++++++++++++++
\mysubsubsection{VisuGeneralContact}
\label{sec:GeneralContact:visualization}
This structure may contains some visualization parameters in future. Currently, all visualization settings are controlled via SC.visualizationSettings

\begin{center}
\footnotesize
\begin{longtable}{| p{8cm} | p{8cm} |} 
\hline
{\bf function/structure name} & {\bf description}\\ \hline
  Reset() & reset visualization parameters to default values\\ \hline 
\end{longtable}
\end{center}

\mysubsection{Data structures}
\label{sec:cinterface:dataStructures}

This section describes a set of special data structures which are used in the Python-C++ interface, 
such as a MatrixContainer for dense/sparse matrices or a list of 3D vectors. 
Note that there are many native data types, such as lists, dicts and numpy arrays (e.g. 3D vectors), 
which are not described here as they are native to Pybind11, but can be passed as arguments when appropriate.

%++++++++++++++++++++
\mysubsubsection{MatrixContainer}
The MatrixContainer is a versatile representation for dense and sparse matrices.

\pythonstyle
\begin{lstlisting}[language=Python, firstnumber=1]

#Create empty MatrixContainer:
mc = MatrixContainer()

#Create MatrixContainer with dense matrix:
#matrix can be a list of lists or a numpy array, e.g.:
matrix = np.eye(6)
mc = MatrixContainer(matrix)

#Set with dense pyArray (a numpy array): 
mc.SetWithDenseMatrix(pyArray, useDenseMatrix = True)
\end{lstlisting}

\begin{center}
\footnotesize
\begin{longtable}{| p{8cm} | p{8cm} |} 
\hline
{\bf function/structure name} & {\bf description}\\ \hline
  SetWithDenseMatrix(pyArray, useDenseMatrix = False) & set MatrixContainer with dense numpy array; array (=matrix) contains values and matrix size information; if useDenseMatrix=True, matrix will be stored internally as dense matrix, otherwise it will be converted and stored as sparse matrix (which may speed up computations for larger problems)\\ \hline 
  SetWithSparseMatrixCSR(numberOfRowsInit, numberOfColumnsInit, pyArrayCSR, useDenseMatrix = True) & set with sparse CSR matrix format: numpy array 'pyArrayCSR' contains sparse triplet (row, col, value) per row; numberOfRows and numberOfColumns given extra; if useDenseMatrix=True, matrix will be converted and stored internally as dense matrix, otherwise it will be stored as sparse matrix\\ \hline 
  GetPythonObject() & convert MatrixContainer to numpy array (dense) or dictionary (sparse): containing nr. of rows, nr. of columns, numpy matrix with sparse triplets\\ \hline 
  Convert2DenseMatrix() & convert MatrixContainer to dense numpy array (SLOW and may fail for too large sparse matrices)\\ \hline 
  UseDenseMatrix() & returns True if dense matrix is used, otherwise False\\ \hline 
  \_\_repr\_\_() & return the string representation of the MatrixContainer\\ \hline 
\end{longtable}
\end{center}

%++++++++++++++++++++
\mysubsubsection{Vector3DList}
The Vector3DList is used to represent lists of 3D vectors. This is used to transfer such lists from Python to C++. \\ \\ Usage: \bi
  \item Create empty \texttt{Vector3DList} with \texttt{x = Vector3DList()} 
  \item Create \texttt{Vector3DList} with list of numpy arrays:\\\texttt{x = Vector3DList([ numpy.array([1.,2.,3.]), numpy.array([4.,5.,6.]) ])}
  \item Create \texttt{Vector3DList} with list of lists \texttt{x = Vector3DList([[1.,2.,3.], [4.,5.,6.]])}
  \item Append item: \texttt{x.Append([0.,2.,4.])}
  \item Convert into list of numpy arrays: \texttt{x.GetPythonObject()}
\ei


\begin{center}
\footnotesize
\begin{longtable}{| p{8cm} | p{8cm} |} 
\hline
{\bf function/structure name} & {\bf description}\\ \hline
  Append(pyArray) & add single array or list to Vector3DList; array or list must have appropriate dimension!\\ \hline 
  GetPythonObject() & convert Vector3DList into (copied) list of numpy arrays\\ \hline 
  len(data) & return length of the Vector3DList, using len(data) where data is the Vector3DList\\ \hline 
  data[index]= ... & set list item 'index' with data, write: data[index] = ...\\ \hline 
  ... = data[index] & get copy of list item with 'index' as vector\\ \hline 
  \_\_repr\_\_() & return the string representation of the Vector3DList data, e.g.: print(data)\\ \hline 
\end{longtable}
\end{center}

%++++++++++++++++++++
\mysubsubsection{Vector2DList}
The Vector2DList is used to represent lists of 2D vectors. This is used to transfer such lists from Python to C++. \\ \\ Usage: \bi
  \item Create empty \texttt{Vector2DList} with \texttt{x = Vector2DList()} 
  \item Create \texttt{Vector2DList} with list of numpy arrays:\\\texttt{x = Vector2DList([ numpy.array([1.,2.]), numpy.array([4.,5.]) ])}
  \item Create \texttt{Vector2DList} with list of lists \texttt{x = Vector2DList([[1.,2.], [4.,5.]])}
  \item Append item: \texttt{x.Append([0.,2.])}
  \item Convert into list of numpy arrays: \texttt{x.GetPythonObject()}
  \item similar to Vector3DList !
\ei


\begin{center}
\footnotesize
\begin{longtable}{| p{8cm} | p{8cm} |} 
\hline
{\bf function/structure name} & {\bf description}\\ \hline
  Append(pyArray) & add single array or list to Vector2DList; array or list must have appropriate dimension!\\ \hline 
  GetPythonObject() & convert Vector2DList into (copied) list of numpy arrays\\ \hline 
  len(data) & return length of the Vector2DList, using len(data) where data is the Vector2DList\\ \hline 
  data[index]= ... & set list item 'index' with data, write: data[index] = ...\\ \hline 
  ... = data[index] & get copy of list item with 'index' as vector\\ \hline 
  \_\_repr\_\_() & return the string representation of the Vector2DList data, e.g.: print(data)\\ \hline 
\end{longtable}
\end{center}

%++++++++++++++++++++
\mysubsubsection{Vector6DList}
The Vector6DList is used to represent lists of 6D vectors. This is used to transfer such lists from Python to C++. \\ \\ Usage: \bi
  \item Create empty \texttt{Vector6DList} with \texttt{x = Vector6DList()} 
  \item Convert into list of numpy arrays: \texttt{x.GetPythonObject()}
  \item similar to Vector3DList !
\ei


\begin{center}
\footnotesize
\begin{longtable}{| p{8cm} | p{8cm} |} 
\hline
{\bf function/structure name} & {\bf description}\\ \hline
  Append(pyArray) & add single array or list to Vector6DList; array or list must have appropriate dimension!\\ \hline 
  GetPythonObject() & convert Vector6DList into (copied) list of numpy arrays\\ \hline 
  len(data) & return length of the Vector6DList, using len(data) where data is the Vector6DList\\ \hline 
  data[index]= ... & set list item 'index' with data, write: data[index] = ...\\ \hline 
  ... = data[index] & get copy of list item with 'index' as vector\\ \hline 
  \_\_repr\_\_() & return the string representation of the Vector6DList data, e.g.: print(data)\\ \hline 
\end{longtable}
\end{center}

%++++++++++++++++++++
\mysubsubsection{Matrix3DList}
The Matrix3DList is used to represent lists of 3D Matrices. . This is used to transfer such lists from Python to C++. \\ \\ Usage: \bi
  \item Create empty \texttt{Matrix3DList} with \texttt{x = Matrix3DList()} 
  \item Create \texttt{Matrix3DList} with list of numpy arrays:\\\texttt{x = Matrix3DList([ numpy.eye(3), numpy.array([[1.,2.,3.],[4.,5.,6.],[7.,8.,9.]]) ])}
  \item Append item: \texttt{x.Append(numpy.eye(3))}
  \item Convert into list of numpy arrays: \texttt{x.GetPythonObject()}
  \item similar to Vector3DList !
\ei


\begin{center}
\footnotesize
\begin{longtable}{| p{8cm} | p{8cm} |} 
\hline
{\bf function/structure name} & {\bf description}\\ \hline
  Append(pyArray) & add single 3D array or list of lists to Matrix3DList; array or lists must have appropriate dimension!\\ \hline 
  GetPythonObject() & convert Matrix3DList into (copied) list of 2D numpy arrays\\ \hline 
  len(data) & return length of the Matrix3DList, using len(data) where data is the Matrix3DList\\ \hline 
  data[index]= ... & set list item 'index' with matrix, write: data[index] = ...\\ \hline 
  ... = data[index] & get copy of list item with 'index' as matrix\\ \hline 
  \_\_repr\_\_() & return the string representation of the Matrix3DList data, e.g.: print(data)\\ \hline 
\end{longtable}
\end{center}

%++++++++++++++++++++
\mysubsubsection{Matrix6DList}
The Matrix6DList is used to represent lists of 6D Matrices. . This is used to transfer such lists from Python to C++. \\ \\ Usage: \bi
  \item Create empty \texttt{Matrix6DList} with \texttt{x = Matrix6DList()} 
  \item Create \texttt{Matrix6DList} with list of numpy arrays:\\\texttt{x = Matrix6DList([ numpy.eye(6), 2*numpy.eye(6) ])}
  \item Append item: \texttt{x.Append(numpy.eye(6))}
  \item Convert into list of numpy arrays: \texttt{x.GetPythonObject()}
  \item similar to Matrix3DList !
\ei


\begin{center}
\footnotesize
\begin{longtable}{| p{8cm} | p{8cm} |} 
\hline
{\bf function/structure name} & {\bf description}\\ \hline
  Append(pyArray) & add single 6D array or list of lists to Matrix6DList; array or lists must have appropriate dimension!\\ \hline 
  GetPythonObject() & convert Matrix6DList into (copied) list of 2D numpy arrays\\ \hline 
  len(data) & return length of the Matrix6DList, using len(data) where data is the Matrix6DList\\ \hline 
  data[index]= ... & set list item 'index' with matrix, write: data[index] = ...\\ \hline 
  ... = data[index] & get copy of list item with 'index' as matrix\\ \hline 
  \_\_repr\_\_() & return the string representation of the Matrix6DList data, e.g.: print(data)\\ \hline 
\end{longtable}
\end{center}

\mysubsection{Type definitions}
\label{sec:cinterface:typedef}
This section defines a couple of structures (C++: enum aka enumeration type), which are used to select, e.g., a configuration type or a variable type. In the background, these types are integer numbers, but for safety, the types should be used as type variables. See this examples:

\pythonstyle
\begin{lstlisting}[language=Python, firstnumber=1]

#Conversion to integer is possible: 
x = int(exu.OutputVariableType.Displacement)
#also conversion from integer: 
varType = exu.OutputVariableType(8)
#use in settings:
SC.visualizationSettings.contour.outputVariable = exu.OutputVariableType.StressLocal
#use outputVariableType in sensor:
mbs.AddSensor(SensorBody(bodyNumber=rigid, storeInternal=True,
                         outputVariableType=exu.OutputVariableType.Displacement))
#
\end{lstlisting}


%++++++++++++++++++++
\mysubsubsection{OutputVariableType}
\label{sec:OutputVariableType}
This section shows the OutputVariableType structure, which is used for selecting output values, e.g. for GetObjectOutput(...) or for selecting variables for contour plot.

Available output variables and the interpreation of the output variable can be found at the object definitions. 
 The OutputVariableType does not provide information about the size of the output variable, which can be either scalar or a list (vector). For vector output quantities, the contour plot option offers an additional parameter for selection of the component of the OutputVariableType. The components are usually out of \{0,1,2\}, representing \{x,y,z\} components (e.g., of displacements, velocities, ...), or \{0,1,2,3,4,5\} representing \{xx,yy,zz,yz,xz,xy\} components (e.g., of strain or stress). In order to compute a norm, chose component=-1, which will result in the quadratic norm for other vectors and to a norm specified for stresses (if no norm is defined for an outputVariable, it does not compute anything)


\begin{center}
\footnotesize
\begin{longtable}{| p{8cm} | p{8cm} |} 
\hline
{\bf function/structure name} & {\bf description}\\ \hline
  \_None & no value; used, e.g., to select no output variable in contour plot\\ \hline  
  Distance & e.g., measure distance in spring damper connector\\ \hline  
  Position & measure 3D position, e.g., of node or body\\ \hline  
  Displacement & measure displacement; usually difference between current position and reference position\\ \hline  
  DisplacementLocal & measure local displacement, e.g. in local joint coordinates\\ \hline  
  Velocity & measure (translational) velocity of node or object\\ \hline  
  VelocityLocal & measure local (translational) velocity, e.g. in local body or joint coordinates\\ \hline  
  Acceleration & measure (translational) acceleration of node or object\\ \hline  
  RotationMatrix & measure rotation matrix of rigid body node or object\\ \hline  
  Rotation & measure, e.g., scalar rotation of 2D body, Euler angles of a 3D object or rotation within a joint\\ \hline  
  AngularVelocity & measure angular velocity of node or object\\ \hline  
  AngularVelocityLocal & measure local (body-fixed) angular velocity of node or object\\ \hline  
  AngularAcceleration & measure angular acceleration of node or object\\ \hline  
  Coordinates & measure the coordinates of a node or object; coordinates usually just contain displacements, but not the position values\\ \hline  
  Coordinates\_t & measure the time derivative of coordinates (= velocity coordinates) of a node or object\\ \hline  
  Coordinates\_tt & measure the second time derivative of coordinates (= acceleration coordinates) of a node or object\\ \hline  
  SlidingCoordinate & measure sliding coordinate in sliding joint\\ \hline  
  Director1 & measure a director (e.g. of a rigid body frame), or a slope vector in local 1 or x-direction\\ \hline  
  Director2 & measure a director (e.g. of a rigid body frame), or a slope vector in local 2 or y-direction\\ \hline  
  Director3 & measure a director (e.g. of a rigid body frame), or a slope vector in local 3 or z-direction\\ \hline  
  Force & measure global force, e.g., in joint or beam (resultant force), or generalized forces; see description of according object\\ \hline  
  ForceLocal & measure local force, e.g., in joint or beam (resultant force)\\ \hline  
  Torque & measure torque, e.g., in joint or beam (resultant couple/moment)\\ \hline  
  TorqueLocal & measure local torque, e.g., in joint or beam (resultant couple/moment)\\ \hline  
  StrainLocal & measure local strain, e.g., axial strain in cross section frame of beam or Green-Lagrange strain\\ \hline  
  StressLocal & measure local stress, e.g., axial stress in cross section frame of beam or Second Piola-Kirchoff stress; choosing component==-1 will result in the computation of the Mises stress\\ \hline  
  CurvatureLocal & measure local curvature; may be scalar or vectorial: twist and curvature of beam in cross section frame\\ \hline  
  ConstraintEquation & evaluates constraint equation (=current deviation or drift of constraint equation)\\ \hline  
\end{longtable}
\end{center}

%++++++++++++++++++++
\mysubsubsection{ConfigurationType}
\label{sec:ConfigurationType}
This section shows the ConfigurationType structure, which is used for selecting a configuration for reading or writing information to the module. Specifically, the ConfigurationType.Current configuration is usually used at the end of a solution process, to obtain result values, or the ConfigurationType.Initial is used to set initial values for a solution process.



\begin{center}
\footnotesize
\begin{longtable}{| p{8cm} | p{8cm} |} 
\hline
{\bf function/structure name} & {\bf description}\\ \hline
  \_None & no configuration; usually not valid, but may be used, e.g., if no configurationType is required\\ \hline  
  Initial & initial configuration prior to static or dynamic solver; is computed during mbs.Assemble() or AssembleInitializeSystemCoordinates()\\ \hline  
  Current & current configuration during and at the end of the computation of a step (static or dynamic)\\ \hline  
  Reference & configuration used to define deformable bodies (reference configuration for finite elements) or joints (configuration for which some joints are defined)\\ \hline  
  StartOfStep & during computation, this refers to the solution at the start of the step = end of last step, to which the solver falls back if convergence fails\\ \hline  
  Visualization & this is a state completely de-coupled from computation, used for visualization\\ \hline  
  EndOfEnumList & this marks the end of the list, usually not important to the user\\ \hline  
\end{longtable}
\end{center}

%++++++++++++++++++++
\mysubsubsection{ItemType}
\label{sec:ItemType}
This section shows the ItemType structure, which is used for defining types of indices, e.g., in render window and will be also used in item dictionaries in future.



\begin{center}
\footnotesize
\begin{longtable}{| p{8cm} | p{8cm} |} 
\hline
{\bf function/structure name} & {\bf description}\\ \hline
  \_None & item has no type\\ \hline  
  Node & item or index is of type Node\\ \hline  
  Object & item or index is of type Object\\ \hline  
  Marker & item or index is of type Marker\\ \hline  
  Load & item or index is of type Load\\ \hline  
  Sensor & item or index is of type Sensor\\ \hline  
\end{longtable}
\end{center}

%++++++++++++++++++++
\mysubsubsection{NodeType}
\label{sec:NodeType}
This section shows the NodeType structure, which is used for defining node types for 3D rigid bodies.



\begin{center}
\footnotesize
\begin{longtable}{| p{8cm} | p{8cm} |} 
\hline
{\bf function/structure name} & {\bf description}\\ \hline
  \_None & node has no type\\ \hline  
  Ground & ground node\\ \hline  
  Position2D & 2D position node \\ \hline  
  Orientation2D & node with 2D rotation\\ \hline  
  Point2DSlope1 & 2D node with 1 slope vector\\ \hline  
  Position & 3D position node\\ \hline  
  Orientation & 3D orientation node\\ \hline  
  RigidBody & node that can be used for rigid bodies\\ \hline  
  RotationEulerParameters & node with 3D orientations that are modelled with Euler parameters (unit quaternions)\\ \hline  
  RotationRxyz & node with 3D orientations that are modelled with Tait-Bryan angles\\ \hline  
  RotationRotationVector & node with 3D orientations that are modelled with the rotation vector\\ \hline  
  LieGroupWithDirectUpdate & node to be solved with Lie group methods, without data coordinates\\ \hline  
  LieGroupWithDataCoordinates & node to be solved with Lie group methods, having data coordinates\\ \hline  
  GenericODE2 & node with general ODE2 variables\\ \hline  
  GenericODE1 & node with general ODE1 variables\\ \hline  
  GenericAE & node with general algebraic variables\\ \hline  
  GenericData & node with general data variables\\ \hline  
  Point3DSlope1 & node with 1 slope vector\\ \hline  
  Point3DSlope23 & node with 2 slope vectors in y and z direction\\ \hline  
\end{longtable}
\end{center}

%++++++++++++++++++++
\mysubsubsection{JointType}
\label{sec:JointType}
This section shows the JointType structure, which is used for defining joint types, used in KinematicTree.



\begin{center}
\footnotesize
\begin{longtable}{| p{8cm} | p{8cm} |} 
\hline
{\bf function/structure name} & {\bf description}\\ \hline
  \_None & node has no type\\ \hline  
  RevoluteX & revolute joint type with rotation around local X axis\\ \hline  
  RevoluteY & revolute joint type with rotation around local Y axis\\ \hline  
  RevoluteZ & revolute joint type with rotation around local Z axis\\ \hline  
  PrismaticX & prismatic joint type with translation along local X axis\\ \hline  
  PrismaticY & prismatic joint type with translation along local Y axis\\ \hline  
  PrismaticZ & prismatic joint type with translation along local Z axis\\ \hline  
\end{longtable}
\end{center}

%++++++++++++++++++++
\mysubsubsection{DynamicSolverType}
\label{sec:DynamicSolverType}
This section shows the DynamicSolverType structure, which is used for selecting dynamic solvers for simulation.



\begin{center}
\footnotesize
\begin{longtable}{| p{8cm} | p{8cm} |} 
\hline
{\bf function/structure name} & {\bf description}\\ \hline
  GeneralizedAlpha & an implicit solver for index 3 problems; intended to be used for solving directly the index 3 constraints using the spectralRadius sufficiently small (usually 0.5 .. 1)\\ \hline  
  TrapezoidalIndex2 & an implicit solver for index 3 problems with index2 reduction; uses generalized alpha solver with settings for Newmark with index2 reduction\\ \hline  
  ExplicitEuler & an explicit 1st order solver (generally not compatible with constraints)\\ \hline  
  ExplicitMidpoint & an explicit 2nd order solver (generally not compatible with constraints)\\ \hline  
  RK33 & an explicit 3 stage 3rd order Runge-Kutta method, aka "Heun third order"; (generally not compatible with constraints)\\ \hline  
  RK44 & an explicit 4 stage 4th order Runge-Kutta method, aka "classical Runge Kutta" (generally not compatible with constraints), compatible with Lie group integration and elimination of CoordinateConstraints\\ \hline  
  RK67 & an explicit 7 stage 6th order Runge-Kutta method, see 'On Runge-Kutta Processes of High Order', J. C. Butcher, J. Austr Math Soc 4, (1964); can be used for very accurate (reference) solutions, but without step size control!\\ \hline  
  ODE23 & an explicit Runge Kutta method with automatic step size selection with 3rd order of accuracy and 2nd order error estimation, see Bogacki and Shampine, 1989; also known as ODE23 in MATLAB\\ \hline  
  DOPRI5 & an explicit Runge Kutta method with automatic step size selection with 5th order of accuracy and 4th order error estimation, see  Dormand and Prince, 'A Family of Embedded Runge-Kutta Formulae.', J. Comp. Appl. Math. 6, 1980\\ \hline  
  DVERK6 & [NOT IMPLEMENTED YET] an explicit Runge Kutta solver of 6th order with 5th order error estimation; includes adaptive step selection\\ \hline  
\end{longtable}
\end{center}

%++++++++++++++++++++
\mysubsubsection{KeyCode}
\label{sec:KeyCode}
This section shows the KeyCode structure, which is used for special key codes in keyPressUserFunction.



\begin{center}
\footnotesize
\begin{longtable}{| p{8cm} | p{8cm} |} 
\hline
{\bf function/structure name} & {\bf description}\\ \hline
  SPACE & space key\\ \hline  
  ENTER & enter (return) key\\ \hline  
  TAB & \\ \hline  
  BACKSPACE & \\ \hline  
  RIGHT & cursor right\\ \hline  
  LEFT & cursor left\\ \hline  
  DOWN & cursor down\\ \hline  
  UP & cursor up\\ \hline  
  F1 & function key F1\\ \hline  
  F2 & function key F2\\ \hline  
  F3 & function key F3\\ \hline  
  F4 & function key F4\\ \hline  
  F5 & function key F5\\ \hline  
  F6 & function key F6\\ \hline  
  F7 & function key F7\\ \hline  
  F8 & function key F8\\ \hline  
  F9 & function key F9\\ \hline  
  F10 & function key F10\\ \hline  
\end{longtable}
\end{center}

%++++++++++++++++++++
\mysubsubsection{LinearSolverType}
\label{sec:LinearSolverType}
This section shows the LinearSolverType structure, which is used for selecting linear solver types, which are dense or sparse solvers.



\begin{center}
\footnotesize
\begin{longtable}{| p{8cm} | p{8cm} |} 
\hline
{\bf function/structure name} & {\bf description}\\ \hline
  \_None & no value; used, e.g., if no solver is selected\\ \hline  
  EXUdense & use dense matrices and according solvers for densly populated matrices (usually the CPU time grows cubically with the number of unknowns)\\ \hline  
  EigenSparse & use sparse matrices and according solvers; additional overhead for very small multibody systems; specifically, memory allocation is performed during a factorization process\\ \hline  
  EigenSparseSymmetric & use sparse matrices and according solvers; NOTE: this is the symmetric mode, which assumes symmetric system matrices; this is EXPERIMENTAL and should only be used of user knows that the system matrices are (nearly) symmetric; does not work with scaled GeneralizedAlpha matrices; does not work with constraints, as it must be symmetric positive definite\\ \hline  
\end{longtable}
\end{center}

%++++++++++++++++++++
\mysubsubsection{ContactTypeIndex}
\label{sec:ContactTypeIndex}
This section shows the ContactTypeIndex structure, which is in GeneralContact to select specific contact items, such as spheres, ANCFCable or triangle items.



\begin{center}
\footnotesize
\begin{longtable}{| p{8cm} | p{8cm} |} 
\hline
{\bf function/structure name} & {\bf description}\\ \hline
  IndexSpheresMarkerBased & spheres attached to markers\\ \hline  
  IndexANCFCable2D & ANCFCable2D contact items\\ \hline  
  IndexTrigsRigidBodyBased & triangles attached to rigid body (or rigid body marker)\\ \hline  
  IndexEndOfEnumList & signals end of list\\ \hline  
\end{longtable}
\end{center}
