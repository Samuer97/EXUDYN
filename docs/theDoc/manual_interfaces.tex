% ++++++++++++++++++++++
% description of manual pybind interfaces; generated by Johannes Gerstmayr
% ++++++++++++++++++++++

\mysection{Python-C++ command interface}
\label{sec:PCpp:command:interface}

This chapter lists the basic interface functions which can be used to set up a \codeName\ model in Python.

\mysubsection{General information on Python-C++ interface}
\label{sec:generalPythonInterface}
This chapter lists the basic interface functions which can be used to set up 
a \codeName\ model in Python. Note that some functions or classes will be used in examples, which are explained in detail later on.
In the following, some basic steps and concepts for usage are shown, references to all functions are placed hereafter:

To import the module, just include the \codeName\ module in Python:
\pythonstyle
\begin{lstlisting}[language=Python, firstnumber=1]

import exudyn as exu
\end{lstlisting}


For compatibility with examples and other users, we recommend to use the \texttt{exu} abbreviation throughout. In addition, you may work with a convenient interface for your items, therefore also always include:
\pythonstyle
\begin{lstlisting}[language=Python, firstnumber=1]

from exudyn.itemInterface import *
\end{lstlisting}


Note that including \texttt{exudyn.utilities} will cover \texttt{itemInterface}. Also note that \texttt{from ... import *} is not recommended in general and it will not work in certain cases, e.g., if you like to compute on a cluster. However, it greatly simplifies life for smaller models and you may replace imports in your files afterwards by removing the star import.

The general hub to multibody dynamics models is provided by the classes \texttt{SystemContainer} and \texttt{MainSystem}, except for some very basic system functionality (which is inside the \codeName\ module). 

You can create a new \texttt{SystemContainer}, which is a class that is initialized by assigning a system container to a variable, usually denoted as \texttt{SC}:
\pythonstyle
\begin{lstlisting}[language=Python, firstnumber=1]

SC = exu.SystemContainer()
\end{lstlisting}


Note that creating a second \texttt{exu.SystemContainer()} will be independent of \texttt{SC} and therefore makes no sense if you do not intend to work with two different containers.

To add a MainSystem to system container \texttt{SC} and store as variable \texttt{mbs}, write:
\pythonstyle
\begin{lstlisting}[language=Python, firstnumber=1]

mbs = SC.AddSystem()
\end{lstlisting}


Furthermore, there are a couple of commands available directly in the \texttt{exudyn} module, given in the following subsections. Regarding the \mybold{(basic) module access}, functions are related to the \texttt{exudyn = exu} module, see these examples:
\pythonstyle
\begin{lstlisting}[language=Python, firstnumber=1]

#  import exudyn module:
import exudyn as exu
#  print detailed exudyn version, Python version (at which it is compiled):
exu.GetVersionString(addDetails = True)
#  set precision of C++ output to console
exu.SetOutputPrecision(numberOfDigits)
#  turn on/off output to console
exu.SetWriteToConsole(False)
#  invalid index, may depend on compilation settings:
nInvalid = exu.InvalidIndex() #the invalid index, depends on architecture and version
\end{lstlisting}


Understanding the usage of functions for python object \texttt{SystemContainer} of the module \texttt{exudyn}, the following examples might help:
\pythonstyle
\begin{lstlisting}[language=Python, firstnumber=1]

#import exudyn module:
import exudyn as exu
#  import utilities (includes itemInterface, basicUtilities, 
#                  advancedUtilities, rigidBodyUtilities, graphics):
from exudyn.utilities import *
#  create system container and store in SC:
SC = exu.SystemContainer()
#  add a MainSystem (multibody system) to system container SC and store as mbs:
mbs = SC.AddSystem()
#  add a second MainSystem to system container SC and store as mbs2:
mbs2 = SC.AddSystem()
#  print number of systems available:
nSys = SC.NumberOfSystems()
exu.Print(nSys) #or just print(nSys)
#  delete reference to mbs and mbs2 (usually not necessary):
del mbs, mbs2
#  reset system container (mbs becomes invalid):
SC.Reset()
\end{lstlisting}


If you run a parameter variation (check \texttt{Examples/parameterVariationExample.py}), you may reset or delete the created \texttt{MainSystem} \texttt{mbs} and the \texttt{SystemContainer} \texttt{SC} before creating new instances in order to avoid memory growth.

\mysubsubsection{Item index}
\label{sec:itemIndex}
Many functions will work with node numbers (\texttt{NodeIndex}), object numbers (\texttt{ObjectIndex}),marker numbers (\texttt{MarkerIndex}) and others. These numbers are special Python objects, which have been introduced in order to avoid mixing up, e.g., node and object numbers. 

For example, the command \texttt{mbs.AddNode(...)} returns a \texttt{NodeIndex}. For these indices, the following rules apply:
\bi
  \item[] \texttt{mbs.Add[Node|Object|...](...)} returns a specific \texttt{NodeIndex}, \texttt{ObjectIndex}, ...
  \item[] You can create any item index, e.g., using \texttt{ni = NodeIndex(42)} or \texttt{oi = ObjectIndex(42)}
  \item[] The benefit of these indices comes as they may not be mixed up, e.g., using an object index instead of a node index.
  \item[] You can convert any item index, e.g., NodeIndex \texttt{ni} into an integer number using \texttt{int(ni)} of \texttt{ni.GetIndex()}
  \item[] Still, you can use integers as initialization for item numbers, e.g.:\\\texttt{mbs.AddObject(MassPoint(nodeNumber=13, ...))}\\However, it must be a pure integer type.
  \item[] You can make integer calculations with such indices, e.g., \texttt{oi = 2*ObjectIndex(42)+1} restricing to addition, subtraction and multiplication. Currently, the result of such calculations is a \texttt{int} type andoperating on mixed indices is not checked (but may raise exceptions in future).
  \item[] You can also print item indices, e.g., \texttt{print(ni)} as it converts to string by default.
  \item[] If you are unsure about the type of an index, use \texttt{ni.GetTypeString()} to show the index type.
\ei
\mysubsubsection{Copying and referencing C++ objects}
\label{sec:generalPythonInterface:copyref}
As a key concept to working with \codeName\ , most data which is retrieved by C++ interface functions is copied.
Experienced Python users may know that it is a key concept to Python to often use references instead of copying, which is
sometimes error-prone but offers a computationally efficient behavior.
There are only a few very important cases where data is referenced in \codeName\ , the main ones are 
\texttt{SystemContainer}, 
\texttt{MainSystem}, 
\texttt{VisualizationSettings}, and
\texttt{SimulationSettings} which are always references to internal C++ classes.
The following code snippets and comments should explain this behavior:
\pythonstyle
\begin{lstlisting}[language=Python, firstnumber=1]

import copy                        #for real copying
import exudyn as exu
from exudyn.utilities import *
#create system container, referenced from SC:
SC = exu.SystemContainer()
SC2 = SC                           #this will only put a reference to SC
                                   #SC2 and SC represent the SAME C++ object
#add a MainSystem (multibody system):
mbs = SC.AddSystem()               #get reference mbs to C++ system
mbs2=mbs                           #again, mbs2 and mbs refer to the same C++ object
og = mbs.AddObject(ObjectGround()) #copy data of ObjectGround() into C++
o0 = mbs.GetObject(0)              #get copy of internal data as dictionary

mbsCopy=copy.copy(mbs)             #mbsCopy is now a real copy of mbs; uses pickle; experimental!
SC.Append(mbsCopy)                 #this is needed to work with mbsCopy

del o0                             #delete the local dictionary; C++ data not affected
del mbs, mbs2                      #references to mbs deleted (C++ data still available)
del mbsCopy                        #now also copy of mbs destroyed
del SC                             #references to SystemContainer deleted
#at this point, mbs and SC are not available any more (data will be cleaned up by Python)
\end{lstlisting}


\mysubsubsection{Exceptions and Error Messages}
\label{sec:cinterface:exceptions}
There are several levels of type and argument checks, leading to different types of errors and exceptions. The according error messages are non-unique, because they may be raised in Python modules or in C++, and they may be raised on different levels of the code. Error messages depend on Python version and on your iPython console. Very often the exception may be called \texttt{ValueError}, but it mustnot mean that it is a wrong error, but it could also be, e.g., a wrong order of function calls.

As an example, a type conversion error is raised when providing wrong argument types, e.g., try \texttt{exu.GetVersionString('abc')}:
\pythonstyle
\begin{lstlisting}[language=Python, firstnumber=1]

Traceback (most recent call last):

File "C:\Users\username\AppData\Local\Temp\ipykernel_24988\2212168679.py", line 1, in <module>
    exu.GetVersionString('abc')

TypeError: GetVersionString(): incompatible function arguments. The following argument types are supported:
    1. (addDetails: bool = False) -> str

Invoked with: 'abc'
\end{lstlisting}


Note that your particular error message may be different.

Another error results from internal type and range checking, saying User ERROR, as it is due to a wrong input of the user. For this, we try
\pythonstyle
\begin{lstlisting}[language=Python, firstnumber=1]
mbs.AddObject('abc')
\end{lstlisting}


Which results in an error message similar to:
\pythonstyle
\begin{lstlisting}[language=Python, firstnumber=1]

=========================================
User ERROR [file 'C:\Users\username\AppData\Local\Temp\ipykernel_24988\2838049308.py', line 1]: 
Error in AddObject(...):
Check your python code (negative indices, invalid or undefined parameters, ...)

=========================================

Traceback (most recent call last):

  File "C:\Users\username\AppData\Local\Temp\ipykernel_24988\2838049308.py", line 1, in <module>
    mbs.AddObject('abc')

RuntimeError: Exudyn: parsing of Python file terminated due to Python (user) error

\end{lstlisting}


Finally, there may be system errors. They may be caused due to previous wrong input, but if there is no reason seen, it may be appropriate to report this error on \exuUrl{https://github.com/jgerstmayr/EXUDYN}{github.com/jgerstmayr/EXUDYN/} .

Be careful in reading and interpreting such error messages. You should \mybold{read them from top to bottom}, as the cause may be in the beginning. Often files and line numbers of errors are provided (e.g., if you have a longer script). In the ultimate case, try to comment parts of your code or deactivate items to see where the error comes from. See also section on Trouble shooting and FAQ.

%++++++++++++++++++++
\mysubsection{\codeName}



These are the access functions to the \codeName\ module. General usage is explained in \refSection{sec:generalPythonInterface} and examples are provided there. The C++ module \texttt{exudyn} is the root level object linked between Python and C++.In the installed site-packages, the according file is usually denoted as \texttt{exudynCPP.pyd} for the regular module, \texttt{exudynCPPfast.pyd} for the module without range checks and \texttt{exudynCPPnoAVX.pyd} for the module compiled without AVX vector extensions (may depend on your installation).
\pythonstyle
\begin{lstlisting}[language=Python, firstnumber=1]

#import exudyn module:
import exudyn as exu
#create systemcontainer and mbs:
SC = exu.SystemContainer()
mbs = SC.AddSystem()
\end{lstlisting}

\begin{center}
\footnotesize
\begin{longtable}{| p{8cm} | p{8cm} |} 
\hline
{\bf function/structure name} & {\bf description}\\ \hline
  GetVersionString(addDetails = False) & Get Exudyn built version as string (if addDetails=True, adds more information on compilation Python version, platform, etc.; the Python micro version may differ from that you are working with; AVX2 shows that you are running a AVX2 compiled version)\\ \hline 
  Help() & Show basic help information\\ \hline 
  RequireVersion(requiredVersionString) & Checks if the installed version is according to the required version. Major, micro and minor version must agree the required level. This function is defined in the \texttt{\_\_init\_\_.py} file\tabnewline 
    \textcolor{steelblue}{{\bf EXAMPLE}: \tabnewline 
    \texttt{exu.RequireVersion("1.0.31")}}\\ \hline 
  StartRenderer(verbose = 0) & Start OpenGL rendering engine (in separate thread) for visualization of rigid or flexible multibody system; use verbose=1 to output information during OpenGL window creation; verbose=2 produces more output and verbose=3 gives a debug level; some of the information will only be seen in windows command (powershell) windows or linux shell, but not inside iPython of e.g. Spyder\\ \hline 
  StopRenderer() & Stop OpenGL rendering engine\\ \hline 
  IsRendererActive() & returns True if GLFW renderer is available and running; otherwise False\\ \hline 
  DoRendererIdleTasks(waitSeconds = 0) & Call this function in order to interact with Renderer window; use waitSeconds in order to run this idle tasks while animating a model (e.g. waitSeconds=0.04), use waitSeconds=0 without waiting, or use waitSeconds=-1 to wait until window is closed\\ \hline 
  SolveStatic(mbs, simulationSettings = exudyn.SimulationSettings(), updateInitialValues = False, storeSolver = True) & Static solver function, mapped from module \texttt{solver}, to solve static equations (without inertia terms) of constrained rigid or flexible multibody system; for details on the Python interface see \refSection{sec:mainsystemextensions:SolveStatic}; for background on solvers, see \refSection{sec:solvers}\\ \hline 
  SolveDynamic(mbs, simulationSettings = exudyn.SimulationSettings(), solverType = exudyn.DynamicSolverType.GeneralizedAlpha, updateInitialValues = False, storeSolver = True) & Dynamic solver function, mapped from module \texttt{solver}, to solve equations of motion of constrained rigid or flexible multibody system; for details on the Python interface see \refSection{sec:mainsystemextensions:SolveDynamic}; for background on solvers, see \refSection{sec:solvers}\\ \hline 
  ComputeODE2Eigenvalues(mbs, simulationSettings = exudyn.SimulationSettings(), useSparseSolver = False, numberOfEigenvalues = -1, setInitialValues = True, convert2Frequencies = False) & Simple interface to scipy eigenvalue solver for eigenvalue analysis of the second order differential equations part in mbs, mapped from module \texttt{solver}; for details on the Python interface see \refSection{sec:mainsystemextensions:ComputeODE2Eigenvalues}\\ \hline 
  SetOutputPrecision(numberOfDigits) & Set the precision (integer) for floating point numbers written to console (reset when simulation is started!); NOTE: this affects only floats converted to strings inside C++ exudyn; if you print a float from Python, it is usually printed with 16 digits; if printing numpy arrays, 8 digits are used as standard, to be changed with numpy.set\_printoptions(precision=16); alternatively convert into a list\\ \hline 
  SetLinalgOutputFormatPython(flagPythonFormat) & True: use Python format for output of vectors and matrices; False: use matlab format\\ \hline 
  SetWriteToConsole(flag) & set flag to write (True) or not write to console; default = True\\ \hline 
  SetWriteToFile(filename, flagWriteToFile = True, flagAppend = False) & set flag to write (True) or not write to console; default value of flagWriteToFile = False; flagAppend appends output to file, if set True; in order to finalize the file, write \texttt{exu.SetWriteToFile('', False)} to close the output file\tabnewline 
    \textcolor{steelblue}{{\bf EXAMPLE}: \tabnewline 
    \texttt{exu.SetWriteToConsole(False) \#no output to console\tabnewline
    exu.SetWriteToFile(filename={\textquotesingle}testOutput.log{\textquotesingle}, flagWriteToFile=True, flagAppend=False)\tabnewline
    exu.Print({\textquotesingle}print this to file{\textquotesingle})\tabnewline
    exu.SetWriteToFile({\textquotesingle}{\textquotesingle}, False) \#terminate writing to file which closes the file}}\\ \hline 
  SetPrintDelayMilliSeconds(delayMilliSeconds) & add some delay (in milliSeconds) to printing to console, in order to let Spyder process the output; default = 0\\ \hline 
  Print(*args) & this allows printing via exudyn with similar syntax as in Python print(args) except for keyword arguments: print('test=',42); allows to redirect all output to file given by SetWriteToFile(...); does not output in case that SetWriteToConsole is set to False\\ \hline 
  SuppressWarnings(flag) & set flag to suppress (=True) or enable (=False) warnings\\ \hline 
  InfoStat(writeOutput = True) & Retrieve list of global information on memory allocation and other counts as list:[array\_new\_counts, array\_delete\_counts, vector\_new\_counts, vector\_delete\_counts, matrix\_new\_counts, matrix\_delete\_counts, linkedDataVectorCast\_counts]; May be extended in future; if writeOutput==True, it additionally prints the statistics; counts for new vectors and matrices should not depend on numberOfSteps, except for some objects such as ObjectGenericODE2 and for (sensor) output to files; Not available if code is compiled with \_\_FAST\_EXUDYN\_LINALG flag\\ \hline 
  Go() & Creates a SystemContainer SC and a main multibody system mbs\\ \hline 
  Demo1(showAll) & Run simple demo without graphics to check functionality, see exudyn/demos.py\\ \hline 
  Demo2(showAll) & Run advanced demo without graphics to check functionality, see exudyn/demos.py\\ \hline 
  InvalidIndex() & This function provides the invalid index, which may depend on the kind of 32-bit, 64-bit signed or unsigned integer; e.g. node index or item index in list; currently, the InvalidIndex() gives -1, but it may be changed in future versions, therefore you should use this function\\ \hline 
  \_\_version\_\_ & stores the current version of the Exudyn package\\ \hline  
  symbolic & the symbolic submodule for creating symbolic variables in Python, see documentation of Symbolic; For details, see Section Symbolic.\\ \hline  
  experimental & Experimental features, not intended for regular users; for available features, see the C++ code class PyExperimental\\ \hline  
  special & special attributes and functions, such as global (solver) flags or helper functions; not intended for regular users; for available features, see the C++ code class PySpecial\\ \hline  
  special.solver & special solver attributes and functions; not intended for regular users; for available features, see the C++ code class PySpecialSolver\\ \hline  
  special.solver.timeout & if >= 0, the solver stops after reaching accoring CPU time specified with timeout; makes sense for parameter variation, automatic testing or for long-running simulations; default=-1 (no timeout)\\ \hline  
  variables & this dictionary may be used by the user to store exudyn-wide data in order to avoid global Python variables; usage: exu.variables["myvar"] = 42 \\ \hline  
  sys & this dictionary is used and reserved by the system, e.g. for testsuite, graphics or system function to store module-wide data in order to avoid global Python variables; the variable exu.sys['renderState'] contains the last render state after exu.StopRenderer() and can be used for subsequent simulations \\ \hline  
\end{longtable}
\end{center}

%++++++++++++++++++++
\mysubsection{SystemContainer}



The SystemContainer is the top level of structures in \codeName. The container holds all (multibody) systems, solvers and all other data structures for computation. A SystemContainer is created by \texttt{SC = exu.SystemContainer()}, understanding \texttt{exu.SystemContainer} as a class like Python's internal list class, creating a list instance with \texttt{x=list()}. Currently, only one container shall be used. In future, multiple containers might be usable at the same time. Regarding the \mybold{(basic) module access}, functions are related to the \texttt{exudyn = exu} module, see also the introduction of this chapter and this example:
\pythonstyle
\begin{lstlisting}[language=Python, firstnumber=1]

import exudyn as exu
#create system container and store by reference in SC:
SC = exu.SystemContainer() 
#add MainSystem to SC:
mbs = SC.AddSystem()
\end{lstlisting}

\begin{center}
\footnotesize
\begin{longtable}{| p{8cm} | p{8cm} |} 
\hline
{\bf function/structure name} & {\bf description}\\ \hline
  Reset() & delete all multibody systems and reset SystemContainer (including graphics); this also releases SystemContainer from the renderer, which requires SC.AttachToRenderEngine() to be called in order to reconnect to rendering; a safer way is to delete the current SystemContainer and create a new one (SC=SystemContainer() )\\ \hline 
  AddSystem() & add a new computational system\\ \hline 
  Append(mainSystem) & append an exsiting computational system to the system container; returns the number of MainSystem in system container\\ \hline 
  NumberOfSystems() & obtain number of multibody systems available in system container\\ \hline 
  GetSystem(systemNumber) & obtain multibody systems with index from system container\\ \hline 
  visualizationSettings & this structure is read/writeable and contains visualization settings, which are immediately applied to the rendering window. \tabnewline
    EXAMPLE:\tabnewline
    SC = exu.SystemContainer()\tabnewline
    SC.visualizationSettings.autoFitScene=False  \\ \hline  
  GetDictionary() & [UNDER DEVELOPMENT]: return the dictionary of the system container data, e.g., to copy the system or for pickling\\ \hline 
  SetDictionary(systemDict) & [UNDER DEVELOPMENT]: set system container data from given dictionary; used for pickling\\ \hline 
  GetRenderState() & Get dictionary with current render state (openGL zoom, modelview, etc.); will have no effect if GLFW\_GRAPHICS is deactivated\tabnewline 
    \textcolor{steelblue}{{\bf EXAMPLE}: \tabnewline 
    \texttt{SC = exu.SystemContainer()\tabnewline
    renderState = SC.GetRenderState() \tabnewline
    print(renderState[{\textquotesingle}zoom{\textquotesingle}])}}\\ \hline 
  SetRenderState(renderState) & Set current render state (openGL zoom, modelview, etc.) with given dictionary; usually, this dictionary has been obtained with GetRenderState; will have no effect if GLFW\_GRAPHICS is deactivated\tabnewline 
    \textcolor{steelblue}{{\bf EXAMPLE}: \tabnewline 
    \texttt{SC = exu.SystemContainer()\tabnewline
    SC.SetRenderState(renderState)}}\\ \hline 
  RedrawAndSaveImage() & Redraw openGL scene and save image (command waits until process is finished)\\ \hline 
  WaitForRenderEngineStopFlag() & Wait for user to stop render engine (Press 'Q' or Escape-key); this command is used to have active response of the render window, e.g., to open the visualization dialog or use the right-mouse-button; behaves similar as mbs.WaitForUserToContinue()\\ \hline 
  RenderEngineZoomAll() & Send zoom all signal, which will perform zoom all at next redraw request\\ \hline 
  AttachToRenderEngine() & Links the SystemContainer to the render engine, such that the changes in the graphics structure drawn upon updates, etc.; done automatically on creation of SystemContainer; return False, if no renderer exists (e.g., compiled without GLFW) or cannot be linked (if other SystemContainer already linked)\\ \hline 
  DetachFromRenderEngine() & Releases the SystemContainer from the render engine; return True if successfully released, False if no GLFW available or detaching failed\\ \hline 
  SendRedrawSignal() & This function is used to send a signal to the renderer that all MainSystems (mbs) shall be redrawn\\ \hline 
  GetCurrentMouseCoordinates(useOpenGLcoordinates = False) & Get current mouse coordinates as list [x, y]; x and y being floats, as returned by GLFW, measured from top left corner of window; use GetCurrentMouseCoordinates(useOpenGLcoordinates=True) to obtain OpenGLcoordinates of projected plane\\ \hline 
\end{longtable}
\end{center}

%++++++++++++++++++++
\mysubsection{MainSystem}



This is the class which defines a (multibody) system. The MainSystem shall only be created by \texttt{SC.AddSystem()}, not with \texttt{exu.MainSystem()}, as the latter one would not be linked to a SystemContainer. In some cases, you may use SC.Append(mbs). In C++, there is a MainSystem (the part which links to Python) and a System (computational part). For that reason, the name is MainSystem on the Python side, but it is often just called 'system'. For compatibility, it is recommended to denote the variable holding this system as mbs, the multibody dynamics system. It can be created, visualized and computed. Use the following functions for system manipulation.
\pythonstyle
\begin{lstlisting}[language=Python, firstnumber=1]

import exudyn as exu
SC = exu.SystemContainer()
mbs = SC.AddSystem()
\end{lstlisting}

\begin{center}
\footnotesize
\begin{longtable}{| p{8cm} | p{8cm} |} 
\hline
{\bf function/structure name} & {\bf description}\\ \hline
  Assemble() & assemble items (nodes, bodies, markers, loads, ...) of multibody system; Calls CheckSystemIntegrity(...), AssembleCoordinates(), AssembleLTGLists(), AssembleInitializeSystemCoordinates(), and AssembleSystemInitialize()\\ \hline 
  AssembleCoordinates() & assemble coordinates: assign computational coordinates to nodes and constraints (algebraic variables)\\ \hline 
  AssembleLTGLists() & build \ac{LTG} coordinate lists for objects (used to build global ODE2RHS, MassMatrix, etc. vectors and matrices) and store special object lists (body, connector, constraint, ...)\\ \hline 
  AssembleInitializeSystemCoordinates() & initialize all system-wide coordinates based on initial values given in nodes\\ \hline 
  AssembleSystemInitialize() & initialize some system data, e.g., generalContact objects (searchTree, etc.)\\ \hline 
  Reset() & reset all lists of items (nodes, bodies, markers, loads, ...) and temporary vectors; deallocate memory\\ \hline 
  GetSystemContainer() & return the systemContainer where the mainSystem (mbs) was created\\ \hline 
  WaitForUserToContinue(printMessage = True) & interrupt further computation until user input --> 'pause' function; this command runs a loop in the background to have active response of the render window, e.g., to open the visualization dialog or use the right-mouse-button; behaves similar as SC.WaitForRenderEngineStopFlagthis()\\ \hline 
  SendRedrawSignal() & this function is used to send a signal to the renderer that the scene shall be redrawn because the visualization state has been updated\\ \hline 
  GetRenderEngineStopFlag() & get the current stop simulation flag; True=user wants to stop simulation\\ \hline 
  SetRenderEngineStopFlag(stopFlag) & set the current stop simulation flag; set to False, in order to continue a previously user-interrupted simulation\\ \hline 
  ActivateRendering(flag = True) & activate (flag=True) or deactivate (flag=False) rendering for this system\\ \hline 
  SetPreStepUserFunction(value) & Sets a user function PreStepUserFunction(mbs, t) executed at beginning of every computation step; in normal case return True; return False to stop simulation after current step; set to 0 (integer) in order to erase user function. Note that the time returned is already the end of the step, which allows to compute forces consistently with trapezoidal integrators; for higher order Runge-Kutta methods, step time will be available only in object-user functions.\tabnewline 
    \textcolor{steelblue}{{\bf EXAMPLE}: \tabnewline 
    \texttt{def PreStepUserFunction(mbs, t):\tabnewline
     \phantom{XXXX} print(mbs.systemData.NumberOfNodes())\tabnewline
     \phantom{XXXX} if(t>1): \tabnewline
     \phantom{XXXX}  \phantom{XXXX} return False \tabnewline
     \phantom{XXXX} return True \tabnewline
    mbs.SetPreStepUserFunction(PreStepUserFunction)}}\\ \hline 
  GetPreStepUserFunction(asDict = False) & Returns the preStepUserFunction.\\ \hline 
  SetPostStepUserFunction(value) & Sets a user function PostStepUserFunction(mbs, t) executed at beginning of every computation step; in normal case return True; return False to stop simulation after current step; set to 0 (integer) in order to erase user function.\tabnewline 
    \textcolor{steelblue}{{\bf EXAMPLE}: \tabnewline 
    \texttt{def PostStepUserFunction(mbs, t):\tabnewline
     \phantom{XXXX} print(mbs.systemData.NumberOfNodes())\tabnewline
     \phantom{XXXX} if(t>1): \tabnewline
     \phantom{XXXX}  \phantom{XXXX} return False \tabnewline
     \phantom{XXXX} return True \tabnewline
    mbs.SetPostStepUserFunction(PostStepUserFunction)}}\\ \hline 
  GetPostStepUserFunction(asDict = False) & Returns the postStepUserFunction.\\ \hline 
  SetPostNewtonUserFunction(value) & Sets a user function PostNewtonUserFunction(mbs, t) executed after successful Newton iteration in implicit or static solvers and after step update of explicit solvers, but BEFORE PostNewton functions are called by the solver; function returns list [discontinuousError, recommendedStepSize], containing a error of the PostNewtonStep, which is compared to [solver].discontinuous.iterationTolerance. The recommendedStepSize shall be negative, if no recommendation is given, 0 in order to enforce minimum step size or a specific value to which the current step size will be reduced and the step will be repeated; use this function, e.g., to reduce step size after impact or change of data variables; set to 0 (integer) in order to erase user function. Similar described by Flores and Ambrosio, https://doi.org/10.1007/s11044-010-9209-8\tabnewline 
    \textcolor{steelblue}{{\bf EXAMPLE}: \tabnewline 
    \texttt{def PostNewtonUserFunction(mbs, t):\tabnewline
     \phantom{XXXX} if(t>1): \tabnewline
     \phantom{XXXX}  \phantom{XXXX} return [0, 1e-6] \tabnewline
     \phantom{XXXX} return [0,0] \tabnewline
    mbs.SetPostNewtonUserFunction(PostNewtonUserFunction)}}\\ \hline 
  GetPostNewtonUserFunction(asDict = False) & Returns the postNewtonUserFunction.\\ \hline 
  AddGeneralContact() & add a new general contact, used to enable efficient contact computation between objects (nodes or markers)\\ \hline 
  GetGeneralContact(generalContactNumber) & get read/write access to GeneralContact with index generalContactNumber stored in mbs; Examples shows how to access the GeneralContact object added with last AddGeneralContact() command:\tabnewline 
    \textcolor{steelblue}{{\bf EXAMPLE}: \tabnewline 
    \texttt{gc=mbs.GetGeneralContact(mbs.NumberOfGeneralContacts()-1)}}\\ \hline 
  DeleteGeneralContact(generalContactNumber) & delete GeneralContact with index generalContactNumber in mbs; other general contacts are resorted (index changes!)\\ \hline 
  NumberOfGeneralContacts() & Return number of GeneralContact objects in mbs\\ \hline 
  GetAvailableFactoryItems() & get all available items to be added (nodes, objects, etc.); this is useful in particular in case of additional user elements to check if they are available; the available items are returned as dictionary, containing lists of strings for Node, Object, etc.\\ \hline 
  GetDictionary() & [UNDER DEVELOPMENT]: return the dictionary of the system data (todo: and state), e.g., to copy the system or for pickling\\ \hline 
  SetDictionary(systemDict) & [UNDER DEVELOPMENT]: set system data (todo: and state) from given dictionary; used for pickling\\ \hline 
  \_\_repr\_\_() & return the representation of the system, which can be, e.g., printed\tabnewline 
    \textcolor{steelblue}{{\bf EXAMPLE}: \tabnewline 
    \texttt{print(mbs)}}\\ \hline 
  systemIsConsistent & this flag is used by solvers to decide, whether the system is in a solvable state; this flag is set to False as long as Assemble() has not been called; any modification to the system, such as Add...(), Modify...(), etc. will set the flag to False again; this flag can be modified (set to True), if a change of e.g.~an object (change of stiffness) or load (change of force) keeps the system consistent, but would normally lead to systemIsConsistent=False\\ \hline  
  interactiveMode & set this flag to True in order to invoke a Assemble() command in every system modification, e.g. AddNode, AddObject, ModifyNode, ...; this helps that the system can be visualized in interactive mode.\\ \hline  
  variables & this dictionary may be used by the user to store model-specific data, in order to avoid global Python variables in complex models; mbs.variables["myvar"] = 42 \\ \hline  
  sys & this dictionary is used by exudyn Python libraries, e.g., solvers, to avoid global Python variables \\ \hline  
  solverSignalJacobianUpdate & this flag is used by solvers to decide, whether the jacobian should be updated; at beginning of simulation and after jacobian computation, this flag is set automatically to False; use this flag to indicate system changes, e.g. during time integration  \\ \hline  
  systemData & Access to SystemData structure; enables access to number of nodes, objects, ... and to (current, initial, reference, ...) state variables (ODE2, AE, Data,...)\\ \hline  
\end{longtable}
\end{center}

%++++++++++++++++++++
\mysubsubsection{MainSystem extensions (create)}
\label{sec:mainsystem:pythonExtensionsCreate}
This section represents extensions to MainSystem, which are direct calls to Python functions; the 'create' extensions to simplify the creation of multibody systems, such as CreateMassPoint(...); these extensions allow a more intuitive interaction with the MainSystem class, see the following example. For activation, import \texttt{exudyn.mainSystemExtensions} or \texttt{exudyn.utilities}

\pythonstyle
\begin{lstlisting}[language=Python, firstnumber=1]

import exudyn as exu           
from exudyn.utilities import * 
#alternative: import exudyn.mainSystemExtensions
SC = exu.SystemContainer()
mbs = SC.AddSystem()
#
#create rigid body
b1=mbs.CreateRigidBody(inertia = InertiaCuboid(density=5000, sideLengths=[0.1,0.1,1]),
                       referencePosition = [1,0,0], 
                       gravity = [0,0,-9.81])
\end{lstlisting}

\begin{flushleft}
\noindent {def {\bf \exuUrl{https://github.com/jgerstmayr/EXUDYN/blob/master/main/pythonDev/exudyn/mainSystemExtensions.py\#L147}{CreateGround}{}}}\label{sec:mainsystemextensions:CreateGround}
({\it name}= '', {\it referencePosition}= [0.,0.,0.], {\it referenceRotationMatrix}= np.eye(3), {\it graphicsDataList}= [], {\it graphicsDataUserFunction}= 0, {\it show}= True)
\end{flushleft}
\setlength{\itemindent}{0.7cm}
\begin{itemize}[leftmargin=0.7cm]
\item[--]
{\bf function description}: \vspace{-6pt}
\begin{itemize}[leftmargin=1.2cm]
\setlength{\itemindent}{-0.7cm}
\item[]helper function to create a ground object, using arguments of ObjectGround; this function is mainly added for consistency with other mainSystemExtensions
\item[]- NOTE that this function is added to MainSystem via Python function MainSystemCreateGround.
\end{itemize}
\item[--]
{\bf input}: \vspace{-6pt}
\begin{itemize}[leftmargin=1.2cm]
\setlength{\itemindent}{-0.7cm}
\item[]{\it name}: name string for object
\item[]{\it referencePosition}: reference coordinates for point node (always a 3D vector, no matter if 2D or 3D mass)
\item[]{\it referenceRotationMatrix}: reference rotation matrix for rigid body node (always 3D matrix, no matter if 2D or 3D body)
\item[]{\it graphicsDataList}: list of GraphicsData for optional ground visualization
\item[]{\it graphicsDataUserFunction}: a user function graphicsDataUserFunction(mbs, itemNumber)->BodyGraphicsData (list of GraphicsData), which can be used to draw user-defined graphics; this is much slower than regular GraphicsData
\item[]{\it color}: color of node
\item[]{\it show}: True: show ground object;
\end{itemize}
\item[--]
{\bf output}: ObjectIndex; returns ground object index
\item[--]
{\bf example}: \vspace{-12pt}\ei\begin{lstlisting}[language=Python, xleftmargin=36pt]
  import exudyn as exu
  from exudyn.utilities import * #includes itemInterface and rigidBodyUtilities
  import numpy as np
  SC = exu.SystemContainer()
  mbs = SC.AddSystem()
  ground=mbs.CreateGround(referencePosition = [2,0,0],
                          graphicsDataList = [exu.graphics.CheckerBoard(point=[0,0,0], normal=[0,1,0],size=4)])
\end{lstlisting}\vspace{-24pt}\bi\item[]\vspace{-24pt}\vspace{12pt}\end{itemize}
%

%
\noindent For examples on CreateGround see Relevant Examples (Ex) and TestModels (TM) with weblink to github:
\bi
 \item \footnotesize \exuUrl{https://github.com/jgerstmayr/EXUDYN/blob/master/main/pythonDev/Examples/basicTutorial2024.py}{\texttt{basicTutorial2024.py}} (Ex), 
\exuUrl{https://github.com/jgerstmayr/EXUDYN/blob/master/main/pythonDev/Examples/beamTutorial.py}{\texttt{beamTutorial.py}} (Ex), 
\exuUrl{https://github.com/jgerstmayr/EXUDYN/blob/master/main/pythonDev/Examples/bicycleIftommBenchmark.py}{\texttt{bicycleIftommBenchmark.py}} (Ex), 
\\ \exuUrl{https://github.com/jgerstmayr/EXUDYN/blob/master/main/pythonDev/Examples/bungeeJump.py}{\texttt{bungeeJump.py}} (Ex), 
\exuUrl{https://github.com/jgerstmayr/EXUDYN/blob/master/main/pythonDev/Examples/cartesianSpringDamper.py}{\texttt{cartesianSpringDamper.py}} (Ex), 
 ...
, 
\exuUrl{https://github.com/jgerstmayr/EXUDYN/blob/master/main/pythonDev/TestModels/contactSphereSphereTest.py}{\texttt{contactSphereSphereTest.py}} (TM), 
\\ \exuUrl{https://github.com/jgerstmayr/EXUDYN/blob/master/main/pythonDev/TestModels/contactSphereSphereTestEAPM.py}{\texttt{contactSphereSphereTestEAPM.py}} (TM), 
\exuUrl{https://github.com/jgerstmayr/EXUDYN/blob/master/main/pythonDev/TestModels/ConvexContactTest.py}{\texttt{ConvexContactTest.py}} (TM), 
 ...

\ei

%
\begin{flushleft}
\noindent {def {\bf \exuUrl{https://github.com/jgerstmayr/EXUDYN/blob/master/main/pythonDev/exudyn/mainSystemExtensions.py\#L216}{CreateMassPoint}{}}}\label{sec:mainsystemextensions:CreateMassPoint}
({\it name}= '', {\it referencePosition}= [0.,0.,0.], {\it initialDisplacement}= [0.,0.,0.], {\it initialVelocity}= [0.,0.,0.], {\it physicsMass}= 0, {\it gravity}= [0.,0.,0.], {\it graphicsDataList}= [], {\it drawSize}= -1, {\it color}= [-1.,-1.,-1.,-1.], {\it show}= True, {\it create2D}= False, {\it returnDict}= False)
\end{flushleft}
\setlength{\itemindent}{0.7cm}
\begin{itemize}[leftmargin=0.7cm]
\item[--]
{\bf function description}: \vspace{-6pt}
\begin{itemize}[leftmargin=1.2cm]
\setlength{\itemindent}{-0.7cm}
\item[]helper function to create 2D or 3D mass point object and node, using arguments as in NodePoint and MassPoint
\item[]- NOTE that this function is added to MainSystem via Python function MainSystemCreateMassPoint.
\end{itemize}
\item[--]
{\bf input}: \vspace{-6pt}
\begin{itemize}[leftmargin=1.2cm]
\setlength{\itemindent}{-0.7cm}
\item[]{\it name}: name string for object, node is 'Node:'+name
\item[]{\it referencePosition}: reference coordinates for point node (always a 3D vector, no matter if 2D or 3D mass)
\item[]{\it initialDisplacement}: initial displacements for point node (always a 3D vector, no matter if 2D or 3D mass)
\item[]{\it initialVelocity}: initial velocities for point node (always a 3D vector, no matter if 2D or 3D mass)
\item[]{\it physicsMass}: mass of mass point
\item[]{\it gravity}: gravity vevtor applied (always a 3D vector, no matter if 2D or 3D mass)
\item[]{\it graphicsDataList}: list of GraphicsData for optional mass visualization
\item[]{\it drawSize}: general drawing size of node
\item[]{\it color}: color of node
\item[]{\it show}: True: if graphicsData list is empty, node is shown, otherwise body is shown; otherwise, nothing is shown
\item[]{\it create2D}: if True, create NodePoint2D and MassPoint2D
\item[]{\it returnDict}: if False, returns object index; if True, returns dict of all information on created object and node
\end{itemize}
\item[--]
{\bf output}: Union[dict, ObjectIndex]; returns mass point object index or dict with all data on request (if returnDict=True)
\item[--]
{\bf example}: \vspace{-12pt}\ei\begin{lstlisting}[language=Python, xleftmargin=36pt]
  import exudyn as exu
  from exudyn.utilities import * #includes itemInterface and rigidBodyUtilities
  import numpy as np
  SC = exu.SystemContainer()
  mbs = SC.AddSystem()
  b0=mbs.CreateMassPoint(referencePosition = [0,0,0],
                         initialVelocity = [2,5,0],
                         physicsMass = 1, gravity = [0,-9.81,0],
                         drawSize = 0.5, color=exu.graphics.color.blue)
  mbs.Assemble()
  simulationSettings = exu.SimulationSettings() #takes currently set values or default values
  simulationSettings.timeIntegration.numberOfSteps = 1000
  simulationSettings.timeIntegration.endTime = 2
  mbs.SolveDynamic(simulationSettings = simulationSettings)
\end{lstlisting}\vspace{-24pt}\bi\item[]\vspace{-24pt}\vspace{12pt}\end{itemize}
%

%
\noindent For examples on CreateMassPoint see Relevant Examples (Ex) and TestModels (TM) with weblink to github:
\bi
 \item \footnotesize \exuUrl{https://github.com/jgerstmayr/EXUDYN/blob/master/main/pythonDev/Examples/basicTutorial2024.py}{\texttt{basicTutorial2024.py}} (Ex), 
\exuUrl{https://github.com/jgerstmayr/EXUDYN/blob/master/main/pythonDev/Examples/cartesianSpringDamper.py}{\texttt{cartesianSpringDamper.py}} (Ex), 
\exuUrl{https://github.com/jgerstmayr/EXUDYN/blob/master/main/pythonDev/Examples/cartesianSpringDamperUserFunction.py}{\texttt{cartesianSpringDamperUserFunction.py}} (Ex), 
\\ \exuUrl{https://github.com/jgerstmayr/EXUDYN/blob/master/main/pythonDev/Examples/chatGPTupdate.py}{\texttt{chatGPTupdate.py}} (Ex), 
\exuUrl{https://github.com/jgerstmayr/EXUDYN/blob/master/main/pythonDev/Examples/serialRobotURDF.py}{\texttt{serialRobotURDF.py}} (Ex), 
 ...
, 
\exuUrl{https://github.com/jgerstmayr/EXUDYN/blob/master/main/pythonDev/TestModels/loadUserFunctionTest.py}{\texttt{loadUserFunctionTest.py}} (TM), 
\\ \exuUrl{https://github.com/jgerstmayr/EXUDYN/blob/master/main/pythonDev/TestModels/mainSystemExtensionsTests.py}{\texttt{mainSystemExtensionsTests.py}} (TM), 
\exuUrl{https://github.com/jgerstmayr/EXUDYN/blob/master/main/pythonDev/TestModels/mainSystemUserFunctionsTest.py}{\texttt{mainSystemUserFunctionsTest.py}} (TM), 
 ...

\ei

%
\begin{flushleft}
\noindent {def {\bf \exuUrl{https://github.com/jgerstmayr/EXUDYN/blob/master/main/pythonDev/exudyn/mainSystemExtensions.py\#L347}{CreateRigidBody}{}}}\label{sec:mainsystemextensions:CreateRigidBody}
({\it name}= '', {\it referencePosition}= [0.,0.,0.], {\it referenceRotationMatrix}= np.eye(3), {\it initialVelocity}= [0.,0.,0.], {\it initialAngularVelocity}= [0.,0.,0.], {\it initialDisplacement}= None, {\it initialRotationMatrix}= None, {\it inertia}= None, {\it gravity}= [0.,0.,0.], {\it nodeType}= exudyn.NodeType.RotationEulerParameters, {\it graphicsDataList}= [], {\it graphicsDataUserFunction}= 0, {\it drawSize}= -1, {\it color}= [-1.,-1.,-1.,-1.], {\it show}= True, {\it create2D}= False, {\it returnDict}= False)
\end{flushleft}
\setlength{\itemindent}{0.7cm}
\begin{itemize}[leftmargin=0.7cm]
\item[--]
{\bf function description}: \vspace{-6pt}
\begin{itemize}[leftmargin=1.2cm]
\setlength{\itemindent}{-0.7cm}
\item[]helper function to create 3D (or 2D) rigid body object and node; all quantities are global (angular velocity, etc.)
\item[]- NOTE that this function is added to MainSystem via Python function MainSystemCreateRigidBody.
\end{itemize}
\item[--]
{\bf input}: \vspace{-6pt}
\begin{itemize}[leftmargin=1.2cm]
\setlength{\itemindent}{-0.7cm}
\item[]{\it name}: name string for object, node is 'Node:'+name
\item[]{\it referencePosition}: reference position vector for rigid body node (always a 3D vector, no matter if 2D or 3D body)
\item[]{\it referenceRotationMatrix}: reference rotation matrix for rigid body node (always 3D matrix, no matter if 2D or 3D body)
\item[]{\it initialVelocity}: initial translational velocity vector for node (always a 3D vector, no matter if 2D or 3D body)
\item[]{\it initialAngularVelocity}: initial angular velocity vector for node (always a 3D vector, no matter if 2D or 3D body)
\item[]{\it initialDisplacement}: initial translational displacement vector for node (always a 3D vector, no matter if 2D or 3D body); these displacements are deviations from reference position, e.g. for a finite element node [None: unused]
\item[]{\it initialRotationMatrix}: initial rotation provided as matrix (always a 3D matrix, no matter if 2D or 3D body); this rotation is superimposed to reference rotation [None: unused]
\item[]{\it inertia}: an instance of class RigidBodyInertia, see rigidBodyUtilities; may also be from derived class (InertiaCuboid, InertiaMassPoint, InertiaCylinder, ...)
\item[]{\it gravity}: gravity vevtor applied (always a 3D vector, no matter if 2D or 3D mass)
\item[]{\it graphicsDataList}: list of GraphicsData for rigid body visualization; use exudyn.graphics functions to create GraphicsData for basic solids
\item[]{\it graphicsDataUserFunction}: a user function graphicsDataUserFunction(mbs, itemNumber)->BodyGraphicsData (list of GraphicsData), which can be used to draw user-defined graphics; this is much slower than regular GraphicsData
\item[]{\it drawSize}: general drawing size of node
\item[]{\it color}: color of node
\item[]{\it show}: True: if graphicsData list is empty, node is shown, otherwise body is shown; False: nothing is shown
\item[]{\it create2D}: if True, create NodeRigidBody2D and ObjectRigidBody2D
\item[]{\it returnDict}: if False, returns object index; if True, returns dict of all information on created object and node
\end{itemize}
\item[--]
{\bf output}: Union[dict, ObjectIndex]; returns rigid body object index (or dict with 'nodeNumber', 'objectNumber' and possibly 'loadNumber' and 'markerBodyMass' if returnDict=True)
\item[--]
{\bf example}: \vspace{-12pt}\ei\begin{lstlisting}[language=Python, xleftmargin=36pt]
  import exudyn as exu
  from exudyn.utilities import * #includes itemInterface and rigidBodyUtilities
  import numpy as np
  SC = exu.SystemContainer()
  mbs = SC.AddSystem()
  b0 = mbs.CreateRigidBody(inertia = InertiaCuboid(density=5000,
                                                   sideLengths=[1,0.1,0.1]),
                           referencePosition = [1,0,0],
                           initialVelocity = [2,5,0],
                           initialAngularVelocity = [5,0.5,0.7],
                           gravity = [0,-9.81,0],
                           graphicsDataList = [exu.graphics.Brick(size=[1,0.1,0.1],
                                                                        color=exu.graphics.color.red)])
  mbs.Assemble()
  simulationSettings = exu.SimulationSettings() #takes currently set values or default values
  simulationSettings.timeIntegration.numberOfSteps = 1000
  simulationSettings.timeIntegration.endTime = 2
  mbs.SolveDynamic(simulationSettings = simulationSettings)
\end{lstlisting}\vspace{-24pt}\bi\item[]\vspace{-24pt}\vspace{12pt}\end{itemize}
%

%
\noindent For examples on CreateRigidBody see Relevant Examples (Ex) and TestModels (TM) with weblink to github:
\bi
 \item \footnotesize \exuUrl{https://github.com/jgerstmayr/EXUDYN/blob/master/main/pythonDev/Examples/addPrismaticJoint.py}{\texttt{addPrismaticJoint.py}} (Ex), 
\exuUrl{https://github.com/jgerstmayr/EXUDYN/blob/master/main/pythonDev/Examples/addRevoluteJoint.py}{\texttt{addRevoluteJoint.py}} (Ex), 
\exuUrl{https://github.com/jgerstmayr/EXUDYN/blob/master/main/pythonDev/Examples/ANCFrotatingCable2D.py}{\texttt{ANCFrotatingCable2D.py}} (Ex), 
\\ \exuUrl{https://github.com/jgerstmayr/EXUDYN/blob/master/main/pythonDev/Examples/bicycleIftommBenchmark.py}{\texttt{bicycleIftommBenchmark.py}} (Ex), 
\exuUrl{https://github.com/jgerstmayr/EXUDYN/blob/master/main/pythonDev/Examples/bungeeJump.py}{\texttt{bungeeJump.py}} (Ex), 
 ...
, 
\exuUrl{https://github.com/jgerstmayr/EXUDYN/blob/master/main/pythonDev/TestModels/bricardMechanism.py}{\texttt{bricardMechanism.py}} (TM), 
\\ \exuUrl{https://github.com/jgerstmayr/EXUDYN/blob/master/main/pythonDev/TestModels/carRollingDiscTest.py}{\texttt{carRollingDiscTest.py}} (TM), 
\exuUrl{https://github.com/jgerstmayr/EXUDYN/blob/master/main/pythonDev/TestModels/complexEigenvaluesTest.py}{\texttt{complexEigenvaluesTest.py}} (TM), 
 ...

\ei

%
\begin{flushleft}
\noindent {def {\bf \exuUrl{https://github.com/jgerstmayr/EXUDYN/blob/master/main/pythonDev/exudyn/mainSystemExtensions.py\#L575}{CreateSpringDamper}{}}}\label{sec:mainsystemextensions:CreateSpringDamper}
({\it name}= '', {\it bodyNumbers}= [None, None], {\it localPosition0}= [0.,0.,0.], {\it localPosition1}= [0.,0.,0.], {\it referenceLength}= None, {\it stiffness}= 0., {\it damping}= 0., {\it force}= 0., {\it velocityOffset}= 0., {\it springForceUserFunction}= 0, {\it bodyOrNodeList}= [None, None], {\it bodyList}= [None, None], {\it show}= True, {\it drawSize}= -1, {\it color}= exudyn.graphics.color.default)
\end{flushleft}
\setlength{\itemindent}{0.7cm}
\begin{itemize}[leftmargin=0.7cm]
\item[--]
{\bf function description}: \vspace{-6pt}
\begin{itemize}[leftmargin=1.2cm]
\setlength{\itemindent}{-0.7cm}
\item[]helper function to create SpringDamper connector, using arguments from ObjectConnectorSpringDamper; similar interface as CreateDistanceConstraint(...), see there for for further information
\item[]- NOTE that this function is added to MainSystem via Python function MainSystemCreateSpringDamper.
\end{itemize}
\item[--]
{\bf input}: \vspace{-6pt}
\begin{itemize}[leftmargin=1.2cm]
\setlength{\itemindent}{-0.7cm}
\item[]{\it name}: name string for connector; markers get Marker0:name and Marker1:name
\item[]{\it bodyNumbers}: a list of two body numbers (ObjectIndex) to be connected
\item[]{\it localPosition0}: local position (as 3D list or numpy array) on body0, if not a node number
\item[]{\it localPosition1}: local position (as 3D list or numpy array) on body1, if not a node number
\item[]{\it referenceLength}: if None, length is computed from reference position of bodies or nodes; if not None, this scalar reference length is used for spring
\item[]{\it stiffness}: scalar stiffness coefficient
\item[]{\it damping}: scalar damping coefficient
\item[]{\it force}: scalar additional force applied
\item[]{\it velocityOffset}: scalar offset: if referenceLength is changed over time, the velocityOffset may be changed accordingly to emulate a reference motion
\item[]{\it springForceUserFunction}: a user function springForceUserFunction(mbs, t, itemNumber, deltaL, deltaL\_t, stiffness, damping, force)->float ; this function replaces the internal connector force computation
\item[]{\it bodyOrNodeList}: alternative to bodyNumbers; a list of object numbers (with specific localPosition0/1) or node numbers; may alse be mixed types; to use this case, set bodyNumbers = [None,None]
\item[]{\it show}: if True, connector visualization is drawn
\item[]{\it drawSize}: general drawing size of connector
\item[]{\it color}: color of connector
\end{itemize}
\item[--]
{\bf output}: ObjectIndex; returns index of newly created object
\item[--]
{\bf example}: \vspace{-12pt}\ei\begin{lstlisting}[language=Python, xleftmargin=36pt]
  import exudyn as exu
  from exudyn.utilities import * #includes itemInterface and rigidBodyUtilities
  import numpy as np
  SC = exu.SystemContainer()
  mbs = SC.AddSystem()
  b0 = mbs.CreateMassPoint(referencePosition = [2,0,0],
                           initialVelocity = [2,5,0],
                           physicsMass = 1, gravity = [0,-9.81,0],
                           drawSize = 0.5, color=exu.graphics.color.blue)
  oGround = mbs.AddObject(ObjectGround())
  #add vertical spring
  oSD = mbs.CreateSpringDamper(bodyNumbers=[oGround, b0],
                               localPosition0=[2,1,0],
                               localPosition1=[0,0,0],
                               stiffness=1e4, damping=1e2,
                               drawSize=0.2)
  mbs.Assemble()
  simulationSettings = exu.SimulationSettings() #takes currently set values or default values
  simulationSettings.timeIntegration.numberOfSteps = 1000
  simulationSettings.timeIntegration.endTime = 2
  SC.visualizationSettings.nodes.drawNodesAsPoint=False
  mbs.SolveDynamic(simulationSettings = simulationSettings)
\end{lstlisting}\vspace{-24pt}\bi\item[]\vspace{-24pt}\vspace{12pt}\end{itemize}
%

%
\noindent For examples on CreateSpringDamper see Relevant Examples (Ex) and TestModels (TM) with weblink to github:
\bi
 \item \footnotesize \exuUrl{https://github.com/jgerstmayr/EXUDYN/blob/master/main/pythonDev/Examples/basicTutorial2024.py}{\texttt{basicTutorial2024.py}} (Ex), 
\exuUrl{https://github.com/jgerstmayr/EXUDYN/blob/master/main/pythonDev/Examples/chatGPTupdate.py}{\texttt{chatGPTupdate.py}} (Ex), 
\exuUrl{https://github.com/jgerstmayr/EXUDYN/blob/master/main/pythonDev/Examples/springDamperTutorialNew.py}{\texttt{springDamperTutorialNew.py}} (Ex), 
\\ \exuUrl{https://github.com/jgerstmayr/EXUDYN/blob/master/main/pythonDev/Examples/springMassFriction.py}{\texttt{springMassFriction.py}} (Ex), 
\exuUrl{https://github.com/jgerstmayr/EXUDYN/blob/master/main/pythonDev/Examples/symbolicUserFunctionMasses.py}{\texttt{symbolicUserFunctionMasses.py}} (Ex), 
 ...
, 
\exuUrl{https://github.com/jgerstmayr/EXUDYN/blob/master/main/pythonDev/TestModels/loadUserFunctionTest.py}{\texttt{loadUserFunctionTest.py}} (TM), 
\\ \exuUrl{https://github.com/jgerstmayr/EXUDYN/blob/master/main/pythonDev/TestModels/mainSystemExtensionsTests.py}{\texttt{mainSystemExtensionsTests.py}} (TM), 
\exuUrl{https://github.com/jgerstmayr/EXUDYN/blob/master/main/pythonDev/TestModels/symbolicUserFunctionTest.py}{\texttt{symbolicUserFunctionTest.py}} (TM), 
 ...

\ei

%
\begin{flushleft}
\noindent {def {\bf \exuUrl{https://github.com/jgerstmayr/EXUDYN/blob/master/main/pythonDev/exudyn/mainSystemExtensions.py\#L709}{CreateCartesianSpringDamper}{}}}\label{sec:mainsystemextensions:CreateCartesianSpringDamper}
({\it name}= '', {\it bodyNumbers}= [None, None], {\it localPosition0}= [0.,0.,0.], {\it localPosition1}= [0.,0.,0.], {\it stiffness}= [0.,0.,0.], {\it damping}= [0.,0.,0.], {\it offset}= [0.,0.,0.], {\it springForceUserFunction}= 0, {\it bodyOrNodeList}= [None, None], {\it bodyList}= [None, None], {\it show}= True, {\it drawSize}= -1, {\it color}= exudyn.graphics.color.default)
\end{flushleft}
\setlength{\itemindent}{0.7cm}
\begin{itemize}[leftmargin=0.7cm]
\item[--]
{\bf function description}: \vspace{-6pt}
\begin{itemize}[leftmargin=1.2cm]
\setlength{\itemindent}{-0.7cm}
\item[]helper function to create CartesianSpringDamper connector, using arguments from ObjectConnectorCartesianSpringDamper
\item[]- NOTE that this function is added to MainSystem via Python function MainSystemCreateCartesianSpringDamper.
\end{itemize}
\item[--]
{\bf input}: \vspace{-6pt}
\begin{itemize}[leftmargin=1.2cm]
\setlength{\itemindent}{-0.7cm}
\item[]{\it name}: name string for connector; markers get Marker0:name and Marker1:name
\item[]{\it bodyNumbers}: a list of two body numbers (ObjectIndex) to be connected
\item[]{\it localPosition0}: local position (as 3D list or numpy array) on body0, if not a node number
\item[]{\it localPosition1}: local position (as 3D list or numpy array) on body1, if not a node number
\item[]{\it stiffness}: stiffness coefficients (as 3D list or numpy array)
\item[]{\it damping}: damping coefficients (as 3D list or numpy array)
\item[]{\it offset}: offset vector (as 3D list or numpy array)
\item[]{\it springForceUserFunction}: a user function springForceUserFunction(mbs, t, itemNumber, displacement, velocity, stiffness, damping, offset)->[float,float,float] ; this function replaces the internal connector force computation
\item[]{\it bodyOrNodeList}: alternative to bodyNumbers; a list of object numbers (with specific localPosition0/1) or node numbers; may alse be mixed types; to use this case, set bodyNumbers = [None,None]
\item[]{\it show}: if True, connector visualization is drawn
\item[]{\it drawSize}: general drawing size of connector
\item[]{\it color}: color of connector
\end{itemize}
\item[--]
{\bf output}: ObjectIndex; returns index of newly created object
\item[--]
{\bf example}: \vspace{-12pt}\ei\begin{lstlisting}[language=Python, xleftmargin=36pt]
  import exudyn as exu
  from exudyn.utilities import * #includes itemInterface and rigidBodyUtilities
  import numpy as np
  SC = exu.SystemContainer()
  mbs = SC.AddSystem()
  b0 = mbs.CreateMassPoint(referencePosition = [7,0,0],
                            physicsMass = 1, gravity = [0,-9.81,0],
                            drawSize = 0.5, color=exu.graphics.color.blue)
  oGround = mbs.AddObject(ObjectGround())
  oSD = mbs.CreateCartesianSpringDamper(bodyNumbers=[oGround, b0],
                                localPosition0=[7.5,1,0],
                                localPosition1=[0,0,0],
                                stiffness=[200,2000,0], damping=[2,20,0],
                                drawSize=0.2)
  mbs.Assemble()
  simulationSettings = exu.SimulationSettings() #takes currently set values or default values
  simulationSettings.timeIntegration.numberOfSteps = 1000
  simulationSettings.timeIntegration.endTime = 2
  SC.visualizationSettings.nodes.drawNodesAsPoint=False
  mbs.SolveDynamic(simulationSettings = simulationSettings)
\end{lstlisting}\vspace{-24pt}\bi\item[]\vspace{-24pt}\vspace{12pt}\end{itemize}
%

%
\noindent For examples on CreateCartesianSpringDamper see Relevant Examples (Ex) and TestModels (TM) with weblink to github:
\bi
 \item \footnotesize \exuUrl{https://github.com/jgerstmayr/EXUDYN/blob/master/main/pythonDev/Examples/cartesianSpringDamper.py}{\texttt{cartesianSpringDamper.py}} (Ex), 
\exuUrl{https://github.com/jgerstmayr/EXUDYN/blob/master/main/pythonDev/Examples/cartesianSpringDamperUserFunction.py}{\texttt{cartesianSpringDamperUserFunction.py}} (Ex), 
\exuUrl{https://github.com/jgerstmayr/EXUDYN/blob/master/main/pythonDev/Examples/chatGPTupdate.py}{\texttt{chatGPTupdate.py}} (Ex), 
\\ \exuUrl{https://github.com/jgerstmayr/EXUDYN/blob/master/main/pythonDev/TestModels/complexEigenvaluesTest.py}{\texttt{complexEigenvaluesTest.py}} (TM), 
\exuUrl{https://github.com/jgerstmayr/EXUDYN/blob/master/main/pythonDev/TestModels/computeODE2AEeigenvaluesTest.py}{\texttt{computeODE2AEeigenvaluesTest.py}} (TM), 
\exuUrl{https://github.com/jgerstmayr/EXUDYN/blob/master/main/pythonDev/TestModels/mainSystemExtensionsTests.py}{\texttt{mainSystemExtensionsTests.py}} (TM), 
\\ \exuUrl{https://github.com/jgerstmayr/EXUDYN/blob/master/main/pythonDev/TestModels/mainSystemUserFunctionsTest.py}{\texttt{mainSystemUserFunctionsTest.py}} (TM)
\ei

%
\begin{flushleft}
\noindent {def {\bf \exuUrl{https://github.com/jgerstmayr/EXUDYN/blob/master/main/pythonDev/exudyn/mainSystemExtensions.py\#L798}{CreateRigidBodySpringDamper}{}}}\label{sec:mainsystemextensions:CreateRigidBodySpringDamper}
({\it name}= '', {\it bodyNumbers}= [None, None], {\it localPosition0}= [0.,0.,0.], {\it localPosition1}= [0.,0.,0.], {\it stiffness}= np.zeros((6,6)), {\it damping}= np.zeros((6,6)), {\it offset}= [0.,0.,0.,0.,0.,0.], {\it rotationMatrixJoint}= np.eye(3), {\it useGlobalFrame}= True, {\it intrinsicFormulation}= True, {\it springForceTorqueUserFunction}= 0, {\it postNewtonStepUserFunction}= 0, {\it bodyOrNodeList}= [None, None], {\it bodyList}= [None, None], {\it show}= True, {\it drawSize}= -1, {\it color}= exudyn.graphics.color.default)
\end{flushleft}
\setlength{\itemindent}{0.7cm}
\begin{itemize}[leftmargin=0.7cm]
\item[--]
{\bf function description}: \vspace{-6pt}
\begin{itemize}[leftmargin=1.2cm]
\setlength{\itemindent}{-0.7cm}
\item[]helper function to create RigidBodySpringDamper connector, using arguments from ObjectConnectorRigidBodySpringDamper, see there for the full documentation
\item[]- NOTE that this function is added to MainSystem via Python function MainSystemCreateRigidBodySpringDamper.
\end{itemize}
\item[--]
{\bf input}: \vspace{-6pt}
\begin{itemize}[leftmargin=1.2cm]
\setlength{\itemindent}{-0.7cm}
\item[]{\it name}: name string for connector; markers get Marker0:name and Marker1:name
\item[]{\it bodyNumbers}: a list of two body numbers (ObjectIndex) to be connected
\item[]{\it localPosition0}: local position (as 3D list or numpy array) on body0, if not a node number
\item[]{\it localPosition1}: local position (as 3D list or numpy array) on body1, if not a node number
\item[]{\it stiffness}: stiffness coefficients (as 6D matrix or numpy array)
\item[]{\it damping}: damping coefficients (as 6D matrix or numpy array)
\item[]{\it offset}: offset vector (as 6D list or numpy array)
\item[]{\it rotationMatrixJoint}: additional rotation matrix; in case  useGlobalFrame=False, it transforms body0/node0 local frame to joint frame; if useGlobalFrame=True, it transforms global frame to joint frame
\item[]{\it useGlobalFrame}: if False, the rotationMatrixJoint is defined in the local coordinate system of body0
\item[]{\it intrinsicFormulation}: if True, uses intrinsic formulation of Maserati and Morandini, which uses matrix logarithm and is independent of order of markers (preferred formulation); otherwise, Tait-Bryan angles are used for computation of torque, see documentation
\item[]{\it springForceTorqueUserFunction}: a user function springForceTorqueUserFunction(mbs, t, itemNumber, displacement, rotation, velocity, angularVelocity, stiffness, damping, rotJ0, rotJ1, offset)->[float,float,float, float,float,float] ; this function replaces the internal connector force / torque computation
\item[]{\it postNewtonStepUserFunction}: a special user function postNewtonStepUserFunction(mbs, t, Index itemIndex, dataCoordinates, displacement, rotation, velocity, angularVelocity, stiffness, damping, rotJ0, rotJ1, offset)->[PNerror, recommendedStepSize, data[0], data[1], ...] ; for details, see RigidBodySpringDamper for full docu
\item[]{\it bodyOrNodeList}: alternative to bodyNumbers; a list of object numbers (with specific localPosition0/1) or node numbers; may alse be mixed types; to use this case, set bodyNumbers = [None,None]
\item[]{\it show}: if True, connector visualization is drawn
\item[]{\it drawSize}: general drawing size of connector
\item[]{\it color}: color of connector
\end{itemize}
\item[--]
{\bf output}: ObjectIndex; returns index of newly created object
\item[--]
{\bf example}: \vspace{-12pt}\ei\begin{lstlisting}[language=Python, xleftmargin=36pt]
  #coming later
\end{lstlisting}\vspace{-24pt}\bi\item[]\vspace{-24pt}\vspace{12pt}\end{itemize}
%

%
\noindent For examples on CreateRigidBodySpringDamper see Relevant Examples (Ex) and TestModels (TM) with weblink to github:
\bi
 \item \footnotesize \exuUrl{https://github.com/jgerstmayr/EXUDYN/blob/master/main/pythonDev/TestModels/bricardMechanism.py}{\texttt{bricardMechanism.py}} (TM), 
\exuUrl{https://github.com/jgerstmayr/EXUDYN/blob/master/main/pythonDev/TestModels/rigidBodySpringDamperIntrinsic.py}{\texttt{rigidBodySpringDamperIntrinsic.py}} (TM)
\ei

%
\begin{flushleft}
\noindent {def {\bf \exuUrl{https://github.com/jgerstmayr/EXUDYN/blob/master/main/pythonDev/exudyn/mainSystemExtensions.py\#L930}{CreateTorsionalSpringDamper}{}}}\label{sec:mainsystemextensions:CreateTorsionalSpringDamper}
({\it name}= '', {\it bodyNumbers}= [None, None], {\it position}= [0.,0.,0.], {\it axis}= [0.,0.,0.], {\it stiffness}= 0., {\it damping}= 0., {\it offset}= 0., {\it velocityOffset}= 0., {\it torque}= 0., {\it useGlobalFrame}= True, {\it springTorqueUserFunction}= 0, {\it unlimitedRotations}= True, {\it show}= True, {\it drawSize}= -1, {\it color}= exudyn.graphics.color.default)
\end{flushleft}
\setlength{\itemindent}{0.7cm}
\begin{itemize}[leftmargin=0.7cm]
\item[--]
{\bf function description}: \vspace{-6pt}
\begin{itemize}[leftmargin=1.2cm]
\setlength{\itemindent}{-0.7cm}
\item[]helper function to create TorsionalSpringDamper connector, using arguments from ObjectConnectorTorsionalSpringDamper, see there for the full documentation
\item[]- NOTE that this function is added to MainSystem via Python function MainSystemCreateTorsionalSpringDamper.
\end{itemize}
\item[--]
{\bf input}: \vspace{-6pt}
\begin{itemize}[leftmargin=1.2cm]
\setlength{\itemindent}{-0.7cm}
\item[]{\it name}: name string for connector; markers get Marker0:name and Marker1:name
\item[]{\it bodyNumbers}: a list of two body numbers (ObjectIndex) to be connected
\item[]{\it position}: a 3D vector as list or np.array: if useGlobalFrame=True it describes the global position of the joint in reference configuration; else: local position in body0
\item[]{\it axis}: a 3D vector as list or np.array containing the axis around which the spring acts, either in local body0 coordinates (useGlobalFrame=False), or in global reference configuration (useGlobalFrame=True)
\item[]{\it stiffness}: scalar stiffness of spring
\item[]{\it damping}: scalar damping added to spring
\item[]{\it offset}: scalar offset, which can be used to realize a P-controlled actuator
\item[]{\it velocityOffset}: scalar velocity offset, which can be used to realize a D-controlled actuator
\item[]{\it torque}: additional constant torque added to spring-damper, acting between the two bodies
\item[]{\it useGlobalFrame}: if False, the position and axis vectors are defined in the local coordinate system of body0, otherwise in global (reference) coordinates
\item[]springTorqueUserFunction : a user function springTorqueUserFunction(mbs, t, itemNumber, rotation, angularVelocity, stiffness, damping, offset)->float ; this function replaces the internal connector torque computation
\item[]{\it unlimitedRotations}: if True, an additional generic data node is added to enable measurement of rotations beyond +/- pi; this also allows the spring to cope with multiple turns.
\item[]{\it show}: if True, connector visualization is drawn
\item[]{\it drawSize}: general drawing size of connector
\item[]{\it color}: color of connector
\end{itemize}
\item[--]
{\bf output}: ObjectIndex; returns index of newly created object
\item[--]
{\bf example}: \vspace{-12pt}\ei\begin{lstlisting}[language=Python, xleftmargin=36pt]
  #coming later
\end{lstlisting}\vspace{-24pt}\bi\item[]\vspace{-24pt}\vspace{12pt}\end{itemize}
%

\begin{flushleft}
\noindent {def {\bf \exuUrl{https://github.com/jgerstmayr/EXUDYN/blob/master/main/pythonDev/exudyn/mainSystemExtensions.py\#L1089}{CreateRevoluteJoint}{}}}\label{sec:mainsystemextensions:CreateRevoluteJoint}
({\it name}= '', {\it bodyNumbers}= [None, None], {\it position}= [], {\it axis}= [], {\it useGlobalFrame}= True, {\it show}= True, {\it axisRadius}= 0.1, {\it axisLength}= 0.4, {\it color}= exudyn.graphics.color.default)
\end{flushleft}
\setlength{\itemindent}{0.7cm}
\begin{itemize}[leftmargin=0.7cm]
\item[--]
{\bf function description}: \vspace{-6pt}
\begin{itemize}[leftmargin=1.2cm]
\setlength{\itemindent}{-0.7cm}
\item[]Create revolute joint between two bodies; definition of joint position and axis in global coordinates (alternatively in body0 local coordinates) for reference configuration of bodies; all markers, markerRotation and other quantities are automatically computed
\item[]- NOTE that this function is added to MainSystem via Python function MainSystemCreateRevoluteJoint.
\end{itemize}
\item[--]
{\bf input}: \vspace{-6pt}
\begin{itemize}[leftmargin=1.2cm]
\setlength{\itemindent}{-0.7cm}
\item[]{\it name}: name string for joint; markers get Marker0:name and Marker1:name
\item[]{\it bodyNumbers}: a list of object numbers for body0 and body1; must be rigid body or ground object
\item[]{\it position}: a 3D vector as list or np.array: if useGlobalFrame=True it describes the global position of the joint in reference configuration; else: local position in body0
\item[]{\it axis}: a 3D vector as list or np.array containing the joint axis either in local body0 coordinates (useGlobalFrame=False), or in global reference configuration (useGlobalFrame=True)
\item[]{\it useGlobalFrame}: if False, the position and axis vectors are defined in the local coordinate system of body0, otherwise in global (reference) coordinates
\item[]{\it show}: if True, connector visualization is drawn
\item[]{\it axisRadius}: radius of axis for connector graphical representation
\item[]{\it axisLength}: length of axis for connector graphical representation
\item[]{\it color}: color of connector
\end{itemize}
\item[--]
{\bf output}: ObjectIndex; returns index of created joint
\item[--]
{\bf example}: \vspace{-12pt}\ei\begin{lstlisting}[language=Python, xleftmargin=36pt]
  import exudyn as exu
  from exudyn.utilities import * #includes itemInterface and rigidBodyUtilities
  import numpy as np
  SC = exu.SystemContainer()
  mbs = SC.AddSystem()
  b0 = mbs.CreateRigidBody(inertia = InertiaCuboid(density=5000,
                                                   sideLengths=[1,0.1,0.1]),
                           referencePosition = [3,0,0],
                           gravity = [0,-9.81,0],
                           graphicsDataList = [exu.graphics.Brick(size=[1,0.1,0.1],
                                                                        color=exu.graphics.color.steelblue)])
  oGround = mbs.AddObject(ObjectGround())
  mbs.CreateRevoluteJoint(bodyNumbers=[oGround, b0], position=[2.5,0,0], axis=[0,0,1],
                          useGlobalFrame=True, axisRadius=0.02, axisLength=0.14)
  mbs.Assemble()
  simulationSettings = exu.SimulationSettings() #takes currently set values or default values
  simulationSettings.timeIntegration.numberOfSteps = 1000
  simulationSettings.timeIntegration.endTime = 2
  mbs.SolveDynamic(simulationSettings = simulationSettings)
\end{lstlisting}\vspace{-24pt}\bi\item[]\vspace{-24pt}\vspace{12pt}\end{itemize}
%

%
\noindent For examples on CreateRevoluteJoint see Relevant Examples (Ex) and TestModels (TM) with weblink to github:
\bi
 \item \footnotesize \exuUrl{https://github.com/jgerstmayr/EXUDYN/blob/master/main/pythonDev/Examples/addRevoluteJoint.py}{\texttt{addRevoluteJoint.py}} (Ex), 
\exuUrl{https://github.com/jgerstmayr/EXUDYN/blob/master/main/pythonDev/Examples/bicycleIftommBenchmark.py}{\texttt{bicycleIftommBenchmark.py}} (Ex), 
\exuUrl{https://github.com/jgerstmayr/EXUDYN/blob/master/main/pythonDev/Examples/chatGPTupdate.py}{\texttt{chatGPTupdate.py}} (Ex), 
\\ \exuUrl{https://github.com/jgerstmayr/EXUDYN/blob/master/main/pythonDev/Examples/chatGPTupdate2.py}{\texttt{chatGPTupdate2.py}} (Ex), 
\exuUrl{https://github.com/jgerstmayr/EXUDYN/blob/master/main/pythonDev/Examples/multiMbsTest.py}{\texttt{multiMbsTest.py}} (Ex), 
 ...
, 
\exuUrl{https://github.com/jgerstmayr/EXUDYN/blob/master/main/pythonDev/TestModels/bricardMechanism.py}{\texttt{bricardMechanism.py}} (TM), 
\\ \exuUrl{https://github.com/jgerstmayr/EXUDYN/blob/master/main/pythonDev/TestModels/createRollingDiscPenaltyTest.py}{\texttt{createRollingDiscPenaltyTest.py}} (TM), 
\exuUrl{https://github.com/jgerstmayr/EXUDYN/blob/master/main/pythonDev/TestModels/mainSystemExtensionsTests.py}{\texttt{mainSystemExtensionsTests.py}} (TM), 
 ...

\ei

%
\begin{flushleft}
\noindent {def {\bf \exuUrl{https://github.com/jgerstmayr/EXUDYN/blob/master/main/pythonDev/exudyn/mainSystemExtensions.py\#L1191}{CreatePrismaticJoint}{}}}\label{sec:mainsystemextensions:CreatePrismaticJoint}
({\it name}= '', {\it bodyNumbers}= [None, None], {\it position}= [], {\it axis}= [], {\it useGlobalFrame}= True, {\it show}= True, {\it axisRadius}= 0.1, {\it axisLength}= 0.4, {\it color}= exudyn.graphics.color.default)
\end{flushleft}
\setlength{\itemindent}{0.7cm}
\begin{itemize}[leftmargin=0.7cm]
\item[--]
{\bf function description}: \vspace{-6pt}
\begin{itemize}[leftmargin=1.2cm]
\setlength{\itemindent}{-0.7cm}
\item[]Create prismatic joint between two bodies; definition of joint position and axis in global coordinates (alternatively in body0 local coordinates) for reference configuration of bodies; all markers, markerRotation and other quantities are automatically computed
\item[]- NOTE that this function is added to MainSystem via Python function MainSystemCreatePrismaticJoint.
\end{itemize}
\item[--]
{\bf input}: \vspace{-6pt}
\begin{itemize}[leftmargin=1.2cm]
\setlength{\itemindent}{-0.7cm}
\item[]{\it name}: name string for joint; markers get Marker0:name and Marker1:name
\item[]{\it bodyNumbers}: a list of object numbers for body0 and body1; must be rigid body or ground object
\item[]{\it position}: a 3D vector as list or np.array: if useGlobalFrame=True it describes the global position of the joint in reference configuration; else: local position in body0
\item[]{\it axis}: a 3D vector as list or np.array containing the joint axis either in local body0 coordinates (useGlobalFrame=False), or in global reference configuration (useGlobalFrame=True)
\item[]{\it useGlobalFrame}: if False, the position and axis vectors are defined in the local coordinate system of body0, otherwise in global (reference) coordinates
\item[]{\it show}: if True, connector visualization is drawn
\item[]{\it axisRadius}: radius of axis for connector graphical representation
\item[]{\it axisLength}: length of axis for connector graphical representation
\item[]{\it color}: color of connector
\end{itemize}
\item[--]
{\bf output}: ObjectIndex; returns index of created joint
\item[--]
{\bf example}: \vspace{-12pt}\ei\begin{lstlisting}[language=Python, xleftmargin=36pt]
  import exudyn as exu
  from exudyn.utilities import * #includes itemInterface and rigidBodyUtilities
  import numpy as np
  SC = exu.SystemContainer()
  mbs = SC.AddSystem()
  b0 = mbs.CreateRigidBody(inertia = InertiaCuboid(density=5000,
                                                   sideLengths=[1,0.1,0.1]),
                           referencePosition = [4,0,0],
                           initialVelocity = [0,4,0],
                           gravity = [0,-9.81,0],
                           graphicsDataList = [exu.graphics.Brick(size=[1,0.1,0.1],
                                                                        color=exu.graphics.color.steelblue)])
  oGround = mbs.AddObject(ObjectGround())
  mbs.CreatePrismaticJoint(bodyNumbers=[oGround, b0], position=[3.5,0,0], axis=[0,1,0],
                           useGlobalFrame=True, axisRadius=0.02, axisLength=1)
  mbs.Assemble()
  simulationSettings = exu.SimulationSettings() #takes currently set values or default values
  simulationSettings.timeIntegration.numberOfSteps = 1000
  simulationSettings.timeIntegration.endTime = 2
  mbs.SolveDynamic(simulationSettings = simulationSettings)
\end{lstlisting}\vspace{-24pt}\bi\item[]\vspace{-24pt}\vspace{12pt}\end{itemize}
%

%
\noindent For examples on CreatePrismaticJoint see Relevant Examples (Ex) and TestModels (TM) with weblink to github:
\bi
 \item \footnotesize \exuUrl{https://github.com/jgerstmayr/EXUDYN/blob/master/main/pythonDev/Examples/addPrismaticJoint.py}{\texttt{addPrismaticJoint.py}} (Ex), 
\exuUrl{https://github.com/jgerstmayr/EXUDYN/blob/master/main/pythonDev/Examples/chatGPTupdate.py}{\texttt{chatGPTupdate.py}} (Ex), 
\exuUrl{https://github.com/jgerstmayr/EXUDYN/blob/master/main/pythonDev/Examples/chatGPTupdate2.py}{\texttt{chatGPTupdate2.py}} (Ex), 
\\ \exuUrl{https://github.com/jgerstmayr/EXUDYN/blob/master/main/pythonDev/TestModels/mainSystemExtensionsTests.py}{\texttt{mainSystemExtensionsTests.py}} (TM), 
\exuUrl{https://github.com/jgerstmayr/EXUDYN/blob/master/main/pythonDev/TestModels/pickleCopyMbs.py}{\texttt{pickleCopyMbs.py}} (TM)
\ei

%
\begin{flushleft}
\noindent {def {\bf \exuUrl{https://github.com/jgerstmayr/EXUDYN/blob/master/main/pythonDev/exudyn/mainSystemExtensions.py\#L1286}{CreateSphericalJoint}{}}}\label{sec:mainsystemextensions:CreateSphericalJoint}
({\it name}= '', {\it bodyNumbers}= [None, None], {\it position}= [], {\it constrainedAxes}= [1,1,1], {\it useGlobalFrame}= True, {\it show}= True, {\it jointRadius}= 0.1, {\it color}= exudyn.graphics.color.default)
\end{flushleft}
\setlength{\itemindent}{0.7cm}
\begin{itemize}[leftmargin=0.7cm]
\item[--]
{\bf function description}: \vspace{-6pt}
\begin{itemize}[leftmargin=1.2cm]
\setlength{\itemindent}{-0.7cm}
\item[]Create spherical joint between two bodies; definition of joint position in global coordinates (alternatively in body0 local coordinates) for reference configuration of bodies; all markers are automatically computed
\item[]- NOTE that this function is added to MainSystem via Python function MainSystemCreateSphericalJoint.
\end{itemize}
\item[--]
{\bf input}: \vspace{-6pt}
\begin{itemize}[leftmargin=1.2cm]
\setlength{\itemindent}{-0.7cm}
\item[]{\it name}: name string for joint; markers get Marker0:name and Marker1:name
\item[]{\it bodyNumbers}: a list of object numbers for body0 and body1; must be mass point, rigid body or ground object
\item[]{\it position}: a 3D vector as list or np.array: if useGlobalFrame=True it describes the global position of the joint in reference configuration; else: local position in body0
\item[]{\it constrainedAxes}: flags, which determines which (global) translation axes are constrained; each entry may only be 0 (=free) axis or 1 (=constrained axis)
\item[]{\it useGlobalFrame}: if False, the point and axis vectors are defined in the local coordinate system of body0
\item[]{\it show}: if True, connector visualization is drawn
\item[]{\it jointRadius}: radius of sphere for connector graphical representation
\item[]{\it color}: color of connector
\end{itemize}
\item[--]
{\bf output}: ObjectIndex; returns index of created joint
\item[--]
{\bf example}: \vspace{-12pt}\ei\begin{lstlisting}[language=Python, xleftmargin=36pt]
  import exudyn as exu
  from exudyn.utilities import * #includes itemInterface and rigidBodyUtilities
  import numpy as np
  SC = exu.SystemContainer()
  mbs = SC.AddSystem()
  b0 = mbs.CreateRigidBody(inertia = InertiaCuboid(density=5000,
                                                   sideLengths=[1,0.1,0.1]),
                           referencePosition = [5,0,0],
                           initialAngularVelocity = [5,0,0],
                           gravity = [0,-9.81,0],
                           graphicsDataList = [exu.graphics.Brick(size=[1,0.1,0.1],
                                                                        color=exu.graphics.color.orange)])
  oGround = mbs.AddObject(ObjectGround())
  mbs.CreateSphericalJoint(bodyNumbers=[oGround, b0], position=[5.5,0,0],
                           useGlobalFrame=True, jointRadius=0.06)
  mbs.Assemble()
  simulationSettings = exu.SimulationSettings() #takes currently set values or default values
  simulationSettings.timeIntegration.numberOfSteps = 1000
  simulationSettings.timeIntegration.endTime = 2
  mbs.SolveDynamic(simulationSettings = simulationSettings)
\end{lstlisting}\vspace{-24pt}\bi\item[]\vspace{-24pt}\vspace{12pt}\end{itemize}
%

%
\noindent For examples on CreateSphericalJoint see Relevant Examples (Ex) and TestModels (TM) with weblink to github:
\bi
 \item \footnotesize \exuUrl{https://github.com/jgerstmayr/EXUDYN/blob/master/main/pythonDev/TestModels/driveTrainTest.py}{\texttt{driveTrainTest.py}} (TM), 
\exuUrl{https://github.com/jgerstmayr/EXUDYN/blob/master/main/pythonDev/TestModels/mainSystemExtensionsTests.py}{\texttt{mainSystemExtensionsTests.py}} (TM)
\ei

%
\begin{flushleft}
\noindent {def {\bf \exuUrl{https://github.com/jgerstmayr/EXUDYN/blob/master/main/pythonDev/exudyn/mainSystemExtensions.py\#L1376}{CreateGenericJoint}{}}}\label{sec:mainsystemextensions:CreateGenericJoint}
({\it name}= '', {\it bodyNumbers}= [None, None], {\it position}= [], {\it rotationMatrixAxes}= np.eye(3), {\it constrainedAxes}= [1,1,1, 1,1,1], {\it useGlobalFrame}= True, {\it offsetUserFunction}= 0, {\it offsetUserFunction\_t}= 0, {\it show}= True, {\it axesRadius}= 0.1, {\it axesLength}= 0.4, {\it color}= exudyn.graphics.color.default)
\end{flushleft}
\setlength{\itemindent}{0.7cm}
\begin{itemize}[leftmargin=0.7cm]
\item[--]
{\bf function description}: \vspace{-6pt}
\begin{itemize}[leftmargin=1.2cm]
\setlength{\itemindent}{-0.7cm}
\item[]Create generic joint between two bodies; definition of joint position (position) and axes (rotationMatrixAxes) in global coordinates (useGlobalFrame=True) or in local coordinates of body0 (useGlobalFrame=False), where rotationMatrixAxes is an additional rotation to body0; all markers, markerRotation and other quantities are automatically computed
\item[]- NOTE that this function is added to MainSystem via Python function MainSystemCreateGenericJoint.
\end{itemize}
\item[--]
{\bf input}: \vspace{-6pt}
\begin{itemize}[leftmargin=1.2cm]
\setlength{\itemindent}{-0.7cm}
\item[]{\it name}: name string for joint; markers get Marker0:name and Marker1:name
\item[]{\it bodyNumber0}: a object number for body0, must be rigid body or ground object
\item[]{\it bodyNumber1}: a object number for body1, must be rigid body or ground object
\item[]{\it position}: a 3D vector as list or np.array: if useGlobalFrame=True it describes the global position of the joint in reference configuration; else: local position in body0
\item[]{\it rotationMatrixAxes}: rotation matrix which defines orientation of constrainedAxes; if useGlobalFrame, this rotation matrix is global, else the rotation matrix is post-multiplied with the rotation of body0, identical with rotationMarker0 in the joint
\item[]{\it constrainedAxes}: flag, which determines which translation (0,1,2) and rotation (3,4,5) axes are constrained; each entry may only be 0 (=free) axis or 1 (=constrained axis); ALL constrained Axes are defined relative to reference rotation of body0 times rotation0
\item[]{\it useGlobalFrame}: if False, the position is defined in the local coordinate system of body0, otherwise it is defined in global coordinates
\item[]{\it offsetUserFunction}: a user function offsetUserFunction(mbs, t, itemNumber, offsetUserFunctionParameters)->float ; this function replaces the internal (constant) by a user-defined offset. This allows to realize rheonomic joints and allows kinematic simulation
\item[]{\it offsetUserFunction\_t}: a user function offsetUserFunction\_t(mbs, t, itemNumber, offsetUserFunctionParameters)->float ; this function replaces the internal (constant) by a user-defined offset velocity; this function is used instead of offsetUserFunction, if velocityLevel (index2) time integration
\item[]{\it show}: if True, connector visualization is drawn
\item[]{\it axesRadius}: radius of axes for connector graphical representation
\item[]{\it axesLength}: length of axes for connector graphical representation
\item[]{\it color}: color of connector
\end{itemize}
\item[--]
{\bf output}: ObjectIndex; returns index of created joint
\item[--]
{\bf example}: \vspace{-12pt}\ei\begin{lstlisting}[language=Python, xleftmargin=36pt]
  import exudyn as exu
  from exudyn.utilities import * #includes itemInterface and rigidBodyUtilities
  import numpy as np
  SC = exu.SystemContainer()
  mbs = SC.AddSystem()
  b0 = mbs.CreateRigidBody(inertia = InertiaCuboid(density=5000,
                                                   sideLengths=[1,0.1,0.1]),
                           referencePosition = [6,0,0],
                           initialAngularVelocity = [0,8,0],
                           gravity = [0,-9.81,0],
                           graphicsDataList = [exu.graphics.Brick(size=[1,0.1,0.1],
                                                                        color=exu.graphics.color.orange)])
  oGround = mbs.AddObject(ObjectGround())
  mbs.CreateGenericJoint(bodyNumbers=[oGround, b0], position=[5.5,0,0],
                         constrainedAxes=[1,1,1, 1,0,0],
                         rotationMatrixAxes=RotationMatrixX(0.125*pi), #tilt axes
                         useGlobalFrame=True, axesRadius=0.02, axesLength=0.2)
  mbs.Assemble()
  simulationSettings = exu.SimulationSettings() #takes currently set values or default values
  simulationSettings.timeIntegration.numberOfSteps = 1000
  simulationSettings.timeIntegration.endTime = 2
  mbs.SolveDynamic(simulationSettings = simulationSettings)
\end{lstlisting}\vspace{-24pt}\bi\item[]\vspace{-24pt}\vspace{12pt}\end{itemize}
%

%
\noindent For examples on CreateGenericJoint see Relevant Examples (Ex) and TestModels (TM) with weblink to github:
\bi
 \item \footnotesize \exuUrl{https://github.com/jgerstmayr/EXUDYN/blob/master/main/pythonDev/Examples/bungeeJump.py}{\texttt{bungeeJump.py}} (Ex), 
\exuUrl{https://github.com/jgerstmayr/EXUDYN/blob/master/main/pythonDev/Examples/pistonEngine.py}{\texttt{pistonEngine.py}} (Ex), 
\exuUrl{https://github.com/jgerstmayr/EXUDYN/blob/master/main/pythonDev/Examples/universalJoint.py}{\texttt{universalJoint.py}} (Ex), 
\\ \exuUrl{https://github.com/jgerstmayr/EXUDYN/blob/master/main/pythonDev/TestModels/bricardMechanism.py}{\texttt{bricardMechanism.py}} (TM), 
\exuUrl{https://github.com/jgerstmayr/EXUDYN/blob/master/main/pythonDev/TestModels/complexEigenvaluesTest.py}{\texttt{complexEigenvaluesTest.py}} (TM), 
\exuUrl{https://github.com/jgerstmayr/EXUDYN/blob/master/main/pythonDev/TestModels/computeODE2AEeigenvaluesTest.py}{\texttt{computeODE2AEeigenvaluesTest.py}} (TM), 
\\ \exuUrl{https://github.com/jgerstmayr/EXUDYN/blob/master/main/pythonDev/TestModels/driveTrainTest.py}{\texttt{driveTrainTest.py}} (TM), 
\exuUrl{https://github.com/jgerstmayr/EXUDYN/blob/master/main/pythonDev/TestModels/generalContactImplicit2.py}{\texttt{generalContactImplicit2.py}} (TM), 
 ...

\ei

%
\begin{flushleft}
\noindent {def {\bf \exuUrl{https://github.com/jgerstmayr/EXUDYN/blob/master/main/pythonDev/exudyn/mainSystemExtensions.py\#L1490}{CreateDistanceConstraint}{}}}\label{sec:mainsystemextensions:CreateDistanceConstraint}
({\it name}= '', {\it bodyNumbers}= [None, None], {\it localPosition0}= [0.,0.,0.], {\it localPosition1}= [0.,0.,0.], {\it distance}= None, {\it bodyOrNodeList}= [None, None], {\it bodyList}= [None, None], {\it show}= True, {\it drawSize}= -1., {\it color}= exudyn.graphics.color.default)
\end{flushleft}
\setlength{\itemindent}{0.7cm}
\begin{itemize}[leftmargin=0.7cm]
\item[--]
{\bf function description}: \vspace{-6pt}
\begin{itemize}[leftmargin=1.2cm]
\setlength{\itemindent}{-0.7cm}
\item[]Create distance joint between two bodies; definition of joint positions in local coordinates of bodies or nodes; if distance=None, it is computed automatically from reference length; all markers are automatically computed
\item[]- NOTE that this function is added to MainSystem via Python function MainSystemCreateDistanceConstraint.
\end{itemize}
\item[--]
{\bf input}: \vspace{-6pt}
\begin{itemize}[leftmargin=1.2cm]
\setlength{\itemindent}{-0.7cm}
\item[]{\it name}: name string for joint; markers get Marker0:name and Marker1:name
\item[]{\it bodyNumbers}: a list of two body numbers (ObjectIndex) to be constrained
\item[]{\it localPosition0}: local position (as 3D list or numpy array) on body0, if not a node number
\item[]{\it localPosition1}: local position (as 3D list or numpy array) on body1, if not a node number
\item[]{\it distance}: if None, distance is computed from reference position of bodies or nodes; if not None, this distance is prescribed between the two positions; if distance = 0, it will create a SphericalJoint as this case is not possible with a DistanceConstraint
\item[]{\it bodyOrNodeList}: alternative to bodyNumbers; a list of object numbers (with specific localPosition0/1) or node numbers; may alse be mixed types; to use this case, set bodyNumbers = [None,None]
\item[]{\it show}: if True, connector visualization is drawn
\item[]{\it drawSize}: general drawing size of node
\item[]{\it color}: color of connector
\end{itemize}
\item[--]
{\bf output}: ObjectIndex; returns index of created joint
\item[--]
{\bf example}: \vspace{-12pt}\ei\begin{lstlisting}[language=Python, xleftmargin=36pt]
  import exudyn as exu
  from exudyn.utilities import * #includes itemInterface and rigidBodyUtilities
  import numpy as np
  SC = exu.SystemContainer()
  mbs = SC.AddSystem()
  b0 = mbs.CreateRigidBody(inertia = InertiaCuboid(density=5000,
                                                    sideLengths=[1,0.1,0.1]),
                            referencePosition = [6,0,0],
                            gravity = [0,-9.81,0],
                            graphicsDataList = [exu.graphics.Brick(size=[1,0.1,0.1],
                                                                        color=exu.graphics.color.orange)])
  m1 = mbs.CreateMassPoint(referencePosition=[5.5,-1,0],
                           physicsMass=1, drawSize = 0.2)
  n1 = mbs.GetObject(m1)['nodeNumber']
  oGround = mbs.AddObject(ObjectGround())
  mbs.CreateDistanceConstraint(bodyNumbers=[oGround, b0],
                               localPosition0 = [6.5,1,0],
                               localPosition1 = [0.5,0,0],
                               distance=None, #automatically computed
                               drawSize=0.06)
  mbs.CreateDistanceConstraint(bodyOrNodeList=[b0, n1],
                               localPosition0 = [-0.5,0,0],
                               localPosition1 = [0.,0.,0.], #must be [0,0,0] for Node
                               distance=None, #automatically computed
                               drawSize=0.06)
  mbs.Assemble()
  simulationSettings = exu.SimulationSettings() #takes currently set values or default values
  simulationSettings.timeIntegration.numberOfSteps = 1000
  simulationSettings.timeIntegration.endTime = 2
  mbs.SolveDynamic(simulationSettings = simulationSettings)
\end{lstlisting}\vspace{-24pt}\bi\item[]\vspace{-24pt}\vspace{12pt}\end{itemize}
%

%
\noindent For examples on CreateDistanceConstraint see Relevant Examples (Ex) and TestModels (TM) with weblink to github:
\bi
 \item \footnotesize \exuUrl{https://github.com/jgerstmayr/EXUDYN/blob/master/main/pythonDev/Examples/chatGPTupdate.py}{\texttt{chatGPTupdate.py}} (Ex), 
\exuUrl{https://github.com/jgerstmayr/EXUDYN/blob/master/main/pythonDev/Examples/chatGPTupdate2.py}{\texttt{chatGPTupdate2.py}} (Ex), 
\exuUrl{https://github.com/jgerstmayr/EXUDYN/blob/master/main/pythonDev/Examples/newtonsCradle.py}{\texttt{newtonsCradle.py}} (Ex), 
\\ \exuUrl{https://github.com/jgerstmayr/EXUDYN/blob/master/main/pythonDev/TestModels/mainSystemExtensionsTests.py}{\texttt{mainSystemExtensionsTests.py}} (TM), 
\exuUrl{https://github.com/jgerstmayr/EXUDYN/blob/master/main/pythonDev/TestModels/taskmanagerTest.py}{\texttt{taskmanagerTest.py}} (TM)
\ei

%
\begin{flushleft}
\noindent {def {\bf \exuUrl{https://github.com/jgerstmayr/EXUDYN/blob/master/main/pythonDev/exudyn/mainSystemExtensions.py\#L1621}{CreateRollingDisc}{}}}\label{sec:mainsystemextensions:CreateRollingDisc}
({\it name}= '', {\it bodyNumbers}= [None, None], {\it axisPosition}= [], {\it axisVector}= [1,0,0], {\it discRadius}= 0., {\it planePosition}= [0,0,0], {\it planeNormal}= [0,0,1], {\it constrainedAxes}= [1,1,1], {\it show}= True, {\it discWidth}= 0.1, {\it color}= exudyn.graphics.color.default)
\end{flushleft}
\setlength{\itemindent}{0.7cm}
\begin{itemize}[leftmargin=0.7cm]
\item[--]
{\bf function description}: \vspace{-6pt}
\begin{itemize}[leftmargin=1.2cm]
\setlength{\itemindent}{-0.7cm}
\item[]Create an ideal rolling disc joint between wheel rigid body and ground; the disc is infinitely thin and the ground is a perfectly flat plane; the wheel may lift off; definition of joint position and axis in global coordinates (alternatively in wheel (body1) local coordinates) for reference configuration of bodies; all markers and other quantities are automatically computed; some constraint conditions may be deactivated, e.g. to resolve redundancy of constraints for multi-wheel vehicles
\item[]- NOTE that this function is added to MainSystem via Python function MainSystemCreateRollingDisc.
\end{itemize}
\item[--]
{\bf input}: \vspace{-6pt}
\begin{itemize}[leftmargin=1.2cm]
\setlength{\itemindent}{-0.7cm}
\item[]{\it name}: name string for joint; markers get Marker0:name and Marker1:name
\item[]{\it bodyNumbers}: a list of object numbers for body0=ground and body1=wheel; must be rigid body or ground object
\item[]{\it axisPosition}: a 3D vector as list or np.array: position of wheel axis in local body1=wheel coordinates
\item[]{\it axisVector}: a 3D vector as list or np.array containing the joint (=wheel) axis in local body1=wheel coordinates
\item[]{\it discRadius}: radius of the disc
\item[]{\it planePosition}: any 3D position vector of plane in ground object; given as local coordinates in ground object
\item[]{\it planeNormal}: 3D normal vector of the rolling (contact) plane on ground; given as local coordinates in ground object
\item[]{\it constrainedAxes}: [j0,j1,j2] flags, which determine which constraints are active, in which j0 represents the constraint for lateral motion, j1 longitudinal (forward/backward) motion and j2 represents the normal (contact) direction
\item[]{\it show}: if True, connector visualization is drawn
\item[]{\it discWidth}: disc with, only used for drawing
\item[]{\it color}: color of connector
\end{itemize}
\item[--]
{\bf output}: ObjectIndex; returns index of created joint
\item[--]
{\bf example}: \vspace{-12pt}\ei\begin{lstlisting}[language=Python, xleftmargin=36pt]
  import exudyn as exu
  from exudyn.utilities import * #includes itemInterface and rigidBodyUtilities
  import numpy as np
  SC = exu.SystemContainer()
  mbs = SC.AddSystem()
  r = 0.2
  oDisc = mbs.CreateRigidBody(inertia = InertiaCylinder(density=5000, length=0.1, outerRadius=r, axis=0),
                            referencePosition = [1,0,r],
                            initialAngularVelocity = [-3*2*pi,0,0],
                            initialVelocity = [0,r*3*2*pi,0],
                            gravity = [0,0,-9.81],
                            graphicsDataList = [exu.graphics.Cylinder(pAxis = [-0.05,0,0], vAxis = [0.1,0,0], radius = r*0.99,
                                                                      color=exu.graphics.color.blue),
                                                exu.graphics.Basis(length=2*r)])
  oGround = mbs.CreateGround(graphicsDataList=[exu.graphics.CheckerBoard(size=4)])
  mbs.CreateRollingDisc(bodyNumbers=[oGround, oDisc],
                        axisPosition=[0,0,0], axisVector=[1,0,0], #on local wheel frame
                        planePosition = [0,0,0], planeNormal = [0,0,1],  #in ground frame
                        discRadius = r,
                        discWidth=0.01, color=exu.graphics.color.steelblue)
  mbs.Assemble()
  simulationSettings = exu.SimulationSettings()
  simulationSettings.timeIntegration.numberOfSteps = 1000
  simulationSettings.timeIntegration.endTime = 2
  mbs.SolveDynamic(simulationSettings = simulationSettings)
\end{lstlisting}\vspace{-24pt}\bi\item[]\vspace{-24pt}\vspace{12pt}\end{itemize}
%

%
\noindent For examples on CreateRollingDisc see Relevant Examples (Ex) and TestModels (TM) with weblink to github:
\bi
 \item \footnotesize \exuUrl{https://github.com/jgerstmayr/EXUDYN/blob/master/main/pythonDev/TestModels/createRollingDiscTest.py}{\texttt{createRollingDiscTest.py}} (TM)
\ei

%
\begin{flushleft}
\noindent {def {\bf \exuUrl{https://github.com/jgerstmayr/EXUDYN/blob/master/main/pythonDev/exudyn/mainSystemExtensions.py\#L1729}{CreateRollingDiscPenalty}{}}}\label{sec:mainsystemextensions:CreateRollingDiscPenalty}
({\it name}= '', {\it bodyNumbers}= [None, None], {\it axisPosition}= [], {\it axisVector}= [1,0,0], {\it discRadius}= 0., {\it planePosition}= [0,0,0], {\it planeNormal}= [0,0,1], {\it contactStiffness}= 0., {\it contactDamping}= 0., {\it dryFriction}= [0,0], {\it dryFrictionAngle}= 0., {\it dryFrictionProportionalZone}= 0., {\it viscousFriction}= [0,0], {\it rollingFrictionViscous}= 0., {\it useLinearProportionalZone}= False, {\it #activeConnector}= True, {\it show}= True, {\it discWidth}= 0.1, {\it color}= exudyn.graphics.color.default)
\end{flushleft}
\setlength{\itemindent}{0.7cm}
\begin{itemize}[leftmargin=0.7cm]
\item[--]
{\bf function description}: \vspace{-6pt}
\begin{itemize}[leftmargin=1.2cm]
\setlength{\itemindent}{-0.7cm}
\item[]Create penalty-based rolling disc joint between wheel rigid body and ground; the disc is infinitely thin and the ground is a perfectly flat plane; the wheel may lift off; definition of joint position and axis in global coordinates (alternatively in wheel (body1) local coordinates) for reference configuration of bodies; all markers and other quantities are automatically computed
\item[]- NOTE that this function is added to MainSystem via Python function MainSystemCreateRollingDiscPenalty.
\end{itemize}
\item[--]
{\bf input}: \vspace{-6pt}
\begin{itemize}[leftmargin=1.2cm]
\setlength{\itemindent}{-0.7cm}
\item[]{\it name}: name string for joint; markers get Marker0:name and Marker1:name
\item[]{\it bodyNumbers}: a list of object numbers for body0=ground and body1=wheel; must be rigid body or ground object
\item[]{\it axisPosition}: a 3D vector as list or np.array: position of wheel axis in local body1=wheel coordinates
\item[]{\it axisVector}: a 3D vector as list or np.array containing the joint (=wheel) axis in local body1=wheel coordinates
\item[]{\it discRadius}: radius of the disc
\item[]{\it planePosition}: any 3D position vector of plane in ground object; given as local coordinates in ground object
\item[]{\it planeNormal}: 3D normal vector of the rolling (contact) plane on ground; given as local coordinates in ground object
\item[]{\it dryFrictionAngle}: angle (radiant) which defines a rotation of the local tangential coordinates dry friction; this allows to model Mecanum wheels with specified roll angle
\item[]{\it contactStiffness}: normal contact stiffness
\item[]{\it contactDamping}: normal contact damping
\item[]{\it dryFriction}: 2D list of friction parameters; dry friction coefficients in local wheel coordinates, where for dryFrictionAngle=0, the first parameter refers to forward direction and the second parameter to lateral direction
\item[]{\it viscousFriction}: 2D list of viscous friction coefficients [SI:1/(m/s)] in local wheel coordinates; proportional to slipping velocity, leading to increasing slipping friction force for increasing slipping velocity; directions are same as in dryFriction
\item[]{\it dryFrictionProportionalZone}: limit velocity [m/s] up to which the friction is proportional to velocity (for regularization / avoid numerical oscillations)
\item[]{\it rollingFrictionViscous}: rolling friction [SI:1], which acts against the velocity of the trail on ground and leads to a force proportional to the contact normal force;
\item[]{\it useLinearProportionalZone}: if True, a linear proportional zone is used; the linear zone performs better in implicit time integration as the Jacobian has a constant tangent in the sticking case
\item[]{\it show}: if True, connector visualization is drawn
\item[]{\it discWidth}: disc with, only used for drawing
\item[]{\it color}: color of connector
\end{itemize}
\item[--]
{\bf output}: ObjectIndex; returns index of created joint
\item[--]
{\bf example}: \vspace{-12pt}\ei\begin{lstlisting}[language=Python, xleftmargin=36pt]
  import exudyn as exu
  from exudyn.utilities import * #includes itemInterface and rigidBodyUtilities
  import numpy as np
  SC = exu.SystemContainer()
  mbs = SC.AddSystem()
  r = 0.2
  oDisc = mbs.CreateRigidBody(inertia = InertiaCylinder(density=5000, length=0.1, outerRadius=r, axis=0),
                            referencePosition = [1,0,r],
                            initialAngularVelocity = [-3*2*pi,0,0],
                            initialVelocity = [0,r*3*2*pi,0],
                            gravity = [0,0,-9.81],
                            graphicsDataList = [exu.graphics.Cylinder(pAxis = [-0.05,0,0], vAxis = [0.1,0,0], radius = r*0.99,
                                                                      color=exu.graphics.color.blue),
                                                exu.graphics.Basis(length=2*r)])
  oGround = mbs.CreateGround(graphicsDataList=[exu.graphics.CheckerBoard(size=4)])
  mbs.CreateRollingDiscPenalty(bodyNumbers=[oGround, oDisc], axisPosition=[0,0,0], axisVector=[1,0,0],
                                discRadius = r, planePosition = [0,0,0], planeNormal = [0,0,1],
                                dryFriction = [0.2,0.2],
                                contactStiffness = 1e5, contactDamping = 2e3,
                                discWidth=0.01, color=exu.graphics.color.steelblue)
  mbs.Assemble()
  simulationSettings = exu.SimulationSettings()
  simulationSettings.timeIntegration.numberOfSteps = 1000
  simulationSettings.timeIntegration.endTime = 2
  mbs.SolveDynamic(simulationSettings = simulationSettings)
\end{lstlisting}\vspace{-24pt}\bi\item[]\vspace{-24pt}\vspace{12pt}\end{itemize}
%

%
\noindent For examples on CreateRollingDiscPenalty see Relevant Examples (Ex) and TestModels (TM) with weblink to github:
\bi
 \item \footnotesize \exuUrl{https://github.com/jgerstmayr/EXUDYN/blob/master/main/pythonDev/TestModels/createRollingDiscPenaltyTest.py}{\texttt{createRollingDiscPenaltyTest.py}} (TM)
\ei

%
\begin{flushleft}
\noindent {def {\bf \exuUrl{https://github.com/jgerstmayr/EXUDYN/blob/master/main/pythonDev/exudyn/mainSystemExtensions.py\#L1840}{CreateForce}{}}}\label{sec:mainsystemextensions:CreateForce}
({\it name}= '', {\it bodyNumber}= None, {\it loadVector}= [0.,0.,0.], {\it localPosition}= [0.,0.,0.], {\it bodyFixed}= False, {\it loadVectorUserFunction}= 0, {\it show}= True)
\end{flushleft}
\setlength{\itemindent}{0.7cm}
\begin{itemize}[leftmargin=0.7cm]
\item[--]
{\bf function description}: \vspace{-6pt}
\begin{itemize}[leftmargin=1.2cm]
\setlength{\itemindent}{-0.7cm}
\item[]helper function to create force applied to given body
\item[]- NOTE that this function is added to MainSystem via Python function MainSystemCreateForce.
\end{itemize}
\item[--]
{\bf input}: \vspace{-6pt}
\begin{itemize}[leftmargin=1.2cm]
\setlength{\itemindent}{-0.7cm}
\item[]{\it name}: name string for object
\item[]{\it bodyNumber}: body number (ObjectIndex) at which the force is applied to
\item[]{\it loadVector}: force vector (as 3D list or numpy array)
\item[]{\it localPosition}: local position (as 3D list or numpy array) where force is applied
\item[]{\it bodyFixed}: if True, the force is corotated with the body; else, the force is global
\item[]{\it loadVectorUserFunction}: A Python function f(mbs, t, load)->loadVector which defines the time-dependent load and replaces loadVector in every time step; the arg load is the static loadVector
\item[]{\it show}: if True, load is drawn
\end{itemize}
\item[--]
{\bf output}: LoadIndex; returns load index
\item[--]
{\bf example}: \vspace{-12pt}\ei\begin{lstlisting}[language=Python, xleftmargin=36pt]
  import exudyn as exu
  from exudyn.utilities import * #includes itemInterface and rigidBodyUtilities
  import numpy as np
  SC = exu.SystemContainer()
  mbs = SC.AddSystem()
  b0=mbs.CreateMassPoint(referencePosition = [0,0,0],
                         initialVelocity = [2,5,0],
                         physicsMass = 1, gravity = [0,-9.81,0],
                         drawSize = 0.5, color=exu.graphics.color.blue)
  f0=mbs.CreateForce(bodyNumber=b0, loadVector=[100,0,0],
                     localPosition=[0,0,0])
  mbs.Assemble()
  simulationSettings = exu.SimulationSettings() #takes currently set values or default values
  simulationSettings.timeIntegration.numberOfSteps = 1000
  simulationSettings.timeIntegration.endTime = 2
  mbs.SolveDynamic(simulationSettings = simulationSettings)
\end{lstlisting}\vspace{-24pt}\bi\item[]\vspace{-24pt}\vspace{12pt}\end{itemize}
%

%
\noindent For examples on CreateForce see Relevant Examples (Ex) and TestModels (TM) with weblink to github:
\bi
 \item \footnotesize \exuUrl{https://github.com/jgerstmayr/EXUDYN/blob/master/main/pythonDev/Examples/cartesianSpringDamper.py}{\texttt{cartesianSpringDamper.py}} (Ex), 
\exuUrl{https://github.com/jgerstmayr/EXUDYN/blob/master/main/pythonDev/Examples/cartesianSpringDamperUserFunction.py}{\texttt{cartesianSpringDamperUserFunction.py}} (Ex), 
\exuUrl{https://github.com/jgerstmayr/EXUDYN/blob/master/main/pythonDev/Examples/chatGPTupdate.py}{\texttt{chatGPTupdate.py}} (Ex), 
\\ \exuUrl{https://github.com/jgerstmayr/EXUDYN/blob/master/main/pythonDev/Examples/chatGPTupdate2.py}{\texttt{chatGPTupdate2.py}} (Ex), 
\exuUrl{https://github.com/jgerstmayr/EXUDYN/blob/master/main/pythonDev/Examples/rigidBodyTutorial3.py}{\texttt{rigidBodyTutorial3.py}} (Ex), 
 ...
, 
\exuUrl{https://github.com/jgerstmayr/EXUDYN/blob/master/main/pythonDev/TestModels/loadUserFunctionTest.py}{\texttt{loadUserFunctionTest.py}} (TM), 
\\ \exuUrl{https://github.com/jgerstmayr/EXUDYN/blob/master/main/pythonDev/TestModels/mainSystemExtensionsTests.py}{\texttt{mainSystemExtensionsTests.py}} (TM), 
\exuUrl{https://github.com/jgerstmayr/EXUDYN/blob/master/main/pythonDev/TestModels/simulatorCouplingTwoMbs.py}{\texttt{simulatorCouplingTwoMbs.py}} (TM), 
 ...

\ei

%
\begin{flushleft}
\noindent {def {\bf \exuUrl{https://github.com/jgerstmayr/EXUDYN/blob/master/main/pythonDev/exudyn/mainSystemExtensions.py\#L1918}{CreateTorque}{}}}\label{sec:mainsystemextensions:CreateTorque}
({\it name}= '', {\it bodyNumber}= None, {\it loadVector}= [0.,0.,0.], {\it localPosition}= [0.,0.,0.], {\it bodyFixed}= False, {\it loadVectorUserFunction}= 0, {\it show}= True)
\end{flushleft}
\setlength{\itemindent}{0.7cm}
\begin{itemize}[leftmargin=0.7cm]
\item[--]
{\bf function description}: \vspace{-6pt}
\begin{itemize}[leftmargin=1.2cm]
\setlength{\itemindent}{-0.7cm}
\item[]helper function to create torque applied to given body
\item[]- NOTE that this function is added to MainSystem via Python function MainSystemCreateTorque.
\end{itemize}
\item[--]
{\bf input}: \vspace{-6pt}
\begin{itemize}[leftmargin=1.2cm]
\setlength{\itemindent}{-0.7cm}
\item[]{\it name}: name string for object
\item[]{\it bodyNumber}: body number (ObjectIndex) at which the torque is applied to
\item[]{\it loadVector}: torque vector (as 3D list or numpy array)
\item[]{\it localPosition}: local position (as 3D list or numpy array) where torque is applied
\item[]{\it bodyFixed}: if True, the torque is corotated with the body; else, the torque is global
\item[]{\it loadVectorUserFunction}: A Python function f(mbs, t, load)->loadVector which defines the time-dependent load and replaces loadVector in every time step; the arg load is the static loadVector
\item[]{\it show}: if True, load is drawn
\end{itemize}
\item[--]
{\bf output}: LoadIndex; returns load index
\item[--]
{\bf example}: \vspace{-12pt}\ei\begin{lstlisting}[language=Python, xleftmargin=36pt]
  import exudyn as exu
  from exudyn.utilities import * #includes itemInterface and rigidBodyUtilities
  import numpy as np
  SC = exu.SystemContainer()
  mbs = SC.AddSystem()
  b0 = mbs.CreateRigidBody(inertia = InertiaCuboid(density=5000,
                                                   sideLengths=[1,0.1,0.1]),
                           referencePosition = [1,3,0],
                           gravity = [0,-9.81,0],
                           graphicsDataList = [exu.graphics.Brick(size=[1,0.1,0.1],
                                                                        color=exu.graphics.color.red)])
  f0=mbs.CreateTorque(bodyNumber=b0, loadVector=[0,100,0])
  mbs.Assemble()
  simulationSettings = exu.SimulationSettings() #takes currently set values or default values
  simulationSettings.timeIntegration.numberOfSteps = 1000
  simulationSettings.timeIntegration.endTime = 2
  mbs.SolveDynamic(simulationSettings = simulationSettings)
\end{lstlisting}\vspace{-24pt}\bi\item[]\vspace{-24pt}\vspace{12pt}\end{itemize}
%

%
\noindent For examples on CreateTorque see Relevant Examples (Ex) and TestModels (TM) with weblink to github:
\bi
 \item \footnotesize \exuUrl{https://github.com/jgerstmayr/EXUDYN/blob/master/main/pythonDev/Examples/chatGPTupdate.py}{\texttt{chatGPTupdate.py}} (Ex), 
\exuUrl{https://github.com/jgerstmayr/EXUDYN/blob/master/main/pythonDev/Examples/chatGPTupdate2.py}{\texttt{chatGPTupdate2.py}} (Ex), 
\exuUrl{https://github.com/jgerstmayr/EXUDYN/blob/master/main/pythonDev/Examples/rigidBodyTutorial3.py}{\texttt{rigidBodyTutorial3.py}} (Ex), 
\\ \exuUrl{https://github.com/jgerstmayr/EXUDYN/blob/master/main/pythonDev/TestModels/mainSystemExtensionsTests.py}{\texttt{mainSystemExtensionsTests.py}} (TM), 
\exuUrl{https://github.com/jgerstmayr/EXUDYN/blob/master/main/pythonDev/TestModels/pickleCopyMbs.py}{\texttt{pickleCopyMbs.py}} (TM), 
\exuUrl{https://github.com/jgerstmayr/EXUDYN/blob/master/main/pythonDev/TestModels/simulatorCouplingTwoMbs.py}{\texttt{simulatorCouplingTwoMbs.py}} (TM)
\ei

%



%++++++++++++++++++++
\mysubsubsection{MainSystem extensions (general)}
\label{sec:mainsystem:pythonExtensions}
This section represents general extensions to MainSystem, which are direct calls to Python functions, such as PlotSensor or SolveDynamic; these extensions allow a more intuitive interaction with the MainSystem class, see the following example. For activation, import \texttt{exudyn.mainSystemExtensions} or \texttt{exudyn.utilities}

\pythonstyle
\begin{lstlisting}[language=Python, firstnumber=1]

#this example sketches the usage 
#for complete examples see Examples/ or TestModels/ folders
#create some multibody system (mbs) first:
# ... 
#
#compute system degree of freedom: 
mbs.ComputeSystemDegreeOfFreedom(verbose=True)
#
#call solver function directly from mbs:
mbs.SolveDynamic(exu.SimulationSettings())
#
#plot sensor directly from mbs:
mbs.PlotSensor(...)
\end{lstlisting}

\begin{flushleft}
\noindent {def {\bf \exuUrl{https://github.com/jgerstmayr/EXUDYN/blob/master/main/pythonDev/exudyn/interactive.py\#L754}{SolutionViewer}{}}}\label{sec:mainsystemextensions:SolutionViewer}
({\it solution}= [], {\it rowIncrement}= 1, {\it timeout}= 0.04, {\it runOnStart}= True, {\it runMode}= 2, {\it fontSize}= 12, {\it title}= '', {\it checkRenderEngineStopFlag}= True)
\end{flushleft}
\setlength{\itemindent}{0.7cm}
\begin{itemize}[leftmargin=0.7cm]
\item[--]
{\bf function description}: \vspace{-6pt}
\begin{itemize}[leftmargin=1.2cm]
\setlength{\itemindent}{-0.7cm}
\item[]open interactive dialog and visulation (animate) solution loaded with LoadSolutionFile(...); Change slider 'Increment' to change the automatic increment of time frames; Change mode between continuous run, one cycle (fits perfect for animation recording) or 'Static' (to change Solution steps manually with the mouse); update period also lets you change the speed of animation; Press Run / Stop button to start/stop interactive mode (updating of grpahics)
\item[]- NOTE that this function is added to MainSystem via Python function SolutionViewer.
\end{itemize}
\item[--]
{\bf input}: \vspace{-6pt}
\begin{itemize}[leftmargin=1.2cm]
\setlength{\itemindent}{-0.7cm}
\item[]{\it solution}: solution dictionary previously loaded with exudyn.utilities.LoadSolutionFile(...); will be played from first to last row; if solution=='', it tries to load the file coordinatesSolutionFileName as stored in mbs.sys['simulationSettings'], which are the simulationSettings of the previous simulation
\item[]{\it rowIncrement}: can be set larger than 1 in order to skip solution frames: e.g. rowIncrement=10 visualizes every 10th row (frame)
\item[]{\it timeout}: in seconds is used between frames in order to limit the speed of animation; e.g. use timeout=0.04 to achieve approximately 25 frames per second
\item[]{\it runOnStart}: immediately go into 'Run' mode
\item[]{\it runMode}: 0=continuous run, 1=one cycle, 2=static (use slider/mouse to vary time steps)
\item[]{\it fontSize}: define font size for labels in InteractiveDialog
\item[]{\it title}: if empty, it uses default; otherwise define specific title
\item[]{\it checkRenderEngineStopFlag}: if True, stopping renderer (pressing Q or Escape) also causes stopping the interactive dialog
\end{itemize}
\item[--]
{\bf output}: None; updates current visualization state, renders the scene continuously (after pressing button 'Run')
\item[--]
{\bf example}: \vspace{-12pt}\ei\begin{lstlisting}[language=Python, xleftmargin=36pt]
  #HERE, mbs must contain same model as solution stored in coordinatesSolution.txt
  #adjust autoFitScence, otherwise it may lead to unwanted fit to scene
  SC.visualizationSettings.general.autoFitScene = False
  from exudyn.interactive import SolutionViewer #import function
  sol = LoadSolutionFile('coordinatesSolution.txt') #load solution: adjust to your file name
  mbs.SolutionViewer(sol) #call via MainSystem
\end{lstlisting}\vspace{-24pt}\bi\item[]\vspace{-24pt}\vspace{12pt}\end{itemize}
%

%
\noindent For examples on SolutionViewer see Relevant Examples (Ex) and TestModels (TM) with weblink to github:
\bi
 \item \footnotesize \exuUrl{https://github.com/jgerstmayr/EXUDYN/blob/master/main/pythonDev/Examples/addPrismaticJoint.py}{\texttt{addPrismaticJoint.py}} (Ex), 
\exuUrl{https://github.com/jgerstmayr/EXUDYN/blob/master/main/pythonDev/Examples/addRevoluteJoint.py}{\texttt{addRevoluteJoint.py}} (Ex), 
\exuUrl{https://github.com/jgerstmayr/EXUDYN/blob/master/main/pythonDev/Examples/beltDriveALE.py}{\texttt{beltDriveALE.py}} (Ex), 
\\ \exuUrl{https://github.com/jgerstmayr/EXUDYN/blob/master/main/pythonDev/Examples/beltDriveReevingSystem.py}{\texttt{beltDriveReevingSystem.py}} (Ex), 
\exuUrl{https://github.com/jgerstmayr/EXUDYN/blob/master/main/pythonDev/Examples/beltDrivesComparison.py}{\texttt{beltDrivesComparison.py}} (Ex), 
 ...
, 
\exuUrl{https://github.com/jgerstmayr/EXUDYN/blob/master/main/pythonDev/TestModels/ACFtest.py}{\texttt{ACFtest.py}} (TM), 
\\ \exuUrl{https://github.com/jgerstmayr/EXUDYN/blob/master/main/pythonDev/TestModels/ANCFbeltDrive.py}{\texttt{ANCFbeltDrive.py}} (TM), 
\exuUrl{https://github.com/jgerstmayr/EXUDYN/blob/master/main/pythonDev/TestModels/ANCFgeneralContactCircle.py}{\texttt{ANCFgeneralContactCircle.py}} (TM), 
 ...

\ei

%
\begin{flushleft}
\noindent {def {\bf \exuUrl{https://github.com/jgerstmayr/EXUDYN/blob/master/main/pythonDev/exudyn/mainSystemExtensions.py\#L118}{CreateMassPoint}{}}}\label{sec:mainsystemextensions:CreateMassPoint}
({\it name}= '', {\it referenceCoordinates}= [0.,0.,0.], {\it initialCoordinates}= [0.,0.,0.], {\it initialVelocities}= [0.,0.,0.], {\it physicsMass}= 0, {\it gravity}= [0.,0.,0.], {\it graphicsDataList}= [], {\it drawSize}= -1, {\it color}= [-1.,-1.,-1.,-1.], {\it show}= True, {\it create2D}= False, {\it returnDict}= False)
\end{flushleft}
\setlength{\itemindent}{0.7cm}
\begin{itemize}[leftmargin=0.7cm]
\item[--]
{\bf function description}: \vspace{-6pt}
\begin{itemize}[leftmargin=1.2cm]
\setlength{\itemindent}{-0.7cm}
\item[]helper function to create 2D or 3D mass point object and node, using arguments as in NodePoint and MassPoint
\item[]- NOTE that this function is added to MainSystem via Python function MainSystemCreateMassPoint.
\end{itemize}
\item[--]
{\bf input}: \vspace{-6pt}
\begin{itemize}[leftmargin=1.2cm]
\setlength{\itemindent}{-0.7cm}
\item[]{\it name}: name string for object, node is 'Node:'+name
\item[]{\it referenceCoordinates}: reference coordinates for point node (always a 3D vector, no matter if 2D or 3D mass)
\item[]{\it initialCoordinates}: initial displacements for point node (always a 3D vector, no matter if 2D or 3D mass)
\item[]{\it initialVelocities}: initial velocities for point node (always a 3D vector, no matter if 2D or 3D mass)
\item[]{\it physicsMass}: mass of mass point
\item[]{\it gravity}: gravity vevtor applied (always a 3D vector, no matter if 2D or 3D mass)
\item[]{\it graphicsDataList}: list of GraphicsData for optional mass visualization
\item[]{\it drawSize}: general drawing size of node
\item[]{\it color}: color of node
\item[]{\it show}: True: if graphicsData list is empty, node is shown, otherwise body is shown; otherwise, nothing is shown
\item[]{\it create2D}: if False, create NodePoint2D and MassPoint2D
\item[]{\it returnDict}: if False, returns object index; if True, returns dict of all information on created object and node
\end{itemize}
\item[--]
{\bf output}: Union[dict, ObjectIndex]; returns mass point object index or dict with all data on request (if returnDict=True)
\item[--]
{\bf example}: \vspace{-12pt}\ei\begin{lstlisting}[language=Python, xleftmargin=36pt]
  import exudyn as exu
  from exudyn.utilities import * #includes itemInterface, graphicsDataUtilities and rigidBodyUtilities
  import numpy as np
  SC = exu.SystemContainer()
  mbs = SC.AddSystem()
  b0=mbs.CreateMassPoint(referenceCoordinates = [0,0,0],
                         initialVelocities = [2,5,0],
                         physicsMass = 1, gravity = [0,-9.81,0],
                         drawSize = 0.5, color=color4blue)
  mbs.Assemble()
  simulationSettings = exu.SimulationSettings() #takes currently set values or default values
  simulationSettings.timeIntegration.numberOfSteps = 1000
  simulationSettings.timeIntegration.endTime = 2
  mbs.SolveDynamic(simulationSettings = simulationSettings)
\end{lstlisting}\vspace{-24pt}\bi\item[]\vspace{-24pt}\vspace{12pt}\end{itemize}
%

%
\noindent For examples on CreateMassPoint see Relevant Examples (Ex) and TestModels (TM) with weblink to github:
\bi
 \item \footnotesize \exuUrl{https://github.com/jgerstmayr/EXUDYN/blob/master/main/pythonDev/Examples/springDamperTutorialNew.py}{\texttt{springDamperTutorialNew.py}} (Ex), 
\exuUrl{https://github.com/jgerstmayr/EXUDYN/blob/master/main/pythonDev/TestModels/mainSystemExtensionsTests.py}{\texttt{mainSystemExtensionsTests.py}} (TM)
\ei

%
\begin{flushleft}
\noindent {def {\bf \exuUrl{https://github.com/jgerstmayr/EXUDYN/blob/master/main/pythonDev/exudyn/mainSystemExtensions.py\#L247}{CreateRigidBody}{}}}\label{sec:mainsystemextensions:CreateRigidBody}
({\it name}= '', {\it referencePosition}= [0.,0.,0.], {\it referenceRotationMatrix}= np.eye(3), {\it initialVelocity}= [0.,0.,0.], {\it initialAngularVelocity}= [0.,0.,0.], {\it initialDisplacement}= None, {\it initialRotationMatrix}= None, {\it inertia}= None, {\it gravity}= [0.,0.,0.], {\it nodeType}= exudyn.NodeType.RotationEulerParameters, {\it graphicsDataList}= [], {\it drawSize}= -1, {\it color}= [-1.,-1.,-1.,-1.], {\it show}= True, {\it create2D}= False, {\it returnDict}= False)
\end{flushleft}
\setlength{\itemindent}{0.7cm}
\begin{itemize}[leftmargin=0.7cm]
\item[--]
{\bf function description}: \vspace{-6pt}
\begin{itemize}[leftmargin=1.2cm]
\setlength{\itemindent}{-0.7cm}
\item[]helper function to create 3D (or 2D) rigid body object and node; all quantities are global (angular velocity, etc.)
\item[]- NOTE that this function is added to MainSystem via Python function MainSystemCreateRigidBody.
\end{itemize}
\item[--]
{\bf input}: \vspace{-6pt}
\begin{itemize}[leftmargin=1.2cm]
\setlength{\itemindent}{-0.7cm}
\item[]{\it name}: name string for object, node is 'Node:'+name
\item[]{\it referencePosition}: reference position vector for rigid body node (always a 3D vector, no matter if 2D or 3D body)
\item[]{\it referenceRotationMatrix}: reference rotation matrix for rigid body node (always 3D matrix, no matter if 2D or 3D body)
\item[]{\it initialVelocity}: initial translational velocity vector for node (always a 3D vector, no matter if 2D or 3D body)
\item[]{\it initialAngularVelocity}: initial angular velocity vector for node (always a 3D vector, no matter if 2D or 3D body)
\item[]{\it initialDisplacement}: initial translational displacement vector for node (always a 3D vector, no matter if 2D or 3D body); these displacements are deviations from reference position, e.g. for a finite element node [None: unused]
\item[]{\it initialRotationMatrix}: initial rotation provided as matrix (always a 3D matrix, no matter if 2D or 3D body); this rotation is superimposed to reference rotation [None: unused]
\item[]{\it inertia}: an instance of class RigidBodyInertia, see rigidBodyUtilities; may also be from derived class (InertiaCuboid, InertiaMassPoint, InertiaCylinder, ...)
\item[]{\it gravity}: gravity vevtor applied (always a 3D vector, no matter if 2D or 3D mass)
\item[]{\it graphicsDataList}: list of GraphicsData for rigid body visualization; use graphicsDataUtilities function GraphicsData...(...)
\item[]{\it drawSize}: general drawing size of node
\item[]{\it color}: color of node
\item[]{\it show}: True: if graphicsData list is empty, node is shown, otherwise body is shown; False: nothing is shown
\item[]{\it create2D}: if False, create NodePoint2D and MassPoint2D
\item[]{\it returnDict}: if False, returns object index; if True, returns dict of all information on created object and node
\end{itemize}
\item[--]
{\bf output}: Union[dict, ObjectIndex]; returns rigid body object index (or dict with 'nodeNumber', 'objectNumber' and possibly 'loadNumber' and 'markerBodyMass' if returnDict=True)
\item[--]
{\bf example}: \vspace{-12pt}\ei\begin{lstlisting}[language=Python, xleftmargin=36pt]
  import exudyn as exu
  from exudyn.utilities import * #includes itemInterface, graphicsDataUtilities and rigidBodyUtilities
  import numpy as np
  SC = exu.SystemContainer()
  mbs = SC.AddSystem()
  b0 = mbs.CreateRigidBody(inertia = InertiaCuboid(density=5000,
                                                   sideLengths=[1,0.1,0.1]),
                           referencePosition = [1,0,0],
                           initialVelocity = [2,5,0],
                           initialAngularVelocity = [5,0.5,0.7],
                           gravity = [0,-9.81,0],
                           graphicsDataList = [GraphicsDataOrthoCubePoint(size=[1,0.1,0.1],
                                                                        color=color4red)])
  mbs.Assemble()
  simulationSettings = exu.SimulationSettings() #takes currently set values or default values
  simulationSettings.timeIntegration.numberOfSteps = 1000
  simulationSettings.timeIntegration.endTime = 2
  mbs.SolveDynamic(simulationSettings = simulationSettings)
\end{lstlisting}\vspace{-24pt}\bi\item[]\vspace{-24pt}\vspace{12pt}\end{itemize}
%

%
\noindent For examples on CreateRigidBody see Relevant Examples (Ex) and TestModels (TM) with weblink to github:
\bi
 \item \footnotesize \exuUrl{https://github.com/jgerstmayr/EXUDYN/blob/master/main/pythonDev/Examples/addPrismaticJoint.py}{\texttt{addPrismaticJoint.py}} (Ex), 
\exuUrl{https://github.com/jgerstmayr/EXUDYN/blob/master/main/pythonDev/Examples/addRevoluteJoint.py}{\texttt{addRevoluteJoint.py}} (Ex), 
\exuUrl{https://github.com/jgerstmayr/EXUDYN/blob/master/main/pythonDev/Examples/graphicsDataExample.py}{\texttt{graphicsDataExample.py}} (Ex), 
\\ \exuUrl{https://github.com/jgerstmayr/EXUDYN/blob/master/main/pythonDev/Examples/rigidBodyTutorial3.py}{\texttt{rigidBodyTutorial3.py}} (Ex), 
\exuUrl{https://github.com/jgerstmayr/EXUDYN/blob/master/main/pythonDev/Examples/rigidBodyTutorial3withMarkers.py}{\texttt{rigidBodyTutorial3withMarkers.py}} (Ex), 
 ...
, 
\exuUrl{https://github.com/jgerstmayr/EXUDYN/blob/master/main/pythonDev/TestModels/driveTrainTest.py}{\texttt{driveTrainTest.py}} (TM), 
\\ \exuUrl{https://github.com/jgerstmayr/EXUDYN/blob/master/main/pythonDev/TestModels/mainSystemExtensionsTests.py}{\texttt{mainSystemExtensionsTests.py}} (TM), 
\exuUrl{https://github.com/jgerstmayr/EXUDYN/blob/master/main/pythonDev/TestModels/perf3DRigidBodies.py}{\texttt{perf3DRigidBodies.py}} (TM), 
 ...

\ei

%
\begin{flushleft}
\noindent {def {\bf \exuUrl{https://github.com/jgerstmayr/EXUDYN/blob/master/main/pythonDev/exudyn/mainSystemExtensions.py\#L469}{CreateSpringDamper}{}}}\label{sec:mainsystemextensions:CreateSpringDamper}
({\it name}= '', {\it bodyOrNodeList}= [None, None], {\it localPosition0}= [0.,0.,0.], {\it localPosition1}= [0.,0.,0.], {\it referenceLength}= None, {\it stiffness}= 0., {\it damping}= 0., {\it force}= 0., {\it velocityOffset}= 0., {\it show}= True, {\it drawSize}= -1, {\it color}= color4default)
\end{flushleft}
\setlength{\itemindent}{0.7cm}
\begin{itemize}[leftmargin=0.7cm]
\item[--]
{\bf function description}: \vspace{-6pt}
\begin{itemize}[leftmargin=1.2cm]
\setlength{\itemindent}{-0.7cm}
\item[]helper function to create SpringDamper connector, using arguments from ObjectConnectorSpringDamper; similar interface as CreateDistanceConstraint(...)
\item[]- NOTE that this function is added to MainSystem via Python function MainSystemCreateSpringDamper.
\end{itemize}
\item[--]
{\bf input}: \vspace{-6pt}
\begin{itemize}[leftmargin=1.2cm]
\setlength{\itemindent}{-0.7cm}
\item[]{\it name}: name string for connector; markers get Marker0:name and Marker1:name
\item[]{\it bodyOrNodeList}: a list of object numbers (with specific localPosition0/1) or node numbers; may also be of mixed types
\item[]{\it localPosition0}: local position (as 3D list or numpy array) on body0, if not a node number
\item[]{\it localPosition1}: local position (as 3D list or numpy array) on body1, if not a node number
\item[]{\it referenceLength}: if None, length is computed from reference position of bodies or nodes; if not None, this scalar reference length is used for spring
\item[]{\it stiffness}: scalar stiffness coefficient
\item[]{\it damping}: scalar damping coefficient
\item[]{\it force}: scalar additional force applied
\item[]{\it velocityOffset}: scalar offset: if referenceLength is changed over time, the velocityOffset may be changed accordingly to emulate a reference motion
\item[]{\it show}: if True, connector visualization is drawn
\item[]{\it drawSize}: general drawing size of connector
\item[]{\it color}: color of connector
\end{itemize}
\item[--]
{\bf output}: ObjectIndex; returns index of newly created object
\item[--]
{\bf example}: \vspace{-12pt}\ei\begin{lstlisting}[language=Python, xleftmargin=36pt]
  import exudyn as exu
  from exudyn.utilities import * #includes itemInterface, graphicsDataUtilities and rigidBodyUtilities
  import numpy as np
  SC = exu.SystemContainer()
  mbs = SC.AddSystem()
  b0 = mbs.CreateMassPoint(referenceCoordinates = [2,0,0],
                           initialVelocities = [2,5,0],
                           physicsMass = 1, gravity = [0,-9.81,0],
                           drawSize = 0.5, color=color4blue)
  oGround = mbs.AddObject(ObjectGround())
  #add vertical spring
  oSD = mbs.CreateSpringDamper(bodyOrNodeList=[oGround, b0],
                               localPosition0=[2,1,0],
                               localPosition1=[0,0,0],
                               stiffness=1e4, damping=1e2,
                               drawSize=0.2)
  mbs.Assemble()
  simulationSettings = exu.SimulationSettings() #takes currently set values or default values
  simulationSettings.timeIntegration.numberOfSteps = 1000
  simulationSettings.timeIntegration.endTime = 2
  SC.visualizationSettings.nodes.drawNodesAsPoint=False
  mbs.SolveDynamic(simulationSettings = simulationSettings)
\end{lstlisting}\vspace{-24pt}\bi\item[]\vspace{-24pt}\vspace{12pt}\end{itemize}
%

%
\noindent For examples on CreateSpringDamper see Relevant Examples (Ex) and TestModels (TM) with weblink to github:
\bi
 \item \footnotesize \exuUrl{https://github.com/jgerstmayr/EXUDYN/blob/master/main/pythonDev/Examples/springDamperTutorialNew.py}{\texttt{springDamperTutorialNew.py}} (Ex), 
\exuUrl{https://github.com/jgerstmayr/EXUDYN/blob/master/main/pythonDev/TestModels/mainSystemExtensionsTests.py}{\texttt{mainSystemExtensionsTests.py}} (TM)
\ei

%
\begin{flushleft}
\noindent {def {\bf \exuUrl{https://github.com/jgerstmayr/EXUDYN/blob/master/main/pythonDev/exudyn/mainSystemExtensions.py\#L605}{CreateCartesianSpringDamper}{}}}\label{sec:mainsystemextensions:CreateCartesianSpringDamper}
({\it name}= '', {\it bodyOrNodeList}= [None, None], {\it localPosition0}= [0.,0.,0.], {\it localPosition1}= [0.,0.,0.], {\it stiffness}= [0.,0.,0.], {\it damping}= [0.,0.,0.], {\it offset}= [0.,0.,0.], {\it show}= True, {\it drawSize}= -1, {\it color}= color4default)
\end{flushleft}
\setlength{\itemindent}{0.7cm}
\begin{itemize}[leftmargin=0.7cm]
\item[--]
{\bf function description}: \vspace{-6pt}
\begin{itemize}[leftmargin=1.2cm]
\setlength{\itemindent}{-0.7cm}
\item[]helper function to create CartesianSpringDamper connector, using arguments from ObjectConnectorCartesianSpringDamper
\item[]- NOTE that this function is added to MainSystem via Python function MainSystemCreateCartesianSpringDamper.
\end{itemize}
\item[--]
{\bf input}: \vspace{-6pt}
\begin{itemize}[leftmargin=1.2cm]
\setlength{\itemindent}{-0.7cm}
\item[]{\it name}: name string for connector; markers get Marker0:name and Marker1:name
\item[]{\it bodyOrNodeList}: a list of object numbers (with specific localPosition0/1) or node numbers; may also be of mixed types
\item[]{\it localPosition0}: local position (as 3D list or numpy array) on body0, if not a node number
\item[]{\it localPosition1}: local position (as 3D list or numpy array) on body1, if not a node number
\item[]{\it stiffness}: stiffness coefficients (as 3D list or numpy array)
\item[]{\it damping}: damping coefficients (as 3D list or numpy array)
\item[]{\it offset}: offset vector (as 3D list or numpy array)
\item[]{\it show}: if True, connector visualization is drawn
\item[]{\it drawSize}: general drawing size of connector
\item[]{\it color}: color of connector
\end{itemize}
\item[--]
{\bf output}: ObjectIndex; returns index of newly created object
\item[--]
{\bf example}: \vspace{-12pt}\ei\begin{lstlisting}[language=Python, xleftmargin=36pt]
  import exudyn as exu
  from exudyn.utilities import * #includes itemInterface, graphicsDataUtilities and rigidBodyUtilities
  import numpy as np
  SC = exu.SystemContainer()
  mbs = SC.AddSystem()
  b0 = mbs.CreateMassPoint(referenceCoordinates = [7,0,0],
                            physicsMass = 1, gravity = [0,-9.81,0],
                            drawSize = 0.5, color=color4blue)
  oGround = mbs.AddObject(ObjectGround())
  oSD = mbs.CreateCartesianSpringDamper(bodyOrNodeList=[oGround, b0],
                                localPosition0=[7.5,1,0],
                                localPosition1=[0,0,0],
                                stiffness=[200,2000,0], damping=[2,20,0],
                                drawSize=0.2)
  mbs.Assemble()
  simulationSettings = exu.SimulationSettings() #takes currently set values or default values
  simulationSettings.timeIntegration.numberOfSteps = 1000
  simulationSettings.timeIntegration.endTime = 2
  SC.visualizationSettings.nodes.drawNodesAsPoint=False
  mbs.SolveDynamic(simulationSettings = simulationSettings)
\end{lstlisting}\vspace{-24pt}\bi\item[]\vspace{-24pt}\vspace{12pt}\end{itemize}
%

%
\noindent For examples on CreateCartesianSpringDamper see Relevant Examples (Ex) and TestModels (TM) with weblink to github:
\bi
 \item \footnotesize \exuUrl{https://github.com/jgerstmayr/EXUDYN/blob/master/main/pythonDev/TestModels/mainSystemExtensionsTests.py}{\texttt{mainSystemExtensionsTests.py}} (TM)
\ei

%
\begin{flushleft}
\noindent {def {\bf \exuUrl{https://github.com/jgerstmayr/EXUDYN/blob/master/main/pythonDev/exudyn/mainSystemExtensions.py\#L712}{CreateRevoluteJoint}{}}}\label{sec:mainsystemextensions:CreateRevoluteJoint}
({\it name}= '', {\it bodyNumbers}= [None, None], {\it position}= [], {\it axis}= [], {\it useGlobalFrame}= True, {\it show}= True, {\it axisRadius}= 0.1, {\it axisLength}= 0.4, {\it color}= color4default)
\end{flushleft}
\setlength{\itemindent}{0.7cm}
\begin{itemize}[leftmargin=0.7cm]
\item[--]
{\bf function description}: \vspace{-6pt}
\begin{itemize}[leftmargin=1.2cm]
\setlength{\itemindent}{-0.7cm}
\item[]Create revolute joint between two bodies; definition of joint position and axis in global coordinates (alternatively in body0 local coordinates) for reference configuration of bodies; all markers, markerRotation and other quantities are automatically computed
\item[]- NOTE that this function is added to MainSystem via Python function MainSystemCreateRevoluteJoint.
\end{itemize}
\item[--]
{\bf input}: \vspace{-6pt}
\begin{itemize}[leftmargin=1.2cm]
\setlength{\itemindent}{-0.7cm}
\item[]{\it name}: name string for joint; markers get Marker0:name and Marker1:name
\item[]{\it bodyNumbers}: a list of object numbers for body0 and body1; must be rigid body or ground object
\item[]{\it position}: a 3D vector as list or np.array: if useGlobalFrame=True it describes the global position of the joint in reference configuration; else: local position in body0
\item[]{\it axis}: a 3D vector as list or np.array: if  useGlobalFrame=True it describes the global rotation axis of the joint in reference configuration; else: local axis in body0
\item[]{\it useGlobalFrame}: if False, the point and axis vectors are defined in the local coordinate system of body0
\item[]{\it show}: if True, connector visualization is drawn
\item[]{\it axisRadius}: radius of axis for connector graphical representation
\item[]{\it axisLength}: length of axis for connector graphical representation
\item[]{\it color}: color of connector
\end{itemize}
\item[--]
{\bf output}: [ObjectIndex, MarkerIndex, MarkerIndex]; returns list [oJoint, mBody0, mBody1], containing the joint object number, and the two rigid body markers on body0/1 for the joint
\item[--]
{\bf example}: \vspace{-12pt}\ei\begin{lstlisting}[language=Python, xleftmargin=36pt]
  import exudyn as exu
  from exudyn.utilities import * #includes itemInterface, graphicsDataUtilities and rigidBodyUtilities
  import numpy as np
  SC = exu.SystemContainer()
  mbs = SC.AddSystem()
  b0 = mbs.CreateRigidBody(inertia = InertiaCuboid(density=5000,
                                                   sideLengths=[1,0.1,0.1]),
                           referencePosition = [3,0,0],
                           gravity = [0,-9.81,0],
                           graphicsDataList = [GraphicsDataOrthoCubePoint(size=[1,0.1,0.1],
                                                                        color=color4steelblue)])
  oGround = mbs.AddObject(ObjectGround())
  mbs.CreateRevoluteJoint(bodyNumbers=[oGround, b0], position=[2.5,0,0], axis=[0,0,1],
                          useGlobalFrame=True, axisRadius=0.02, axisLength=0.14)
  mbs.Assemble()
  simulationSettings = exu.SimulationSettings() #takes currently set values or default values
  simulationSettings.timeIntegration.numberOfSteps = 1000
  simulationSettings.timeIntegration.endTime = 2
  mbs.SolveDynamic(simulationSettings = simulationSettings)
\end{lstlisting}\vspace{-24pt}\bi\item[]\vspace{-24pt}\vspace{12pt}\end{itemize}
%

%
\noindent For examples on CreateRevoluteJoint see Relevant Examples (Ex) and TestModels (TM) with weblink to github:
\bi
 \item \footnotesize \exuUrl{https://github.com/jgerstmayr/EXUDYN/blob/master/main/pythonDev/Examples/addRevoluteJoint.py}{\texttt{addRevoluteJoint.py}} (Ex), 
\exuUrl{https://github.com/jgerstmayr/EXUDYN/blob/master/main/pythonDev/Examples/rigidBodyTutorial3.py}{\texttt{rigidBodyTutorial3.py}} (Ex), 
\exuUrl{https://github.com/jgerstmayr/EXUDYN/blob/master/main/pythonDev/Examples/solutionViewerTest.py}{\texttt{solutionViewerTest.py}} (Ex), 
\\ \exuUrl{https://github.com/jgerstmayr/EXUDYN/blob/master/main/pythonDev/TestModels/mainSystemExtensionsTests.py}{\texttt{mainSystemExtensionsTests.py}} (TM), 
\exuUrl{https://github.com/jgerstmayr/EXUDYN/blob/master/main/pythonDev/TestModels/perf3DRigidBodies.py}{\texttt{perf3DRigidBodies.py}} (TM)
\ei

%
\begin{flushleft}
\noindent {def {\bf \exuUrl{https://github.com/jgerstmayr/EXUDYN/blob/master/main/pythonDev/exudyn/mainSystemExtensions.py\#L810}{CreatePrismaticJoint}{}}}\label{sec:mainsystemextensions:CreatePrismaticJoint}
({\it name}= '', {\it bodyNumbers}= [None, None], {\it position}= [], {\it axis}= [], {\it useGlobalFrame}= True, {\it show}= True, {\it axisRadius}= 0.1, {\it axisLength}= 0.4, {\it color}= color4default)
\end{flushleft}
\setlength{\itemindent}{0.7cm}
\begin{itemize}[leftmargin=0.7cm]
\item[--]
{\bf function description}: \vspace{-6pt}
\begin{itemize}[leftmargin=1.2cm]
\setlength{\itemindent}{-0.7cm}
\item[]Create prismatic joint between two bodies; definition of joint position and axis in global coordinates (alternatively in body0 local coordinates) for reference configuration of bodies; all markers, markerRotation and other quantities are automatically computed
\item[]- NOTE that this function is added to MainSystem via Python function MainSystemCreatePrismaticJoint.
\end{itemize}
\item[--]
{\bf input}: \vspace{-6pt}
\begin{itemize}[leftmargin=1.2cm]
\setlength{\itemindent}{-0.7cm}
\item[]{\it name}: name string for joint; markers get Marker0:name and Marker1:name
\item[]{\it bodyNumbers}: a list of object numbers for body0 and body1; must be rigid body or ground object
\item[]{\it position}: a 3D vector as list or np.array: if useGlobalFrame=True it describes the global position of the joint in reference configuration; else: local position in body0
\item[]{\it axis}: a 3D vector as list or np.array containing the global translation axis of the joint in reference configuration
\item[]{\it useGlobalFrame}: if False, the point and axis vectors are defined in the local coordinate system of body0
\item[]{\it show}: if True, connector visualization is drawn
\item[]{\it axisRadius}: radius of axis for connector graphical representation
\item[]{\it axisLength}: length of axis for connector graphical representation
\item[]{\it color}: color of connector
\end{itemize}
\item[--]
{\bf output}: [ObjectIndex, MarkerIndex, MarkerIndex]; returns list [oJoint, mBody0, mBody1], containing the joint object number, and the two rigid body markers on body0/1 for the joint
\item[--]
{\bf example}: \vspace{-12pt}\ei\begin{lstlisting}[language=Python, xleftmargin=36pt]
  import exudyn as exu
  from exudyn.utilities import * #includes itemInterface, graphicsDataUtilities and rigidBodyUtilities
  import numpy as np
  SC = exu.SystemContainer()
  mbs = SC.AddSystem()
  b0 = mbs.CreateRigidBody(inertia = InertiaCuboid(density=5000,
                                                   sideLengths=[1,0.1,0.1]),
                           referencePosition = [4,0,0],
                           initialVelocity = [0,4,0],
                           gravity = [0,-9.81,0],
                           graphicsDataList = [GraphicsDataOrthoCubePoint(size=[1,0.1,0.1],
                                                                        color=color4steelblue)])
  oGround = mbs.AddObject(ObjectGround())
  mbs.CreatePrismaticJoint(bodyNumbers=[oGround, b0], position=[3.5,0,0], axis=[0,1,0],
                           useGlobalFrame=True, axisRadius=0.02, axisLength=1)
  mbs.Assemble()
  simulationSettings = exu.SimulationSettings() #takes currently set values or default values
  simulationSettings.timeIntegration.numberOfSteps = 1000
  simulationSettings.timeIntegration.endTime = 2
  mbs.SolveDynamic(simulationSettings = simulationSettings)
\end{lstlisting}\vspace{-24pt}\bi\item[]\vspace{-24pt}\vspace{12pt}\end{itemize}
%

%
\noindent For examples on CreatePrismaticJoint see Relevant Examples (Ex) and TestModels (TM) with weblink to github:
\bi
 \item \footnotesize \exuUrl{https://github.com/jgerstmayr/EXUDYN/blob/master/main/pythonDev/Examples/addPrismaticJoint.py}{\texttt{addPrismaticJoint.py}} (Ex), 
\exuUrl{https://github.com/jgerstmayr/EXUDYN/blob/master/main/pythonDev/TestModels/mainSystemExtensionsTests.py}{\texttt{mainSystemExtensionsTests.py}} (TM)
\ei

%
\begin{flushleft}
\noindent {def {\bf \exuUrl{https://github.com/jgerstmayr/EXUDYN/blob/master/main/pythonDev/exudyn/mainSystemExtensions.py\#L900}{CreateSphericalJoint}{}}}\label{sec:mainsystemextensions:CreateSphericalJoint}
({\it name}= '', {\it bodyNumbers}= [None, None], {\it position}= [], {\it constrainedAxes}= [1,1,1], {\it useGlobalFrame}= True, {\it show}= True, {\it jointRadius}= 0.1, {\it color}= color4default)
\end{flushleft}
\setlength{\itemindent}{0.7cm}
\begin{itemize}[leftmargin=0.7cm]
\item[--]
{\bf function description}: \vspace{-6pt}
\begin{itemize}[leftmargin=1.2cm]
\setlength{\itemindent}{-0.7cm}
\item[]Create spherical joint between two bodies; definition of joint position in global coordinates (alternatively in body0 local coordinates) for reference configuration of bodies; all markers are automatically computed
\item[]- NOTE that this function is added to MainSystem via Python function MainSystemCreateSphericalJoint.
\end{itemize}
\item[--]
{\bf input}: \vspace{-6pt}
\begin{itemize}[leftmargin=1.2cm]
\setlength{\itemindent}{-0.7cm}
\item[]{\it name}: name string for joint; markers get Marker0:name and Marker1:name
\item[]{\it bodyNumbers}: a list of object numbers for body0 and body1; must be mass point, rigid body or ground object
\item[]{\it position}: a 3D vector as list or np.array: if useGlobalFrame=True it describes the global position of the joint in reference configuration; else: local position in body0
\item[]{\it constrainedAxes}: flags, which determines which (global) translation axes are constrained; each entry may only be 0 (=free) axis or 1 (=constrained axis)
\item[]{\it useGlobalFrame}: if False, the point and axis vectors are defined in the local coordinate system of body0
\item[]{\it show}: if True, connector visualization is drawn
\item[]{\it jointRadius}: radius of sphere for connector graphical representation
\item[]{\it color}: color of connector
\end{itemize}
\item[--]
{\bf output}: [ObjectIndex, MarkerIndex, MarkerIndex]; returns list [oJoint, mBody0, mBody1], containing the joint object number, and the two rigid body markers on body0/1 for the joint
\item[--]
{\bf example}: \vspace{-12pt}\ei\begin{lstlisting}[language=Python, xleftmargin=36pt]
  import exudyn as exu
  from exudyn.utilities import * #includes itemInterface, graphicsDataUtilities and rigidBodyUtilities
  import numpy as np
  SC = exu.SystemContainer()
  mbs = SC.AddSystem()
  b0 = mbs.CreateRigidBody(inertia = InertiaCuboid(density=5000,
                                                   sideLengths=[1,0.1,0.1]),
                           referencePosition = [5,0,0],
                           initialAngularVelocity = [5,0,0],
                           gravity = [0,-9.81,0],
                           graphicsDataList = [GraphicsDataOrthoCubePoint(size=[1,0.1,0.1],
                                                                        color=color4orange)])
  oGround = mbs.AddObject(ObjectGround())
  mbs.CreateSphericalJoint(bodyNumbers=[oGround, b0], position=[5.5,0,0],
                           useGlobalFrame=True, jointRadius=0.06)
  mbs.Assemble()
  simulationSettings = exu.SimulationSettings() #takes currently set values or default values
  simulationSettings.timeIntegration.numberOfSteps = 1000
  simulationSettings.timeIntegration.endTime = 2
  mbs.SolveDynamic(simulationSettings = simulationSettings)
\end{lstlisting}\vspace{-24pt}\bi\item[]\vspace{-24pt}\vspace{12pt}\end{itemize}
%

%
\noindent For examples on CreateSphericalJoint see Relevant Examples (Ex) and TestModels (TM) with weblink to github:
\bi
 \item \footnotesize \exuUrl{https://github.com/jgerstmayr/EXUDYN/blob/master/main/pythonDev/TestModels/driveTrainTest.py}{\texttt{driveTrainTest.py}} (TM), 
\exuUrl{https://github.com/jgerstmayr/EXUDYN/blob/master/main/pythonDev/TestModels/mainSystemExtensionsTests.py}{\texttt{mainSystemExtensionsTests.py}} (TM)
\ei

%
\begin{flushleft}
\noindent {def {\bf \exuUrl{https://github.com/jgerstmayr/EXUDYN/blob/master/main/pythonDev/exudyn/mainSystemExtensions.py\#L983}{CreateGenericJoint}{}}}\label{sec:mainsystemextensions:CreateGenericJoint}
({\it name}= '', {\it bodyNumbers}= [None, None], {\it position}= [], {\it rotationMatrixAxes}= np.eye(3), {\it constrainedAxes}= [1,1,1, 1,1,1], {\it useGlobalFrame}= True, {\it show}= True, {\it axesRadius}= 0.1, {\it axesLength}= 0.4, {\it color}= color4default)
\end{flushleft}
\setlength{\itemindent}{0.7cm}
\begin{itemize}[leftmargin=0.7cm]
\item[--]
{\bf function description}: \vspace{-6pt}
\begin{itemize}[leftmargin=1.2cm]
\setlength{\itemindent}{-0.7cm}
\item[]Create generic joint between two bodies; definition of joint position (position) and axes (rotationMatrixAxes) in global coordinates (useGlobalFrame=True) or in local coordinates of body0 (useGlobalFrame=False), where rotationMatrixAxes is an additional rotation to body0; all markers, markerRotation and other quantities are automatically computed
\item[]- NOTE that this function is added to MainSystem via Python function MainSystemCreateGenericJoint.
\end{itemize}
\item[--]
{\bf input}: \vspace{-6pt}
\begin{itemize}[leftmargin=1.2cm]
\setlength{\itemindent}{-0.7cm}
\item[]{\it name}: name string for joint; markers get Marker0:name and Marker1:name
\item[]{\it bodyNumber0}: a object number for body0, must be rigid body or ground object
\item[]{\it bodyNumber1}: a object number for body1, must be rigid body or ground object
\item[]{\it position}: a 3D vector as list or np.array: if useGlobalFrame=True it describes the global position of the joint in reference configuration; else: local position in body0
\item[]{\it rotationMatrixAxes}: rotation matrix which defines orientation of constrainedAxes; if useGlobalFrame, this rotation matrix is global, else the rotation matrix is post-multiplied with the rotation of body0, identical with rotationMarker0 in the joint
\item[]{\it constrainedAxes}: flag, which determines which translation (0,1,2) and rotation (3,4,5) axes are constrained; each entry may only be 0 (=free) axis or 1 (=constrained axis); ALL constrained Axes are defined relative to reference rotation of body0 times rotation0
\item[]{\it useGlobalFrame}: if False, the position is defined in the local coordinate system of body0, otherwise it is defined in global coordinates
\item[]{\it show}: if True, connector visualization is drawn
\item[]{\it axesRadius}: radius of axes for connector graphical representation
\item[]{\it axesLength}: length of axes for connector graphical representation
\item[]{\it color}: color of connector
\end{itemize}
\item[--]
{\bf output}: [ObjectIndex, MarkerIndex, MarkerIndex]; returns list [oJoint, mBody0, mBody1], containing the joint object number, and the two rigid body markers on body0/1 for the joint
\item[--]
{\bf example}: \vspace{-12pt}\ei\begin{lstlisting}[language=Python, xleftmargin=36pt]
  import exudyn as exu
  from exudyn.utilities import * #includes itemInterface, graphicsDataUtilities and rigidBodyUtilities
  import numpy as np
  SC = exu.SystemContainer()
  mbs = SC.AddSystem()
  b0 = mbs.CreateRigidBody(inertia = InertiaCuboid(density=5000,
                                                   sideLengths=[1,0.1,0.1]),
                           referencePosition = [6,0,0],
                           initialAngularVelocity = [0,8,0],
                           gravity = [0,-9.81,0],
                           graphicsDataList = [GraphicsDataOrthoCubePoint(size=[1,0.1,0.1],
                                                                        color=color4orange)])
  oGround = mbs.AddObject(ObjectGround())
  mbs.CreateGenericJoint(bodyNumbers=[oGround, b0], position=[5.5,0,0],
                         constrainedAxes=[1,1,1, 1,0,0],
                         rotationMatrixAxes=RotationMatrixX(0.125*pi), #tilt axes
                         useGlobalFrame=True, axesRadius=0.02, axesLength=0.2)
  mbs.Assemble()
  simulationSettings = exu.SimulationSettings() #takes currently set values or default values
  simulationSettings.timeIntegration.numberOfSteps = 1000
  simulationSettings.timeIntegration.endTime = 2
  mbs.SolveDynamic(simulationSettings = simulationSettings)
\end{lstlisting}\vspace{-24pt}\bi\item[]\vspace{-24pt}\vspace{12pt}\end{itemize}
%

%
\noindent For examples on CreateGenericJoint see Relevant Examples (Ex) and TestModels (TM) with weblink to github:
\bi
 \item \footnotesize \exuUrl{https://github.com/jgerstmayr/EXUDYN/blob/master/main/pythonDev/TestModels/driveTrainTest.py}{\texttt{driveTrainTest.py}} (TM), 
\exuUrl{https://github.com/jgerstmayr/EXUDYN/blob/master/main/pythonDev/TestModels/mainSystemExtensionsTests.py}{\texttt{mainSystemExtensionsTests.py}} (TM), 
\exuUrl{https://github.com/jgerstmayr/EXUDYN/blob/master/main/pythonDev/TestModels/rigidBodyCOMtest.py}{\texttt{rigidBodyCOMtest.py}} (TM)
\ei

%
\begin{flushleft}
\noindent {def {\bf \exuUrl{https://github.com/jgerstmayr/EXUDYN/blob/master/main/pythonDev/exudyn/mainSystemExtensions.py\#L1086}{CreateDistanceConstraint}{}}}\label{sec:mainsystemextensions:CreateDistanceConstraint}
({\it name}= '', {\it bodyOrNodeList}= [None, None], {\it localPosition0}= [0.,0.,0.], {\it localPosition1}= [0.,0.,0.], {\it distance}= None, {\it show}= True, {\it drawSize}= -1., {\it color}= color4default)
\end{flushleft}
\setlength{\itemindent}{0.7cm}
\begin{itemize}[leftmargin=0.7cm]
\item[--]
{\bf function description}: \vspace{-6pt}
\begin{itemize}[leftmargin=1.2cm]
\setlength{\itemindent}{-0.7cm}
\item[]Create distance joint between two bodies; definition of joint positions in local coordinates of bodies or nodes; if distance=None, it is computed automatically from reference length; all markers are automatically computed
\item[]- NOTE that this function is added to MainSystem via Python function MainSystemCreateDistanceConstraint.
\end{itemize}
\item[--]
{\bf input}: \vspace{-6pt}
\begin{itemize}[leftmargin=1.2cm]
\setlength{\itemindent}{-0.7cm}
\item[]{\it name}: name string for joint; markers get Marker0:name and Marker1:name
\item[]{\it bodyOrNodeList}: a list of object numbers (with specific localPosition0/1) or node numbers; may also be of mixed types
\item[]{\it localPosition0}: local position (as 3D list or numpy array) on body0, if not a node number
\item[]{\it localPosition1}: local position (as 3D list or numpy array) on body1, if not a node number
\item[]{\it distance}: if None, distance is computed from reference position of bodies or nodes; if not None, this distance (which must be always larger than zero) is prescribed between the two positions
\item[]{\it show}: if True, connector visualization is drawn
\item[]{\it drawSize}: general drawing size of node
\item[]{\it color}: color of connector
\end{itemize}
\item[--]
{\bf output}: [ObjectIndex, MarkerIndex, MarkerIndex]; returns list [oJoint, mBody0, mBody1], containing the joint object number, and the two rigid body markers on body0/1 for the joint
\item[--]
{\bf example}: \vspace{-12pt}\ei\begin{lstlisting}[language=Python, xleftmargin=36pt]
  import exudyn as exu
  from exudyn.utilities import * #includes itemInterface, graphicsDataUtilities and rigidBodyUtilities
  import numpy as np
  SC = exu.SystemContainer()
  mbs = SC.AddSystem()
  b0 = mbs.CreateRigidBody(inertia = InertiaCuboid(density=5000,
                                                    sideLengths=[1,0.1,0.1]),
                            referencePosition = [6,0,0],
                            gravity = [0,-9.81,0],
                            graphicsDataList = [GraphicsDataOrthoCubePoint(size=[1,0.1,0.1],
                                                                        color=color4orange)])
  m1 = mbs.CreateMassPoint(referenceCoordinates=[5.5,-1,0],
                           physicsMass=1, drawSize = 0.2)
  n1 = mbs.GetObject(m1)['nodeNumber']
  oGround = mbs.AddObject(ObjectGround())
  mbs.CreateDistanceConstraint(bodyOrNodeList=[oGround, b0],
                               localPosition0 = [6.5,1,0],
                               localPosition1 = [0.5,0,0],
                               distance=None, #automatically computed
                               drawSize=0.06)
  mbs.CreateDistanceConstraint(bodyOrNodeList=[b0, n1],
                               localPosition0 = [-0.5,0,0],
                               localPosition1 = [0.,0.,0.], #must be [0,0,0] for Node
                               distance=None, #automatically computed
                               drawSize=0.06)
  mbs.Assemble()
  simulationSettings = exu.SimulationSettings() #takes currently set values or default values
  simulationSettings.timeIntegration.numberOfSteps = 1000
  simulationSettings.timeIntegration.endTime = 2
  mbs.SolveDynamic(simulationSettings = simulationSettings)
\end{lstlisting}\vspace{-24pt}\bi\item[]\vspace{-24pt}\vspace{12pt}\end{itemize}
%

%
\noindent For examples on CreateDistanceConstraint see Relevant Examples (Ex) and TestModels (TM) with weblink to github:
\bi
 \item \footnotesize \exuUrl{https://github.com/jgerstmayr/EXUDYN/blob/master/main/pythonDev/TestModels/mainSystemExtensionsTests.py}{\texttt{mainSystemExtensionsTests.py}} (TM)
\ei

%
\begin{flushleft}
\noindent {def {\bf \exuUrl{https://github.com/jgerstmayr/EXUDYN/blob/master/main/pythonDev/exudyn/plot.py\#L226}{PlotSensor}{}}}\label{sec:mainsystemextensions:PlotSensor}
({\it sensorNumbers}= [], {\it components}= 0, {\it xLabel}= 'time (s)', {\it yLabel}= None, {\it labels}= [], {\it colorCodeOffset}= 0, {\it newFigure}= True, {\it closeAll}= False, {\it componentsX}= [], {\it title}= '', {\it figureName}= '', {\it fontSize}= 16, {\it colors}= [], {\it lineStyles}= [], {\it lineWidths}= [], {\it markerStyles}= [], {\it markerSizes}= [], {\it markerDensity}= 0.08, {\it rangeX}= [], {\it rangeY}= [], {\it majorTicksX}= 10, {\it majorTicksY}= 10, {\it offsets}= [], {\it factors}= [], {\it subPlot}= [], {\it sizeInches}= [6.4,4.8], {\it fileName}= '', {\it useXYZcomponents}= True, {\it **kwargs})
\end{flushleft}
\setlength{\itemindent}{0.7cm}
\begin{itemize}[leftmargin=0.7cm]
\item[--]
{\bf function description}: \vspace{-6pt}
\begin{itemize}[leftmargin=1.2cm]
\setlength{\itemindent}{-0.7cm}
\item[]Helper function for direct and easy visualization of sensor outputs, without need for loading text files, etc.; PlotSensor can be used to simply plot, e.g., the measured x-Position over time in a figure. PlotSensor provides an interface to matplotlib (which needs to be installed). Default values of many function arguments can be changed using the exudyn.plot function PlotSensorDefaults(), see there for usage.
\item[]- NOTE that this function is added to MainSystem via Python function PlotSensor.
\end{itemize}
\item[--]
{\bf input}: \vspace{-6pt}
\begin{itemize}[leftmargin=1.2cm]
\setlength{\itemindent}{-0.7cm}
\item[]{\it sensorNumbers}: consists of one or a list of sensor numbers (type SensorIndex or int) as returned by the mbs function AddSensor(...); sensors need to set writeToFile=True and/or storeInternal=True for PlotSensor to work; alternatively, it may contain FILENAMES (incl. path) to stored sensor or solution files OR a numpy array instead of sensor numbers; the format of data (file or numpy array) must contain per row the time and according solution values in columns; if components is a list and sensorNumbers is a scalar, sensorNumbers is adjusted automatically to the components
\item[]{\it components}: consists of one or a list of components according to the component of the sensor to be plotted at y-axis; if components is a list and sensorNumbers is a scalar, sensorNumbers is adjusted automatically to the components; as always, components are zero-based, meaning 0=X, 1=Y, etc.; for regular sensor files, time will be component=-1; to show the norm (e.g., of a force vector), use component=[plot.componentNorm] for according sensors; norm will consider all values of sensor except time (for 3D force, it will be $\sqrt{f_0^2+f_1^2+f_2^2}$); offsets and factors are mapped on norm (plot value=factor*(norm(values) + offset) ), not on component values
\item[]{\it componentsX}: default componentsX=[] uses time in files; otherwise provide componentsX as list of components (or scalar) representing x components of sensors in plotted curves; DON'T forget to change xLabel accordingly!
\item[]Using componentsX=[...] with a list of column indices specifies the respective columns used for the x-coordinates in all sensors; by default, values are plotted against the first column in the files, which is time; according to counting in PlotSensor, this represents componentX=-1;
\item[]plotting y over x in a position sensor thus reads: components=[1], componentsX=[0];
\item[]plotting time over x reads: components=[-1], componentsX=[0];
\item[]the default value reads componentsX=[-1,-1,...]
\item[]{\it xLabel}: string for text at x-axis
\item[]{\it yLabel}: string for text at y-axis (default: None==> label is automatically computed from sensor value types)
\item[]{\it labels}: string (for one sensor) or list of strings (according to number of sensors resp. components) representing the labels used in legend; if labels=[], automatically generated legend is used
\item[]{\it rangeX}: default rangeX=[]: computes range automatically; otherwise use rangeX to set range (limits) for x-axis provided as sorted list of two floats, e.g., rangeX=[0,4]
\item[]{\it rangeY}: default rangeY=[]: computes range automatically; otherwise use rangeY to set range (limits) for y-axis provided as sorted list of two floats, e.g., rangeY=[-1,1]
\item[]{\it figureName}: optional name for figure, if newFigure=True
\item[]{\it fontSize}: change general fontsize of axis, labels, etc. (matplotlib default is 12, default in PlotSensor: 16)
\item[]{\it title}: optional string representing plot title
\item[]{\it offsets}: provide as scalar, list of scalars (per sensor) or list of 2D numpy.arrays (per sensor, having same rows/columns as sensor data; in this case it will also influence x-axis if componentsX is different from -1) to add offset to each sensor output; for an original value fOrig, the new value reads fNew = factor*(fOrig+offset); for offset provided as numpy array (with same time values), the 'time' column is ignored in the offset computation; can be used to compute difference of sensors; if offsets=[], no offset is used
\item[]{\it factors}: provide as scalar or list (per sensor) to add factor to each sensor output; for an original value fOrig, the new value reads fNew = factor*(fOrig+offset); if factor=[], no factor is used
\item[]{\it majorTicksX}: number of major ticks on x-axis; default: 10
\item[]{\it majorTicksY}: number of major ticks on y-axis; default: 10
\item[]{\it colorCodeOffset}: int offset for color code, color codes going from 0 to 27 (see PlotLineCode(...)); automatic line/color codes are used if no colors and lineStyles are used
\item[]{\it colors}: color is automatically selected from colorCodeOffset if colors=[]; otherwise chose from 'b', 'g', 'r', 'c', 'm', 'y', 'k' and many other colors see https://matplotlib.org/stable/gallery/color/named\_colors.html
\item[]{\it lineStyles}: line style is automatically selected from colorCodeOffset if lineStyles=[]; otherwise define for all lines with string or with list of strings, chosing from '-', '--', '-.', ':', or ''
\item[]{\it lineWidths}: float to define line width by float (default=1); either use single float for all sensors or list of floats with length >= number of sensors
\item[]{\it markerStyles}: if different from [], marker styles are defined as list of marker style strings or single string for one sensor; chose from '.', 'o', 'x', '+' ... check listMarkerStylesFilled and listMarkerStyles in exudyn.plot and see https://matplotlib.org/stable/api/markers\_api.html ; ADD a space to markers to make them empty (transparent), e.g. 'o ' will create an empty circle
\item[]{\it markerSizes}: float to define marker size by float (default=6); either use single float for all sensors or list of floats with length >= number of sensors
\item[]{\it markerDensity}: if int, it defines approx. the total number of markers used along each graph; if float, this defines the distance of markers relative to the diagonal of the plot (default=0.08); if None, it adds a marker to every data point if marker style is specified for sensor
\item[]{\it newFigure}: if True, a new matplotlib.pyplot figure is created; otherwise, existing figures are overwritten
\item[]{\it subPlot}: given as list [nx, ny, position] with nx, ny being the number of subplots in x and y direction (nx=cols, ny=rows), and position in [1,..., nx*ny] gives the position in the subplots; use the same structure for first PlotSensor (with newFigure=True) and all subsequent PlotSensor calls with newFigure=False, which creates the according subplots; default=[](no subplots)
\item[]{\it sizeInches}: given as list [sizeX, sizeY] with the sizes per (sub)plot given in inches; default: [6.4, 4.8]; in case of sub plots, the total size of the figure is computed from nx*sizeInches[0] and ny*sizeInches[1]
\item[]{\it fileName}: if this string is non-empty, figure will be saved to given path and filename (use figName.pdf to safe as PDF or figName.png to save as PNG image); use matplotlib.use('Agg') in order not to open figures if you just want to save them
\item[]{\it useXYZcomponents}: of True, it will use X, Y and Z for sensor components, e.g., measuring Position, Velocity, etc. wherever possible
\item[]{\it closeAll}: if True, close all figures before opening new one (do this only in first PlotSensor command!)
\item[]{\it [*kwargs]}:
\item[]{\it minorTicksXon}: if True, turn minor ticks for x-axis on
\item[]{\it minorTicksYon}: if True, turn minor ticks for y-axis on
\item[]{\it fileCommentChar}: if exists, defines the comment character in files (\#, %, ...)
\item[]{\it fileDelimiterChar}: if exists, defines the character indicating the columns for data (',', ' ', ';', ...)
\end{itemize}
\item[--]
{\bf output}: [Any, Any, Any, Any]; plots the sensor data; returns [plt, fig, ax, line] in which plt is matplotlib.pyplot, fig is the figure (or None), ax is the axis (or None) and line is the return value of plt.plot (or None) which could be changed hereafter
\item[--]
{\bf notes}: adjust default values by modifying the variables exudyn.plot.plotSensorDefault..., e.g., exudyn.plot.plotSensorDefaultFontSize
\item[--]
{\bf example}: \vspace{-12pt}\ei\begin{lstlisting}[language=Python, xleftmargin=36pt]
  #assume to have some position-based nodes 0 and 1:
  s0=mbs.AddSensor(SensorNode(nodeNumber=0, fileName='s0.txt',
                              outputVariableType=exu.OutputVariableType.Position))
  s1=mbs.AddSensor(SensorNode(nodeNumber=1, fileName='s1.txt',
                              outputVariableType=exu.OutputVariableType.Position))
  mbs.PlotSensor(s0, 0) #plot x-coordinate
  #plot x for s0 and z for s1:
  mbs.PlotSensor(sensorNumbers=[s0,s1], components=[0,2], yLabel='this is the position in meter')
  mbs.PlotSensor(sensorNumbers=s0, components=plot.componentNorm) #norm of position
  mbs.PlotSensor(sensorNumbers=s0, components=[0,1,2], factors=1000., title='Answers to the big questions')
  mbs.PlotSensor(sensorNumbers=s0, components=[0,1,2,3],
             yLabel='Coordantes with offset 1\nand scaled with $\\frac{1}{1000}$',
             factors=1e-3, offsets=1,fontSize=12, closeAll=True)
  #assume to have body sensor sBody, marker sensor sMarker:
  mbs.PlotSensor(sensorNumbers=[sBody]*3+[sMarker]*3, components=[0,1,2,0,1,2],
             colorCodeOffset=3, newFigure=False, fontSize=10,
             yLabel='Rotation $\\alpha, \\beta, \\gamma$ and\n Position $x,y,z$',
             title='compare marker and body sensor')
  #assume having file plotSensorNode.txt:
  mbs.PlotSensor(sensorNumbers=[s0]*3+ [filedir+'plotSensorNode.txt']*3,
             components=[0,1,2]*2)
  #plot y over x:
  mbs.PlotSensor(sensorNumbers=s0, componentsX=[0], components=[1], xLabel='x-Position', yLabel='y-Position')
  #for further examples, see also Examples/plotSensorExamples.py
\end{lstlisting}\vspace{-24pt}\bi\item[]\vspace{-24pt}\vspace{12pt}\end{itemize}
%

%
\noindent For examples on PlotSensor see Relevant Examples (Ex) and TestModels (TM) with weblink to github:
\bi
 \item \footnotesize \exuUrl{https://github.com/jgerstmayr/EXUDYN/blob/master/main/pythonDev/Examples/ANCFALEtest.py}{\texttt{ANCFALEtest.py}} (Ex), 
\exuUrl{https://github.com/jgerstmayr/EXUDYN/blob/master/main/pythonDev/Examples/beltDriveALE.py}{\texttt{beltDriveALE.py}} (Ex), 
\exuUrl{https://github.com/jgerstmayr/EXUDYN/blob/master/main/pythonDev/Examples/beltDriveReevingSystem.py}{\texttt{beltDriveReevingSystem.py}} (Ex), 
\\ \exuUrl{https://github.com/jgerstmayr/EXUDYN/blob/master/main/pythonDev/Examples/beltDrivesComparison.py}{\texttt{beltDrivesComparison.py}} (Ex), 
\exuUrl{https://github.com/jgerstmayr/EXUDYN/blob/master/main/pythonDev/Examples/bicycleIftommBenchmark.py}{\texttt{bicycleIftommBenchmark.py}} (Ex), 
 ...
, 
\exuUrl{https://github.com/jgerstmayr/EXUDYN/blob/master/main/pythonDev/TestModels/ACFtest.py}{\texttt{ACFtest.py}} (TM), 
\\ \exuUrl{https://github.com/jgerstmayr/EXUDYN/blob/master/main/pythonDev/TestModels/ANCFbeltDrive.py}{\texttt{ANCFbeltDrive.py}} (TM), 
\exuUrl{https://github.com/jgerstmayr/EXUDYN/blob/master/main/pythonDev/TestModels/ANCFgeneralContactCircle.py}{\texttt{ANCFgeneralContactCircle.py}} (TM), 
 ...

\ei

%
\begin{flushleft}
\noindent {def {\bf \exuUrl{https://github.com/jgerstmayr/EXUDYN/blob/master/main/pythonDev/exudyn/solver.py\#L153}{SolveStatic}{}}}\label{sec:mainsystemextensions:SolveStatic}
({\it simulationSettings}= exudyn.SimulationSettings(), {\it updateInitialValues}= False, {\it storeSolver}= True, {\it showHints}= False, {\it showCausingItems}= True)
\end{flushleft}
\setlength{\itemindent}{0.7cm}
\begin{itemize}[leftmargin=0.7cm]
\item[--]
{\bf function description}: \vspace{-6pt}
\begin{itemize}[leftmargin=1.2cm]
\setlength{\itemindent}{-0.7cm}
\item[]solves the static mbs problem using simulationSettings; check theDoc.pdf for MainSolverStatic for further details of the static solver; this function is also available in exudyn (using exudyn.SolveStatic(...))
\item[]- NOTE that this function is added to MainSystem via Python function SolveStatic.
\end{itemize}
\item[--]
{\bf input}: \vspace{-6pt}
\begin{itemize}[leftmargin=1.2cm]
\setlength{\itemindent}{-0.7cm}
\item[]{\it simulationSettings}: specific simulation settings out of exu.SimulationSettings(), as described in \refSection{sec:SolutionSettings}; use options for newton, discontinuous settings, etc., from staticSolver sub-items
\item[]{\it updateInitialValues}: if True, the results are written to initial values, such at a consecutive simulation uses the results of this simulation as the initial values of the next simulation
\item[]{\it storeSolver}: if True, the staticSolver object is stored in the mbs.sys dictionary as mbs.sys['staticSolver'], and simulationSettings are stored as mbs.sys['simulationSettings']
\end{itemize}
\item[--]
{\bf output}: bool; returns True, if successful, False if fails; if storeSolver = True, mbs.sys contains staticSolver, which allows to investigate solver problems (check theDoc.pdf \refSection{sec:solverSubstructures} and the items described in \refSection{sec:MainSolverStatic})
\item[--]
{\bf example}: \vspace{-12pt}\ei\begin{lstlisting}[language=Python, xleftmargin=36pt]
  import exudyn as exu
  from exudyn.itemInterface import *
  SC = exu.SystemContainer()
  mbs = SC.AddSystem()
  #create simple system:
  ground = mbs.AddObject(ObjectGround())
  mbs.AddNode(NodePoint())
  body = mbs.AddObject(MassPoint(physicsMass=1, nodeNumber=0))
  m0 = mbs.AddMarker(MarkerBodyPosition(bodyNumber=ground))
  m1 = mbs.AddMarker(MarkerBodyPosition(bodyNumber=body))
  mbs.AddObject(CartesianSpringDamper(markerNumbers=[m0,m1], stiffness=[100,100,100]))
  mbs.AddLoad(LoadForceVector(markerNumber=m1, loadVector=[10,10,10]))
  mbs.Assemble()
  simulationSettings = exu.SimulationSettings()
  simulationSettings.timeIntegration.endTime = 10
  success = mbs.SolveStatic(simulationSettings, storeSolver = True)
  print("success =", success)
  print("iterations = ", mbs.sys['staticSolver'].it)
  print("pos=", mbs.GetObjectOutputBody(body,localPosition=[0,0,0],
        variableType=exu.OutputVariableType.Position))
\end{lstlisting}\vspace{-24pt}\bi\item[]\vspace{-24pt}\vspace{12pt}\end{itemize}
%

%
\noindent For examples on SolveStatic see Relevant Examples (Ex) and TestModels (TM) with weblink to github:
\bi
 \item \footnotesize \exuUrl{https://github.com/jgerstmayr/EXUDYN/blob/master/main/pythonDev/Examples/3SpringsDistance.py}{\texttt{3SpringsDistance.py}} (Ex), 
\exuUrl{https://github.com/jgerstmayr/EXUDYN/blob/master/main/pythonDev/Examples/ALEANCFpipe.py}{\texttt{ALEANCFpipe.py}} (Ex), 
\exuUrl{https://github.com/jgerstmayr/EXUDYN/blob/master/main/pythonDev/Examples/ANCFALEtest.py}{\texttt{ANCFALEtest.py}} (Ex), 
\\ \exuUrl{https://github.com/jgerstmayr/EXUDYN/blob/master/main/pythonDev/Examples/ANCFcantileverTest.py}{\texttt{ANCFcantileverTest.py}} (Ex), 
\exuUrl{https://github.com/jgerstmayr/EXUDYN/blob/master/main/pythonDev/Examples/ANCFcontactCircle.py}{\texttt{ANCFcontactCircle.py}} (Ex), 
 ...
, 
\exuUrl{https://github.com/jgerstmayr/EXUDYN/blob/master/main/pythonDev/TestModels/ANCFBeamTest.py}{\texttt{ANCFBeamTest.py}} (TM), 
\\ \exuUrl{https://github.com/jgerstmayr/EXUDYN/blob/master/main/pythonDev/TestModels/ANCFbeltDrive.py}{\texttt{ANCFbeltDrive.py}} (TM), 
\exuUrl{https://github.com/jgerstmayr/EXUDYN/blob/master/main/pythonDev/TestModels/ANCFcontactCircleTest.py}{\texttt{ANCFcontactCircleTest.py}} (TM), 
 ...

\ei

%
\begin{flushleft}
\noindent {def {\bf \exuUrl{https://github.com/jgerstmayr/EXUDYN/blob/master/main/pythonDev/exudyn/solver.py\#L218}{SolveDynamic}{}}}\label{sec:mainsystemextensions:SolveDynamic}
({\it simulationSettings}= exudyn.SimulationSettings(), {\it solverType}= exudyn.DynamicSolverType.GeneralizedAlpha, {\it updateInitialValues}= False, {\it storeSolver}= True, {\it showHints}= False, {\it showCausingItems}= True)
\end{flushleft}
\setlength{\itemindent}{0.7cm}
\begin{itemize}[leftmargin=0.7cm]
\item[--]
{\bf function description}: \vspace{-6pt}
\begin{itemize}[leftmargin=1.2cm]
\setlength{\itemindent}{-0.7cm}
\item[]solves the dynamic mbs problem using simulationSettings and solver type; check theDoc.pdf for MainSolverImplicitSecondOrder for further details of the dynamic solver; this function is also available in exudyn (using exudyn.SolveDynamic(...))
\item[]- NOTE that this function is added to MainSystem via Python function SolveDynamic.
\end{itemize}
\item[--]
{\bf input}: \vspace{-6pt}
\begin{itemize}[leftmargin=1.2cm]
\setlength{\itemindent}{-0.7cm}
\item[]{\it simulationSettings}: specific simulation settings out of exu.SimulationSettings(), as described in \refSection{sec:SolutionSettings}; use options for newton, discontinuous settings, etc., from timeIntegration; therein, implicit second order solvers use settings from generalizedAlpha and explict solvers from explicitIntegration; be careful with settings, as the influence accuracy (step size!), convergence and performance (see special \refSection{sec:overview:basics:speedup})
\item[]{\it solverType}: use exudyn.DynamicSolverType to set specific solver (default=generalized alpha)
\item[]{\it updateInitialValues}: if True, the results are written to initial values, such at a consecutive simulation uses the results of this simulation as the initial values of the next simulation
\item[]{\it storeSolver}: if True, the staticSolver object is stored in the mbs.sys dictionary as mbs.sys['staticSolver'], and simulationSettings are stored as mbs.sys['simulationSettings']
\item[]{\it showHints}: show additional hints, if solver fails
\item[]{\it showCausingItems}: if linear solver fails, this option helps to identify objects, etc. which are related to a singularity in the linearized system matrix
\end{itemize}
\item[--]
{\bf output}: bool; returns True, if successful, False if fails; if storeSolver = True, mbs.sys contains staticSolver, which allows to investigate solver problems (check theDoc.pdf \refSection{sec:solverSubstructures} and the items described in \refSection{sec:MainSolverStatic})
\item[--]
{\bf example}: \vspace{-12pt}\ei\begin{lstlisting}[language=Python, xleftmargin=36pt]
  import exudyn as exu
  from exudyn.itemInterface import *
  SC = exu.SystemContainer()
  mbs = SC.AddSystem()
  #create simple system:
  ground = mbs.AddObject(ObjectGround())
  mbs.AddNode(NodePoint())
  body = mbs.AddObject(MassPoint(physicsMass=1, nodeNumber=0))
  m0 = mbs.AddMarker(MarkerBodyPosition(bodyNumber=ground))
  m1 = mbs.AddMarker(MarkerBodyPosition(bodyNumber=body))
  mbs.AddObject(CartesianSpringDamper(markerNumbers=[m0,m1], stiffness=[100,100,100]))
  mbs.AddLoad(LoadForceVector(markerNumber=m1, loadVector=[10,10,10]))
  #
  mbs.Assemble()
  simulationSettings = exu.SimulationSettings()
  simulationSettings.timeIntegration.endTime = 10
  success = mbs.SolveDynamic(simulationSettings, storeSolver = True)
  print("success =", success)
  print("iterations = ", mbs.sys['dynamicSolver'].it)
  print("pos=", mbs.GetObjectOutputBody(body,localPosition=[0,0,0],
        variableType=exu.OutputVariableType.Position))
\end{lstlisting}\vspace{-24pt}\bi\item[]\vspace{-24pt}\vspace{12pt}\end{itemize}
%

%
\noindent For examples on SolveDynamic see Relevant Examples (Ex) and TestModels (TM) with weblink to github:
\bi
 \item \footnotesize \exuUrl{https://github.com/jgerstmayr/EXUDYN/blob/master/main/pythonDev/Examples/3SpringsDistance.py}{\texttt{3SpringsDistance.py}} (Ex), 
\exuUrl{https://github.com/jgerstmayr/EXUDYN/blob/master/main/pythonDev/Examples/addPrismaticJoint.py}{\texttt{addPrismaticJoint.py}} (Ex), 
\exuUrl{https://github.com/jgerstmayr/EXUDYN/blob/master/main/pythonDev/Examples/addRevoluteJoint.py}{\texttt{addRevoluteJoint.py}} (Ex), 
\\ \exuUrl{https://github.com/jgerstmayr/EXUDYN/blob/master/main/pythonDev/Examples/ALEANCFpipe.py}{\texttt{ALEANCFpipe.py}} (Ex), 
\exuUrl{https://github.com/jgerstmayr/EXUDYN/blob/master/main/pythonDev/Examples/ANCFALEtest.py}{\texttt{ANCFALEtest.py}} (Ex), 
 ...
, 
\exuUrl{https://github.com/jgerstmayr/EXUDYN/blob/master/main/pythonDev/TestModels/abaqusImportTest.py}{\texttt{abaqusImportTest.py}} (TM), 
\\ \exuUrl{https://github.com/jgerstmayr/EXUDYN/blob/master/main/pythonDev/TestModels/ACFtest.py}{\texttt{ACFtest.py}} (TM), 
\exuUrl{https://github.com/jgerstmayr/EXUDYN/blob/master/main/pythonDev/TestModels/ANCFBeamEigTest.py}{\texttt{ANCFBeamEigTest.py}} (TM), 
 ...

\ei

%
\begin{flushleft}
\noindent {def {\bf \exuUrl{https://github.com/jgerstmayr/EXUDYN/blob/master/main/pythonDev/exudyn/solver.py\#L364}{ComputeLinearizedSystem}{}}}\label{sec:mainsystemextensions:ComputeLinearizedSystem}
({\it simulationSettings}= exudyn.SimulationSettings(), {\it useSparseSolver}= False)
\end{flushleft}
\setlength{\itemindent}{0.7cm}
\begin{itemize}[leftmargin=0.7cm]
\item[--]
{\bf function description}: \vspace{-6pt}
\begin{itemize}[leftmargin=1.2cm]
\setlength{\itemindent}{-0.7cm}
\item[]compute linearized system of equations for ODE2 part of mbs, not considering the effects of algebraic constraints
\item[]- NOTE that this function is added to MainSystem via Python function ComputeLinearizedSystem.
\end{itemize}
\item[--]
{\bf input}: \vspace{-6pt}
\begin{itemize}[leftmargin=1.2cm]
\setlength{\itemindent}{-0.7cm}
\item[]{\it simulationSettings}: specific simulation settings used for computation of jacobian (e.g., sparse mode in static solver enables sparse computation)
\item[]{\it useSparseSolver}: if False (only for small systems), all eigenvalues are computed in dense mode (slow for large systems!); if True, only the numberOfEigenvalues are computed (numberOfEigenvalues must be set!); Currently, the matrices are exported only in DENSE MODE from mbs! NOTE that the sparsesolver accuracy is much less than the dense solver
\end{itemize}
\item[--]
{\bf output}: [ArrayLike, ArrayLike, ArrayLike]; [M, K, D]; list containing numpy mass matrix M, stiffness matrix K and damping matrix D
\item[--]
{\bf example}: \vspace{-12pt}\ei\begin{lstlisting}[language=Python, xleftmargin=36pt]
  import exudyn as exu
  from exudyn.utilities import * #includes itemInterface, graphicsDataUtilities and rigidBodyUtilities
  import numpy as np
  SC = exu.SystemContainer()
  mbs = SC.AddSystem()
  #
  b0 = mbs.CreateMassPoint(referenceCoordinates = [2,0,0],
                           initialVelocities = [2*0,5,0],
                           physicsMass = 1, gravity = [0,-9.81,0],
                           drawSize = 0.5, color=color4blue)
  #
  oGround = mbs.AddObject(ObjectGround())
  #add vertical spring
  oSD = mbs.CreateSpringDamper(bodyOrNodeList=[oGround, b0],
                               localPosition0=[2,1,0],
                               localPosition1=[0,0,0],
                               stiffness=1e4, damping=1e2,
                               drawSize=0.2)
  #
  mbs.Assemble()
  [M,K,D] = mbs.ComputeLinearizedSystem()
  exu.Print('M=\n',M,'\nK=\n',K,'\nD=\n',D)
\end{lstlisting}\vspace{-24pt}\bi\item[]\vspace{-24pt}\vspace{12pt}\end{itemize}
%

%
\noindent For examples on ComputeLinearizedSystem see Relevant Examples (Ex) and TestModels (TM) with weblink to github:
\bi
 \item \footnotesize \exuUrl{https://github.com/jgerstmayr/EXUDYN/blob/master/main/pythonDev/TestModels/ANCFBeamEigTest.py}{\texttt{ANCFBeamEigTest.py}} (TM), 
\exuUrl{https://github.com/jgerstmayr/EXUDYN/blob/master/main/pythonDev/TestModels/ANCFBeamTest.py}{\texttt{ANCFBeamTest.py}} (TM), 
\exuUrl{https://github.com/jgerstmayr/EXUDYN/blob/master/main/pythonDev/TestModels/geometricallyExactBeamTest.py}{\texttt{geometricallyExactBeamTest.py}} (TM), 
\\ \exuUrl{https://github.com/jgerstmayr/EXUDYN/blob/master/main/pythonDev/TestModels/mainSystemExtensionsTests.py}{\texttt{mainSystemExtensionsTests.py}} (TM)
\ei

%
\begin{flushleft}
\noindent {def {\bf \exuUrl{https://github.com/jgerstmayr/EXUDYN/blob/master/main/pythonDev/exudyn/solver.py\#L437}{ComputeODE2Eigenvalues}{}}}\label{sec:mainsystemextensions:ComputeODE2Eigenvalues}
({\it simulationSettings}= exudyn.SimulationSettings(), {\it useSparseSolver}= False, {\it numberOfEigenvalues}= 0, {\it constrainedCoordinates}= [], {\it convert2Frequencies}= False, {\it useAbsoluteValues}= True)
\end{flushleft}
\setlength{\itemindent}{0.7cm}
\begin{itemize}[leftmargin=0.7cm]
\item[--]
{\bf function description}: \vspace{-6pt}
\begin{itemize}[leftmargin=1.2cm]
\setlength{\itemindent}{-0.7cm}
\item[]compute eigenvalues for unconstrained ODE2 part of mbs, not considering the effects of algebraic constraints; the computation is done for the initial values of the mbs, independently of previous computations. If you would like to use the current state for the eigenvalue computation, you need to copy the current state to the initial state (using GetSystemState,SetSystemState, see \refSection{sec:mbs:systemData}); note that mass and stiffness matrix are computed in dense mode so far, while eigenvalues are computed according to useSparseSolver.
\item[]- NOTE that this function is added to MainSystem via Python function ComputeODE2Eigenvalues.
\end{itemize}
\item[--]
{\bf input}: \vspace{-6pt}
\begin{itemize}[leftmargin=1.2cm]
\setlength{\itemindent}{-0.7cm}
\item[]{\it simulationSettings}: specific simulation settings used for computation of jacobian (e.g., sparse mode in static solver enables sparse computation)
\item[]{\it useSparseSolver}: if False (only for small systems), all eigenvalues are computed in dense mode (slow for large systems!); if True, only the numberOfEigenvalues are computed (numberOfEigenvalues must be set!); Currently, the matrices are exported only in DENSE MODE from mbs! NOTE that the sparsesolver accuracy is much less than the dense solver
\item[]{\it numberOfEigenvalues}: number of eigenvalues and eivenvectors to be computed; if numberOfEigenvalues==0, all eigenvalues will be computed (may be impossible for larger problems!)
\item[]{\it constrainedCoordinates}: if this list is non-empty, the integer indices represent constrained coordinates of the system, which are fixed during eigenvalue/vector computation; according rows/columns of mass and stiffness matrices are erased
\item[]{\it convert2Frequencies}: if True, the eigen values are converted into frequencies (Hz) and the output is [eigenFrequencies, eigenVectors]
\item[]{\it useAbsoluteValues}: if True, abs(eigenvalues) is used, which avoids problems for small (close to zero) eigen values; needed, when converting to frequencies
\end{itemize}
\item[--]
{\bf output}: [ArrayLike, ArrayLike]; [eigenValues, eigenVectors]; eigenValues being a numpy array of eigen values ($\omega_i^2$, being the squared eigen frequencies in ($\omega_i$ in rad/s)!), eigenVectors a numpy array containing the eigenvectors in every column
\item[--]
{\bf example}: \vspace{-12pt}\ei\begin{lstlisting}[language=Python, xleftmargin=36pt]
   #take any example from the Examples or TestModels folder, e.g., 'cartesianSpringDamper.py' and run it
   #specific example:
  import exudyn as exu
  from exudyn.utilities import * #includes itemInterface, graphicsDataUtilities and rigidBodyUtilities
  import numpy as np
  SC = exu.SystemContainer()
  mbs = SC.AddSystem()
  #
  b0 = mbs.CreateMassPoint(referenceCoordinates = [2,0,0],
                           initialVelocities = [2*0,5,0],
                           physicsMass = 1, gravity = [0,-9.81,0],
                           drawSize = 0.5, color=color4blue)
  #
  oGround = mbs.AddObject(ObjectGround())
  #add vertical spring
  oSD = mbs.CreateSpringDamper(bodyOrNodeList=[oGround, b0],
                               localPosition0=[2,1,0],
                               localPosition1=[0,0,0],
                               stiffness=1e4, damping=1e2,
                               drawSize=0.2)
  #
  mbs.Assemble()
  #
  [eigenvalues, eigenvectors] = mbs.ComputeODE2Eigenvalues()
   #==>eigenvalues contain the eigenvalues of the ODE2 part of the system in the current configuration
\end{lstlisting}\vspace{-24pt}\bi\item[]\vspace{-24pt}\vspace{12pt}\end{itemize}
%

%
\noindent For examples on ComputeODE2Eigenvalues see Relevant Examples (Ex) and TestModels (TM) with weblink to github:
\bi
 \item \footnotesize \exuUrl{https://github.com/jgerstmayr/EXUDYN/blob/master/main/pythonDev/Examples/nMassOscillatorInteractive.py}{\texttt{nMassOscillatorInteractive.py}} (Ex), 
\exuUrl{https://github.com/jgerstmayr/EXUDYN/blob/master/main/pythonDev/TestModels/ANCFBeamEigTest.py}{\texttt{ANCFBeamEigTest.py}} (TM), 
\exuUrl{https://github.com/jgerstmayr/EXUDYN/blob/master/main/pythonDev/TestModels/ANCFBeamTest.py}{\texttt{ANCFBeamTest.py}} (TM), 
\\ \exuUrl{https://github.com/jgerstmayr/EXUDYN/blob/master/main/pythonDev/TestModels/computeODE2EigenvaluesTest.py}{\texttt{computeODE2EigenvaluesTest.py}} (TM), 
\exuUrl{https://github.com/jgerstmayr/EXUDYN/blob/master/main/pythonDev/TestModels/geometricallyExactBeamTest.py}{\texttt{geometricallyExactBeamTest.py}} (TM), 
\exuUrl{https://github.com/jgerstmayr/EXUDYN/blob/master/main/pythonDev/TestModels/mainSystemExtensionsTests.py}{\texttt{mainSystemExtensionsTests.py}} (TM)
\ei

%
\begin{flushleft}
\noindent {def {\bf \exuUrl{https://github.com/jgerstmayr/EXUDYN/blob/master/main/pythonDev/exudyn/solver.py\#L558}{ComputeSystemDegreeOfFreedom}{}}}\label{sec:mainsystemextensions:ComputeSystemDegreeOfFreedom}
({\it simulationSettings}= exudyn.SimulationSettings(), {\it threshold}= 1e-12, {\it verbose}= False, {\it useSVD}= False)
\end{flushleft}
\setlength{\itemindent}{0.7cm}
\begin{itemize}[leftmargin=0.7cm]
\item[--]
{\bf function description}: \vspace{-6pt}
\begin{itemize}[leftmargin=1.2cm]
\setlength{\itemindent}{-0.7cm}
\item[]compute system DOF numerically, considering Gr{\"u}bler-Kutzbach formula as well as redundant constraints; uses numpy matrix rank or singular value decomposition of scipy (useSVD=True)
\item[]- NOTE that this function is added to MainSystem via Python function ComputeSystemDegreeOfFreedom.
\end{itemize}
\item[--]
{\bf input}: \vspace{-6pt}
\begin{itemize}[leftmargin=1.2cm]
\setlength{\itemindent}{-0.7cm}
\item[]{\it simulationSettings}: used e.g. for settings regarding numerical differentiation; default settings may be used in most cases
\item[]{\it threshold}: threshold factor for singular values which estimate the redundant constraints
\item[]{\it useSVD}: use singular value decomposition directly, also showing SVD values if verbose=True
\item[]{\it verbose}: if True, it will show the singular values and one may decide if the threshold shall be adapted
\end{itemize}
\item[--]
{\bf output}: List[int]; returns list of [dof, nRedundant, nODE2, nODE1, nAE, nPureAE], where: dof = the degree of freedom computed numerically, nRedundant=the number of redundant constraints, nODE2=number of ODE2 coordinates, nODE1=number of ODE1 coordinates, nAE=total number of constraints, nPureAE=number of constraints on algebraic variables (e.g., lambda=0) that are not coupled to ODE2 coordinates
\item[--]
{\bf notes}: this approach may not always work! Currently only works with dense matrices, thus it will be slow for larger systems
\item[--]
{\bf example}: \vspace{-12pt}\ei\begin{lstlisting}[language=Python, xleftmargin=36pt]
  import exudyn as exu
  from exudyn.utilities import * #includes itemInterface, graphicsDataUtilities and rigidBodyUtilities
  import numpy as np
  SC = exu.SystemContainer()
  mbs = SC.AddSystem()
  #
  b0 = mbs.CreateRigidBody(inertia = InertiaCuboid(density=5000,
                                                   sideLengths=[1,0.1,0.1]),
                           referencePosition = [6,0,0],
                           initialAngularVelocity = [0,8,0],
                           gravity = [0,-9.81,0],
                           graphicsDataList = [GraphicsDataOrthoCubePoint(size=[1,0.1,0.1],
                                                                        color=color4orange)])
  oGround = mbs.AddObject(ObjectGround())
  mbs.CreateGenericJoint(bodyNumbers=[oGround, b0], position=[5.5,0,0],
                         constrainedAxes=[1,1,1, 1,0,0],
                         rotationMatrixAxes=RotationMatrixX(0.125*pi), #tilt axes
                         useGlobalFrame=True, axesRadius=0.02, axesLength=0.2)
  #
  mbs.Assemble()
  res = mbs.ComputeSystemDegreeOfFreedom(verbose=1) #print out details
\end{lstlisting}\vspace{-24pt}\bi\item[]\vspace{-24pt}\vspace{12pt}\end{itemize}
%

%
\noindent For examples on ComputeSystemDegreeOfFreedom see Relevant Examples (Ex) and TestModels (TM) with weblink to github:
\bi
 \item \footnotesize \exuUrl{https://github.com/jgerstmayr/EXUDYN/blob/master/main/pythonDev/Examples/fourBarMechanism3D.py}{\texttt{fourBarMechanism3D.py}} (Ex), 
\exuUrl{https://github.com/jgerstmayr/EXUDYN/blob/master/main/pythonDev/Examples/rigidBodyTutorial3.py}{\texttt{rigidBodyTutorial3.py}} (Ex), 
\exuUrl{https://github.com/jgerstmayr/EXUDYN/blob/master/main/pythonDev/TestModels/mainSystemExtensionsTests.py}{\texttt{mainSystemExtensionsTests.py}} (TM)
\ei

%
\begin{flushleft}
\noindent {def {\bf \exuUrl{https://github.com/jgerstmayr/EXUDYN/blob/master/main/pythonDev/exudyn/utilities.py\#L160}{CreateDistanceSensorGeometry}{}}}\label{sec:mainsystemextensions:CreateDistanceSensorGeometry}
({\it meshPoints}, {\it meshTrigs}, {\it rigidBodyMarkerIndex}, {\it searchTreeCellSize}= [8,8,8])
\end{flushleft}
\setlength{\itemindent}{0.7cm}
\begin{itemize}[leftmargin=0.7cm]
\item[--]
{\bf function description}: \vspace{-6pt}
\begin{itemize}[leftmargin=1.2cm]
\setlength{\itemindent}{-0.7cm}
\item[]Add geometry for distance sensor given by points and triangles (point indices) to mbs; use a rigid body marker where the geometry is put on;
\item[]Creates a GeneralContact for efficient search on background. If you have several sets of points and trigs, first merge them or add them manually to the contact
\item[]- NOTE that this function is added to MainSystem via Python function CreateDistanceSensorGeometry.
\end{itemize}
\item[--]
{\bf input}: \vspace{-6pt}
\begin{itemize}[leftmargin=1.2cm]
\setlength{\itemindent}{-0.7cm}
\item[]{\it meshPoints}: list of points (3D), as returned by GraphicsData2PointsAndTrigs()
\item[]{\it meshTrigs}: list of trigs (3 node indices each), as returned by GraphicsData2PointsAndTrigs()
\item[]{\it rigidBodyMarkerIndex}: rigid body marker to which the triangles are fixed on (ground or moving object)
\item[]{\it searchTreeCellSize}: size of search tree (X,Y,Z); use larger values in directions where more triangles are located
\end{itemize}
\item[--]
{\bf output}: int; returns ngc, which is the number of GeneralContact in mbs, to be used in CreateDistanceSensor(...); keep the gContact as deletion may corrupt data
\item[--]
{\bf notes}: should be used by CreateDistanceSensor(...) and AddLidar(...) for simple initialization of GeneralContact; old name: DistanceSensorSetupGeometry(...)
\vspace{12pt}\end{itemize}
%

%
\noindent For examples on CreateDistanceSensorGeometry see Relevant Examples (Ex) and TestModels (TM) with weblink to github:
\bi
 \item \footnotesize \exuUrl{https://github.com/jgerstmayr/EXUDYN/blob/master/main/pythonDev/TestModels/laserScannerTest.py}{\texttt{laserScannerTest.py}} (TM)
\ei

%
\begin{flushleft}
\noindent {def {\bf \exuUrl{https://github.com/jgerstmayr/EXUDYN/blob/master/main/pythonDev/exudyn/utilities.py\#L193}{CreateDistanceSensor}{}}}\label{sec:mainsystemextensions:CreateDistanceSensor}
({\it generalContactIndex}, {\it positionOrMarker}, {\it dirSensor}, {\it minDistance}= -1e7, {\it maxDistance}= 1e7, {\it cylinderRadius}= 0, {\it selectedTypeIndex}= exudyn.ContactTypeIndex.IndexEndOfEnumList, {\it storeInternal}= False, {\it fileName}= '', {\it measureVelocity}= False, {\it addGraphicsObject}= False, {\it drawDisplaced}= True, {\it color}= color4red)
\end{flushleft}
\setlength{\itemindent}{0.7cm}
\begin{itemize}[leftmargin=0.7cm]
\item[--]
{\bf function description}: \vspace{-6pt}
\begin{itemize}[leftmargin=1.2cm]
\setlength{\itemindent}{-0.7cm}
\item[]Function to create distance sensor based on GeneralContact in mbs; sensor can be either placed on absolute position or attached to rigid body marker; in case of marker, dirSensor is relative to the marker
\item[]- NOTE that this function is added to MainSystem via Python function CreateDistanceSensor.
\end{itemize}
\item[--]
{\bf input}: \vspace{-6pt}
\begin{itemize}[leftmargin=1.2cm]
\setlength{\itemindent}{-0.7cm}
\item[]{\it generalContactIndex}: the number of the GeneralContact object in mbs; the index of the GeneralContact object which has been added with last AddGeneralContact(...) command is generalContactIndex=mbs.NumberOfGeneralContacts()-1
\item[]{\it positionOrMarker}: either a 3D position as list or np.array, or a MarkerIndex with according rigid body marker
\item[]{\it dirSensor}: the direction (no need to normalize) along which the distance is measured (must not be normalized); in case of marker, the direction is relative to marker orientation if marker contains orientation (BodyRigid, NodeRigid)
\item[]{\it minDistance}: the minimum distance which is accepted; smaller distance will be ignored
\item[]{\it maxDistance}: the maximum distance which is accepted; items being at maxDistance or futher are ignored; if no items are found, the function returns maxDistance
\item[]{\it cylinderRadius}: in case of spheres (selectedTypeIndex=ContactTypeIndex.IndexSpheresMarkerBased), a cylinder can be used which measures the shortest distance at a certain radius (geometrically interpreted as cylinder)
\item[]{\it selectedTypeIndex}: either this type has default value, meaning that all items in GeneralContact are measured, or there is a specific type index, which is the only type that is considered during measurement
\item[]{\it storeInternal}: like with any SensorUserFunction, setting to True stores sensor data internally
\item[]{\it fileName}: if defined, recorded data of SensorUserFunction is written to specified file
\item[]{\it measureVelocity}: if True, the sensor measures additionally the velocity (component 0=distance, component 1=velocity); velocity is the velocity in direction 'dirSensor' and does not account for changes in geometry, thus it may be different from the time derivative of the distance!
\item[]{\it addGraphicsObject}: if True, the distance sensor is also visualized graphically in a simplified manner with a red line having the length of dirSensor; NOTE that updates are ONLY performed during computation, not in visualization; for this reason, solutionSettings.sensorsWritePeriod should be accordingly small
\item[]{\it drawDisplaced}: if True, the red line is drawn backwards such that it moves along the measured surface; if False, the beam is fixed to marker or position
\item[]{\it color}: optional color for 'laser beam' to be drawn
\end{itemize}
\item[--]
{\bf output}: SensorIndex; creates sensor and returns according sensor number of SensorUserFunction
\item[--]
{\bf notes}: use generalContactIndex = CreateDistanceSensorGeometry(...) before to create GeneralContact module containing geometry; old name: AddDistanceSensor(...)
\vspace{12pt}\end{itemize}
%

%
\noindent For examples on CreateDistanceSensor see Relevant Examples (Ex) and TestModels (TM) with weblink to github:
\bi
 \item \footnotesize \exuUrl{https://github.com/jgerstmayr/EXUDYN/blob/master/main/pythonDev/TestModels/distanceSensor.py}{\texttt{distanceSensor.py}} (TM), 
\exuUrl{https://github.com/jgerstmayr/EXUDYN/blob/master/main/pythonDev/TestModels/laserScannerTest.py}{\texttt{laserScannerTest.py}} (TM)
\ei

%
\begin{flushleft}
\noindent {def {\bf \exuUrl{https://github.com/jgerstmayr/EXUDYN/blob/master/main/pythonDev/exudyn/utilities.py\#L847}{DrawSystemGraph}{}}}\label{sec:mainsystemextensions:DrawSystemGraph}
({\it showLoads}= True, {\it showSensors}= True, {\it useItemNames}= False, {\it useItemTypes}= False, {\it addItemTypeNames}= True, {\it multiLine}= True, {\it fontSizeFactor}= 1., {\it layoutDistanceFactor}= 3., {\it layoutIterations}= 100, {\it showLegend}= True)
\end{flushleft}
\setlength{\itemindent}{0.7cm}
\begin{itemize}[leftmargin=0.7cm]
\item[--]
{\bf function description}: \vspace{-6pt}
\begin{itemize}[leftmargin=1.2cm]
\setlength{\itemindent}{-0.7cm}
\item[]helper function which draws system graph of a MainSystem (mbs); several options let adjust the appearance of the graph; the graph visualization uses randomizer, which results in different graphs after every run!
\item[]- NOTE that this function is added to MainSystem via Python function DrawSystemGraph.
\end{itemize}
\item[--]
{\bf input}: \vspace{-6pt}
\begin{itemize}[leftmargin=1.2cm]
\setlength{\itemindent}{-0.7cm}
\item[]{\it showLoads}: toggle appearance of loads in mbs
\item[]{\it showSensors}: toggle appearance of sensors in mbs
\item[]{\it useItemNames}: if True, object names are shown instead of basic object types (Node, Load, ...)
\item[]{\it useItemTypes}: if True, object type names (MassPoint, JointRevolute, ...) are shown instead of basic object types (Node, Load, ...); Note that Node, Object, is omitted at the beginning of itemName (as compared to theDoc.pdf); item classes become clear from the legend
\item[]{\it addItemTypeNames}: if True, type nymes (Node, Load, etc.) are added
\item[]{\it multiLine}: if True, labels are multiline, improving readability
\item[]{\it fontSizeFactor}: use this factor to scale fonts, allowing to fit larger graphs on the screen with values < 1
\item[]{\it showLegend}: shows legend for different item types
\item[]{\it layoutDistanceFactor}: this factor influences the arrangement of labels; larger distance values lead to circle-like results
\item[]{\it layoutIterations}: more iterations lead to better arrangement of the layout, but need more time for larger systems (use 1000-10000 to get good results)
\end{itemize}
\item[--]
{\bf output}: [Any, Any, Any]; returns [networkx, G, items] with nx being networkx, G the graph and item what is returned by nx.draw\_networkx\_labels(...)
\vspace{12pt}\end{itemize}
%

%
\noindent For examples on DrawSystemGraph see Relevant Examples (Ex) and TestModels (TM) with weblink to github:
\bi
 \item \footnotesize \exuUrl{https://github.com/jgerstmayr/EXUDYN/blob/master/main/pythonDev/Examples/fourBarMechanism3D.py}{\texttt{fourBarMechanism3D.py}} (Ex), 
\exuUrl{https://github.com/jgerstmayr/EXUDYN/blob/master/main/pythonDev/Examples/rigidBodyTutorial3.py}{\texttt{rigidBodyTutorial3.py}} (Ex), 
\exuUrl{https://github.com/jgerstmayr/EXUDYN/blob/master/main/pythonDev/Examples/rigidBodyTutorial3withMarkers.py}{\texttt{rigidBodyTutorial3withMarkers.py}} (Ex), 
\\ \exuUrl{https://github.com/jgerstmayr/EXUDYN/blob/master/main/pythonDev/TestModels/mainSystemExtensionsTests.py}{\texttt{mainSystemExtensionsTests.py}} (TM)
\ei

%



%++++++++++++++++++++
\mysubsubsection{MainSystem: Node}
\label{sec:mainsystem:node}



This section provides functions for adding, reading and modifying nodes. Nodes are used to define coordinates (unknowns to the static system and degrees of freedom if constraints are not present). Nodes can provide various types of coordinates for second/first order differential equations (ODE2/ODE1), algebraic equations (AE) and for data (history) variables -- which are not providing unknowns in the nonlinear solver but will be solved in an additional nonlinear iteration for e.g. contact, friction or plasticity.
\pythonstyle
\begin{lstlisting}[language=Python, firstnumber=1]

import exudyn as exu               #EXUDYN package including C++ core part
from exudyn.itemInterface import * #conversion of data to exudyn dictionaries
SC = exu.SystemContainer()         #container of systems
mbs = SC.AddSystem()               #add a new system to work with
nMP = mbs.AddNode(NodePoint2D(referenceCoordinates=[0,0]))
\end{lstlisting}

\begin{center}
\footnotesize
\begin{longtable}{| p{8cm} | p{8cm} |} 
\hline
{\bf function/structure name} & {\bf description}\\ \hline
  AddNode(pyObject) & add a node with nodeDefinition from Python node class; returns (global) node index (type NodeIndex) of newly added node; use int(nodeIndex) to convert to int, if needed (but not recommended in order not to mix up index types of nodes, objects, markers, ...)\tabnewline 
    \textcolor{steelblue}{{\bf EXAMPLE}: \tabnewline 
    \texttt{item = Rigid2D( referenceCoordinates= [1,0.5,0], initialVelocities= [10,0,0]) \tabnewline
    mbs.AddNode(item) \tabnewline
    nodeDict = \{{\textquotesingle}nodeType{\textquotesingle}: {\textquotesingle}Point{\textquotesingle}, \tabnewline
    {\textquotesingle}referenceCoordinates{\textquotesingle}: [1.0, 0.0, 0.0], \tabnewline
    {\textquotesingle}initialCoordinates{\textquotesingle}: [0.0, 2.0, 0.0], \tabnewline
    {\textquotesingle}name{\textquotesingle}: {\textquotesingle}example node{\textquotesingle}\} \tabnewline
    mbs.AddNode(nodeDict)}}\\ \hline 
  GetNodeNumber(nodeName) & get node's number by name (string)\tabnewline 
    \textcolor{steelblue}{{\bf EXAMPLE}: \tabnewline 
    \texttt{n = mbs.GetNodeNumber({\textquotesingle}example node{\textquotesingle})}}\\ \hline 
  GetNode(nodeNumber) & get node's dictionary by node number (type NodeIndex)\tabnewline 
    \textcolor{steelblue}{{\bf EXAMPLE}: \tabnewline 
    \texttt{nodeDict = mbs.GetNode(0)}}\\ \hline 
  ModifyNode(nodeNumber, nodeDict) & modify node's dictionary by node number (type NodeIndex)\tabnewline 
    \textcolor{steelblue}{{\bf EXAMPLE}: \tabnewline 
    \texttt{mbs.ModifyNode(nodeNumber, nodeDict)}}\\ \hline 
  GetNodeDefaults(typeName) & get node's default values for a certain nodeType as (dictionary)\tabnewline 
    \textcolor{steelblue}{{\bf EXAMPLE}: \tabnewline 
    \texttt{nodeType = {\textquotesingle}Point{\textquotesingle}\tabnewline
    nodeDict = mbs.GetNodeDefaults(nodeType)}}\\ \hline 
  GetNodeOutput(nodeNumber, variableType, configuration = exu.ConfigurationType.Current) & get the ouput of the node specified with the OutputVariableType; output may be scalar or array (e.g. displacement vector)\tabnewline 
    \textcolor{steelblue}{{\bf EXAMPLE}: \tabnewline 
    \texttt{mbs.GetNodeOutput(nodeNumber=0, variableType=exu.OutputVariableType.Displacement)}}\\ \hline 
  GetNodeODE2Index(nodeNumber) & get index in the global ODE2 coordinate vector for the first node coordinate of the specified node\tabnewline 
    \textcolor{steelblue}{{\bf EXAMPLE}: \tabnewline 
    \texttt{mbs.GetNodeODE2Index(nodeNumber=0)}}\\ \hline 
  GetNodeODE1Index(nodeNumber) & get index in the global ODE1 coordinate vector for the first node coordinate of the specified node\tabnewline 
    \textcolor{steelblue}{{\bf EXAMPLE}: \tabnewline 
    \texttt{mbs.GetNodeODE1Index(nodeNumber=0)}}\\ \hline 
  GetNodeAEIndex(nodeNumber) & get index in the global AE coordinate vector for the first node coordinate of the specified node\tabnewline 
    \textcolor{steelblue}{{\bf EXAMPLE}: \tabnewline 
    \texttt{mbs.GetNodeAEIndex(nodeNumber=0)}}\\ \hline 
  GetNodeParameter(nodeNumber, parameterName) & get nodes's parameter from node number (type NodeIndex) and parameterName; parameter names can be found for the specific items in the reference manual; for visualization parameters, use a 'V' as a prefix\tabnewline 
    \textcolor{steelblue}{{\bf EXAMPLE}: \tabnewline 
    \texttt{mbs.GetNodeParameter(0, {\textquotesingle}referenceCoordinates{\textquotesingle})}}\\ \hline 
  SetNodeParameter(nodeNumber, parameterName, value) & set parameter 'parameterName' of node with node number (type NodeIndex) to value; parameter names can be found for the specific items in the reference manual; for visualization parameters, use a 'V' as a prefix\tabnewline 
    \textcolor{steelblue}{{\bf EXAMPLE}: \tabnewline 
    \texttt{mbs.SetNodeParameter(0, {\textquotesingle}Vshow{\textquotesingle}, True)}}\\ \hline 
\end{longtable}
\end{center}

%++++++++++++++++++++
\mysubsubsection{MainSystem: Object}
\label{sec:mainsystem:object}



This section provides functions for adding, reading and modifying objects, which can be bodies (mass point, rigid body, finite element, ...), connectors (spring-damper or joint) or general objects. Objects provided terms to the residual of equations resulting from every coordinate given by the nodes. Single-noded objects (e.g.~mass point) provides exactly residual terms for its nodal coordinates. Connectors constrain or penalize two markers, which can be, e.g., position, rigid or coordinate markers. Thus, the dependence of objects is either on the coordinates of the marker-objects/nodes or on nodes which the objects possess themselves.
\pythonstyle
\begin{lstlisting}[language=Python, firstnumber=1]

import exudyn as exu               #EXUDYN package including C++ core part
from exudyn.itemInterface import * #conversion of data to exudyn dictionaries
SC = exu.SystemContainer()         #container of systems
mbs = SC.AddSystem()               #add a new system to work with
nMP = mbs.AddNode(NodePoint2D(referenceCoordinates=[0,0]))
mbs.AddObject(ObjectMassPoint2D(physicsMass=10, nodeNumber=nMP ))
\end{lstlisting}

\begin{center}
\footnotesize
\begin{longtable}{| p{8cm} | p{8cm} |} 
\hline
{\bf function/structure name} & {\bf description}\\ \hline
  AddObject(pyObject) & add an object with objectDefinition from Python object class; returns (global) object number (type ObjectIndex) of newly added object\tabnewline 
    \textcolor{steelblue}{{\bf EXAMPLE}: \tabnewline 
    \texttt{item = MassPoint(name={\textquotesingle}heavy object{\textquotesingle}, nodeNumber=0, physicsMass=100) \tabnewline
    mbs.AddObject(item) \tabnewline
    objectDict = \{{\textquotesingle}objectType{\textquotesingle}: {\textquotesingle}MassPoint{\textquotesingle}, \tabnewline
    {\textquotesingle}physicsMass{\textquotesingle}: 10, \tabnewline
    {\textquotesingle}nodeNumber{\textquotesingle}: 0, \tabnewline
    {\textquotesingle}name{\textquotesingle}: {\textquotesingle}example object{\textquotesingle}\} \tabnewline
    mbs.AddObject(objectDict)}}\\ \hline 
  GetObjectNumber(objectName) & get object's number by name (string)\tabnewline 
    \textcolor{steelblue}{{\bf EXAMPLE}: \tabnewline 
    \texttt{n = mbs.GetObjectNumber({\textquotesingle}heavy object{\textquotesingle})}}\\ \hline 
  GetObject(objectNumber, addGraphicsData = False) & get object's dictionary by object number (type ObjectIndex); NOTE: visualization parameters have a prefix 'V'; in order to also get graphicsData written, use addGraphicsData=True (which is by default False, as it would spoil the information)\tabnewline 
    \textcolor{steelblue}{{\bf EXAMPLE}: \tabnewline 
    \texttt{objectDict = mbs.GetObject(0)}}\\ \hline 
  ModifyObject(objectNumber, objectDict) & modify object's dictionary by object number (type ObjectIndex); NOTE: visualization parameters have a prefix 'V'\tabnewline 
    \textcolor{steelblue}{{\bf EXAMPLE}: \tabnewline 
    \texttt{mbs.ModifyObject(objectNumber, objectDict)}}\\ \hline 
  GetObjectDefaults(typeName) & get object's default values for a certain objectType as (dictionary)\tabnewline 
    \textcolor{steelblue}{{\bf EXAMPLE}: \tabnewline 
    \texttt{objectType = {\textquotesingle}MassPoint{\textquotesingle}\tabnewline
    objectDict = mbs.GetObjectDefaults(objectType)}}\\ \hline 
  GetObjectOutput(objectNumber, variableType, configuration = exu.ConfigurationType.Current) & get object's current output variable from object number (type ObjectIndex) and OutputVariableType; for connectors, it can only be computed for exu.ConfigurationType.Current configuration!\\ \hline 
  GetObjectOutputBody(objectNumber, variableType, localPosition = [0,0,0], configuration = exu.ConfigurationType.Current) & get body's output variable from object number (type ObjectIndex) and OutputVariableType, using the localPosition as defined in the body, and as used in MarkerBody and SensorBody\tabnewline 
    \textcolor{steelblue}{{\bf EXAMPLE}: \tabnewline 
    \texttt{u = mbs.GetObjectOutputBody(objectNumber = 1, variableType = exu.OutputVariableType.Position, localPosition=[1,0,0], configuration = exu.ConfigurationType.Initial)}}\\ \hline 
  GetObjectOutputSuperElement(objectNumber, variableType, meshNodeNumber, configuration = exu.ConfigurationType.Current) & get output variable from mesh node number of object with type SuperElement (GenericODE2, FFRF, FFRFreduced - CMS) with specific OutputVariableType; the meshNodeNumber is the object's local node number, not the global node number!\tabnewline 
    \textcolor{steelblue}{{\bf EXAMPLE}: \tabnewline 
    \texttt{u = mbs.GetObjectOutputSuperElement(objectNumber = 1, variableType = exu.OutputVariableType.Position, meshNodeNumber = 12, configuration = exu.ConfigurationType.Initial)}}\\ \hline 
  GetObjectParameter(objectNumber, parameterName) & get objects's parameter from object number (type ObjectIndex) and parameterName; parameter names can be found for the specific items in the reference manual; for visualization parameters, use a 'V' as a prefix; NOTE that BodyGraphicsData cannot be get or set, use dictionary access instead\tabnewline 
    \textcolor{steelblue}{{\bf EXAMPLE}: \tabnewline 
    \texttt{mbs.GetObjectParameter(objectNumber = 0, parameterName = {\textquotesingle}nodeNumber{\textquotesingle})}}\\ \hline 
  SetObjectParameter(objectNumber, parameterName, value) & set parameter 'parameterName' of object with object number (type ObjectIndex) to value;; parameter names can be found for the specific items in the reference manual; for visualization parameters, use a 'V' as a prefix; NOTE that BodyGraphicsData cannot be get or set, use dictionary access instead\tabnewline 
    \textcolor{steelblue}{{\bf EXAMPLE}: \tabnewline 
    \texttt{mbs.SetObjectParameter(objectNumber = 0, parameterName = {\textquotesingle}Vshow{\textquotesingle}, value=True)}}\\ \hline 
\end{longtable}
\end{center}

%++++++++++++++++++++
\mysubsubsection{MainSystem: Marker}
\label{sec:mainsystem:marker}



This section provides functions for adding, reading and modifying markers. Markers define how to measure primal kinematical quantities on objects or nodes (e.g., position, orientation or coordinates themselves), and how to act on the quantities which are dual to the kinematical quantities (e.g., force, torque and generalized forces). Markers provide unique interfaces for loads, sensors and constraints in order to address these quantities independently of the structure of the object or node (e.g., rigid or flexible body).
\pythonstyle
\begin{lstlisting}[language=Python, firstnumber=1]

import exudyn as exu               #EXUDYN package including C++ core part
from exudyn.itemInterface import * #conversion of data to exudyn dictionaries
SC = exu.SystemContainer()         #container of systems
mbs = SC.AddSystem()               #add a new system to work with
nMP = mbs.AddNode(NodePoint2D(referenceCoordinates=[0,0]))
mbs.AddObject(ObjectMassPoint2D(physicsMass=10, nodeNumber=nMP ))
mMP = mbs.AddMarker(MarkerNodePosition(nodeNumber = nMP))
\end{lstlisting}

\begin{center}
\footnotesize
\begin{longtable}{| p{8cm} | p{8cm} |} 
\hline
{\bf function/structure name} & {\bf description}\\ \hline
  AddMarker(pyObject) & add a marker with markerDefinition from Python marker class; returns (global) marker number (type MarkerIndex) of newly added marker\tabnewline 
    \textcolor{steelblue}{{\bf EXAMPLE}: \tabnewline 
    \texttt{item = MarkerNodePosition(name={\textquotesingle}my marker{\textquotesingle},nodeNumber=1) \tabnewline
    mbs.AddMarker(item)\tabnewline
    markerDict = \{{\textquotesingle}markerType{\textquotesingle}: {\textquotesingle}NodePosition{\textquotesingle}, \tabnewline
      {\textquotesingle}nodeNumber{\textquotesingle}: 0, \tabnewline
      {\textquotesingle}name{\textquotesingle}: {\textquotesingle}position0{\textquotesingle}\}\tabnewline
    mbs.AddMarker(markerDict)}}\\ \hline 
  GetMarkerNumber(markerName) & get marker's number by name (string)\tabnewline 
    \textcolor{steelblue}{{\bf EXAMPLE}: \tabnewline 
    \texttt{n = mbs.GetMarkerNumber({\textquotesingle}my marker{\textquotesingle})}}\\ \hline 
  GetMarker(markerNumber) & get marker's dictionary by index\tabnewline 
    \textcolor{steelblue}{{\bf EXAMPLE}: \tabnewline 
    \texttt{markerDict = mbs.GetMarker(0)}}\\ \hline 
  ModifyMarker(markerNumber, markerDict) & modify marker's dictionary by index\tabnewline 
    \textcolor{steelblue}{{\bf EXAMPLE}: \tabnewline 
    \texttt{mbs.ModifyMarker(markerNumber, markerDict)}}\\ \hline 
  GetMarkerDefaults(typeName) & get marker's default values for a certain markerType as (dictionary)\tabnewline 
    \textcolor{steelblue}{{\bf EXAMPLE}: \tabnewline 
    \texttt{markerType = {\textquotesingle}NodePosition{\textquotesingle}\tabnewline
    markerDict = mbs.GetMarkerDefaults(markerType)}}\\ \hline 
  GetMarkerParameter(markerNumber, parameterName) & get markers's parameter from markerNumber and parameterName; parameter names can be found for the specific items in the reference manual\\ \hline 
  SetMarkerParameter(markerNumber, parameterName, value) & set parameter 'parameterName' of marker with markerNumber to value; parameter names can be found for the specific items in the reference manual\\ \hline 
  GetMarkerOutput(markerNumber, variableType, configuration = exu.ConfigurationType.Current) & get the ouput of the marker specified with the OutputVariableType; currently only provides Displacement, Position and Velocity for position based markers, and RotationMatrix, Rotation and AngularVelocity(Local) for markers providing orientation; Coordinates and Coordinates\_t available for coordinate markers\tabnewline 
    \textcolor{steelblue}{{\bf EXAMPLE}: \tabnewline 
    \texttt{mbs.GetMarkerOutput(markerNumber=0, variableType=exu.OutputVariableType.Position)}}\\ \hline 
\end{longtable}
\end{center}

%++++++++++++++++++++
\mysubsubsection{MainSystem: Load}
\label{sec:mainsystem:load}



This section provides functions for adding, reading and modifying operating loads. Loads are used to act on the quantities which are dual to the primal kinematic quantities, such as displacement and rotation. Loads represent, e.g., forces, torques or generalized forces.
\pythonstyle
\begin{lstlisting}[language=Python, firstnumber=1]

import exudyn as exu               #EXUDYN package including C++ core part
from exudyn.itemInterface import * #conversion of data to exudyn dictionaries
SC = exu.SystemContainer()         #container of systems
mbs = SC.AddSystem()               #add a new system to work with
nMP = mbs.AddNode(NodePoint2D(referenceCoordinates=[0,0]))
mbs.AddObject(ObjectMassPoint2D(physicsMass=10, nodeNumber=nMP ))
mMP = mbs.AddMarker(MarkerNodePosition(nodeNumber = nMP))
mbs.AddLoad(Force(markerNumber = mMP, loadVector=[0.001,0,0]))
\end{lstlisting}

\begin{center}
\footnotesize
\begin{longtable}{| p{8cm} | p{8cm} |} 
\hline
{\bf function/structure name} & {\bf description}\\ \hline
  AddLoad(pyObject) & add a load with loadDefinition from Python load class; returns (global) load number (type LoadIndex) of newly added load\tabnewline 
    \textcolor{steelblue}{{\bf EXAMPLE}: \tabnewline 
    \texttt{item = mbs.AddLoad(LoadForceVector(loadVector=[1,0,0], markerNumber=0, name={\textquotesingle}heavy load{\textquotesingle})) \tabnewline
    mbs.AddLoad(item)\tabnewline
    loadDict = \{{\textquotesingle}loadType{\textquotesingle}: {\textquotesingle}ForceVector{\textquotesingle},\tabnewline
      {\textquotesingle}markerNumber{\textquotesingle}: 0,\tabnewline
      {\textquotesingle}loadVector{\textquotesingle}: [1.0, 0.0, 0.0],\tabnewline
      {\textquotesingle}name{\textquotesingle}: {\textquotesingle}heavy load{\textquotesingle}\} \tabnewline
    mbs.AddLoad(loadDict)}}\\ \hline 
  GetLoadNumber(loadName) & get load's number by name (string)\tabnewline 
    \textcolor{steelblue}{{\bf EXAMPLE}: \tabnewline 
    \texttt{n = mbs.GetLoadNumber({\textquotesingle}heavy load{\textquotesingle})}}\\ \hline 
  GetLoad(loadNumber) & get load's dictionary by index\tabnewline 
    \textcolor{steelblue}{{\bf EXAMPLE}: \tabnewline 
    \texttt{loadDict = mbs.GetLoad(0)}}\\ \hline 
  ModifyLoad(loadNumber, loadDict) & modify load's dictionary by index\tabnewline 
    \textcolor{steelblue}{{\bf EXAMPLE}: \tabnewline 
    \texttt{mbs.ModifyLoad(loadNumber, loadDict)}}\\ \hline 
  GetLoadDefaults(typeName) & get load's default values for a certain loadType as (dictionary)\tabnewline 
    \textcolor{steelblue}{{\bf EXAMPLE}: \tabnewline 
    \texttt{loadType = {\textquotesingle}ForceVector{\textquotesingle}\tabnewline
    loadDict = mbs.GetLoadDefaults(loadType)}}\\ \hline 
  GetLoadValues(loadNumber) & Get current load values, specifically if user-defined loads are used; can be scalar or vector-valued return value\\ \hline 
  GetLoadParameter(loadNumber, parameterName) & get loads's parameter from loadNumber and parameterName; parameter names can be found for the specific items in the reference manual\\ \hline 
  SetLoadParameter(loadNumber, parameterName, value) & set parameter 'parameterName' of load with loadNumber to value; parameter names can be found for the specific items in the reference manual\\ \hline 
\end{longtable}
\end{center}

%++++++++++++++++++++
\mysubsubsection{MainSystem: Sensor}
\label{sec:mainsystem:sensor}



This section provides functions for adding, reading and modifying operating sensors. Sensors are used to measure information in nodes, objects, markers, and loads for output in a file.
\pythonstyle
\begin{lstlisting}[language=Python, firstnumber=1]

import exudyn as exu               #EXUDYN package including C++ core part
from exudyn.itemInterface import * #conversion of data to exudyn dictionaries
SC = exu.SystemContainer()         #container of systems
mbs = SC.AddSystem()               #add a new system to work with
nMP = mbs.AddNode(NodePoint(referenceCoordinates=[0,0,0]))
mbs.AddObject(ObjectMassPoint(physicsMass=10, nodeNumber=nMP ))
mMP = mbs.AddMarker(MarkerNodePosition(nodeNumber = nMP))
mbs.AddLoad(Force(markerNumber = mMP, loadVector=[2,0,5]))
sMP = mbs.AddSensor(SensorNode(nodeNumber=nMP, storeInternal=True,
                               outputVariableType=exu.OutputVariableType.Position))
mbs.Assemble()
exu.SolveDynamic(mbs, exu.SimulationSettings())
from exudyn.plot import PlotSensor
PlotSensor(mbs, sMP, components=[0,1,2])
\end{lstlisting}

\begin{center}
\footnotesize
\begin{longtable}{| p{8cm} | p{8cm} |} 
\hline
{\bf function/structure name} & {\bf description}\\ \hline
  AddSensor(pyObject) & add a sensor with sensor definition from Python sensor class; returns (global) sensor number (type SensorIndex) of newly added sensor\tabnewline 
    \textcolor{steelblue}{{\bf EXAMPLE}: \tabnewline 
    \texttt{item = mbs.AddSensor(SensorNode(sensorType= exu.SensorType.Node, nodeNumber=0, name={\textquotesingle}test sensor{\textquotesingle})) \tabnewline
    mbs.AddSensor(item)\tabnewline
    sensorDict = \{{\textquotesingle}sensorType{\textquotesingle}: {\textquotesingle}Node{\textquotesingle},\tabnewline
      {\textquotesingle}nodeNumber{\textquotesingle}: 0,\tabnewline
      {\textquotesingle}fileName{\textquotesingle}: {\textquotesingle}sensor.txt{\textquotesingle},\tabnewline
      {\textquotesingle}name{\textquotesingle}: {\textquotesingle}test sensor{\textquotesingle}\} \tabnewline
    mbs.AddSensor(sensorDict)}}\\ \hline 
  GetSensorNumber(sensorName) & get sensor's number by name (string)\tabnewline 
    \textcolor{steelblue}{{\bf EXAMPLE}: \tabnewline 
    \texttt{n = mbs.GetSensorNumber({\textquotesingle}test sensor{\textquotesingle})}}\\ \hline 
  GetSensor(sensorNumber) & get sensor's dictionary by index\tabnewline 
    \textcolor{steelblue}{{\bf EXAMPLE}: \tabnewline 
    \texttt{sensorDict = mbs.GetSensor(0)}}\\ \hline 
  ModifySensor(sensorNumber, sensorDict) & modify sensor's dictionary by index\tabnewline 
    \textcolor{steelblue}{{\bf EXAMPLE}: \tabnewline 
    \texttt{mbs.ModifySensor(sensorNumber, sensorDict)}}\\ \hline 
  GetSensorDefaults(typeName) & get sensor's default values for a certain sensorType as (dictionary)\tabnewline 
    \textcolor{steelblue}{{\bf EXAMPLE}: \tabnewline 
    \texttt{sensorType = {\textquotesingle}Node{\textquotesingle}\tabnewline
    sensorDict = mbs.GetSensorDefaults(sensorType)}}\\ \hline 
  GetSensorValues(sensorNumber, configuration = exu.ConfigurationType.Current) & get sensors's values for configuration; can be a scalar or vector-valued return value!\\ \hline 
  GetSensorStoredData(sensorNumber) & get sensors's internally stored data as matrix (all time points stored); rows are containing time and sensor values as obtained by sensor (e.g., time, and x, y, and z value of position)\\ \hline 
  GetSensorParameter(sensorNumber, parameterName) & get sensors's parameter from sensorNumber and parameterName; parameter names can be found for the specific items in the reference manual\\ \hline 
  SetSensorParameter(sensorNumber, parameterName, value) & set parameter 'parameterName' of sensor with sensorNumber to value; parameter names can be found for the specific items in the reference manual\\ \hline 
\end{longtable}
\end{center}

%++++++++++++++++++++
\mysubsection{SystemData}
\label{sec:mbs:systemData}



This is the data structure of a system which contains Objects (bodies/constraints/...), Nodes, Markers and Loads. The SystemData structure allows advanced access to this data, which HAS TO BE USED WITH CARE, as unexpected results and system crash might happen.
\pythonstyle
\begin{lstlisting}[language=Python, firstnumber=1]

import exudyn as exu               #EXUDYN package including C++ core part
from exudyn.itemInterface import * #conversion of data to exudyn dictionaries
SC = exu.SystemContainer()         #container of systems
mbs = SC.AddSystem()               #add a new system to work with
nMP = mbs.AddNode(NodePoint(referenceCoordinates=[0,0,0]))
mbs.AddObject(ObjectMassPoint(physicsMass=10, nodeNumber=nMP ))
mMP = mbs.AddMarker(MarkerNodePosition(nodeNumber = nMP))
mbs.AddLoad(Force(markerNumber = mMP, loadVector=[2,0,5]))
mbs.Assemble()
exu.SolveDynamic(mbs, exu.SimulationSettings())

#obtain current ODE2 system vector (e.g. after static simulation finished):
u = mbs.systemData.GetODE2Coordinates()
#set initial ODE2 vector for next simulation:
mbs.systemData.SetODE2Coordinates(coordinates=u,
               configuration=exu.ConfigurationType.Initial)
#print detailed information on items:
mbs.systemData.Info()
#print LTG lists for objects and loads:
mbs.systemData.InfoLTG()
\end{lstlisting}

\begin{center}
\footnotesize
\begin{longtable}{| p{8cm} | p{8cm} |} 
\hline
{\bf function/structure name} & {\bf description}\\ \hline
  NumberOfLoads() & return number of loads in system\tabnewline 
    \textcolor{steelblue}{{\bf EXAMPLE}: \tabnewline 
    \texttt{print(mbs.systemData.NumberOfLoads())}}\\ \hline 
  NumberOfMarkers() & return number of markers in system\tabnewline 
    \textcolor{steelblue}{{\bf EXAMPLE}: \tabnewline 
    \texttt{print(mbs.systemData.NumberOfMarkers())}}\\ \hline 
  NumberOfNodes() & return number of nodes in system\tabnewline 
    \textcolor{steelblue}{{\bf EXAMPLE}: \tabnewline 
    \texttt{print(mbs.systemData.NumberOfNodes())}}\\ \hline 
  NumberOfObjects() & return number of objects in system\tabnewline 
    \textcolor{steelblue}{{\bf EXAMPLE}: \tabnewline 
    \texttt{print(mbs.systemData.NumberOfObjects())}}\\ \hline 
  NumberOfSensors() & return number of sensors in system\tabnewline 
    \textcolor{steelblue}{{\bf EXAMPLE}: \tabnewline 
    \texttt{print(mbs.systemData.NumberOfSensors())}}\\ \hline 
  ODE2Size(configurationType = exu.ConfigurationType.Current) & get size of ODE2 coordinate vector for given configuration (only works correctly after mbs.Assemble() )\tabnewline 
    \textcolor{steelblue}{{\bf EXAMPLE}: \tabnewline 
    \texttt{print({\textquotesingle}ODE2 size={\textquotesingle},mbs.systemData.ODE2Size())}}\\ \hline 
  ODE1Size(configurationType = exu.ConfigurationType.Current) & get size of ODE1 coordinate vector for given configuration (only works correctly after mbs.Assemble() )\tabnewline 
    \textcolor{steelblue}{{\bf EXAMPLE}: \tabnewline 
    \texttt{print({\textquotesingle}ODE1 size={\textquotesingle},mbs.systemData.ODE1Size())}}\\ \hline 
  AEsize(configurationType = exu.ConfigurationType.Current) & get size of AE coordinate vector for given configuration (only works correctly after mbs.Assemble() )\tabnewline 
    \textcolor{steelblue}{{\bf EXAMPLE}: \tabnewline 
    \texttt{print({\textquotesingle}AE size={\textquotesingle},mbs.systemData.AEsize())}}\\ \hline 
  DataSize(configurationType = exu.ConfigurationType.Current) & get size of Data coordinate vector for given configuration (only works correctly after mbs.Assemble() )\tabnewline 
    \textcolor{steelblue}{{\bf EXAMPLE}: \tabnewline 
    \texttt{print({\textquotesingle}Data size={\textquotesingle},mbs.systemData.DataSize())}}\\ \hline 
  SystemSize(configurationType = exu.ConfigurationType.Current) & get size of System coordinate vector for given configuration (only works correctly after mbs.Assemble() )\tabnewline 
    \textcolor{steelblue}{{\bf EXAMPLE}: \tabnewline 
    \texttt{print({\textquotesingle}System size={\textquotesingle},mbs.systemData.SystemSize())}}\\ \hline 
  GetTime(configurationType = exu.ConfigurationType.Current) & get configuration dependent time.\tabnewline 
    \textcolor{steelblue}{{\bf EXAMPLE}: \tabnewline 
    \texttt{mbs.systemData.GetTime(exu.ConfigurationType.Initial)}}\\ \hline 
  SetTime(newTime, configurationType = exu.ConfigurationType.Current) & set configuration dependent time; use this access with care, e.g. in user-defined solvers.\tabnewline 
    \textcolor{steelblue}{{\bf EXAMPLE}: \tabnewline 
    \texttt{mbs.systemData.SetTime(10., exu.ConfigurationType.Initial)}}\\ \hline 
  AddODE2LoadDependencies(loadNumber, globalODE2coordinates) & advanced function for adding special dependencies of loads onto ODE2 coordinates, taking a list / numpy array of global ODE2 coordinates; this function needs to be called after Assemble() and needs to contain global ODE2 coordinate indices; this list only affects implicit or static solvers if timeIntegration.computeLoadsJacobian or staticSolver.computeLoadsJacobian is set to 1 (ODE2) or 2 (ODE2 and ODE2\_t dependencies); if set, it may greatly improve convergence if loads with user functions depend on some system states, such as in a load with feedback control loop; the additional dependencies are not required, if doSystemWideDifferentiation=True, however the latter option being much less efficient. For more details, consider the file doublePendulum2DControl.py in the examples directory.\tabnewline 
    \textcolor{steelblue}{{\bf EXAMPLE}: \tabnewline 
    \texttt{mbs.systemData.AddODE2LoadDependencies(0,[0,1,2])\tabnewline
    \#add dependency of load 5 onto node 2 coordinates:\tabnewline
    nodeLTG2 = mbs.systemData.GetNodeLTGODE2(2)\tabnewline
    mbs.systemData.AddODE2LoadDependencies(5,nodeLTG2)}}\\ \hline 
  Info() & print detailed information on every item; for short information use print(mbs)\tabnewline 
    \textcolor{steelblue}{{\bf EXAMPLE}: \tabnewline 
    \texttt{mbs.systemData.Info()}}\\ \hline 
  InfoLTG() & print LTG information of objects and load dependencies\tabnewline 
    \textcolor{steelblue}{{\bf EXAMPLE}: \tabnewline 
    \texttt{mbs.systemData.InfoLTG()}}\\ \hline 
\end{longtable}
\end{center}

%++++++++++++++++++++
\mysubsubsection{SystemData: Access coordinates}
\label{sec:mbs:systemData:coordinates}



This section provides access functions to global coordinate vectors. Assigning invalid values or using wrong vector size might lead to system crash and unexpected results.
\begin{center}
\footnotesize
\begin{longtable}{| p{8cm} | p{8cm} |} 
\hline
{\bf function/structure name} & {\bf description}\\ \hline
  GetODE2Coordinates(configuration = exu.ConfigurationType.Current) & get ODE2 system coordinates (displacements) for given configuration (default: exu.Configuration.Current)\tabnewline 
    \textcolor{steelblue}{{\bf EXAMPLE}: \tabnewline 
    \texttt{uCurrent = mbs.systemData.GetODE2Coordinates()}}\\ \hline 
  SetODE2Coordinates(coordinates, configuration = exu.ConfigurationType.Current) & set ODE2 system coordinates (displacements) for given configuration (default: exu.Configuration.Current); invalid vector size may lead to system crash!\tabnewline 
    \textcolor{steelblue}{{\bf EXAMPLE}: \tabnewline 
    \texttt{mbs.systemData.SetODE2Coordinates(uCurrent)}}\\ \hline 
  GetODE2Coordinates\_t(configuration = exu.ConfigurationType.Current) & get ODE2 system coordinates (velocities) for given configuration (default: exu.Configuration.Current)\tabnewline 
    \textcolor{steelblue}{{\bf EXAMPLE}: \tabnewline 
    \texttt{vCurrent = mbs.systemData.GetODE2Coordinates\_t()}}\\ \hline 
  SetODE2Coordinates\_t(coordinates, configuration = exu.ConfigurationType.Current) & set ODE2 system coordinates (velocities) for given configuration (default: exu.Configuration.Current); invalid vector size may lead to system crash!\tabnewline 
    \textcolor{steelblue}{{\bf EXAMPLE}: \tabnewline 
    \texttt{mbs.systemData.SetODE2Coordinates\_t(vCurrent)}}\\ \hline 
  GetODE2Coordinates\_tt(configuration = exu.ConfigurationType.Current) & get ODE2 system coordinates (accelerations) for given configuration (default: exu.Configuration.Current)\tabnewline 
    \textcolor{steelblue}{{\bf EXAMPLE}: \tabnewline 
    \texttt{vCurrent = mbs.systemData.GetODE2Coordinates\_tt()}}\\ \hline 
  SetODE2Coordinates\_tt(coordinates, configuration = exu.ConfigurationType.Current) & set ODE2 system coordinates (accelerations) for given configuration (default: exu.Configuration.Current); invalid vector size may lead to system crash!\tabnewline 
    \textcolor{steelblue}{{\bf EXAMPLE}: \tabnewline 
    \texttt{mbs.systemData.SetODE2Coordinates\_tt(aCurrent)}}\\ \hline 
  GetODE1Coordinates(configuration = exu.ConfigurationType.Current) & get ODE1 system coordinates (displacements) for given configuration (default: exu.Configuration.Current)\tabnewline 
    \textcolor{steelblue}{{\bf EXAMPLE}: \tabnewline 
    \texttt{qCurrent = mbs.systemData.GetODE1Coordinates()}}\\ \hline 
  SetODE1Coordinates(coordinates, configuration = exu.ConfigurationType.Current) & set ODE1 system coordinates (velocities) for given configuration (default: exu.Configuration.Current); invalid vector size may lead to system crash!\tabnewline 
    \textcolor{steelblue}{{\bf EXAMPLE}: \tabnewline 
    \texttt{mbs.systemData.SetODE1Coordinates\_t(qCurrent)}}\\ \hline 
  GetODE1Coordinates\_t(configuration = exu.ConfigurationType.Current) & get ODE1 system coordinates (velocities) for given configuration (default: exu.Configuration.Current)\tabnewline 
    \textcolor{steelblue}{{\bf EXAMPLE}: \tabnewline 
    \texttt{qCurrent = mbs.systemData.GetODE1Coordinates\_t()}}\\ \hline 
  SetODE1Coordinates\_t(coordinates, configuration = exu.ConfigurationType.Current) & set ODE1 system coordinates (displacements) for given configuration (default: exu.Configuration.Current); invalid vector size may lead to system crash!\tabnewline 
    \textcolor{steelblue}{{\bf EXAMPLE}: \tabnewline 
    \texttt{mbs.systemData.SetODE1Coordinates(qCurrent)}}\\ \hline 
  GetAECoordinates(configuration = exu.ConfigurationType.Current) & get algebraic equations (AE) system coordinates for given configuration (default: exu.Configuration.Current)\tabnewline 
    \textcolor{steelblue}{{\bf EXAMPLE}: \tabnewline 
    \texttt{lambdaCurrent = mbs.systemData.GetAECoordinates()}}\\ \hline 
  SetAECoordinates(coordinates, configuration = exu.ConfigurationType.Current) & set algebraic equations (AE) system coordinates for given configuration (default: exu.Configuration.Current); invalid vector size may lead to system crash!\tabnewline 
    \textcolor{steelblue}{{\bf EXAMPLE}: \tabnewline 
    \texttt{mbs.systemData.SetAECoordinates(lambdaCurrent)}}\\ \hline 
  GetDataCoordinates(configuration = exu.ConfigurationType.Current) & get system data coordinates for given configuration (default: exu.Configuration.Current)\tabnewline 
    \textcolor{steelblue}{{\bf EXAMPLE}: \tabnewline 
    \texttt{dataCurrent = mbs.systemData.GetDataCoordinates()}}\\ \hline 
  SetDataCoordinates(coordinates, configuration = exu.ConfigurationType.Current) & set system data coordinates for given configuration (default: exu.Configuration.Current); invalid vector size may lead to system crash!\tabnewline 
    \textcolor{steelblue}{{\bf EXAMPLE}: \tabnewline 
    \texttt{mbs.systemData.SetDataCoordinates(dataCurrent)}}\\ \hline 
  GetSystemState(configuration = exu.ConfigurationType.Current) & get system state for given configuration (default: exu.Configuration.Current); state vectors do not include the non-state derivatives ODE1\_t and ODE2\_tt and the time; function is copying data - not highly efficient; format of pyList: [ODE2Coords, ODE2Coords\_t, ODE1Coords, AEcoords, dataCoords]\tabnewline 
    \textcolor{steelblue}{{\bf EXAMPLE}: \tabnewline 
    \texttt{sysStateList = mbs.systemData.GetSystemState()}}\\ \hline 
  SetSystemState(systemStateList, configuration = exu.ConfigurationType.Current) & set system data coordinates for given configuration (default: exu.Configuration.Current); invalid list of vectors / vector size may lead to system crash; write access to state vectors (but not the non-state derivatives ODE1\_t and ODE2\_tt and the time); function is copying data - not highly efficient; format of pyList: [ODE2Coords, ODE2Coords\_t, ODE1Coords, AEcoords, dataCoords]\tabnewline 
    \textcolor{steelblue}{{\bf EXAMPLE}: \tabnewline 
    \texttt{mbs.systemData.SetSystemState(sysStateList, configuration = exu.ConfigurationType.Initial)}}\\ \hline 
\end{longtable}
\end{center}

%++++++++++++++++++++
\mysubsubsection{SystemData: Get object LTG coordinate mappings}
\label{sec:systemData:ObjectLTG}



This section provides access functions the \ac{LTG}-lists for every object (body, constraint, ...) in the system. For details on the \ac{LTG} mapping, see \refSection{sec:overview:ltgmapping}.
\begin{center}
\footnotesize
\begin{longtable}{| p{8cm} | p{8cm} |} 
\hline
{\bf function/structure name} & {\bf description}\\ \hline
  GetObjectLTGODE2(objectNumber) & get object local-to-global coordinate mapping (list of global coordinate indices) for ODE2 coordinates; only available after Assemble()\tabnewline 
    \textcolor{steelblue}{{\bf EXAMPLE}: \tabnewline 
    \texttt{ltgObject4 = mbs.systemData.GetObjectLTGODE2(4)}}\\ \hline 
  GetObjectLTGODE1(objectNumber) & get object local-to-global coordinate mapping (list of global coordinate indices) for ODE1 coordinates; only available after Assemble()\tabnewline 
    \textcolor{steelblue}{{\bf EXAMPLE}: \tabnewline 
    \texttt{ltgObject4 = mbs.systemData.GetObjectLTGODE1(4)}}\\ \hline 
  GetObjectLTGAE(objectNumber) & get object local-to-global coordinate mapping (list of global coordinate indices) for algebraic equations (AE) coordinates; only available after Assemble()\tabnewline 
    \textcolor{steelblue}{{\bf EXAMPLE}: \tabnewline 
    \texttt{ltgObject4 = mbs.systemData.GetObjectLTGAE(4)}}\\ \hline 
  GetObjectLTGData(objectNumber) & get object local-to-global coordinate mapping (list of global coordinate indices) for data coordinates; only available after Assemble()\tabnewline 
    \textcolor{steelblue}{{\bf EXAMPLE}: \tabnewline 
    \texttt{ltgObject4 = mbs.systemData.GetObjectLTGData(4)}}\\ \hline 
  GetNodeLTGODE2(nodeNumber) & get node local-to-global coordinate mapping (list of global coordinate indices) for ODE2 coordinates; only available after Assemble()\tabnewline 
    \textcolor{steelblue}{{\bf EXAMPLE}: \tabnewline 
    \texttt{ltgNode4 = mbs.systemData.GetNodeLTGODE2(4)}}\\ \hline 
  GetNodeLTGODE1(nodeNumber) & get node local-to-global coordinate mapping (list of global coordinate indices) for ODE1 coordinates; only available after Assemble()\tabnewline 
    \textcolor{steelblue}{{\bf EXAMPLE}: \tabnewline 
    \texttt{ltgNode4 = mbs.systemData.GetNodeLTGODE1(4)}}\\ \hline 
  GetNodeLTGAE(nodeNumber) & get node local-to-global coordinate mapping (list of global coordinate indices) for AE coordinates; only available after Assemble()\tabnewline 
    \textcolor{steelblue}{{\bf EXAMPLE}: \tabnewline 
    \texttt{ltgNode4 = mbs.systemData.GetNodeLTGAE(4)}}\\ \hline 
  GetNodeLTGData(nodeNumber) & get node local-to-global coordinate mapping (list of global coordinate indices) for Data coordinates; only available after Assemble()\tabnewline 
    \textcolor{steelblue}{{\bf EXAMPLE}: \tabnewline 
    \texttt{ltgNode4 = mbs.systemData.GetNodeLTGData(4)}}\\ \hline 
\end{longtable}
\end{center}

\mysubsection{Symbolic}
\label{sec:cinterface:symbolic}
The Symbolic sub-module in \texttt{exudyn.symbolic} allows limited symbolic manipulations in \codeName\ and is currently under development In particular, symbolic user functions can be created, which allow significant speedup of Python user functions. However, \mybold{always veryfy your symbolic expressions or user functions}, as behavior may be unexpected in some cases. 

%++++++++++++++++++++
\mysubsubsection{symbolic.Real}



The symbolic Real type allows to replace Python's float by a symbolic quantity. The \texttt{symbolic.Real} may be directly set to a float and be evaluated as float. However, turing on recording by using 	exttt{exudyn.symbolic.SetRecording(True)} (on by default), results are stored as expression trees, which may be evaluated in C++ or Python, in particular in user functions, see the following example:
\pythonstyle
\begin{lstlisting}[language=Python, firstnumber=1]

import exudyn as exu
esym = exu.symbolic     #abbreviation
SymReal = esym.Real     #abbreviation

#create some variables
a = SymReal('a',42.)    #use named expression
b = SymReal(13)         #b is 13
c = a+b*7.+1.-3         #c stores expression tree
d = c                   #d and c are containing same tree!
print('a: ',a,' = ',a.Evaluate())
print('c: ',c,' = ',c.Evaluate())

#use special functions:
d = a+b*esym.sin(a)+esym.cos(SymReal(7))
print('d: ',d,' = ',d.Evaluate())

a.SetValue(14)          #variable a set to new value; influences d
print('d: ',d,' = ',d.Evaluate())

a = SymReal(1000)       #a is now a new variable; not updated in d!
print('d: ',d,' = ',d.Evaluate())

#compute derivatives (automatic differentiation):
x = SymReal("x",0.5)
f = a+b*esym.sin(x)+esym.cos(SymReal(7))+x**4
print('f=',f.Evaluate(), ', diff=',f.Diff(x))

#turn off recording of trees (globally for all symbolic.Real!):
esym.SetRecording(False)
x = SymReal(42) #now, only represents a value
y = x/3.       #directly evaluates to 14
\end{lstlisting}


To create a symbolic Real, use \texttt{aa=symbolic.Real(1.23)} to build a Python object aa with value 1.23. In order to use a named value, use \texttt{pi=symbolic.Real('pi',3.14)}. Note that in the following, we use the abbreviation \texttt{SymReal=exudyn.symbolic.Real}. Member functions of \texttt{SymReal}, which are \mybold{not recorded}, are:
\begin{center}
\footnotesize
\begin{longtable}{| p{8cm} | p{8cm} |} 
\hline
{\bf function/structure name} & {\bf description}\\ \hline
  \_\_init\_\_(value) & Construct symbolic.Real from float.\\ \hline 
  \_\_init\_\_(name, value) & Construct named symbolic.Real from name and float.\\ \hline 
  SetValue(valueInit) & Set either internal float value or value of named expression; cannot change symbolic expressions.\tabnewline 
    \textcolor{steelblue}{{\bf EXAMPLE}: \tabnewline 
    \texttt{b = SymReal(13)\tabnewline
    b.SetValue(14) \#now b is 14\tabnewline
    \#b.SetValue(a+3.) \#not possible!}}\\ \hline 
  Evaluate() & return evaluated expression (prioritized) or stored Real value.\\ \hline 
  Diff(var) & (UNTESTED!) return derivative of stored expression with respect to given symbolic named variable; NOTE: when defining the expression of the variable which shall be differentiated, the variable may only be changed with the SetValue(...) method hereafter!\tabnewline 
    \textcolor{steelblue}{{\bf EXAMPLE}: \tabnewline 
    \texttt{x=SymReal({\textquotesingle}x{\textquotesingle},2)\tabnewline
    f=3*x+x**2*sin(x)\tabnewline
    f.Diff(x) \#evaluate derivative w.r.t. x}}\\ \hline 
  value & access to internal float value, which is used in case that symbolic.Real has been built from a float (but without a name and without symbolic expression)\\ \hline  
  operator \_\_float\_\_() & evaluation of expression and conversion of symbolic.Real to Python float\\ \hline  
  operator \_\_str\_\_() & conversion of symbolic.Real to string\\ \hline  
  operator \_\_repr\_\_() & representation of symbolic.Real in Python\\ \hline  
\end{longtable}
\end{center}

The remaining operators and mathematical functions are recorded within expressions. Main mathematical operators for \texttt{SymReal} exist, similar to Python, such as:
\pythonstyle
\begin{lstlisting}[language=Python, firstnumber=1]

a = SymReal(1)
b = SymReal(2)

r1 = a+b
r1 = a-b
r1 = a*b
r1 = a/b
r1 = -a
r1 = a**b

c = SymReal(3.3)
c += b
c -= b
c *= b
c /= b

c = (a == b)
c = (a != b)
c = (a < b)
c = (a > b)
c = (a <= b)
c = (a >= b)

#in most cases, we can also mix with float:
c = a*7 + SymReal.sin(8)
\end{lstlisting}


Mathematical functions may be called with an \texttt{SymReal} or with a \texttt{float}. Most standard mathematical functions exist for \texttt{symbolic}, e.g., as \texttt{symbolic.abs}. \mybold{HINT}: function names are lower-case for compatibility with Python's math library. Thus, you can easily exchange math.sin with esym.sin, and you may want to use a generic name, such as myMath=symbolic in order to switch between Python and symbolic user functions. The following functions exist:
\begin{center}
\footnotesize
\begin{longtable}{| p{8cm} | p{8cm} |} 
\hline
{\bf function/structure name} & {\bf description}\\ \hline
  isfinite(x) & according to specification of C++ std::isfinite\\ \hline 
  abs(x) & according to specification of C++ std::fabs\\ \hline 
  round(x) & according to specification of C++ std::round\\ \hline 
  ceil(x) & according to specification of C++ std::ceil\\ \hline 
  floor(x) & according to specification of C++ std::floor\\ \hline 
  sqrt(x) & according to specification of C++ std::sqrt\\ \hline 
  exp(x) & according to specification of C++ std::exp\\ \hline 
  log(x) & according to specification of C++ std::log\\ \hline 
  sin(x) & according to specification of C++ std::sin\\ \hline 
  cos(x) & according to specification of C++ std::cos\\ \hline 
  tan(x) & according to specification of C++ std::tan\\ \hline 
  asin(x) & according to specification of C++ std::asin\\ \hline 
  acos(x) & according to specification of C++ std::acos\\ \hline 
  atan(x) & according to specification of C++ std::atan\\ \hline 
  sinh(x) & according to specification of C++ std::sinh\\ \hline 
  cosh(x) & according to specification of C++ std::cosh\\ \hline 
  tanh(x) & according to specification of C++ std::tanh\\ \hline 
  asinh(x) & according to specification of C++ std::asinh\\ \hline 
  acosh(x) & according to specification of C++ std::acosh\\ \hline 
  atanh(x) & according to specification of C++ std::atanh\\ \hline 
\end{longtable}
\end{center}

The following table lists special functions for \texttt{SymReal}: 
\begin{center}
\footnotesize
\begin{longtable}{| p{8cm} | p{8cm} |} 
\hline
{\bf function/structure name} & {\bf description}\\ \hline
  sign(x) & returns 0 for x=0, -1 for x<0 and 1 for x>1.\\ \hline 
  Not(x) & returns logical not of expression, equal to Python's 'not'. Not(True)=False, Not(0.)=True, Not(-0.1)=False\\ \hline 
  min(x, y) & return minimum of x and y. \\ \hline 
  max(x, y) & return maximum of x and y. \\ \hline 
  mod(x, y) & return floating-point remainder of the division operation x / y. For example, mod(5.1, 3) gives 2.1 as a remainder.\\ \hline 
  pow(x, y) & return $x^y$. \\ \hline 
  max(x, y) & return maximum of x and y. \\ \hline 
  IfThenElse(condition, ifTrue, ifFalse) & Symbolic function for conditional evaluation. If the condition evaluates to True, the expression ifTrue is evaluated, while otherwise expression ifFalse is evaluated\tabnewline 
    \textcolor{steelblue}{{\bf EXAMPLE}: \tabnewline 
    \texttt{x=SymReal(-1)\tabnewline
    y=SymReal(2,{\textquotesingle}y{\textquotesingle})\tabnewline
    a=SymReal.IfThenElse(x<0, y+1, y-1))}}\\ \hline 
  SetRecording(flag) & Set current (global / module-wide) status of expression recording. By default, recording is on.\tabnewline 
    \textcolor{steelblue}{{\bf EXAMPLE}: \tabnewline 
    \texttt{SymReal.SetRecording(True)}}\\ \hline 
  GetRecording() & Get current (global / module-wide) status of expression recording.\tabnewline 
    \textcolor{steelblue}{{\bf EXAMPLE}: \tabnewline 
    \texttt{symbolic.Real.GetRecording()}}\\ \hline 
\end{longtable}
\end{center}

%++++++++++++++++++++
\mysubsubsection{symbolic.Vector}



A symbolic Vector type to replace Python's (1D) numpy array in symbolic expressions. The \texttt{symbolic.Vector} may be directly set to a list of floats or (1D) numpy array and be evaluated as array. However, turing on recording by using 	exttt{exudyn.symbolic.SetRecording(True)} (on by default), results are stored as expression trees, which may be evaluated in C++ or Python, in particular in user functions, see the following example:
\pythonstyle
\begin{lstlisting}[language=Python, firstnumber=1]

import exudyn as exu
esym = exu.symbolic

SymVector = esym.Vector
SymReal = esym.Real 

a = SymReal('a',42.)
b = SymReal(13)
c = a-3*b

#create from list:
v1 = SymVector([1,3,2])
print('v1: ',v1)

#create from numpy array:
v2 = SymVector(np.array([1,3,2]))
print('v2 initial: ',v2)

#create from list, mixing symbolic expressions and numbers:
v2 = SymVector([a,42,c])

print('v2 now: ',v2,"=",v2.Evaluate())
print('v1+v2: ',v1+v2,"=",(v1+v2).Evaluate()) #evaluate as vector

print('v1*v2: ',v1*v2,"=",(v1*v2).Evaluate()) #evaluate as Real

#access of vector component:
print('v1[2]: ',v1[2],"=",v1[2].Evaluate())   #evaluate as Real
\end{lstlisting}


To create a symbolic Vector, use \texttt{aa=symbolic.Vector([3,4.2,5]} to build a Python object aa with values [3,4.2,5]. In order to use a named vector, use \texttt{v=symbolic.Vector('myVec',[3,4.2,5])}. Vectors can be also created from mixed symbolic expressions and numbers, such as \texttt{v=symbolic.Vector([x,x**2,3.14])}, however, this cannot become a named vector as it contains expressions. There is a significance difference to numpy, such that '*' represents the scalar vector multplication which gives a scalar. Furthermore, the comparison operator '==' gives only True, if all components are equal, and the operator '!=' gives True, if any component is unequal. Note that in the following, we use the abbreviation \texttt{SymVector=exudyn.symbolic.Vector}. Note that only functions are able to be recorded. Member functions of \texttt{SymVector} are:
\begin{center}
\footnotesize
\begin{longtable}{| p{8cm} | p{8cm} |} 
\hline
{\bf function/structure name} & {\bf description}\\ \hline
  \_\_init\_\_(vector) & Construct symbolic.Vector from vector represented as numpy array or list (which may contain symbolic expressions).\\ \hline 
  \_\_init\_\_(name, vector) & Construct named symbolic.Vector from name and vector represented as numpy array or list (which may contain symbolic expressions).\\ \hline 
  Evaluate() & Return evaluated expression (prioritized) or stored vector value. (not recorded)\\ \hline 
  SetVector(vector) & Set stored vector or named vector expression to new given (non-symbolic) vector. Only works, if SymVector contains no expression. (may lead to inconsistencies in recording)\\ \hline 
  NumberOfItems() & Get size of Vector (may require to evaluate expression; not recording)\\ \hline 
  operator \_\_setitem\_\_(index) & bracket [] operator for setting a component of the vector. Only works, if SymVector contains no expression. (may lead to inconsistencies in recording)\\ \hline  
  NormL2() & return (symbolic) L2-norm of vector.\tabnewline 
    \textcolor{steelblue}{{\bf EXAMPLE}: \tabnewline 
    \texttt{v1 = SymVector([1,4,8])\tabnewline
    length = v1.NormL2() \#gives 9.}}\\ \hline 
  MultComponents(other) & Perform component-wise multiplication of vector times other vector and return result. This corresponds to the numpy multiplication using '*'.\tabnewline 
    \textcolor{steelblue}{{\bf EXAMPLE}: \tabnewline 
    \texttt{v1 = SymVector([1,2,4])\tabnewline
    v2 = SymVector([1,0.5,0.25])\tabnewline
    v3 = v1.MultComponents(v2)}}\\ \hline 
  operator \_\_getitem\_\_(index) & bracket [] operator to return (symbolic) component of vector, allowing read-access. Index may also evaluate from an expression.\\ \hline  
  operator \_\_str\_\_() & conversion of SymVector to string\\ \hline  
  operator \_\_repr\_\_() & representation of SymVector in Python\\ \hline  
\end{longtable}
\end{center}

Standard vector operators are available for \texttt{SymVector}, see the following examples:
\pythonstyle
\begin{lstlisting}[language=Python, firstnumber=1]

v = SymVector([1,3,2])
w = SymVector([3.3,2.2,1.1])

u = v+w
u = v-w
u = -v
#scalar multiplication; evaluates to SymReal:
x = v*w 
#NOTE: component-wise multiplication, returns SymVector:
u = v.MultComponents(w)

#inplace operators:
v += w
v -= w
v *= SymReal(0.5)
\end{lstlisting}


%++++++++++++++++++++
\mysubsubsection{symbolic.Matrix}



A symbolic Matrix type to replace Python's (2D) numpy array in symbolic expressions. The \texttt{symbolic.Matrix} may be directly set to a list of list of floats or (2D) numpy array and be evaluated as array. However, turing on recording by using 	exttt{exudyn.symbolic.SetRecording(True)} (on by default), results are stored as expression trees, which may be evaluated in C++ or Python, in particular in user functions, see the following example:
\pythonstyle
\begin{lstlisting}[language=Python, firstnumber=1]

import exudyn as exu
import numpy as np
esym = exu.symbolic

SymMatrix = esym.Matrix
SymReal = esym.Real 

a = SymReal('a',42.)
b = SymReal(13)

#create matrix from list of lists
m1 = SymMatrix([[1,3,2],[4,5,6]])

#create symbolic matrix from list of lists
m3 = SymMatrix([[a,3*b,2],[4,5,6]])

#create from numpy array
m2 = SymMatrix(np.ones((3,3))-np.eye(3))

m1 += m3
m1 *= 3
m1 -= 3*m3
print('m1: ',m1)
print('m2: ',m2)

\end{lstlisting}


To create a symbolic Matrix, use \texttt{aa=symbolic.Matrix([[3,4.2],[3.3,1.2]]} to build a Python object aa. In order to use a named matrix, use \texttt{v=symbolic.Matrix('myMat',[3,4.2,5])}. Matrixs can be also created from mixed symbolic expressions and numbers, such as \texttt{v=symbolic.Matrix([x,x**2,3.14])}, however, this cannot become a named matrix as it contains expressions. There is a significance difference to numpy, such that '*' represents the matrix multplication (compute components from row times column operations). Note that in the following, we use the abbreviation \texttt{SymMatrix=exudyn.symbolic.Matrix}. Member functions of \texttt{SymMatrix} are:
\begin{center}
\footnotesize
\begin{longtable}{| p{8cm} | p{8cm} |} 
\hline
{\bf function/structure name} & {\bf description}\\ \hline
  \_\_init\_\_(matrix) & Construct symbolic.Matrix from vector represented as numpy array or list of lists (which may contain symbolic expressions).\\ \hline 
  \_\_init\_\_(name, matrix) & Construct named symbolic.Matrix from name and vector represented as numpy array or list of lists (which may contain symbolic expressions).\\ \hline 
  Evaluate() & Return evaluated expression (prioritized) or stored Matrix value. (not recorded)\\ \hline 
  SetMatrix(matrix) & Set stored Matrix or named Matrix expression to new given (non-symbolic) Matrix. Only works, if SymMatrix contains no expression. (may lead to inconsistencies in recording)\\ \hline 
  NumberOfRows() & Get number of rows (may require to evaluate expression; not recording)\\ \hline 
  NumberOfColumns() & Get number of columns (may require to evaluate expression; not recording)\\ \hline 
  operator \_\_setitem\_\_(row, column) & bracket [] operator for (symbolic) component of Matrix (write-access). Only works, if SymMatrix contains no expression. (may lead to inconsistencies in recording)\\ \hline  
  operator \_\_getitem\_\_(row, column) & bracket [] operator for (symbolic) component of Matrix (read-access). Row and column may also evaluate from an expression.\\ \hline  
  operator \_\_str\_\_() & conversion of SymMatrix to string\\ \hline  
  operator \_\_repr\_\_() & representation of SymMatrix in Python\\ \hline  
\end{longtable}
\end{center}

Standard Matrix operators are available for \texttt{SymMatrix}, see the following examples:
\pythonstyle
\begin{lstlisting}[language=Python, firstnumber=1]

m1 = SymMatrix([[1,7],[4,5]])
m2 = SymMatrix([[1,2.2],[4,4.3]])
v = SymVector([1.5,3])

m3 = m1+m2
m3 = m1-m2
m3 = m1*m2

#multiply with scalar
m3 = 13*m2
m3 = m2*3.14

#multiply with vector
m3 = m2*v

#transposed:
m3 = v*m2 #equals numpy operation m2.T @ v

#inplace operators:
m1 += m1
m1 -= m1
m1 *= 3.14
\end{lstlisting}


%++++++++++++++++++++
\mysubsubsection{symbolic.VariableSet}



A container for symbolic variables, in particular for exchange between user functions and the model. For details, see the following example:
\pythonstyle
\begin{lstlisting}[language=Python, firstnumber=1]

import exudyn as exu
import math
SymReal = exu.symbolic.Real

#use global variable set:
variables = exu.symbolic.variables

#create a named Real
a = SymReal('a',42.)

#regular way to add variable:
variables.Add('pi', math.pi)

#add named variable (doesn't need a name):
variables.Add(a)

#print current variable set
print(variables)

print('pi=',variables.Get('pi').Evaluate()) #3.14
print('a=',variables.Get('a')) #prints 'a'

x=variables.Get('a')
print('x=',x.Evaluate()) #x=42

#override a
variables.Set('a',3.33)

#x is depending on a:
print('x:',x,"=",x.Evaluate()) #3.33

#create your own variable set
mySet = esym.VariableSet()
\end{lstlisting}

\begin{center}
\footnotesize
\begin{longtable}{| p{8cm} | p{8cm} |} 
\hline
{\bf function/structure name} & {\bf description}\\ \hline
  Add(name, value) & Add a variable with name and value (name may not exist)\\ \hline 
  Add(namedReal) & Add a variable with named real (name may not exist)\\ \hline 
  Set(name, value) & Set a variable with name and value (adds new or overrides existing)\\ \hline 
  Get(name) & Get a variable by name\\ \hline 
  Exists(name) & Return True, if variable name exists\\ \hline 
  Reset() & Erase all variables and reset VariableSet\\ \hline 
  NumberOfItems(name) & Return True, if variable name exists\\ \hline 
  GetNames() & Get list of stored variable names\\ \hline 
  data[index]= ...name, value & bracket [] operator for setting a variable to a specific value\\ \hline 
  ... = data[index]name & bracket [] operator for getting a specific variable by name\\ \hline 
  operator \_\_str\_\_() & create string of set of variables\\ \hline  
  operator \_\_repr\_\_() & representation of SymMatrix in Python\\ \hline  
\end{longtable}
\end{center}

%++++++++++++++++++++
\mysubsubsection{symbolic.UserFunction}



A class for creating and handling symbolic user functions in C++. Use these functions for high performance extensions, e.g., of existing objects or loadsFor details, see the following example:
\pythonstyle
\begin{lstlisting}[language=Python, firstnumber=1]

import exudyn as exu
esym = exu.symbolic
from exudyn.utilities import * #advancedUtilities with user function utilities included
SymReal = exu.symbolic.Real

SC = exu.SystemContainer()
mbs = SC.AddSystem()

#regular Python user function with esym math functions
def UFload(mbs, t, load):
    return load*esym.sin(10*(2*pi)*t)

#create symbolic user function from Python user function:
symFuncLoad = CreateSymbolicUserFunction(mbs, UFload, load, 'loadUserFunction',verbose=1)

#add ground and mass point:
oGround = mbs.CreateGround()
oMassPoint = mbs.CreateMassPoint(referencePosition=[1.+0.05,0,0], physicsMass=1)

#add marker and load:
mc = mbs.AddMarker(MarkerNodeCoordinate(nodeNumber=mbs.GetObject(oMassPoint)['nodeNumber'], coordinate=0))
load = mbs.AddLoad(LoadCoordinate(markerNumber=mc, load=10,
                                  loadUserFunction=symFuncLoad))

#print string of symbolic expression of user function (to check if it looks ok):
print('load user function: ',symFuncLoad)

#test evaluate user function; requires args of user function:
print('load user function: ',symFuncLoad.Evaluate(mbs, 0.025, 10.))
    
#now you could add further items or simulate ...
\end{lstlisting}

\begin{center}
\footnotesize
\begin{longtable}{| p{8cm} | p{8cm} |} 
\hline
{\bf function/structure name} & {\bf description}\\ \hline
  Evaluate() & Evaluate symbolic function with test values; requires exactly same args as Python user functions; this is slow and only intended for testing\\ \hline 
  SetUserFunctionFromDict(mainSystem, fcnDict, itemIndex, userFunctionName) & Create C++ std::function (as requested in C++ item) with symbolic user function as recorded in given dictionary, as created with ConvertFunctionToSymbolic(...).\\ \hline 
  operator \_\_repr\_\_() & Representation of Symbolic function\\ \hline  
  operator \_\_str\_\_() & Convert stored symbolic function to string\\ \hline  
\end{longtable}
\end{center}

%++++++++++++++++++++
\mysubsection{GeneralContact}
\label{sec:GeneralContact}



Structure to define general and highly efficient contact functionality in multibody systems\footnote{Note that GeneralContact is still developed, use with care.}. For further explanations and theoretical backgrounds, see \refSection{secContactTheory}. Internally, the contacts are stored with global indices, which are in the following list: [numberOfSpheresMarkerBased, numberOfANCFCable2D, numberOfTrigsRigidBodyBased], see alsothe output of GetPythonObject().
\pythonstyle
\begin{lstlisting}[language=Python, firstnumber=1]

#...
#code snippet, must be placed anywhere before mbs.Assemble()
#Add GeneralContact to mbs:
gContact = mbs.AddGeneralContact()
#Add contact elements, e.g.:
gContact.AddSphereWithMarker(...) #use appropriate arguments
gContact.SetFrictionPairings(...) #set friction pairings and adjust searchTree if needed.
\end{lstlisting}

\begin{center}
\footnotesize
\begin{longtable}{| p{8cm} | p{8cm} |} 
\hline
{\bf function/structure name} & {\bf description}\\ \hline
  GetPythonObject() & convert member variables of GeneralContact into dictionary; use this for debug only!\\ \hline 
  Reset(freeMemory = True) & remove all contact objects and reset contact parameters\\ \hline 
  isActive & default = True (compute contact); if isActive=False, no contact computation is performed for this contact set \\ \hline  
  verboseMode & default = 0; verboseMode = 1 or higher outputs useful information on the contact creation and computation \\ \hline  
  visualization & access visualization data structure \\ \hline  
  resetSearchTreeInterval & (default=10000) number of search tree updates (contact computation steps) after which the search tree cells are re-created; this costs some time, will free memory in cells that are not needed any more \\ \hline  
  sphereSphereContact & activate/deactivate contact between spheres \\ \hline  
  sphereSphereFrictionRecycle & False: compute static friction force based on tangential velocity; True: recycle friction from previous PostNewton step, which greatly improves convergence, but may lead to unphysical artifacts; will be solved in future by step reduction \\ \hline  
  minRelDistanceSpheresTriangles & (default=1e-10) tolerance (relative to sphere radiues) below which the contact between triangles and spheres is ignored; used for spheres directly attached to triangles \\ \hline  
  frictionProportionalZone & (default=0.001) velocity $v\_\{\mu,reg\}$ upon which the dry friction coefficient is interpolated linearly (regularized friction model); must be greater 0; very small values cause oscillations in friction force \\ \hline  
  frictionVelocityPenalty & (default=1e3) regularization factor for friction [N/(m$^2 \cdot$m/s) ];$k\_\{\mu,reg\}$, multiplied with tangential velocity to compute friciton force as long as it is smaller than $\mu$ times contact force; large values cause oscillations in friction force \\ \hline  
  excludeOverlappingTrigSphereContacts & (default=True) for consistent, closed meshes, we can exclude overlapping contact triangles (which would cause holes if mesh is overlapping and not consistent!!!) \\ \hline  
  excludeDuplicatedTrigSphereContactPoints & (default=False) run additional checks for double contacts at edges or vertices, being more accurate but can cause additional costs if many contacts \\ \hline  
  computeContactForces & (default=False) if True, additional system vector is computed which contains all contact force and torque contributions. In order to recover forces on a single rigid body, the respective LTG-vector has to be used and forces need to be extracted from this system vector; may slow down computations.\\ \hline  
  ancfCableUseExactMethod & (default=True) if True, uses exact computation of intersection of 3rd order polynomials and contacting circles \\ \hline  
  ancfCableNumberOfContactSegments & (default=1) number of segments to be used in case that ancfCableUseExactMethod=False; maximum number of segments=3 \\ \hline  
  ancfCableMeasuringSegments & (default=20) number of segments used to approximate geometry for ANCFCable2D elements for measuring with ShortestDistanceAlongLine; with 20 segments the relative error due to approximation as compared to 10 segments usually stays below 1e-8 \\ \hline  
  SetFrictionPairings(frictionPairings) & set Coulomb friction coefficients for pairings of materials (e.g., use material 0,1, then the entries (0,1) and (1,0) define the friction coefficients for this pairing); matrix should be symmetric!\tabnewline 
    \textcolor{steelblue}{{\bf EXAMPLE}: \tabnewline 
    \texttt{\#set 3 surface friction types, all being 0.1:\tabnewline
    gContact.SetFrictionPairings(0.1*np.ones((3,3)));}}\\ \hline 
  SetFrictionProportionalZone(frictionProportionalZone) & regularization for friction (m/s); used for all contacts\\ \hline 
  SetSearchTreeCellSize(numberOfCells) & set number of cells of search tree (boxed search) in x, y and z direction\tabnewline 
    \textcolor{steelblue}{{\bf EXAMPLE}: \tabnewline 
    \texttt{gContact.SetSearchTreeInitSize([10,10,10])}}\\ \hline 
  SetSearchTreeBox(pMin, pMax) & set geometric dimensions of searchTreeBox (point with minimum coordinates and point with maximum coordinates); if this box becomes smaller than the effective contact objects, contact computations may slow down significantly\tabnewline 
    \textcolor{steelblue}{{\bf EXAMPLE}: \tabnewline 
    \texttt{gContact.SetSearchTreeBox(pMin=[-1,-1,-1],\tabnewline
     \phantom{XXXX} pMax=[1,1,1])}}\\ \hline 
  AddSphereWithMarker(markerIndex, radius, contactStiffness, contactDamping, frictionMaterialIndex) & add contact object using a marker (Position or Rigid), radius and contact/friction parameters and return localIndex of the contact item in GeneralContact; frictionMaterialIndex refers to frictionPairings in GeneralContact; contact is possible between spheres (circles in 2D) (if intraSphereContact = True), spheres and triangles and between sphere (=circle) and ANCFCable2D; contactStiffness is computed as serial spring between contacting objects, while damping is computed as a parallel damper\\ \hline 
  AddANCFCable(objectIndex, halfHeight, contactStiffness, contactDamping, frictionMaterialIndex) & add contact object for an ANCF cable element, using the objectIndex of the cable element and the cable's half height as an additional distance to contacting objects (currently not causing additional torque in case of friction), and return localIndex of the contact item in GeneralContact; currently only contact with spheres (circles in 2D) possible; contact computed using exact geometry of elements, finding max 3 intersecting contact regions\\ \hline 
  AddTrianglesRigidBodyBased(rigidBodyMarkerIndex, contactStiffness, contactDamping, frictionMaterialIndex, pointList, triangleList) & add contact object using a rigidBodyMarker (of a body), contact/friction parameters, a list of points (as 3D numpy arrays or lists; coordinates relative to rigidBodyMarker) and a list of triangles (3 indices as numpy array or list) according to a mesh attached to the rigidBodyMarker; returns starting local index of trigsRigidBodyBased at which the triangles are stored; mesh can be produced with GraphicsData2TrigsAndPoints(...); contact is possible between sphere (circle) and Triangle but yet not between triangle and triangle; frictionMaterialIndex refers to frictionPairings in GeneralContact; contactStiffness is computed as serial spring between contacting objects, while damping is computed as a parallel damper (otherwise the smaller damper would always dominate); the triangle normal must point outwards, with the normal of a triangle given with local points (p0,p1,p2) defined as n=(p1-p0) x (p2-p0), see function ComputeTriangleNormal(...)\\ \hline 
  GetItemsInBox(pMin, pMax) & Get all items in box defined by minimum coordinates given in pMin and maximum coordinates given by pMax, accepting 3D lists or numpy arrays; in case that no objects are found, False is returned; otherwise, a dictionary is returned, containing numpy arrays with indices of obtained MarkerBasedSpheres, TrigsRigidBodyBased, ANCFCable2D, ...; the indices refer to the local index in GeneralContact which can be evaluated e.g. by GetMarkerBasedSphere(localIndex)\tabnewline 
    \textcolor{steelblue}{{\bf EXAMPLE}: \tabnewline 
    \texttt{gContact.GetItemsInBox(pMin=[0,1,1],\tabnewline
     \phantom{XXXX} pMax=[2,3,2])}}\\ \hline 
  GetSphereMarkerBased(localIndex, addData = False) & Get dictionary with current position, orientation, velocity, angular velocity as computed in last contact iteration; if addData=True, adds stored data of contact element, such as radius, markerIndex and contact parameters; localIndex is the internal index of contact element, as returned e.g. from GetItemsInBox\\ \hline 
  SetSphereMarkerBased(localIndex, contactStiffness = -1., contactDamping = -1., radius = -1., frictionMaterialIndex = -1) & Set data of marker based sphere with localIndex (as internally stored) with given arguments; arguments that are < 0 (default) imply that current values are not overwritten\\ \hline 
  GetTriangleRigidBodyBased(localIndex) & Get dictionary with rigid body index, local position of triangle vertices (nodes) and triangle normal; NOTE: the mesh added to contact is different from this structure, as it contains nodes and connectivity lists; the triangle index corresponds to the order as triangles are added to GeneralContact\\ \hline 
  SetTriangleRigidBodyBased(localIndex, points, contactRigidBodyIndex = -1) & Set data of marker based sphere with localIndex (triangle index); points are provided as 3x3 numpy array, with point coordinates in rows; contactRigidBodyIndex<0 indicates no change of the current index (and changing this index should be handled with care)\\ \hline 
  ShortestDistanceAlongLine(pStart = [0,0,0], direction = [1,0,0], minDistance = -1e-7, maxDistance = 1e7, asDictionary = False, cylinderRadius = 0, typeIndex = Contact.IndexEndOfEnumList) & Find shortest distance to contact objects in GeneralContact along line with pStart (given as 3D list or numpy array) and direction (as 3D list or numpy array with no need to be normalized); the function returns the distance which is >= minDistance and < maxDistance; in case of beam elements, it measures the distance to the beam centerline; the distance is measured from pStart along given direction and can also be negative; if no item is found along line, the maxDistance is returned; if asDictionary=False, the result is a float, while otherwise details are returned as dictionary (including distance, velocityAlongLine (which is the object velocity in given direction and may be different from the time derivative of the distance; works similar to a laser Doppler vibrometer - LDV), itemIndex and itemType in GeneralContact); the cylinderRadius, if not equal to 0, will be used for spheres to find closest sphere along cylinder with given point and direction; the typeIndex can be set to a specific contact type, e.g., which are searched for (otherwise all objects are considered)\\ \hline 
  UpdateContacts(mainSystem) & Update contact sets, e.g. if no contact is simulated (isActive=False) but user functions need up-to-date contact states for GetItemsInBox(...) or for GetActiveContacts(...)\tabnewline 
    \textcolor{steelblue}{{\bf EXAMPLE}: \tabnewline 
    \texttt{gContact.UpdateContacts(mbs)}}\\ \hline 
  GetActiveContacts(typeIndex, itemIndex) & Get list of global item numbers which are in contact with itemIndex of type typeIndex in case that the global itemIndex is smaller than the abs value of the contact pair index; a negative sign indicates that the contacting (spheres) is in Coloumb friction, a positive sign indicates a regularized friction region; in case of itemIndex==-1, it will return the list of numbers of active contacts per item for the contact type; for interpretation of global contact indices, see gContact.GetPythonObject() and documentation; requires either implicit contact computation or UpdateContacts(...) needs to be called prior to this function\tabnewline 
    \textcolor{steelblue}{{\bf EXAMPLE}: \tabnewline 
    \texttt{\#if explicit solver is used, we first need to update contacts:\tabnewline
    gContact.UpdateContacts(mbs)\tabnewline
    \#obtain active contacts of marker based sphere 42:\tabnewline
    gList = gContact.GetActiveContacts(exu.ContactTypeIndex.IndexSpheresMarkerBased, 42)}}\\ \hline 
  GetSystemODE2RhsContactForces() & Get numpy array of system vector, containing contribution of contact forces to system ODE2 Rhs vector; contributions to single objects may be extracted by checking the according LTG-array of according objects (such as rigid bodies); the contact forces vector is computed in each contact iteration;\\ \hline 
  \_\_repr\_\_() & return the string representation of the GeneralContact, containing basic information and statistics\\ \hline 
\end{longtable}
\end{center}

%++++++++++++++++++++
\mysubsubsection{VisuGeneralContact}
\label{sec:GeneralContact:visualization}
This structure may contains some visualization parameters in future. Currently, all visualization settings are controlled via SC.visualizationSettings

\begin{center}
\footnotesize
\begin{longtable}{| p{8cm} | p{8cm} |} 
\hline
{\bf function/structure name} & {\bf description}\\ \hline
  Reset() & reset visualization parameters to default values\\ \hline 
\end{longtable}
\end{center}

\mysubsection{Data structures}
\label{sec:cinterface:dataStructures}

This section describes a set of special data structures which are used in the Python-C++ interface, 
such as a MatrixContainer for dense/sparse matrices or a list of 3D vectors. 
Note that there are many native data types, such as lists, dicts and numpy arrays (e.g. 3D vectors), 
which are not described here as they are native to Pybind11, but can be passed as arguments when appropriate.

%++++++++++++++++++++
\mysubsubsection{MatrixContainer}
\label{sec:MatrixContainer}
The MatrixContainer is a versatile representation for dense and sparse matrices. NOTE: if the MatrixContainer is constructed from a numpy array or a list of lists, both representing a dense matrix, it will go into dense mode; if it is initialized with a scipy sparse csr matrix, it will go into sparse mode

\pythonstyle
\begin{lstlisting}[language=Python, firstnumber=1]

#Create empty MatrixContainer:
from scipy.sparse import csr_matrix
from exudyn import MatrixContainer
mc = MatrixContainer() #empty matrix, dense mode

#Create MatrixContainer with dense matrix:
#matrix can be initialized with a dense matrix, using list of lists or a numpy array, e.g.:
matrix = np.eye(3)
mcDense1 = MatrixContainer(matrix)
mcDense2 = MatrixContainer([[1,2],[3,4]])

#Set with dense pyArray (a numpy array): 
pyArray = np.array(matrix)
mc.SetWithDenseMatrix(pyArray, useDenseMatrix = True)

#Set empty matrix:
mc.SetWithDenseMatrix([[]], useDenseMatrix = True)

#Set with list of lists, stored as sparse matrix:
mc.SetWithDenseMatrix([[1,2],[3,4]], useDenseMatrix = False)

#Set with sparse triplets (list of lists or numpy array):
mc.SetWithSparseMatrix([[0,0,13.3],[1,1,4.2],[1,2,42.]], 
                       numberOfRows=2, numberOfColumns=3, 
                       useDenseMatrix=True)

print(mc)
#gives dense matrix:
#[[13.3  0.   0. ]
# [ 0.   4.2 42. ]]

#Set with scipy matrix:
#WARNING: only use csr_matrix
#         csc_matrix would basically run, but gives the transposed!!!
spmat = csr_matrix(matrix) 
mc.SetWithSparseMatrix(spmat) #takes rows and column format automatically

#initialize and add triplets later on
mc.Initialize(3,3,useDenseMatrix=False)
mc.AddSparseMatrix(spmat, factor=1)
#can also add smaller matrix
mc.AddSparseMatrix(csr_matrix(np.eye(2)), factor=0.5)
print('mc8=',mc)

\end{lstlisting}

\begin{center}
\footnotesize
\begin{longtable}{| p{8cm} | p{8cm} |} 
\hline
{\bf function/structure name} & {\bf description}\\ \hline
  Initialize(numberOfRows, numberOfColumns, useDenseMatrix = True) & initialize MatrixContainer with number of rows and columns and set dense/sparse mode\\ \hline 
  SetWithDenseMatrix(pyArray, useDenseMatrix = False, factor = 1.) & set MatrixContainer with dense numpy array of size (n x m); array (=matrix) contains values and matrix size information; if useDenseMatrix=True, matrix will be stored internally as dense matrix, otherwise it will be converted and stored as sparse matrix (which may speed up computations for larger problems); pyArray is multiplied with given factor\\ \hline 
  SetWithSparseMatrix(sparseMatrix, numberOfRows = invalid (-1), numberOfColumns = invalid (-1), useDenseMatrix = False, factor = 1.) & set with scipy sparse csr\_matrix (NOT: csc\_matrix!) or with internal sparse triplet format (denoted as CSR): 'sparseMatrix' either contains a scipy matrix create with csr\_matrix or a list of lists of sparse triplets (row, col, value) or the list of lists converted into numpy array; numberOfRowsInit and numberOfColumnsInit denote the size of the matrices, which are ignored in case of a scipy sparse matrix; if useDenseMatrix=True, matrix will be converted and stored internally as dense matrix, otherwise it will be stored as sparse matrix triplets; the values of sparseMatrix are multiplied with the given factor before storing\\ \hline 
  AddSparseMatrix(sparseMatrix, factor = 1.) & add scipy sparse csr\_matrix with factor to already initilized MatrixContainer; sparseMatrix must contain according scipy csr format, otherwise the behavior is undefined! This function allows to efficiently add submatrices to the MatrixContainer\\ \hline 
  GetPythonObject() & convert MatrixContainer to numpy array (dense) or dictionary (sparse): containing nr. of rows, nr. of columns, numpy matrix with sparse triplets\\ \hline 
  Convert2DenseMatrix() & convert MatrixContainer to dense numpy array (SLOW and may fail for too large sparse matrices)\\ \hline 
  UseDenseMatrix() & returns True if dense matrix is used, otherwise False\\ \hline 
  SetAllZero() & Set all values to zero; dense mode: set all matrix entries to zero (slow); sparse mode: set number of triplets to zero (fast)\\ \hline 
  SetWithSparseMatrixCSR(numberOfRowsInit, numberOfColumnsInit, pyArrayCSR, useDenseMatrix = False, factor = 1.) & DEPRECATED: set with sparse CSR matrix format: numpy array 'pyArrayCSR' contains sparse triplet (row, col, value) per row; numberOfRows and numberOfColumns given extra; if useDenseMatrix=True, matrix will be converted and stored internally as dense matrix, otherwise it will be stored as sparse matrix; the values of pyArrayCSR are multiplied by the given factor\\ \hline 
  \_\_repr\_\_() & return the string representation of the MatrixContainer\\ \hline 
\end{longtable}
\end{center}

%++++++++++++++++++++
\mysubsubsection{Vector3DList}
The Vector3DList is used to represent lists of 3D vectors. This is used to transfer such lists from Python to C++. \\ \\ Usage: \bi
  \item Create empty \texttt{Vector3DList} with \texttt{x = Vector3DList()} 
  \item Create \texttt{Vector3DList} with list of numpy arrays:\\\texttt{x = Vector3DList([ numpy.array([1.,2.,3.]), numpy.array([4.,5.,6.]) ])}
  \item Create \texttt{Vector3DList} with list of lists \texttt{x = Vector3DList([[1.,2.,3.], [4.,5.,6.]])}
  \item Append item: \texttt{x.Append([0.,2.,4.])}
  \item Convert into list of numpy arrays: \texttt{x.GetPythonObject()}
\ei


\begin{center}
\footnotesize
\begin{longtable}{| p{8cm} | p{8cm} |} 
\hline
{\bf function/structure name} & {\bf description}\\ \hline
  Append(pyArray) & add single array or list to Vector3DList; array or list must have appropriate dimension!\\ \hline 
  GetPythonObject() & convert Vector3DList into (copied) list of numpy arrays\\ \hline 
  len(data) & return length of the Vector3DList, using len(data) where data is the Vector3DList\\ \hline 
  data[index]= ... & set list item 'index' with data, write: data[index] = ...\\ \hline 
  ... = data[index] & get copy of list item with 'index' as vector\\ \hline 
  \_\_copy\_\_() & copy method to be used for copy.copy(...); in fact does already deep copy\\ \hline 
  \_\_deepcopy\_\_() & deepcopy method to be used for copy.copy(...)\\ \hline 
  \_\_repr\_\_() & return the string representation of the Vector3DList data, e.g.: print(data)\\ \hline 
\end{longtable}
\end{center}

%++++++++++++++++++++
\mysubsubsection{Vector2DList}
The Vector2DList is used to represent lists of 2D vectors. This is used to transfer such lists from Python to C++. \\ \\ Usage: \bi
  \item Create empty \texttt{Vector2DList} with \texttt{x = Vector2DList()} 
  \item Create \texttt{Vector2DList} with list of numpy arrays:\\\texttt{x = Vector2DList([ numpy.array([1.,2.]), numpy.array([4.,5.]) ])}
  \item Create \texttt{Vector2DList} with list of lists \texttt{x = Vector2DList([[1.,2.], [4.,5.]])}
  \item Append item: \texttt{x.Append([0.,2.])}
  \item Convert into list of numpy arrays: \texttt{x.GetPythonObject()}
  \item similar to Vector3DList !
\ei


\begin{center}
\footnotesize
\begin{longtable}{| p{8cm} | p{8cm} |} 
\hline
{\bf function/structure name} & {\bf description}\\ \hline
  Append(pyArray) & add single array or list to Vector2DList; array or list must have appropriate dimension!\\ \hline 
  GetPythonObject() & convert Vector2DList into (copied) list of numpy arrays\\ \hline 
  len(data) & return length of the Vector2DList, using len(data) where data is the Vector2DList\\ \hline 
  data[index]= ... & set list item 'index' with data, write: data[index] = ...\\ \hline 
  ... = data[index] & get copy of list item with 'index' as vector\\ \hline 
  \_\_copy\_\_() & copy method to be used for copy.copy(...); in fact does already deep copy\\ \hline 
  \_\_deepcopy\_\_() & deepcopy method to be used for copy.copy(...)\\ \hline 
  \_\_repr\_\_() & return the string representation of the Vector2DList data, e.g.: print(data)\\ \hline 
\end{longtable}
\end{center}

%++++++++++++++++++++
\mysubsubsection{Vector6DList}
The Vector6DList is used to represent lists of 6D vectors. This is used to transfer such lists from Python to C++. \\ \\ Usage: \bi
  \item Create empty \texttt{Vector6DList} with \texttt{x = Vector6DList()} 
  \item Convert into list of numpy arrays: \texttt{x.GetPythonObject()}
  \item similar to Vector3DList !
\ei


\begin{center}
\footnotesize
\begin{longtable}{| p{8cm} | p{8cm} |} 
\hline
{\bf function/structure name} & {\bf description}\\ \hline
  Append(pyArray) & add single array or list to Vector6DList; array or list must have appropriate dimension!\\ \hline 
  GetPythonObject() & convert Vector6DList into (copied) list of numpy arrays\\ \hline 
  len(data) & return length of the Vector6DList, using len(data) where data is the Vector6DList\\ \hline 
  data[index]= ... & set list item 'index' with data, write: data[index] = ...\\ \hline 
  ... = data[index] & get copy of list item with 'index' as vector\\ \hline 
  \_\_copy\_\_() & copy method to be used for copy.copy(...); in fact does already deep copy\\ \hline 
  \_\_deepcopy\_\_() & deepcopy method to be used for copy.copy(...)\\ \hline 
  \_\_repr\_\_() & return the string representation of the Vector6DList data, e.g.: print(data)\\ \hline 
\end{longtable}
\end{center}

%++++++++++++++++++++
\mysubsubsection{Matrix3DList}
The Matrix3DList is used to represent lists of 3D Matrices. . This is used to transfer such lists from Python to C++. \\ \\ Usage: \bi
  \item Create empty \texttt{Matrix3DList} with \texttt{x = Matrix3DList()} 
  \item Create \texttt{Matrix3DList} with list of numpy arrays:\\\texttt{x = Matrix3DList([ numpy.eye(3), numpy.array([[1.,2.,3.],[4.,5.,6.],[7.,8.,9.]]) ])}
  \item Append item: \texttt{x.Append(numpy.eye(3))}
  \item Convert into list of numpy arrays: \texttt{x.GetPythonObject()}
  \item similar to Vector3DList !
\ei


\begin{center}
\footnotesize
\begin{longtable}{| p{8cm} | p{8cm} |} 
\hline
{\bf function/structure name} & {\bf description}\\ \hline
  Append(pyArray) & add single 3D array or list of lists to Matrix3DList; array or lists must have appropriate dimension!\\ \hline 
  GetPythonObject() & convert Matrix3DList into (copied) list of 3x3 numpy arrays\\ \hline 
  len(data) & return length of the Matrix3DList, using len(data) where data is the Matrix3DList\\ \hline 
  data[index]= ... & set list item 'index' with matrix, write: data[index] = ...\\ \hline 
  ... = data[index] & get copy of list item with 'index' as matrix\\ \hline 
  \_\_repr\_\_() & return the string representation of the Matrix3DList data, e.g.: print(data)\\ \hline 
\end{longtable}
\end{center}

%++++++++++++++++++++
\mysubsubsection{Matrix6DList}
The Matrix6DList is used to represent lists of 6D Matrices. . This is used to transfer such lists from Python to C++. \\ \\ Usage: \bi
  \item Create empty \texttt{Matrix6DList} with \texttt{x = Matrix6DList()} 
  \item Create \texttt{Matrix6DList} with list of numpy arrays:\\\texttt{x = Matrix6DList([ numpy.eye(6), 2*numpy.eye(6) ])}
  \item Append item: \texttt{x.Append(numpy.eye(6))}
  \item Convert into list of numpy arrays: \texttt{x.GetPythonObject()}
  \item similar to Matrix3DList !
\ei


\begin{center}
\footnotesize
\begin{longtable}{| p{8cm} | p{8cm} |} 
\hline
{\bf function/structure name} & {\bf description}\\ \hline
  Append(pyArray) & add single 6D array or list of lists to Matrix6DList; array or lists must have appropriate dimension!\\ \hline 
  GetPythonObject() & convert Matrix6DList into (copied) list of 6x6 numpy arrays\\ \hline 
  len(data) & return length of the Matrix6DList, using len(data) where data is the Matrix6DList\\ \hline 
  data[index]= ... & set list item 'index' with matrix, write: data[index] = ...\\ \hline 
  ... = data[index] & get copy of list item with 'index' as matrix\\ \hline 
  \_\_repr\_\_() & return the string representation of the Matrix6DList data, e.g.: print(data)\\ \hline 
\end{longtable}
\end{center}

\mysubsection{Type definitions}
\label{sec:cinterface:typedef}
This section defines a couple of structures (C++: enum aka enumeration type), which are used to select, e.g., a configuration type or a variable type. In the background, these types are integer numbers, but for safety, the types should be used as type variables. See this examples:

\pythonstyle
\begin{lstlisting}[language=Python, firstnumber=1]

#Conversion to integer is possible: 
x = int(exu.OutputVariableType.Displacement)
#also conversion from integer: 
varType = exu.OutputVariableType(8)
#use in settings:
SC.visualizationSettings.contour.outputVariable = exu.OutputVariableType.StressLocal
#use outputVariableType in sensor:
mbs.AddSensor(SensorBody(bodyNumber=rigid, storeInternal=True,
                         outputVariableType=exu.OutputVariableType.Displacement))
#
\end{lstlisting}


%++++++++++++++++++++
\mysubsubsection{OutputVariableType}
\label{sec:OutputVariableType}
This section shows the OutputVariableType structure, which is used for selecting output values, e.g. for GetObjectOutput(...) or for selecting variables for contour plot.

Available output variables and the interpreation of the output variable can be found at the object definitions. 
 The OutputVariableType does not provide information about the size of the output variable, which can be either scalar or a list (vector). For vector output quantities, the contour plot option offers an additional parameter for selection of the component of the OutputVariableType. The components are usually out of \{0,1,2\}, representing \{x,y,z\} components (e.g., of displacements, velocities, ...), or \{0,1,2,3,4,5\} representing \{xx,yy,zz,yz,xz,xy\} components (e.g., of strain or stress). In order to compute a norm, chose component=-1, which will result in the quadratic norm for other vectors and to a norm specified for stresses (if no norm is defined for an outputVariable, it does not compute anything)


\begin{center}
\footnotesize
\begin{longtable}{| p{8cm} | p{8cm} |} 
\hline
{\bf function/structure name} & {\bf description}\\ \hline
  \_None & no value; used, e.g., to select no output variable in contour plot\\ \hline  
  Distance & e.g., measure distance in spring damper connector\\ \hline  
  Position & measure 3D position, e.g., of node or body\\ \hline  
  Displacement & measure displacement; usually difference between current position and reference position\\ \hline  
  DisplacementLocal & measure local displacement, e.g. in local joint coordinates\\ \hline  
  Velocity & measure (translational) velocity of node or object\\ \hline  
  VelocityLocal & measure local (translational) velocity, e.g. in local body or joint coordinates\\ \hline  
  Acceleration & measure (translational) acceleration of node or object\\ \hline  
  AccelerationLocal & measure (translational) acceleration of node or object in local coordinates\\ \hline  
  RotationMatrix & measure rotation matrix of rigid body node or object\\ \hline  
  Rotation & measure, e.g., scalar rotation of 2D body, Euler angles of a 3D object or rotation within a joint\\ \hline  
  AngularVelocity & measure angular velocity of node or object\\ \hline  
  AngularVelocityLocal & measure local (body-fixed) angular velocity of node or object\\ \hline  
  AngularAcceleration & measure angular acceleration of node or object\\ \hline  
  AngularAccelerationLocal & measure angular acceleration of node or object in local coordinates\\ \hline  
  Coordinates & measure the coordinates of a node or object; coordinates usually just contain displacements, but not the position values\\ \hline  
  Coordinates\_t & measure the time derivative of coordinates (= velocity coordinates) of a node or object\\ \hline  
  Coordinates\_tt & measure the second time derivative of coordinates (= acceleration coordinates) of a node or object\\ \hline  
  SlidingCoordinate & measure sliding coordinate in sliding joint\\ \hline  
  Director1 & measure a director (e.g. of a rigid body frame), or a slope vector in local 1 or x-direction\\ \hline  
  Director2 & measure a director (e.g. of a rigid body frame), or a slope vector in local 2 or y-direction\\ \hline  
  Director3 & measure a director (e.g. of a rigid body frame), or a slope vector in local 3 or z-direction\\ \hline  
  Force & measure global force, e.g., in joint or beam (resultant force), or generalized forces; see description of according object\\ \hline  
  ForceLocal & measure local force, e.g., in joint or beam (resultant force)\\ \hline  
  Torque & measure torque, e.g., in joint or beam (resultant couple/moment)\\ \hline  
  TorqueLocal & measure local torque, e.g., in joint or beam (resultant couple/moment)\\ \hline  
  StrainLocal & measure local strain, e.g., axial strain in cross section frame of beam or Green-Lagrange strain\\ \hline  
  StressLocal & measure local stress, e.g., axial stress in cross section frame of beam or Second Piola-Kirchoff stress; choosing component==-1 will result in the computation of the Mises stress\\ \hline  
  CurvatureLocal & measure local curvature; may be scalar or vectorial: twist and curvature of beam in cross section frame\\ \hline  
  ConstraintEquation & evaluates constraint equation (=current deviation or drift of constraint equation)\\ \hline  
\end{longtable}
\end{center}

%++++++++++++++++++++
\mysubsubsection{ConfigurationType}
\label{sec:ConfigurationType}
This section shows the ConfigurationType structure, which is used for selecting a configuration for reading or writing information to the module. Specifically, the ConfigurationType.Current configuration is usually used at the end of a solution process, to obtain result values, or the ConfigurationType.Initial is used to set initial values for a solution process.



\begin{center}
\footnotesize
\begin{longtable}{| p{8cm} | p{8cm} |} 
\hline
{\bf function/structure name} & {\bf description}\\ \hline
  \_None & no configuration; usually not valid, but may be used, e.g., if no configurationType is required\\ \hline  
  Initial & initial configuration prior to static or dynamic solver; is computed during mbs.Assemble() or AssembleInitializeSystemCoordinates()\\ \hline  
  Current & current configuration during and at the end of the computation of a step (static or dynamic)\\ \hline  
  Reference & configuration used to define deformable bodies (reference configuration for finite elements) or joints (configuration for which some joints are defined)\\ \hline  
  StartOfStep & during computation, this refers to the solution at the start of the step = end of last step, to which the solver falls back if convergence fails\\ \hline  
  Visualization & this is a state completely de-coupled from computation, used for visualization\\ \hline  
  EndOfEnumList & this marks the end of the list, usually not important to the user\\ \hline  
\end{longtable}
\end{center}

%++++++++++++++++++++
\mysubsubsection{ItemType}
\label{sec:ItemType}
This section shows the ItemType structure, which is used for defining types of indices, e.g., in render window and will be also used in item dictionaries in future.



\begin{center}
\footnotesize
\begin{longtable}{| p{8cm} | p{8cm} |} 
\hline
{\bf function/structure name} & {\bf description}\\ \hline
  \_None & item has no type\\ \hline  
  Node & item or index is of type Node\\ \hline  
  Object & item or index is of type Object\\ \hline  
  Marker & item or index is of type Marker\\ \hline  
  Load & item or index is of type Load\\ \hline  
  Sensor & item or index is of type Sensor\\ \hline  
\end{longtable}
\end{center}

%++++++++++++++++++++
\mysubsubsection{NodeType}
\label{sec:NodeType}
This section shows the NodeType structure, which is used for defining node types for 3D rigid bodies.



\begin{center}
\footnotesize
\begin{longtable}{| p{8cm} | p{8cm} |} 
\hline
{\bf function/structure name} & {\bf description}\\ \hline
  \_None & node has no type\\ \hline  
  Ground & ground node\\ \hline  
  Position2D & 2D position node \\ \hline  
  Orientation2D & node with 2D rotation\\ \hline  
  Point2DSlope1 & 2D node with 1 slope vector\\ \hline  
  Position & 3D position node\\ \hline  
  Orientation & 3D orientation node\\ \hline  
  RigidBody & node that can be used for rigid bodies\\ \hline  
  RotationEulerParameters & node with 3D orientations that are modelled with Euler parameters (unit quaternions)\\ \hline  
  RotationRxyz & node with 3D orientations that are modelled with Tait-Bryan angles\\ \hline  
  RotationRotationVector & node with 3D orientations that are modelled with the rotation vector\\ \hline  
  LieGroupWithDirectUpdate & node to be solved with Lie group methods, without data coordinates\\ \hline  
  GenericODE2 & node with general ODE2 variables\\ \hline  
  GenericODE1 & node with general ODE1 variables\\ \hline  
  GenericAE & node with general algebraic variables\\ \hline  
  GenericData & node with general data variables\\ \hline  
  PointSlope1 & node with 1 slope vector\\ \hline  
  PointSlope12 & node with 2 slope vectors in x and y direction\\ \hline  
  PointSlope23 & node with 2 slope vectors in y and z direction\\ \hline  
\end{longtable}
\end{center}

%++++++++++++++++++++
\mysubsubsection{JointType}
\label{sec:JointType}
This section shows the JointType structure, which is used for defining joint types, used in KinematicTree.



\begin{center}
\footnotesize
\begin{longtable}{| p{8cm} | p{8cm} |} 
\hline
{\bf function/structure name} & {\bf description}\\ \hline
  \_None & node has no type\\ \hline  
  RevoluteX & revolute joint type with rotation around local X axis\\ \hline  
  RevoluteY & revolute joint type with rotation around local Y axis\\ \hline  
  RevoluteZ & revolute joint type with rotation around local Z axis\\ \hline  
  PrismaticX & prismatic joint type with translation along local X axis\\ \hline  
  PrismaticY & prismatic joint type with translation along local Y axis\\ \hline  
  PrismaticZ & prismatic joint type with translation along local Z axis\\ \hline  
\end{longtable}
\end{center}

%++++++++++++++++++++
\mysubsubsection{DynamicSolverType}
\label{sec:DynamicSolverType}
This section shows the DynamicSolverType structure, which is used for selecting dynamic solvers for simulation.



\begin{center}
\footnotesize
\begin{longtable}{| p{8cm} | p{8cm} |} 
\hline
{\bf function/structure name} & {\bf description}\\ \hline
  GeneralizedAlpha & an implicit solver for index 3 problems; intended to be used for solving directly the index 3 constraints using the spectralRadius sufficiently small (usually 0.5 .. 1)\\ \hline  
  TrapezoidalIndex2 & an implicit solver for index 3 problems with index2 reduction; uses generalized alpha solver with settings for Newmark with index2 reduction\\ \hline  
  ExplicitEuler & an explicit 1st order solver (generally not compatible with constraints)\\ \hline  
  ExplicitMidpoint & an explicit 2nd order solver (generally not compatible with constraints)\\ \hline  
  RK33 & an explicit 3 stage 3rd order Runge-Kutta method, aka "Heun third order"; (generally not compatible with constraints)\\ \hline  
  RK44 & an explicit 4 stage 4th order Runge-Kutta method, aka "classical Runge Kutta" (generally not compatible with constraints), compatible with Lie group integration and elimination of CoordinateConstraints\\ \hline  
  RK67 & an explicit 7 stage 6th order Runge-Kutta method, see 'On Runge-Kutta Processes of High Order', J. C. Butcher, J. Austr Math Soc 4, (1964); can be used for very accurate (reference) solutions, but without step size control!\\ \hline  
  ODE23 & an explicit Runge Kutta method with automatic step size selection with 3rd order of accuracy and 2nd order error estimation, see Bogacki and Shampine, 1989; also known as ODE23 in MATLAB\\ \hline  
  DOPRI5 & an explicit Runge Kutta method with automatic step size selection with 5th order of accuracy and 4th order error estimation, see  Dormand and Prince, 'A Family of Embedded Runge-Kutta Formulae.', J. Comp. Appl. Math. 6, 1980\\ \hline  
  DVERK6 & [NOT IMPLEMENTED YET] an explicit Runge Kutta solver of 6th order with 5th order error estimation; includes adaptive step selection\\ \hline  
  VelocityVerlet & [TEST phase] a special explicit time integration scheme, the 'velocity Verlet' method (similar to leap frog method), with second order accuracy for conservative second order differential equations, often used for particle dynamics and contact; implementation uses Explicit Euler for ODE1 equations\\ \hline  
\end{longtable}
\end{center}

%++++++++++++++++++++
\mysubsubsection{CrossSectionType}
\label{sec:CrossSectionType}
This section shows the CrossSectionType structure, which is used for defining beam cross section types.



\begin{center}
\footnotesize
\begin{longtable}{| p{8cm} | p{8cm} |} 
\hline
{\bf function/structure name} & {\bf description}\\ \hline
  Polygon & cross section profile defined by polygon\\ \hline  
  Circular & cross section is circle or elliptic\\ \hline  
\end{longtable}
\end{center}

%++++++++++++++++++++
\mysubsubsection{KeyCode}
\label{sec:KeyCode}
This section shows the KeyCode structure, which is used for special key codes in keyPressUserFunction.



\begin{center}
\footnotesize
\begin{longtable}{| p{8cm} | p{8cm} |} 
\hline
{\bf function/structure name} & {\bf description}\\ \hline
  SPACE & space key\\ \hline  
  ENTER & enter (return) key\\ \hline  
  TAB & \\ \hline  
  BACKSPACE & \\ \hline  
  RIGHT & cursor right\\ \hline  
  LEFT & cursor left\\ \hline  
  DOWN & cursor down\\ \hline  
  UP & cursor up\\ \hline  
  F1 & function key F1\\ \hline  
  F2 & function key F2\\ \hline  
  F3 & function key F3\\ \hline  
  F4 & function key F4\\ \hline  
  F5 & function key F5\\ \hline  
  F6 & function key F6\\ \hline  
  F7 & function key F7\\ \hline  
  F8 & function key F8\\ \hline  
  F9 & function key F9\\ \hline  
  F10 & function key F10\\ \hline  
\end{longtable}
\end{center}

%++++++++++++++++++++
\mysubsubsection{LinearSolverType}
\label{sec:LinearSolverType}
This section shows the LinearSolverType structure, which is used for selecting linear solver types, which are dense or sparse solvers.



\begin{center}
\footnotesize
\begin{longtable}{| p{8cm} | p{8cm} |} 
\hline
{\bf function/structure name} & {\bf description}\\ \hline
  \_None & no value; used, e.g., if no solver is selected\\ \hline  
  EXUdense & use dense matrices and according solvers for densly populated matrices (usually the CPU time grows cubically with the number of unknowns)\\ \hline  
  EigenSparse & use sparse matrices and according solvers; additional overhead for very small multibody systems; specifically, memory allocation is performed during a factorization process\\ \hline  
  EigenSparseSymmetric & use sparse matrices and according solvers; NOTE: this is the symmetric mode, which assumes symmetric system matrices; this is EXPERIMENTAL and should only be used of user knows that the system matrices are (nearly) symmetric; does not work with scaled GeneralizedAlpha matrices; does not work with constraints, as it must be symmetric positive definite\\ \hline  
  EigenDense & use Eigen's LU factorization with partial pivoting (faster than EXUdense) or full pivot (if linearSolverSettings.ignoreSingularJacobian=True; is much slower)\\ \hline  
\end{longtable}
\end{center}

%++++++++++++++++++++
\mysubsubsection{ContactTypeIndex}
\label{sec:ContactTypeIndex}
This section shows the ContactTypeIndex structure, which is in GeneralContact to select specific contact items, such as spheres, ANCFCable or triangle items.



\begin{center}
\footnotesize
\begin{longtable}{| p{8cm} | p{8cm} |} 
\hline
{\bf function/structure name} & {\bf description}\\ \hline
  IndexSpheresMarkerBased & spheres attached to markers\\ \hline  
  IndexANCFCable2D & ANCFCable2D contact items\\ \hline  
  IndexTrigsRigidBodyBased & triangles attached to rigid body (or rigid body marker)\\ \hline  
  IndexEndOfEnumList & signals end of list\\ \hline  
\end{longtable}
\end{center}
